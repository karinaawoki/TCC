%\title{LaTeX Portrait Poster Template}
%%%%%%%%%%%%%%%%%%%%%%%%%%%%%%%%%%%%%%%%%
% a0poster Portrait Poster
% LaTeX Template
% Version 1.0 (22/06/13)
%
% The a0poster class was created by:
% Gerlinde Kettl and Matthias Weiser (tex@kettl.de)
% 
% This template has been downloaded from:
% http://www.LaTeXTemplates.com
%
% License:
% CC BY-NC-SA 3.0 (http://creativecommons.org/licenses/by-nc-sa/3.0/)
%
%%%%%%%%%%%%%%%%%%%%%%%%%%%%%%%%%%%%%%%%%

%----------------------------------------------------------------------------------------
%   PACKAGES AND OTHER DOCUMENT CONFIGURATIONS
%----------------------------------------------------------------------------------------

\documentclass[a0,portrait]{a0poster}
\usepackage[utf8]{inputenc}
\usepackage[T1]{fontenc}
\usepackage[svgnames,dvipsnames,x11names]{xcolor} 
\usepackage[brazilian]{babel}
\usepackage{array}
\usepackage[framemethod=TikZ]{mdframed}
\usepackage{amsmath, amsthm, amssymb}
\usepackage{multicol} % This is so we can have multiple columns of text side-by-side
\columnsep=100pt % This is the amount of white space between the columns in the poster
\columnseprule=3.7pt % This is the thickness of the black line between the columns in the poster

\usepackage[svgnames]{xcolor} % Specify colors by their 'svgnames', for a full list of all colors available see here: http://www.latextemplates.com/svgnames-colors

\usepackage{times} % Use the times font
\usepackage{url}
\usepackage{graphicx} % Required for including images
\graphicspath{{figures/}} % Location of the graphics files
\usepackage{booktabs} % Top and bottom rules for table
\usepackage[font=small,labelfont=bf]{caption} % Required for specifying captions to tables and figures
\usepackage{amsfonts, amsmath, amsthm, amssymb} % For math fonts, symbols and environments
\usepackage{wrapfig} % Allows wrapping text around tables and figures
\usepackage{enumitem}
\usepackage{colortbl}
\usepackage{multirow}
\usepackage{mathtools,booktabs}
\DeclarePairedDelimiter\ceil{\lceil}{\rceil}

\newtheorem{teo}{Teorema}
\newtheorem{lem}{Lema}
\newtheorem{prov}{Prova}
\newtheorem{alg}{Análise do algoritmo}
\newtheorem{prob}{Problema}
\newtheorem{coro}{Corolário}

\newcommand{\caminho}{\mathrm{-caminho}}
\newcommand{\dist}{\mathrm{dist}}
\newcommand{\diam}{\mathrm{diam}}
\newcommand{\grau}{\mathrm{grau}}
\newcommand{\rot}{\mathrm{rot}}
\newcommand{\map}{\mathrm{map}}
\newcommand{\Oh}{\mathrm{O}}

\newcolumntype{P}[1]{>{\centering\arraybackslash}p{#1}}


\mdfdefinestyle{MyFrame}{%
    linecolor=blue,
    outerlinewidth=2pt,
    roundcorner=15pt,
    innertopmargin=\baselineskip,
    innerbottommargin=\baselineskip,
    innerrightmargin=15pt,
    innerleftmargin=15pt,
    backgroundcolor=blue!20
}
    
\begin{document}

%----------------------------------------------------------------------------------------
%   POSTER HEADER 
%----------------------------------------------------------------------------------------

% The header is divided into two boxes:
% The first is 75% wide and houses the title, subtitle, names, university/organization and contact information
% The second is 25% wide and houses a logo for your university/organization or a photo of you
% The widths of these boxes can be easily edited to accommodate your content as you see fit

\begin{minipage}[b]{0.15\linewidth}
\includegraphics[width=10.cm]{usp.png}\\
\end{minipage}
\begin{minipage}[b]{0.70\linewidth}
\VeryHuge \center \color{NavyBlue} \textbf{Estudo do Problema da Bissecção Mínima em Árvores} \color{Black}\\ % Title
[0.68cm]
\huge \textbf{Karina Suemi Awoki}\\[0.5cm] % Author(s)
\huge \textbf{Orientadora: Cristina Gomes Fernandes}\\[0.5cm] % Author(s)

\huge Universidade de São Paulo, Instituto de Matemática e Estatística\\[0.2cm] %University/organization
\Large \texttt{karina.awoki@usp.br ---} 
\Large \texttt{\url{https://linux.ime.usp.br/~karina/mac0499/}}\\
\end{minipage}
%
\begin{minipage}[b]{0.15\linewidth}
\includegraphics[width=10.cm]{imeusp.png}\\
\end{minipage}

\vspace{1.cm} % A bit of extra whitespace between the header and poster content

%----------------------------------------------------------------------------------------

\begin{multicols}{3} % This is how many columns your poster will be broken into, a portrait poster is generally split into 2 columns

%----------------------------------------------------------------------------------------
%   ABSTRACT
%----------------------------------------------------------------------------------------

\color{Navy} % Navy color for the abstract

%\begin{abstract}
%The electrical energy from renewables in Algeria contributed about 3.4\% (280 MW) in 2008 of a total power of 8.1 GWe and will reach 5\% by the year 2017 according to the Algerian Electricity and Gas Regulation Commission (CREG). The country’s target is reaching 40\% by 2030. The geothermal resources in Algeria are of low-enthalpy type. Most of these geothermal resources are located in the north of the country and generate a heat discharge of 240 MWt.
%\end{abstract}




  

%----------------------------------------------------------------------------------------
%   INTRODUCTION
%----------------------------------------------------------------------------------------
%\begin{mdframed}[style=MyFrame]
\color{Black} % SaddleBrown color for the introduction
%\color{Navy} % SaddleBrown color for the conclusions to make them stand out
\section*{Introdução}
    O assunto abordado se
    enquadra na área de Otimização Combinatória e diz respeito ao 
    \emph{Problema da Bissecção Mínima}. Para definir o problema 
    precisamente, seguem primeiro algumas definições. 

    Seja~${G=(V,E)}$ um grafo e~$B$ e~$W$ conjuntos de vértices de~$G$.
    Dizemos que~$(B,W)$ é um \textbf{corte}
    de~$G$ se~${B \cup W = V}$ e~${B\cap W =\emptyset}$.
    Denotamos o número de arestas no corte~$(B,W)$ por \textbf{largura}
    do corte ou por~$e_G(B,W)$.
    O corte~$(B,W)$ é uma \textbf{bissecção} se~${|B| =|W|}$
    quando~$|V|$ é par, ou se~${||B|-|W|| =1}$ quando~$|V|$ é ímpar.

    O \emph{Problema da Bissecção Mínima} consiste em, dado um grafo, 
    encontrar uma bissecção no grafo de largura mínima.

    Sabe-se que o problema da bissecção é NP-difícil%~\cite{GareyJS76}
    e a melhor aproximação conhecida para o caso geral do problema tem 
    razão~$O(\lg n)$, onde~$n$ é o número de vértices 
    do grafo. 
    Por outro lado, sabe-se que, para árvores e grafos que de uma 
    certa maneira se assemelham a árvores, há um algoritmo polinomial 
    de programação dinâmica para encontrar uma bissecção mínima, 
    proposto por Jansen, Karpinski, Lingas e Seidel. 
%\end{mdframed}

\noindent\rule[0.5ex]{\linewidth}{1pt}


%----------------------------------------------------------------------------------------
%   Objetivos
%----------------------------------------------------------------------------------------

\color{Black} % DarkSlateGray color for the rest of the content


\section*{Objetivos}

    O principal objetivo deste trabalho foi o estudo, implementação e 
    análise de um algoritmo recente, proposto por Fernandes, Schmidt e 
    Taraz, para encontrar uma bissecção 
    aproximadamente mínima em árvores de grau limitado. Denotaremos
    esse algoritmo por \textbf{FST}~\cite{Schmidt15}. 
    A vantagem deste algoritmo sobre o algoritmo de \textbf{Jansen} et 
    al., que encontra uma bissecção de largura 
    %al.~\cite{JansenKLS01}, que encontra uma bissecção de largura 
    mínima, é que este tem um consumo de tempo linear, enquanto que o 
    segundo tem um consumo de tempo cúbico no número de vértices da 
    árvore. 

\noindent\rule[0.5ex]{\linewidth}{1pt}


%----------------------------------------------------------------------------------------
%   LEMAS
%----------------------------------------------------------------------------------------
\section* {Lemas de Cortes Aproximados}

\begin{lem}[Lema do Corte Aproximado Simples]
    Para toda floresta~$G$ com~$n$ vértices e todo~${m \in [n]}$,
    existe um corte~$(B,W)$ em~$G$ tal 
    que~${\dfrac{m}{2} <|B| \le m}$ e~${e_G(B,W) \le \Delta(G)}$.
    Um corte que satisfaz esses requisitos pode ser computado em
    tempo~$O(n)$.
\end{lem}

\medskip

\begin{lem}[Lema do Corte Aproximado]
    Para toda floresta~$G$ com~$n$ vértices, todo~${m \in [n]}$ e 
    todo~${c \in [0,1)}$, existe um corte~$(B,W)$ em~$G$ tal 
    que~${cm \le |B| \le m}$ 
    e~${e_G(B,W) \le \ceil[\Big]{ \dfrac{2c}{1-c}} \Delta(G)}$.
    Um corte que satisfaz esses requisitos pode ser computado em
    tempo~$O{\Big(\ceil[\Big]{ \dfrac{2c}{1-c}} n+1\Big)}$.
\end{lem}

\noindent\rule[0.5ex]{\linewidth}{1pt}


%----------------------------------------------------------------------------------------
%   ROTULAÇÃO
%----------------------------------------------------------------------------------------

\section*{Rotulação}

    Dada uma árvore~$T$, atribuiremos um 
    rótulo numérico~${x\in [n]}$ distinto para cada um dos vértices 
    da árvore, onde~$n$ é o número de vértices de~$T$. 
    Denotaremos o \textbf{rótulo} de um vértice~$v$ por~$\rot(v)$.

    Primeiramente, calcularemos 
    um caminho~$P$, que é máximo em~$T$. 
    Para todo vértice~${v\in V(P)}$, temos uma 
    árvore~${T_v \in G - E(P)}$ com~$v\in T_v$.

    Dessa forma, teremos sub-árvores enraizadas nos vértices do
    caminho máximo, como o mostrado na figura abaixo, onde denotamos
    os vértices de~$P$ por~${v_0,v_1,\ldots,v_p}$.
    
\begin{center}
\includegraphics[width=1.0\linewidth]{000.png}
\end{center}

    Temos que cada vértice contido no caminho~$P$ receberá 
    uma rotulação de forma que:
    \begin{itemize}
        \item~${\rot(v_i)>\rot(v_j)}$ para todo~${i<j}$ e
        \item~${\rot(v_{i-1})<\rot(v') \le \rot(v_i)}$ para todo 
        vértice~$v'$ contido na sub-árvore~$T{v_i}$. 
    \end{itemize}
    
    Chamaremos de~${\map^+_m}$ a função que devolve o~$m$-ésimo 
    próximo vértice na rotulação, considerando que os rótulos
    estão dispostos de forma circular
    e, de~${\map^-_m}$ a função inversa da primeira. 
    Ou seja,~${\map^-_m}$ devolve o~$m$-ésimo vértice anterior.

\noindent\rule[0.5ex]{\linewidth}{1pt}


%----------------------------------------------------------------------------------------
%   TEOREMA DOBRA DIAMETRO
%----------------------------------------------------------------------------------------

\section*{Teorema da Dobra do Diâmetro}

\begin{teo}[]
    Para toda árvore~$T$ com~${n\ge 3}$ vértices e 
    todo~${m\in [n]}$,
    o conjunto de vértices de~$T$ pode ser particionado em 
    três partes~$B$,~$W$ e~$S$ tal que vale um dos 
    seguintes casos:
    \begin{enumerate}
    \item ${|B|=m}$, ${S=\emptyset}$, e~${e_T(B,W)\le 2}$, ou
    \item ${|B|\le m\le |B|+|S|}$, 
    com~${0<|S|\le\frac{n}{2}}$,
    ${e_T(B,W,S)\le \frac{2\cdot 
    \Delta(T)}{\diam^*(T)}}$, 
    e~${\diam^*(T[S])\ge 2\diam^*(T)}$.
    \end{enumerate}
    Uma partição~$(B,W,S)$ que satisfaz 1 ou 2 pode ser
    computada em tempo~${O\Big(\dfrac{n}{\diam^*(T)}\Big)}$.
\end{teo}


\subsection*{Caso 1.}
    Existe um vértice ~${v\in V(P)}$ tal 
    que~${\map_m^+(v)\in V(P)}$.
\begin{center}
\includegraphics[width=1.0\linewidth]{case_1.png}
\end{center}

\subsection*{Caso 2.}
    Caso contrário, sabemos que para todo vértice~${v\in V(P)}$
    teremos que~${\map_m^+(v)\not\in V(P)}$.

    Dizemos que um vértice~${z\in V(P)}$ é \textbf{b-especial}
    (ou~${z\in B_{esp}}$)
    se existe um vértice~${x\in T_z}$ tal 
    que~${\map^{-}_m(x)\in V(P)}$, e ele é
    \textbf{f-especial} (ou~${z\in F_{esp}}$) 
    se existe um vértice~${x\in T_z}$
    tal que~${\map^{+}_m(x)\in V(P)}$.
    Assumiremos que~$T'_z$ equivale a~$T_z\setminus \{z\}$.

    \bigskip
    \bigskip

    Seja um vértice~${z\in V(P)}$ tal que~$z$ é b-especial e
    sendo~$x$ o menor vértice de~$T_z$ tal 
    que~${\map^-_m(x)\in V(P)}$, temos 
    que~${z^b = \map^-_m(x)}$,
    ~${P_z^b = \map^-_m(V(T_z))\cap V(P)}$
    e~${H_z^f =T_u'}$ para todo 
    vértice~${u\in P_z^b}$ com~${u\ne z^b}$.

    \begin{center}
\includegraphics[width=1.0\linewidth]{bspecial.png}
\end{center}

    \bigskip
    \bigskip

    De forma análoga, para um vértice~${z\in V(P)}$ tal 
    que~$z$ é f-especial,
    temos que~${z^f = \map^+_m(x)}$,
    sendo~$x$ o menor vértice de~$T_z$ tal 
    que~${\map^+_m(x)\in V(P)}$.
    Também temos
    que~${P_z^f = \map^+_m(V(T_z))\cap V(P)}$
    e~${H_z^f =T_u'}$ para todo 
    vértice~${u\in P_z^f}$ com~${u\ne z^f}$.

\begin{center}
    \includegraphics[width=1.0\linewidth]{fspecial.png}
\end{center}
    \subsubsection*{Caso 2.a)}
        É válido que~${\displaystyle\sum_{z\in B_{esp}}
        \Big(|T'_{z}|+|H_z^b|\Big)\le
        \dfrac{1-d}{d}\displaystyle\sum_{z\in B_{esp}}|P_z^b|}$,
        implicando na existência de um vértice~${z\in B_{esp}}$ 
        tal que~${|T'_{z}|+|H_z^b|\le
        \dfrac{1-d}{d}|P_z^b|}$.

\begin{center}
    \includegraphics[width=1.0\linewidth]{5caso2a.png}
\end{center}
    \subsubsection*{Caso 2.b)}
        Caso contrário,~${\displaystyle\sum_{z\in F_{esp}}
        \Big(|T'_{z}|+|H_z^f|\Big)\le
        \dfrac{1-d}{d}\displaystyle\sum_{z\in F_{esp}}|P_z^f|}$
        e, isso implica na existência de um vértice~${z\in F_{esp}}$ 
        tal 
        que~${|T'_{z}|+|H_z^f|\le
        \dfrac{1-d}{d}|P_z^f|}$.

\begin{center}
\includegraphics[width=1.0\linewidth]{6caso2b.png}
\end{center}
\noindent\rule[0.5ex]{\linewidth}{1pt}

%----------------------------------------------------------------------------------------
%   TEOREMA DOBRA DIAMETRO
%----------------------------------------------------------------------------------------
\section*{Teorema de FST}

\begin{teo}[]
\label{teo:corteExato}
    Para todo~${i\in \mathbb{N^+_*}}$, toda árvore~$T$ com~$n$
    vértices e~${\diam^*(T)>\dfrac{1}{2^i}}$, e todo~${m\in[n]}$,
    existe um corte~$(B,W)$ em~$T$ com~${|B|=m}$ 
    e~$e_T(B,W)\le 4\cdot 2^i\cdot \Delta(T)$.
    Um corte que satisfaz esses requisitos pode ser computado
    em tempo~${O\Big(\dfrac{n}{\diam^*(T)}\Big)}$.
\end{teo}

\noindent\rule[0.5ex]{\linewidth}{1pt}


%----------------------------------------------------------------------------------------
%   REDUÇÂO
%----------------------------------------------------------------------------------------
\section*{Redução}

    Estilos de grafos da redução Max Sat2~(Satisfatibilidade
    máxima 
    com no máximo duas literais por cláusula) para
    SimpleMaxCut~(Corte Máximo Simples).
\begin{center}
\includegraphics[width=0.85\linewidth]{8red_2.png}
%\includegraphics[width=0.5\linewidth]{arrumado.png}
\end{center}
%\includegraphics[width=1.0\linewidth]{8red_2.png}

Estilo de grafo da redução SimpleMaxCut para Bissecção Mínima.
    
\begin{center}
\includegraphics[width=0.84\linewidth]{ult.png}
\end{center}
\noindent\rule[0.5ex]{\linewidth}{1pt}

%----------------------------------------------------------------------------------------
%   TABELAS
%----------------------------------------------------------------------------------------
\section*{Resultados}
\subsubsection*{Árvores Ternárias}

    \begin{tabular}{| c | c | >{\columncolor{blue!14}}P{3.6cm} | >{\columncolor{red!14}}P{3.6cm} | >{\columncolor{blue!14}}P{5cm} | >{\columncolor{red!14}}P{5.1cm} | }

    \specialrule{1.7pt}{1pt}{1pt}
        Altura & $|V(T)|$ & $e_T(B,W)$ \textbf{FST} & $e_T(B,W)$ \textbf{Jansen} & Tempo \textbf{FST} (segundos) & Tempo \textbf{Jansen} (segundos)  \\[10pt]

    \specialrule{1.7pt}{1pt}{1pt}
        1 & 4    & 2  & 2  & 0.000091  &   0.000009 \\ [3pt]
        2 & 13   & 8  & 3  & 0.000177  &   0.000033 \\ [3pt]
        3 & 40   & 10 & 4  & 0.000173  &   0.000336 \\ [3pt]
        4 & 121  & 11 & 4  & 0.000450  &   0.006730 \\ [3pt]
        5 & 364  & 10 & 5  & 0.000568  &   0.071686 \\ [3pt]
        6 & 1093 & 11 & 6  & 0.001301  &   1.655212 \\ [3pt]
        7 & 3280 & 13 & 7  & 0.004268  &  44.435073 \\ [3pt]
        8 & 9841 & 14 & 8  & 0.011514  &1193.373235 \\ [3pt]

    \specialrule{1.7pt}{1pt}{1pt}    
\end{tabular}



\subsubsection*{Árvores de Caminho Longo}

\begin{tabular}{| c | >{\columncolor{blue!14}}P{4.6cm} | >{\columncolor{red!14}}P{4.5cm} | >{\columncolor{blue!14}}P{5.5cm} | >{\columncolor{red!14}}P{5.5cm} |}

            \specialrule{1.7pt}{1pt}{1pt}
            $|V(T)|$ & $e_T(B,W)$ \textbf{FST} & $e_T(B,W)$ \textbf{Jansen} & Tempo \textbf{FST} (segundos) & Tempo \textbf{Jansen}   (segundos) \\[10pt]

    \specialrule{1.7pt}{1pt}{1pt}

        100  & 2  &  1  & 0.000080  &   0.000663 \\ [3.2pt] 
        200  & 2  &  1  & 0.000114  &   0.004626 \\ [3.2pt]
        400  & 2  &  1  & 0.000181  &   0.037905 \\ [3.2pt]
        800  & 2  &  1  & 0.000327  &   0.250858 \\ [3.2pt]
        1600 & 2  &  2  & 0.00 584  &   2.136476 \\ [3.2pt]
        3200 & 2  &  2  & 0.001118  &  17.872311 \\ [3.2pt]
        6400 & 2  &  2  & 0.002568  & 130.088130 \\ [3.2pt]
       12800 & 2  &  2  & 0.006036  &1339.149309 \\ [3.2pt]

    \specialrule{1.7pt}{1pt}{1pt}
         
\end{tabular}

\bigskip

\noindent\rule[0.5ex]{\linewidth}{1pt}


%----------------------------------------------------------------------------------------
%   CONCLUSIONS
%----------------------------------------------------------------------------------------

\color{SaddleBrown} % SaddleBrown color for the conclusions to make them stand out

%\section*{Conclusions}


%\color{Black} % Set the color back to DarkSlateGray for the rest of the content

%----------------------------------------------------------------------------------------
\bibliographystyle{plain}
\bibliography{refs}


\end{multicols}
\end{document}