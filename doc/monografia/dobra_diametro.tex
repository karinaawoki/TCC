\section {Algoritmo da dobra do diâmetro}

	\subsection{Teorema da dobra do diâmetro}
	
		\begin{teo}[{\cite[Teorema 4]{Schmidt15}}]
		\label{teo:dobraDiametro}
			Para toda árvore~$T$ com~$n\ge 3$ vértices e 
			todo~$m\in [n]$,
			o conjunto de vértices de~$T$ pode ser particionado em 
			três partes~$B$,~$W$ e~$S$ tal que vale um dos 
			seguintes itens:
			\begin{enumerate}
				\item ${|B|=m}$, ${S=\emptyset}$, e~${e_T(B,W)\le 2}$, ou
				\item ${|B|\le m\le |B|+|S|}$, 
				com~${0<|S|\le\dfrac{n}{2}}$,
				${e_T(B,W,S)\le \dfrac{2\cdot 
				\Delta(T)}{\diam^*(T)}}$, 
				e~${\diam^*(T[S])\ge 2\diam^*(T)}$.
			\end{enumerate}
			Uma partição~$(B,W,S)$ que satisfaz 1 ou 2 pode ser
			computada em tempo~${O\Big(\dfrac{n}{\diam^*(T)}\Big)}$ 
		\end{teo}

	\bigskip
	\bigskip

		% Primeiramente, faremos uma série de definições para que
		% depois possamos explicar melhor como o teorema e o algoritmo
		% funcionam.

		% Usaremos o que foi explicado na seção~\ref{sec:caminhMaximo}
		% para encontrármos um caminho máximo. 
		% Depois, usaremos 

		Primeiramente, temos uma árvore~$T$ com~$n\ge 3$ vértices e um 
		inteiro~$m\in[n]$.
		Seja~$P$ um caminho máximo de~$T$ e sabendo que os vértices
		estão rotulados como o descrito na seção~\ref{sec:rotulacao}. 

	\bigskip
	\bigskip
	\bigskip


	\begin{center} \begin{tikzpicture}
		[scale=.7,auto=left,every node/.style={circle, draw=black,
				fill=white!70}]

		\draw [decorate,decoration={brace,amplitude=10pt,mirror,
		raise=4pt}, draw=brown, line width=1.5pt] 
		(8,5.7) -- (2,5.7);

		\draw [decorate,decoration={brace,amplitude=10pt,mirror,
		raise=4pt}, draw=yellow, line width=1.5pt] 
		(4,2.7) -- (8.5,2.7);

		\draw [draw=yellow, fill=yellow!20] (18.1,3.7) 
		ellipse (1.9cm and 3.7cm); 

		\node [text=blue!65,draw=none, fill=none] at 
		(17.7,6.8) {$T_z$};


		

		\node [scale=1.3,subtree,fill=red!50](y1) at (2.5,5) {};
		\node [subtree, fill=yellow!40]      (y2) at (5,5)   {};
		\node [subtree, fill=yellow!40]      (y3) at (7.5,5) {};
		\node [scale=1.3,subtree,fill=red!50](y4) at (10,5)  {};
		\node [subtree, fill=Emerald]        (y5) at (12.5,5){};
		\node [subtree, fill=Emerald]        (y6) at (15,5)  {};
		\node [scale=1.9, subtree,fill=CornflowerBlue] (y7) at (18.1,5)  {};

		\node (x0) at (0,5)  {...};
		\node [fill=brown](x1) at (2.5,5)    {};
		\node [fill=brown](x2) at (5,5)  {};
		\node [fill=brown](x3) at (7.5,5)    {};
		\node (x4) at (10,5) {};
		\node (x5) at (12.5,5)   {};
		\node (x6) at (15,5) {};
		\node (x7) at (18.1,5) {};
		\node (x8) at (20.6,5) {...};

		\node [draw=none, fill=none] at 
		(2,2) {\footnotesize{$\map^-_w(v)$}};
		\node [draw=none, fill=none](nv) at (2,2){};
		\node [draw=none, fill=none](nvt) at (2.5,3.6){};

		\node [draw=none, fill=none] at 
		(9.5,2) {\footnotesize{$\map^-_w(z)$}};
		\node [draw=none, fill=none](nz) at (9.5,2){};
		\node [draw=none, fill=none](nzt) at (10,3.6){};

		\node [draw=none, fill=none] at 
		(15.5,2.6) {\footnotesize{$x$}};
		\node [draw=none, fill=none](z) at (15.5,2.6){};
		\node [draw=none, fill=none](zt) at (18,2.6){};

		\node [draw=none, fill=none] at (2.6,5.7){$z^b$};
		\node [draw=none, fill=none] at (18.1,5.6){$z$};
		\node [draw=none, fill=none] at (15,5.6){$v$};


		\foreach \from/\to in {x0/x1,x1/x2,x2/x3,x3/x4,x4/x5,x5/x6,x6/x7,x7/x8}
		\draw (\from) -- (\to);




		\path[->, line width=1pt]
		(nv) edge[left=23 ] (nvt);
		\path[->, line width=1pt]
		(z) edge[right=45] (zt);
		\path[->, line width=1pt]
		(nz) edge[right=45] (nzt);




	\end{tikzpicture} \end{center}

	\subsection{Algoritmo da dobra do diâmetro}
	\begin{algorithm}[H]
	\label{alg:dobraDiametro}
		\SetKwInOut{Input}{input}
		\SetKwInOut{Output}{output}

		\caption{}
		\Input{árvore $T$ com $n$ vértices e ${m\in[n]}$}
		\Output{}

	\end{algorithm}	