\section {Algoritmo de Jansen}

O algoritmo de Jansen é um algoritmo de 
programação dinâmica que sempre encontra
a solução ótima para o problema da bissecção
mínima aplicado a grafos planares.
Seu consumo de tempo é~$O(n^3)$.

Como o foco deste trabalho é o problema da 
bissecção mínima aplicado a árvores, e dado
que árvores são um tipo específico de grafos
planares, então, nesta seção, analisaremos a 
fórmula de recorrência específica para encontrar
a solução desse problema em árvores. 

\bigskip

\subsection*{Fórmula de recorrência}

Seja~$T$ uma árvore com~$n$ vértices. 
Para todo~${v\in V(T)}$, todo~${m\in[n]}$
e sendo~$T_v$ a sub-árvore enraizada em~$v$,
temos que a função~$b(v,w)$ representa o valor
de~$e_{T_v}(B,W)$, onde~${w=|B|}$,~${v\in B}$
e~$(B,W)$ é o corte de largura mínima em~$T_v$.
O mesmo vale para~$w(v,m)$, mudando somente
o fato de que, neste caso,~$v\in W$.

Sendo assim, para todo vértice~$v$, temos que:

\begin{itemize}
\item Se~$v$ é uma folha de~$T$, então

\bigskip

$ b(v,m)~ = 
\begin{cases}
	0, &m=1 \\
	\infty, &m\ne 1
\end{cases}$

\medskip

$w(v,m) = 
\begin{cases}
	0, &m=0 \\
	\infty, &m\ne 0,
\end{cases}$

dado que, como~$v$ é uma folha, então

\item Caso contrário, sendo~$u_1,u_2,\cdots,u_k$
filhos de~$v$, teremos que

\end{itemize}