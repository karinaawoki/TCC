\section {Algoritmo de Jansen}

O algoritmo de Jansen é um algoritmo de 
programação dinâmica que sempre encontra
a solução ótima para o problema da bissecção
mínima aplicado a grafos planares.
Seu consumo de tempo é~$O(n^3)$.

Como o foco deste trabalho é o problema da 
bissecção mínima aplicado a árvores, e dado
que árvores são um tipo específico de grafos
planares, então, nesta seção, analisaremos a 
fórmula de recorrência específica para encontrar
a solução desse problema em árvores. 

\bigskip
\bigskip

\subsection*{Fórmula de recorrência}

Seja~$T$ uma árvore com~$n$ vértices. 
Para todo~${v\in V(T)}$, todo~${m\in[n]}$
e sendo~$T_v$ a sub-árvore que contém~$v$
e todos os seus descendentes,
temos que a função~$b(v,w)$ representa o valor
de~$e_{T_v}(B,W)$, onde~${w=|B|}$,~${v\in B}$
e~$(B,W)$ é o corte de largura mínima em~$T_v$.
O mesmo vale para a função~$w(v,m)$, mudando somente
o fato de que, neste caso,~$v\in W$.
Ou seja,~$w(v,w)$ representa o valor
de~$e_{T_v}(B,W)$, onde~${w=|B|}$,~${v\in W}$
e~$(B,W)$ é o corte de largura mínima em~$T_v$.
Temos também que as soluções inviáveis recebem~$\infty$
como valor de corte.

Sendo assim, para todo vértice~$v$, é válido que:

\begin{itemize}
\item Se~$v$ é uma folha de~$T$, então

\bigskip

$ b(v,m)~ = 
\begin{cases}
	0, &m=1 \\
	\infty, &m\ne 1
\end{cases}$

\medskip

$w(v,m) = 
\begin{cases}
	0, &m=0 \\
	\infty, &m\ne 0,
\end{cases}$

dado que, como~$v$ é uma folha,
então os únicos
cortes possíveis em~$T_v$ são~${(B,W) = (\{v\},\emptyset)}$ 
e~${(B,W) = (\emptyset, \{v\})}$.

Temos que a função~$b(v,m)$ será factível somente 
quando~$v\in B$,
então só teremos soluções válidas quando~${m=1}$.
Já a função~$w(v,m)$ só será factível quando~$v\in W$,
logo, o único valor viável para~$m$ é~$0$.
Ambass as soluções não possuem arestas no corte, dado que
a árvores~$T_v$
possui um único vértice.

\item Caso contrário, o vértice~$v$ não é uma folha. 
Sendo assim, temos que~$u_1,u_2,\ldots,u_k$ são
os vértices filhos de~$v$ e 
que~$T_{u_i}$ é a sub-árvore que contém~$u_i$ e 
todos os seus descendentes.
Temos também que~${T_{v^i} = T_{u_i}\cup \{v\}\cup \{v,u_i\}} $,
como mostrado na figura abaixo.
	\begin{center} \begin{tikzpicture}
		[scale=.48,auto=left,every node/.style={circle, draw=black,
				fill=white!70}]

		\draw [draw=Emerald, fill=Emerald!25,line width=3pt]  (8.8-14,4) 
		ellipse (1.8cm and 2.7cm);
		\draw [draw=NavyBlue, fill=NavyBlue!20,line width=3pt](0.2-14,4) 
		ellipse (1.8cm and 2.7cm);
		
		\node [text=NavyBlue,draw=none, fill=none] at 
		(0.2-14,0.5) {$\mathbf{T_{u_1}}$};
		\node [text=Emerald,draw=none, fill=none] at 
		(8.8-14,0.5) {$\mathbf{T_{u_k}}$};

		\draw [draw=YellowOrange,fill=red!25,line width=3pt] (4.5,6.) 
		ellipse (1.1cm and 3.6cm);
		\draw [rotate around={-138:(7,5.8)}, draw=Emerald, fill=Emerald!25,line width=3pt] (7,5.8) 
		ellipse (2.cm and 5.8cm);
		\draw [rotate around={138:(2,5.8)}, draw=NavyBlue, fill=NavyBlue!20,line width=3pt] (2,5.8) 
		ellipse (2.cm and 5.8cm);

		\node [text=NavyBlue,draw=none, fill=none] at 
		(2.3-3,0.5) {$\mathbf{T_{v^1}}$};
		\node [text=Emerald,draw=none, fill=none] at 
		(6.7+3,0.5) {$\mathbf{T_{v^k}}$};

		\draw [draw=red!60,fill=none,line width=3pt] (4.5,6.) 
		ellipse (1.1cm and 3.6cm);
		\draw [rotate around={138:(2,5.8)}, draw=NavyBlue, fill=none,line width=3pt] (2,5.8) 
		ellipse (2.0cm and 5.8cm);
		\draw [rotate around={-138:(7,5.8)}, draw=Emerald, fill=none,line width=3pt] (7,5.8) 
		ellipse (2.0cm and 5.8cm);

		\node [scale=0.9,subtree,fill=NavyBlue!50](y1) at (0.2 , 5) {};
		\node [scale=0.9, subtree,fill=Emerald!50](y7) at (8.8 ,5)  {};

		\node (main) at (4.5, 8){$v$};

		\node (x0) at (0.2 ,5)  {$u_1$};
		\node (x1) at (4.5 ,4) {$...$};
		\node (x3) at (8.8 ,5) {$u_k$};


		\node [scale=0.9,subtree,fill=NavyBlue!50](y1) at (0.2-14 , 5) {};
		\node [scale=0.9, subtree,fill=Emerald!50](y7) at (8.8-14 ,5)  {};

		\node (main_) at (4.5-14, 8){$v$};

		\node (x0_) at (0.2-14 ,5)  {$u_1$};
		\node (x1_) at (4.5-14 ,4) {$...$};
		\node (x3_) at (8.8-14 ,5) {$u_k$};


		\foreach \from/\to in {main/x0, main/x1, %main/x2, 
		main/x3,
		main_/x0_, main_/x1_, %main/x2, 
		main_/x3_}
		\draw (\from) -- (\to);

	\end{tikzpicture} \end{center}
	Para facilitar o entendimento das funções~$b(v,m)$ 
	e~$w(v,m)$, vamos quebrá-las usando mais dois pares 
	de fórmulas.

	O primeiro par é~$\tilde{b}_i(v,w)$ 
	e~$\tilde{w}_i(v,w)$. 
	Essas funções recebem um 
	vértice~$v$ e um inteiro~${m\in [n]}$ como parâmetros 
	e devolvem a largura do corte mínimo~$(B,W)$ 
	em~$T_{v^i}$ tal que~${m = |B|}$. 
	Em~$\tilde{b}_i(v,m)$ temos que~${v\in B}$ e
	em~$\tilde{w}_i(v,m)$ temos que~${v\in W}$.
	
	Essas funções podem ser representadas por
	\begin{align*}
		\tilde{b}_i(v,m) &= \min \{ b(u_i,m-1),\ w(u_i, m-1)+1 \} \nonumber \\
		\tilde{w}_i(v,m) &= \min \{ b(u_i,m)+1,\ w(u_i, m) \}. \nonumber
	\end{align*}
	A função~$\tilde{b}_i(v,m)$ assume que~${v\in B}$, 
	logo, para obter um conjunto~$B$ com~$m$ vértices
	é necessário que o corte~$(B,W)$ em~$T_{v^i}$ 
	seja tal que~$B$ tenha~${m-1}$ vértices. 
	Caso~${v_i\in W}$,
	sabemos que a aresta~${\{ v,u_i \}}$ estará
	no corte~$(B,W)$ (dado que~$v\in B$), e é por isso que 
	acrescentamos~$1$ ao valor de~${w(u_i, m-1)}$.
	Caso contrário,~$v$ e~$u_i$ pertencerão ao
	sub-conjunto~$B$ e nenhuma nova aresta
	será adiciona ao corte.
	Se~${u_i\in B}$, sabemos que a aresta~$\{ v,u_i \}$
	será adicionada ao corte~$(B,W)$ e, caso contrário,
	os dois vértices estarão em~$W$ e nenhuma aresta será
	adicionada.

	Já na função~$\tilde{w}_i(v,m)$, temos que~${v\in W}$.
	Portanto, precisamos encontrar 
	um corte~$(B,W)$ em~$T_{v^i}$ 
	tal que~$B$ tenha exatamente~$m$ vértices.
	Adicionaremos uma aresta ao corte~$(B,W)$ somente
	se~$u_i\in B$.

	\bigskip

	O segundo par de funções é~$\tilde{\tilde{b}}_i(v,m)$
	e~$\tilde{\tilde{w}}_i(v,m)$.
	Essas funções recebem um vértice~$v$ e um~$m\in [n]$ e
	devolvem um corte mínimo na 
	árvore~${T_{v^1}\cup T_{v^2}\cup \cdots \cup T_{v^i}}$
	de forma que~${v\in B}$ na primeira função
	e~${v\in W}$ na segunda.
	A condição inicial de ambas é quando~$i=1$, e nesse caso
	temos que 
	\begin{align*}
		\tilde{\tilde{b}}_1(v,m) =  \tilde{b}_1(v,m)\ \ \text{ e }\ \
		\tilde{\tilde{w}}_1(v,m) =  \tilde{w}_1(v,m), \nonumber
	\end{align*}
	pois o corte devolvido será o referente a árvore~$T_{v^1}$.

	Sendo~$m_i = |B|$ em~$T_{v^i}$,
	temos que as fórmulas gerais dessas funções irão verificar todas as 
	combinações possíveis de valores de~$m_1,m_2,\ldots,m_i$,
	%para cada uma das sub-árvores 
	%contidas em, 
	de forma
	que a união dos sub-conjuntos~$B$ de cada uma das 
	sub-árvores~$ T_{v^1},T_{v^2},\ldots,T_{v^i} $
	seja igual a~$m$.
	%que a soma desses valores seja igual a~$m$. ???
	
	Por exemplo, para calcular o valor 
	de~$\tilde{\tilde{w}}_i(v,m)$ com~${m = 2}$ e~${i=3}$,
	verificaremos todas as possíveis combinações de~$m_1,m_2$ e~$m_3$
	%para cada uma das sub-árvores~$T_{v^1}$,~$T_{v^2}$ e~$T_{v^3}$
	de forma que a união dos sub-conjuntos~$B$ referentes as
	árvores~$T_{v^1}$,~$T_{v^2}$ e~$T_{v^3}$
	tenha exatamente~$m=2$ vértices.
	Segue abaixo uma tabela com as possíveis combinações
	de~~$m_1,m_2$ e~$m_3$ referentes a este exemplo.
	\begin{table}[h]
		\centering
		\begin{tabular}{| c | c | c | c | c|}
		%\begin{tabular}{c | >{\columncolor{blue!14}}c | >{\columncolor{red!14}}c | >{\columncolor{blue!14}}c | >{\columncolor{red!14}}c }
			\specialrule{1.7pt}{1pt}{1pt}
%			& $T_{v^1}$ & $T_{v^2}$ & $T_{v^1}$ & $T_{v^1}\cup T_{v^2}\cup T_{v^3}$  \\[3pt]
			& $m_1$ & $m_2$ & $m_3$ & $m$  \\[3pt]

			\specialrule{1.7pt}{1pt}{1pt}
			  	  & 0  & 0  & 2  & 2 \\ [3pt]
				  & 0  & 2  & 0  & 2 \\ [3pt]
			 $m$  & 2  & 0  & 0  & 2 \\ [3pt]
				  & 1  & 1  & 0  & 2 \\ [3pt]
				  & 1  & 0  & 1  & 2 \\ [3pt]
				  & 0  & 1  & 1  & 2 \\ [3pt]
			\specialrule{1.7pt}{1pt}{1pt}
		 
		\end{tabular}
	\end{table}

	Sendo assim, será devolvida a combinação de valores de~$m_1,m_2,\ldots,m_i$ 
	de forma que o corte~$(B,W)$ em~$T_{v^1}\cup T_{v^2}\cup\cdots\cup T_{v^i}$  
	possua o menor número de arestas no corte.

	Como temos que, para cada valor 
	de~$m$,~$\tilde{\tilde{b}}_{i-1}(v,m)$
	e~$\tilde{\tilde{w}}_{i-1}(v,m)$ representam
	o corte mínimo na 
	árvore~${T_{v^1}\cup T_{v^2}\cup \cdots \cup T_{v^{i-1}}}$,
	logo, sabemos que esse corte mínimo foi
	a melhor combinação de valores de~$m$.
	Com isso, ao invés de testar todas as combinações
	para cada sub-árvore em~$\{ T_{v^1}, T_{v^2},\ldots, T_{v^i} \}$,
	testaremos 
	%Testaremos~$m=0$ em~$T_{v^1}$ com~$m=2$ em~$T_{v^2}$,~$m=1$ 
	%em~$T_{v^1}$ com~$m=1$ em~$T_{v^2}$ 
	%e~$m=2$ em~$T_{v^1}$ com~$m=0$ em~$T_{v^2}$.
	\begin{align*}
		\tilde{\tilde{b}}_{i}(v,m) &= 
			\min_{0\le\ell\le m} \{ \tilde{\tilde{b}}_{i-1}(v,\ell) + 
			\tilde{b}_i(v, m-\ell +1) \} \nonumber \\
		\tilde{\tilde{w}}_{i}(v,m) &= 
			\min_{0\le\ell\le m} \{ \tilde{\tilde{w}}_{i-1}(v,\ell) + 
			\tilde{w}_i(v, m-\ell) \} \nonumber \\
	\end{align*}
	Neste caso, as funções~$b(v,m)$ e~$w(v,m)$ são da forma
	\begin{align*}
		b(v,m) = \tilde{\tilde{b}}_k(v,m)\ \ \text{ e }\ \
		w(v,m) = \tilde{\tilde{w}}_k(v,m) \nonumber
	\end{align*}
	e, sendo~$r$ a raíz de~$T$, temos que
	 o corte mínimo~$(B,W)$ em~$T$, onde~${|B|=m}$, 
	é~$\min\{b(r,m), w(r,m)\}$.
\end{itemize}