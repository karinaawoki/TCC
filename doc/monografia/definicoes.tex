\section{Definições}
	
	\subsection*{Conjunto~$[n]$}
	Denotemos por~$\{1,2,\ldots~n\}$
	o conjunto~$[n]$ .

	\subsection*{Conjunto de vértices e arestas}
	Para um grafo~$G$, denotamos seu
	conjunto de vértices por~$V(G)$ e seu
	conjunto de arestas por~$E(G)$.

	\subsection*{Grau dos vértices}
	Em um grafo~$G$, o número de arestas que incidem em um 
	determinado vértice~$v$ é chamado de 
	\textbf{grau} de~$v$ e será representado
	por~$\grau_G(v)$. O 
	\textbf{grau máximo}, que é o grau de um vértice de
	maior grau em~$G$, será representado por~$\Delta(G)$.

	\subsection*{Caminhos em árvores}
	Um \textbf{caminho} em uma árvore~$T$ é uma sequência de 
	vértices~$v_0, v_1, \ldots,v_{p-1}, v_p$ 
	onde~$v_{k-1}, v_k$ é uma aresta para 
	todo~${k =1,2, \ldots, p}$. 
	O número~$p$ é o \textbf{comprimento} do caminho.
	Como existe um único caminho que conecta quaisquer dois 
	vértices~$v_0$ e~$v_p$ em~$T$, chamaremos tal caminho em~$T$ 
	de~$v_0,v_p\caminho$.

	Um \textbf{caminho máximo} em uma árvore~$T$ é um caminho 
	em~$T$ de comprimento máximo.

	\subsection*{Distâncias entre vértices}
	A \textbf{distância} entre dois vértices~$a$ e~$b$ de~$T$ é 
	representada por~$\dist(a,b)$, e é igual ao comprimento 
	do~$a,b\caminho$ em~$T$.
