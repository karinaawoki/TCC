\section {Algoritmo da dobra do diâmetro}
	
	\subsection{Teorema da dobra do diâmetro}
	
		\begin{teo}[{\cite[Teorema 4]{Schmidt15}}]
		\label{teo:dobraDiametro}
			Para toda árvore~$T$ com~$n\ge 3$ vértices e 
			todo~$m\in [n]$,
			o conjunto de vértices de~$T$ pode ser particionado em 
			três partes~$B$,~$W$ e~$S$ tal que vale um dos 
			seguintes itens:
			\begin{enumerate}
				\item ${|B|=m}$, ${S=\emptyset}$, e~${e_T(B,W)\le 2}$, ou
				\item ${|B|\le m\le |B|+|S|}$, 
				com~${0<|S|\le\dfrac{n}{2}}$,
				${e_T(B,W,S)\le \dfrac{2\cdot 
				\Delta(T)}{\diam^*(T)}}$, 
				e~${\diam^*(T[S])\ge 2\diam^*(T)}$.
			\end{enumerate}
			Uma partição~$(B,W,S)$ que satisfaz 1 ou 2 pode ser
			computada em tempo~${O\Big(\dfrac{n}{\diam^*(T)}\Big)}$ 
		\end{teo}

	\medskip
	\medskip


	\subsection{Algortimo da dobra do diâmetro}
	\begin{algorithm}[H]
	\label{alg:dobraDiametro}
		\SetKwInOut{Input}{input}
		\SetKwInOut{Output}{output}

		\caption{}
		\Input{árvore $T$ com $n$ vértices e ${m\in[n]}$}
		\Output{}

	\end{algorithm}	