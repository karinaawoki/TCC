\section {Rotulação dos vértices da árvore}
\label{sec:rotulacao}

	Mais adiante, iremos nos referir a vértices com uma 
	característica específica, e para isso precisamos atribuir um 
	rótulo numérico~$x\in [n]$ distinto para cada um dos vértices 
	da árvore, onde~$n$ é o número de vértices da árvore~$T$. 
	Denotaremos o \textbf{rótulo} de um vértice~$v$ por~$\rot(v)$.

	Primeiramente, usaremos o método mencionado na 
	seção~\ref{sec:caminhoMaximo} para calcular 
	um caminho~$P$, que é máximo em~$T$. 
	Para todo vértice~$v\in V(P)$, temos uma 
	árvore~$T_v \in G - E(P)$ com~$v\in T_v$.

	Dessa forma, teremos sub-árvores enraizadas nos vértices do
	caminho máximo, como o mostrado na figura abaixo, onde denotamos
	os vértices de~$P$ por~${v_0,v_1,\ldots,v_p}$.
	

 	%% fill=blue!30 -- núemro indica quão escuro é
	\begin{center} \begin{tikzpicture}
		[scale=.7,auto=left,every node/.style={circle, draw=black,
				fill=blue!20}]

		\draw [draw=yellow, fill=yellow!20] (0,4.8) 
		ellipse (1.3cm and 2.4cm); %0
		\draw [draw=yellow, fill=yellow!20] (3,4.4) 
		ellipse (1.3cm and 2.4cm); %1
		\draw [draw=yellow, fill=yellow!20] (6,4.8) 
		ellipse (1.3cm and 2.4cm); %2
		\draw [draw=yellow, fill=yellow!20] (9,4.4) 
		ellipse (1.3cm and 2.4cm); %3
 		\draw [draw=yellow, fill=yellow!20] (12,4.8) 
 		ellipse (1.3cm and 2.4cm); %4
		\draw [draw=yellow, fill=yellow!20] (18,4.8) 
		ellipse (1.3cm and 2.4cm); %5

		\node [text=blue!65,draw=none, fill=none] at 
		(-0.4,6.5) {$T_{v_0}$};
		\node [text=blue!65,draw=none, fill=none] at 
		(2.6,6.1) {$T_{v_1}$};
		\node [text=blue!65,draw=none, fill=none] at 
		(5.6,6.5) {$T_{v_2}$};
		\node [text=blue!65,draw=none, fill=none] at 
		(8.6,6.1) {$T_{v_3}$};
		\node [text=blue!65,draw=none, fill=none] at 
		(11.6,6.5) {$T_{v_4}$};
		\node [text=blue!65,draw=none, fill=none] at 
		(17.6,6.5) {$T_{v_p}$};

		\node [subtree] (y1) at (3,4.6)  {};
		\node [subtree, fill=red!30] (y2) at (6,5)  {};
		\node [subtree] (y3) at (9,4.6)  {};
		\node [subtree] (y4) at (12,5) {};

		\node (x0) at (0,5.4)  {$v_0$};
		\node (x1) at (3,5)    {$v_1$};
		\node (x2) at (6,5.4)  {$v_2$};
		\node (x3) at (9,5)    {$v_3$};
		\node (x4) at (12,5.4) {$v_4$};
		\node (x5) at (15,5)   {...};
		\node (x6) at (18,5.4) {$v_p$};


		\foreach \from/\to in {x0/x1,x1/x2,x2/x3,x3/x4,x4/x5,x5/x6}
		\draw (\from) -- (\to);

	\end{tikzpicture} \end{center}

	\bigskip

	Nota-se que as árvores~$T_{v_0}$ e~$T_{v_p}$ possuem um único
	vértice cada uma. 
	Isso ocorre porque~$v_0$ e~$v_p$ são vértices dos extremos
	do caminho máximo, e se houvesse algum vértice~$v_\ell$ em uma
	dessas sub-árvores,~$v_\ell$ faria parte do caminho máximo
	e~$v_0$ ou~$v_p$ não seria um dos extremos, o que levaria a 
	uma contradição.

	\bigskip

	Temos que cada vértice contido no caminho~$P$ receberá 
	uma rotulação de forma que:
	\begin{itemize}
		\item~$\rot(v_i)>\rot(v_j)$ para todo~$i<j$ e
		\item~${\rot(v_{i-1})<\rot(v') \le \rot(v_i)}$ para todo 
		vértice~$v'$ contido na sub-árvore~$T{v_i}$. 
	\end{itemize}

	\bigskip
	\bigskip

	\subsection{Descrição do algoritmo de rotulação}
	Essa rotulação pode ser obtida facilmente 
	manipulando as listas de adjacências de ~$T$ e em seguida,
	usando uma busca em profundidade.
	Ambas as ações são executadas em tempo linear em~$n$, então
	pode-se rotular uma árvore em tempo~$O(n)$.

	Para cada vértice~$v_i$ contido no caminho~$P$, com~$i>0$,
	percorremos a sua lista de adjacências e, quando encontrarmos o 
	vértice~$v_{i-1}$, o colocamos no início da lista.
	Dessa forma, ao aplicarmos uma busca em profundidade nessa 
	árvore usando o~$v_p$ como raiz, a busca descerá primeiro
	pelos vértices do caminho máximo e atribuirá a rotulação
	a um vértice do caminho
	somente quando terminar de ver todos os vértices da sub-árvore
	do vértice, 
	garantindo assim que para todo~$i\in \{1,\ldots, p\}$, temos 
	que~$\rot(v_{i-1})<\rot(j)\le\rot(v_i)$ para todo~$j$ 
	em~$V(T_{v_i})$.
	
	\bigskip
	\bigskip
	\bigskip

	\subsection{Mapeamento dos rótulos de vértices }
	Vamos definir agora as funções que mapeiam um vértice em 
	outro, utilizando a rotulação como base.
	Chamaremos de~${\map^+_m}$ a função que devolve o~$m$-ésimo 
	próximo vértice na rotulação, considerando que os rótulos
	estão dispostos de forma circular
	e, de~${\map^-_m}$ a função inversa da primeira. 
	Ou seja,~${\map^-_m}$ devolve o~$m$-ésimo vértice anterior.
	Dessa forma, podemos ver que, para um 
	vértice~$v$,~${\map^+_m(\rot(v)) =(\rot(v)+m)\mod n}$ 
	e~${\map^-_m(\rot(v)) =(\rot(v)-m)\mod n}$.

	Para facilitar, identificaremos os vértices pelos seus 
	rótulos. 
	Sendo assim, temos que, para um 
	vértice~$v$, ~${\map^+_m(v) =(v+m)\mod n}$ 
	e~${\map^-_m(v) =(v-m)\mod n}$ como uma simplificação das 
	funções enunciadas anteriormente.

	Note que, para todo $m\in [n]$, caso~$n>2$, se existem
	dois vértices~$v$ e~$v'$, ambos no~$v_0,v_p\caminho$, 
	tal que~${v'= \map^+_m(v)}$, então o corte $(B,W)$ em~$T$ 
	com~${B =\{v+1, v+2,\ldots,~v+m\}}$ é tal que~${|B|=m}$
	e~${e_T(B,W)\le 2\le \Delta(T)}$.
	Isso pode ser visto na figura abaixo.

	\begin{center} \begin{tikzpicture}
		[scale=.7,auto=left,every node/.style={circle, draw=black,
				fill=blue!20}]

		\draw [draw=yellow, fill=yellow!20] (7.5,1.5) rectangle 
		(13.5, 6.5);

		\node [text=blue!65,draw=none, fill=none] at 
		(10.5,2.2) {\Large{$m$ vértices}};
		\node [draw=none, fill=none] at 
		(9,6.5) {\footnotesize{$\map^+_m$}};
		\node [draw=none, fill=none] at 
		(9,7.9) {\footnotesize{$\map^-_m$}};


		\node [subtree] (y1) at (3,4.6)  {};
		\node [subtree, fill=red!30] (y2) at (6,5)  {};
		\node [subtree] (y3) at (9,4.6)  {};
		\node [subtree] (y4) at (12,5) {};

		\node (x0) at (0,5.4)  {$v_0$};
		\node (x1) at (3,5)    {...};
		\node (x2) at (6,5.4)  {$v$};
		\node (x3) at (9,5)    {...};
		\node (x4) at (12,5.4) {$v'$};
		\node (x5) at (15,5)   {...};
		\node (x6) at (18,5.4) {$v_p$};


		\foreach \from/\to in {x0/x1,x1/x2,x3/x4,x5/x6}
		\draw (\from) -- (\to);

		\path[-, line width=1.5pt, draw=red]
		(x2) edge[right] (x3);
		\path[-, line width=1.5pt, draw=red]
		(x4) edge[right] (x5);

		\path[->, line width=1.5pt]
		(x2) edge[bend left=23 ] (x4);
		\path[->, line width=1.5pt]
		(x4) edge[bend right=45, looseness=1.7 ] (x2);


	\end{tikzpicture} \end{center}

	
	Caso~$n\le 2$, 
	é fácil ver que todos os vértices
	da árvore estarão no caminho máximo, portanto, para
	qualquer~$m\in[n]$, teremos um conjunto~$B$ tal
	que~$|B|=m$
	e~$e_T(B,W)\le |E(T)|=n-1=\Delta(T)\le 2$.

	\bigskip
	\bigskip

	\subsubsection*{Algumas definições}

	Para os vértices~${a,b,c\in V(T)}$, dizemos que~$b$
	\textbf{está entre}~$a$ e~$c$ se~${a=b}$,~${b=c}$ 
	ou se chegamos em~$b$ antes
	de~$c$ quando aumentamos o valor de~$a$  
	gradativamente.
	Dados os vértices~${x,z\in V(P)}$, dizemos que 
	se~${x+1\in T_z}$, então, o
	vértice~$z$ \textbf{vem depois de~$x$ em~$P$} 
	e~$x$ \textbf{vem antes de~$z$ em~$P$}.


	Um vértice~${z\in V(P)}$ é \textbf{b-especial}
	se existe um vértice~${x\in T_z}$ tal 
	que~${\map^{-}_m(x)\in V(P)}$, e ele é
	\textbf{f-especial} se existe um vértice~${x\in T_z}$
	tal que~${\map^{+}_m(x)\in V(P)}$.

	\bigskip
	\bigskip
	\bigskip
	\bigskip


%%%%%%%%%%%%%%%%%%%%%%%%%%%%%%%%%%%%%%%%%%%%%%%%%%%%%%%%%%%%%%%%%%%
%%%%%%%%%%%%%%%%%%%%%%%%%%%%%%%%%%%%%%%%%%%%%%%%%%%%%%%%%%%%%%%%%%%
%%%%%%%%%%%%%%%%%%%%%%%%%%%%%%%%%%%%%%%%%%%%%%%%%%%%%%%%%%%%%%%%%%%

	\subsection{Árvores de diâmetro grande}

	Futuramente usaremos os resultados do lema abaixo na prova de 
	um dos teoremas.

	\begin{lem}[{\cite[Lema 5]{Schmidt15}}]
	\label{lema:caminhoLongo}
		Para toda árvore~$T$ com~$n$ 
		vértices e~${\diam^*(T)>\dfrac{1}{2}}$
		e para todo~${m\in[n]}$, existe um corte~$(B,W)$ em~$T$
		com~${|B|=m}$ e~${e_T(B,W)\le 2}$. Um corte que satisfaz esses
		requisitos pode ser computado em tempo~$O(n)$.
	\end{lem}

	\medskip
	\medskip

	Podemos ver que a função~$\map^+_m(x)$ é injetora, dado que numa
	soma, elementos distintos do domínio tem imagens distintas.

	Uma árvore com a propriedade ~${\diam^*(T)>\dfrac{1}{2}}$ é uma
	árvore cujo caminho máximo contém mais que a metade dos 
	vértices da árvore, o que nos leva ao fato de que existem mais 
	vértices no caminho máximo do que fora dele. 

	Como o mostrado anteriormente, se um vértice~$v$ mapeia um
	vértice~$v'$ e ambos estão no caminho máximo, então existe um
	corte~$(B,W)$ em~$T$ com~${e_T(B,W)\le 2}$.
	Dado que~${\diam^*(T)>\dfrac{1}{2}}$, e usando o fato de 
	que~$\map^+_m(x)$ é uma função injetora e que
	existem mais vértices no caminho máximo do que fora dele,
	temos que pelo menos um vértice do caminho máximo irá mapear
	outro vértice do caminho máximo. Isso ocorre porque todos
	os vértices mapeiam algum outro vértice, e não temos um 
	número suficiente de vértices fora do caminho máximo para serem
	mapeados por todos os vértices do caminho máximo, então algum
	vértice do caminho máximo vai acabar mapeando outro do caminho
	máximo, e assim, obtemos um corte~$(B,W)$, com~$e_T(B,W)\le 2$,
	como o descrito na subseção anterior.

	Um algoritmo que encontra corte desse tipo, em árvores com 
	essa propriedade, irá percorrer os vértices do caminho máximo
	e verificar se algum deles mapeia um vértice do caminho máximo.
	Isso pode ser feito em tempo~$O(n)$ se for mantido um vetor 
	binário que indica se cada um dos vértices está ou não no
	caminho máximo.

	%{\color{red}{-O algoritmo é bem simples. Não precisa colocar
	%código né?}}