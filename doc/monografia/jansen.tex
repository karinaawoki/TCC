\section {Algoritmo de Jansen et al.}

O algoritmo de Jansen~et~al.~\cite{JansenKLS01} baseia-se 
em programação dinâmica e sempre encontra
uma solução ótima para o problema da bissecção
mínima quando o grafo dado é planar.
Seu consumo de tempo é~$O(n^3)$, onde~$n$ é
o número de vértices do grafo dado.

Como o foco deste trabalho é o problema da 
bissecção mínima em árvores, e dado
que toda árvore é um grafo
planar, nesta seção, apresentaremos a 
recorrência do algoritmo de Jansen et al.
especializada para árvores. 

\bigskip
\bigskip

%\subsection*{Fórmula de recorrência}

Seja~$T$ uma árvore com~$n$ vértices e com uma raiz~${r\in V(T)}$.
Dado um vértice~$u$, seja~$T_u$ a sub-árvore de~$T$ que contém~$u$
e todos os seus descendentes. 
Para todo~${v\in V(T)}$ e todo~${m\in[n]}$,
a função~$b(v,m)$ representa o valor
de~$e_{T_v}(B,W)$, onde~${m=|B|}$
e~$(B,W)$ é um corte de largura mínima em~$T_v$ com~${v\in B}$.
Se~${m>|V(T_v)|}$, então não há corte~$(B,W)$ em~$T_v$ com~${|B|=m}$
e estabelecemos que~${b(v,m) = \infty}$.
O mesmo vale para a função~$w(v,m)$, mudando somente
o fato de que, neste caso, exigimos que~${v\in W}$.
Ou seja,~$w(v,w)$ representa o valor
de~${e_{T_v}(B,W)}$, onde~${m=|B|}$
e~$(B,W)$ é um corte de largura mínima em~$T_v$ com~${v\in W}$.
Se~${m\ge |V(T_v)|}$, então~${w(v,m) = \infty}$.
% Diremos que~$b(v,m)$ e~$w(v,m)$ tem solução inviável quando
% não é possível encontrar um corte~$(B,W)$ com o valor de~$m$ dado,
% por exemplo, temos uma solução inviáveis
% quando~$m$ é maior que o número de descendentes de~$v$.
% Temos também que as soluções inviáveis recebem~$\infty$
% como valor da largura do corte.

Sendo assim, para todo vértice~$v$, é válido que:
\begin{itemize}
	\item Se~$v$ é uma folha de~$T$, então
	
	\medskip

	$ b(v,m)~ = 
	\begin{cases}
		0,     &\text{se }m=1 \\
		\infty,&\text{se }m\ne 1
	\end{cases}$

	\medskip

	$w(v,m) = 
	\begin{cases}
		0, &\text{se }m=0 \\
		\infty,&\text{se }m\ne 0,
	\end{cases}$

dado que, como~$v$ é uma folha,
os únicos
cortes possíveis em~$T_v$ são~${(B,W) = (\{v\},\emptyset)}$ 
e~${(B,W) = (\emptyset, \{v\})}$.

A função~$b(v,m)$ requer que~${v\in B}$,
então, só temos solução quando~${m=1}$.
Já a função~$w(v,m)$ requer que~${v\in W}$,
logo o único valor viável para~$m$ é~$0$.
Ambas as soluções não possuem arestas de~$T_v$ no corte, 
dado que a árvore~$T_v$
possui um único vértice.

\item Se~$v$ não é uma folha de~$T$,
chame de~$u_1,u_2,\ldots,u_k$
os filhos de~$v$. 
%e que~$T_{u_i}$ é a sub-árvore que contém~$u_i$ e 
%todos os seus descendentes.
Seja~$T_{v^i}$ a árvore~$T_{u_i}$ acrescida do vértice~$v$
e da aresta~$\{v,u_i\}$.
%${T_{u_i}\cup \{v\}\cup \{v,u_i\}} $,
como mostrado na figura abaixo.
	\begin{center} \begin{tikzpicture}
		[scale=.48,auto=left,every node/.style={circle, draw=black,
				fill=white!70}]

		\draw [draw=Emerald, fill=Emerald!25,line width=3pt]  (8.8-14,4) 
		ellipse (1.8cm and 2.7cm);
		\draw [draw=NavyBlue, fill=NavyBlue!20,line width=3pt](0.2-14,4) 
		ellipse (1.8cm and 2.7cm);
		
		\node [text=NavyBlue,draw=none, fill=none] at 
		(0.2-14,0.5) {$T_{u_1}$};
		\node [text=Emerald,draw=none, fill=none] at 
		(8.8-14,0.5) {$T_{u_k}$};

		\draw [draw=YellowOrange,fill=red!25,line width=3pt] (4.5,6.) 
		ellipse (1.1cm and 3.6cm);
		\draw [rotate around={-138:(7,5.8)}, draw=Emerald, fill=Emerald!25,line width=3pt] (7,5.8) 
		ellipse (2.cm and 5.8cm);
		\draw [rotate around={138:(2,5.8)}, draw=NavyBlue, fill=NavyBlue!20,line width=3pt] (2,5.8) 
		ellipse (2.cm and 5.8cm);

		\node [text=NavyBlue,draw=none, fill=none] at 
		(2.3-3,0.5) {$T_{v^1}$};
		\node [text=Emerald,draw=none, fill=none] at 
		(6.7+3,0.5) {$T_{v^k}$};

		\draw [draw=red!60,fill=none,line width=3pt] (4.5,6.) 
		ellipse (1.1cm and 3.6cm);
		\draw [rotate around={138:(2,5.8)}, draw=NavyBlue, fill=none,line width=3pt] (2,5.8) 
		ellipse (2.0cm and 5.8cm);
		\draw [rotate around={-138:(7,5.8)}, draw=Emerald, fill=none,line width=3pt] (7,5.8) 
		ellipse (2.0cm and 5.8cm);

		\node [scale=0.9,subtree,fill=NavyBlue!50](y1) at (0.2 , 5) {};
		\node [scale=0.9, subtree,fill=Emerald!50](y7) at (8.8 ,5)  {};

		\node (main) at (4.5, 8){$v$};

		\node (x0) at (0.2 ,5)  {$u_1$};
		\node (x1) at (4.5 ,4) {$...$};
		\node (x3) at (8.8 ,5) {$u_k$};


		\node [scale=0.9,subtree,fill=NavyBlue!50](y1) at (0.2-14 , 5) {};
		\node [scale=0.9, subtree,fill=Emerald!50](y7) at (8.8-14 ,5)  {};

		\node (main_) at (4.5-14, 8){$v$};

		\node (x0_) at (0.2-14 ,5)  {$u_1$};
		\node (x1_) at (4.5-14 ,4) {$...$};
		\node (x3_) at (8.8-14 ,5) {$u_k$};


		\foreach \from/\to in {main/x0, main/x1, %main/x2, 
		main/x3,
		main_/x0_, main_/x1_, %main/x2, 
		main_/x3_}
		\draw (\from) -- (\to);

	\end{tikzpicture} \end{center}
	Para facilitar o entendimento das funções~$b(v,m)$ 
	e~$w(v,m)$, vamos quebrá-las usando mais dois pares 
	de fórmulas.

	\medskip

	O primeiro par é~$\tilde{b}_i(v,m)$ 
	e~$\tilde{w}_i(v,m)$. 
	%Essas funções recebem um 
	%vértice~$v$ e um inteiro~${m\in [n]}$ como parâmetros 
	O valor dessas funções é a largura de um corte mínimo~$(B,W)$ 
	em~$T_{v^i}$ tal que~${m = |B|}$, sendo que 
	em~$\tilde{b}_i(v,m)$ exigimos que~${v\in B}$ e
	em~$\tilde{w}_i(v,m)$ exigimos que~${v\in W}$.
	
	Essas funções satisfazem o seguinte:
	\begin{align*}
		\tilde{b}_i(v,m) &= \min \{ b(u_i,m-1),\ w(u_i, m-1)+1 \} \nonumber \text{ e}\\ 
		\tilde{w}_i(v,m) &= \min \{ b(u_i,m)+1,\ w(u_i, m) \}. \nonumber
	\end{align*}
	A função~$\tilde{b}_i(v,m)$ requer que~${v\in B}$, 
	logo, para obter um conjunto~$B$ com~$m$ vértices
	é necessário que o corte~$(B',W')$ em~$T_{u_i}$ 
	seja tal que~${|B'|=m-1}$.
	Desta forma, tomamos~${B = B'\cup \{v\}}$ e~${W = W'}$,
	implicando em~${|B| = m}$. 
	Caso~${u_i\in W}$,
	a aresta~${\{ v,u_i \}}$ estará
	no corte~$(B,W)$ (dado que~${v\in B}$), e por isso 
	acrescentamos~$1$ ao valor de~${w(u_i, m-1)}$.
	Caso contrário,~$v$ e~$u_i$ pertencerão ao
	conjunto~$B$ e nenhuma nova aresta
	será adiciona ao corte.

	Já a função~$\tilde{w}_i(v,m)$ requer que~${v\in W}$.
	Para que tenhamos um conjunto~$B$ com~$m$ vértices,
	precisamos encontrar 
	um corte~$(B',W')$ em~$T_{u_i}$ 
	tal que~${|B'|=m}$.
	Assim, tomamos~${B = B'}$ e~${W = W'\cup \{v\}}$.
	Se~${u_i\in B}$, a aresta~$\{ v,u_i \}$
	estará no corte~$(B,W)$ e, caso contrário,
	os dois vértices estarão em~$W$ e nenhuma aresta será
	adicionada.
	
	\bigskip

	O segundo par de funções é~$\tilde{\tilde{b}}_i(v,m)$
	e~$\tilde{\tilde{w}}_i(v,m)$.
	O valor dessas funções 
	%recebem um vértice~$v$ e um~${m\in [n]}$ e devolvem 
	é a largura de um corte mínimo~$(B,W)$ na 
	árvore~${T_{v^1}\cup T_{v^2}\cup \cdots \cup T_{v^i}}$
	com~${v\in B}$ na primeira função
	e~${v\in W}$ na segunda.
	A condição inicial de ambas é quando~${i=1}$, e nesse caso
	temos que 
	\begin{align*}
		\tilde{\tilde{b}}_1(v,m) =  \tilde{b}_1(v,m)\ \ \text{ e }\ \
		\tilde{\tilde{w}}_1(v,m) =  \tilde{w}_1(v,m), \nonumber
	\end{align*}
	pois a árvore em questão é exatamente 
	%a largura do corte devolvido será a referente somente 
	a árvore~$T_{v^1}$.

	Seja~$(B_i,W_i)$ um corte de largura mínima em~$T_{v^i}$ 
	com~${|B_i| = m_i}$.
	%Sendo~$m_i = |B_i|$ em~$T_{v^i}$,
	%As fórmulas gerais dessas funções irão 
	Para calcular as funções~$b(v,m)$ e~$w(v,m)$, poderíamos
	verificar todas as 
	combinações possíveis de valores~${m_1,m_2,\ldots,m_i}$
	tais que~${m_1+m_2+\cdots+m_i = m}$,
	%para cada uma das sub-árvores 
	%contidas em, 
	de forma
	%a união dos subconjuntos~$B$ de cada uma das 
	%sub-árvores~$ T_{v^1},T_{v^2},\ldots,T_{v^i} $
	que~${|B_1\cup B_2\cup\cdots\cup B_i| = m}$.
	%seja igual a~$m$.
	%que a soma desses valores seja igual a~$m$. ???
	
	Por exemplo, para calcular o valor 
	de~$\tilde{\tilde{w}}_3(v,2)$ com~${m_1 + m_2 + m_3 = 2}$,
	verificaríamos todas as possíveis combinações de~$m_1,m_2$ e~$m_3$
	%para cada uma das sub-árvores~$T_{v^1}$,~$T_{v^2}$ e~$T_{v^3}$
	de forma 
	%que a união dos subconjuntos~$B$ referentes as
	%árvores~$T_{v^1}$,~$T_{v^2}$ e~$T_{v^3}$
	%tenha exatamente~$m=2$ vértices.
	que~${|B_1\cup B_2\cup B_3| = m = 2}$.
	Segue abaixo uma tabela com as possíveis combinações
	de~~$m_1,m_2$ e~$m_3$ referentes a este exemplo.
	\begin{table}[h]
		\centering
		\begin{tabular}{| c | c | c | c | c|}
		%\begin{tabular}{c | >{\columncolor{blue!14}}c | >{\columncolor{red!14}}c | >{\columncolor{blue!14}}c | >{\columncolor{red!14}}c }
			\specialrule{1.7pt}{1pt}{1pt}
%			& $T_{v^1}$ & $T_{v^2}$ & $T_{v^1}$ & $T_{v^1}\cup T_{v^2}\cup T_{v^3}$  \\[3pt]
			Combinação& $|B_1|$ & $|B_2|$ & $|B_3|$ & $|B_1\cup B_2\cup B_3|$  \\[3pt]

			\specialrule{1.7pt}{1pt}{1pt}
			  	1ª  & 0  & 0  & 2  & 2 \\ [3pt]
				2ª  & 0  & 2  & 0  & 2 \\ [3pt]
			 	3ª  & 2  & 0  & 0  & 2 \\ [3pt]
				4ª  & 1  & 1  & 0  & 2 \\ [3pt]
				5ª  & 1  & 0  & 1  & 2 \\ [3pt]
				6ª  & 0  & 1  & 1  & 2 \\ [3pt]
			\specialrule{1.7pt}{1pt}{1pt}
		 
		\end{tabular}
	\end{table}

	Assim seria calculado o número de arestas em um
	corte~$(B,W)$ em~${T_{v^1}\cup T_{v^2}\cup\cdots\cup T_{v^i}}$  
	de forma que~${B=B_1\cup B_2\cup \cdots\cup B_i}$ 
	e que a combinação de valores 
	de~$m_1,m_2,\ldots,m_i$ 
	leve ao número mínimo de arestas no corte.

	Há no entanto um outro modo mais eficiente de fazermos isso.

	Seja~$(B_{i-1}^*,W_{i-1}^*)$ um corte de largura mínima
	em~${T_{v^1}\cup T_{v^2}\cup \cdots \cup T_{v^{i-1}}}$
	com~${|B|_{i-1}^* = m_{i-1}^*}$.
	Como temos que, para cada valor 
	de~$m$, as funções~$\tilde{\tilde{b}}_{i-1}(v,m)$
	e~$\tilde{\tilde{w}}_{i-1}(v,m)$ representam
	a largura de um corte mínimo~$(B_{i-1}^*,W_{i-1}^*)$ 
	com~${|B_{i-1}^*|=m}$,~${v\in B_{i-1}^*}$
	e~${v\in W_{i-1}^*}$, respectivamente, 
	%mínimo na árvore~${T_{v^1}\cup T_{v^2}\cup \cdots \cup T_{v^{i-1}}}$,
	sabemos que esse corte de largura mínima vem da
	melhor combinação de valores de~$m_1,m_2,\ldots,m_{i-1}$.
	Com isso, em vez de testar todas as combinações
	de~$m_1,m_2,\ldots,m_{i-1}$
	para cada sub-árvore em~$\{ T_{v^1}, T_{v^2},\ldots, T_{v^i} \}$,
	testaremos apenas combinações de valores de~$m_i$ e~$m_{i-1}^*$
	referentes a~$T_{v^i}$ e 
	a~$ {T_{v^1}\cup T_{v^2}\cup\cdots\cup T_{v^{i-1}}}$, respectivamente.
	Isso irá gerar um corte~$(B,W)$ 
	em~${T_{v^1}\cup T_{v^2}\cup\cdots\cup T_{v^i}}$
	tal que~${B = B_{i-1}^* \cup B_i}$ 
	e~${W = W_{i-1}^* \cup W_i}$,
	onde~$m_i$ e~$m_{i-1}^*$ formam uma melhor combinação de valores.
	Especificamente, as fórmulas gerais de~$\tilde{\tilde{b}}_{i}(v,m)$
	e~$\tilde{\tilde{w}}_{i}(v,m)$ são
	%Testaremos~$m=0$ em~$T_{v^1}$ com~$m=2$ em~$T_{v^2}$,~$m=1$ 
	%em~$T_{v^1}$ com~$m=1$ em~$T_{v^2}$ 
	%e~$m=2$ em~$T_{v^1}$ com~$m=0$ em~$T_{v^2}$.
	\begin{align*}
		\tilde{\tilde{b}}_{i}(v,m) &= 
			\min_{0\le\ell\le m} \{ \tilde{\tilde{b}}_{i-1}(v,\ell) + 
			\tilde{b}_i(v, m-\ell +1) \} \nonumber \text{ e}\\
		\tilde{\tilde{w}}_{i}(v,m) &= 
			\min_{0\le\ell\le m} \{ \tilde{\tilde{w}}_{i-1}(v,\ell) + 
			\tilde{w}_i(v, m-\ell) \}. \nonumber
	\end{align*}
	Pode-se ver que em~$\tilde{\tilde{b}}_{i}(v,m)$
	temos que~${m_i = m-\ell+1}$ 
	e que~${m_{i-1}^* = \ell}$
	e, como~${B = B_{i-1}^* \cup B_i}$ e~${v\in B}$, 
	então~${|B| = m_i + m_{i-1}^* -1 = m}$.


	Na função~$\tilde{\tilde{w}}_{i}(v,m)$, 
	temos que~${m_i = m-\ell}$ e que~${m_{i-1}^* = \ell}$
	e, como~${B = B_{i-1}^* \cup B_i}$ e~${v\in W}$, 
	então~${|B| = m_i + m_{i-1}^*= m}$.

	\bigskip

	Logo, sendo~$k$ o número de filhos de~$v$,
	as funções~$b(v,m)$ e~$w(v,m)$ são dadas por
	\begin{align*}
		b(v,m) = \tilde{\tilde{b}}_k(v,m)\ \ \text{ e }\ \
		w(v,m) = \tilde{\tilde{w}}_k(v,m). \nonumber
	\end{align*}

	Seja~$r$ a raiz de~$T$. 
	A largura de um corte mínimo~$(B,W)$ em~$T$, onde~${|B|=m}$, 
	é~$\min\{b(r,m), w(r,m)\}$.
\end{itemize}


	\bigskip
	\bigskip
	\bigskip
