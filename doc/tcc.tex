\documentclass[a4paper,12pt]{article}

% Para usar a linguagem PT-BR.
\usepackage[utf8]{inputenc}
\usepackage[T1]{fontenc}
\usepackage{tikz}
\usepackage[brazilian]{babel}

\selectlanguage{brazilian} 

\newcommand{\Oh}{\mathrm{O}}

\sloppy

\begin{document}
\begin{center}
   {\large \textbf{UNIVERSIDADE DE SÃO PAULO}} \\[1.4cm]
   
   {\large \textbf{INSTITUTO DE MATEMÁTICA E ESTATÍSTICA}}\\[4.2cm]
   
   {\Huge TRABALHO DE CONCLUSÃO }\\[0.3cm]
   {\Huge DE CURSO }\\[9cm]
   
   {\large { Aluna: Karina Suemi Awoki}}\\[0.3cm]
   
   {\large { Orientadora: Cristina Gomes Fernandes}}
   

\end{center}

\newpage
\section{Introdução}
..
\bigskip

\section{Definições}
    
    \subsection{Caminho máximo em árvores}

    	Um \textbf{caminho} em uma árvore $T=(V,E)$, de 
    	comprimento $p$, é uma sequência de vértices 
    	$v_0, v_1, ...,v_{p-1}, v_p$ 
    	onde {$v_{k-1}, v_k$} é uma aresta para todo 
    	$k = 1,2,..., p$. 
    	Como existe um único caminho que conecta dois vértices,
    	então chamaremos um caminho $v_0, v_1, ...,v_{p-1}, v_p$ de
    	$v_0$,$v_p$-$path$.

    	Portanto, um \textbf{caminho máximo} em uma árvore $T$ é
    	um caminho de $T$ cujo comprimento é máximo.

    	Encontrar um caminho máximo em uma árvore $T$ é uma tarefa
    	computacionalmente simples que pode ser realizada em tempo 
    	$O(n)$. 
    	Primeiramente, é necessário escolher um vértice aleatório 
    	$v \in V$ e encontrar $y_0$,
    	que é o vértice mais distante de $v$.
    	Depois, repetimos o mesmo processo em $y_0$, encontrar
    	$x_0$ que é o vértice mais distante de $y_0$. 
    	Para encontrar o vértice mais distante de algum vértice, 
    	basta usar uma busca em largura.  
    	Feito isso, temos um caminho máximo de $T$, que é o único
    	caminho que liga $x_0$ a $y_0$.


    	Agora provaremos que, de fato, $x_0$,$y_0$-$path$ é um
    	caminho máximo.
    	Suponhamos que $x$,$y$-$path$ seja um caminho máximo e:

    	\begin{itemize}
            \item \textbf{Caso 1:} Os caminhos $x$,$y$-$path$ e 
            $x_0$,$y_0$-$path$ não possuem vértices em comum.

            Logo, sabemos que existe um caminho $a$,$b$-$path$, 
            de comprimento $c \ge 1$,
            que funciona como uma conexão entre os dois caminhos,
            possuindo assim apenas o vértice $a$ em 
            $x$,$y$-$path$ e apenas o vértice $b$ em
            $x_0$,$y_0$-$path$.

            Portanto temos,
            $$ dist(x,y) = dist(x,a) + dist(y,a) $$ e
            $$ dist(x_0,y_0) = dist(x_0,b) + dist(y_0,b).$$

            Como $x$,$y$-$path$ é um caminho máximo,
            $$ dist(x,y)\ge dist(y,x_0) \Rightarrow 
            dist(x,a)+dist(y,a)\ge dist(y,a)+c+dist(x_0,b)
            \Rightarrow$$
            $$ dist(x,a)\ge c+dist(x_0,b) $$

            e como $x_0$ é o vértice mais distante de $y_0$, 
            temos que 
            $$ dist(x_0,y_0)\ge dist(x,y_0) \Rightarrow 
            dist(x_0,b) + dist(y_0,b)\ge dist(y_0,b) + c + 
            dist(x,a) \Rightarrow $$
            $$ dist(x_0,b)\ge c + dist(x,a). $$ 
            Logo, temos que
            $$ dist(x,a)\ge 2c + dist(x, a). $$

            Dado que $c\ge 1$, entramos numa contradição. 
            Portanto, $x$,$y$-$path$ e $x_0$,$y_0$-$path$
            possuem vértices em comum.

            \begin{tikzpicture}
				[scale=.8,auto=left,every node/.style={circle,
				fill=blue!20}]
				\node (x) at (0,4)  {x};
				\node (y) at (12,4) {y};
				\node (a) at (6,4)  {a};
				  
				\node (x0) at (0,0) {$x_0$};
				\node (y0) at (12,0) {$y_0$};
				\node (b)  at (6,0)  {b};

				\node (n1) at (3,4) {...};
				\node (n2) at (9,4) {...};
				\node (n3) at (3,0) {...};
				\node (n4) at (9,0) {...};
				\node (n5) at (6,2) {...};

				\foreach \from/\to in {x/n1,n1/a,a/n2,n2/y,x0/n3,
				n3/b,b/n4,n4/y0,a/n5,b/n5}
				\draw (\from) -- (\to);
			\end{tikzpicture}

			\bigskip
			\bigskip
			\bigskip

			\item \textbf{Caso 2:} Os caminhos $x$,$y$-$path$ e 
            $x_0$,$y_0$-$path$ possuem vértices em comum.
         	
         	Logo, sabemos que existe um caminho $a$,$b$-$path$, 
            de comprimento $c \ge 1$,
            que funciona como uma interseção entre os dois 
            caminhos, de modo que $dist(y,a)\ge dist(y,b)$.
            
            \begin{tikzpicture}
				[scale=.8,auto=left,every node/.style={circle,
				fill=blue!20}]
				\node (x) at (0,4)  {x};
				\node (y) at (12,4) {y};
				\node (a) at (4,4)  {a};
				\node (b) at (8,4)  {b};
				  
				\node (n1) at (2,4) {...};
				\node (n2) at (6,4) {...};
				\node (n3) at (10,4) {...};

				\foreach \from/\to in {x/n1,n1/a,a/n2,n2/b,b/n3,
				n3/y}
				\draw (\from) -- (\to);
			\end{tikzpicture}
            
            \bigskip

            \textbf{Caso 2.1:} Se 
            $dist(x_0,b)\ge dist(x_0,a)$, então a árvore será
            como o ilustrado abaixo:

            \begin{tikzpicture}
				[scale=.8,auto=left,every node/.style={circle,
				fill=blue!20}]
				\node (x) at (0,4)  {x};
				\node (y) at (12,4) {y};
				\node (a) at (4,4)  {a};
				\node (b) at (8,4)  {b};
				\node (x0) at (1.5,6)  {$x_0$};
				\node (y0) at (10.5,2)  {$y_0$};
				  
				\node (n1) at (2,4) {...};
				\node (n2) at (6,4) {...};
				\node (n3) at (10,4) {...};
				\node (n4) at (9.25,3) {...};
				\node (n5) at (2.75,5) {...};

				\foreach \from/\to in {x/n1,n1/a,a/n2,n2/b,b/n3,
				n3/y, a/n5,n5/x0,b/n4,n4/y0}
				\draw (\from) -- (\to);
			\end{tikzpicture}
			
			Então podemos afirmar que 
			$$ dist(x,y) = dist(x,a) + dist(a,b) + dist(y,b) $$ e
            $$ dist(x_0,y_0) = dist(x_0,b) + dist(a,b) + 
            dist(y_0,b).$$

			Nesse caso, como sabemos que $x_0$ é o vértice mais
			distante de $y_0$, então
			$$  dist(x_0,b)\ge dist(x,a) \Rightarrow 
			dist(x_0,a) + dist(a,b)\ge dist(x,a) + dist(a,b)
			\Rightarrow$$ 
			$$ dist(x_0,a)\ge dist(x,a), $$
			e como $x$,$y$-$path$ é o caminho máximo, então
			$$  dist(x,y)\ge dist(x_0,y) \Rightarrow 
			dist(x,a)\ge dist(x_0,a), $$ 
			$$  dist(x,y)\ge dist(x_0,y) \Rightarrow 
			dist(x,a) + dist(a,b)\ge dist(y_0,b)\Rightarrow
			dist(x,b)\ge dist(y_0,b) $$
			$$  dist(x,y)\ge dist(x_0,y) \Rightarrow 
			dist(y,b)\ge dist(y_0,b). $$
			Com isso temos que $$dist(x_0,a)=dist(x,a).$$

			Agora vamos recorrer ao vértice $v$, que é o vértice
			que foi escolhido aleatoriamente no algorítmo.
			Digamos que $v'$ é o vértice que está em
			$x$,$y$-$path\cup x_0$,$y_0$-$path$ que está mais
			perto de $v$.
			Sabemos que $y_0$ é um vértice mais distante de $v$,
			logo $y_0$ é um vértice mais distante de $v'$. 
			
			\textbf{Caso 2.1.1:} Se 
			$v\in x$,$b$-$path\cup x_0$,$b$-$path$, temos que
			$$ dist(v,y_0)\ge dist(v,y) \Rightarrow
			dist(v,b)+dist(y_0,b)\ge dist(v,b) + dist(y,b) 
			\Rightarrow$$
			$$ dist(y_0,b)\ge dist(y,b).$$

			\textbf{Caso 2.1.2:} Já se $v\in b$,$y_0$-$path$, 
			temos que
			$$ dist(v,y_0)\ge dist(v,y) \Rightarrow
			dist(y_0,b)\ge dist(v,y_0)\ge 
			dist(v,b) + dist(y,b)\ge dist(y,b) $$
			$$ dist(y_0,b)\ge dist(y,b) $$

			\textbf{Caso 2.1.3:} Caso contrário, quando 
			$v\in b$,$y$-$path$, temos que
			$$ dist(v,y_0)\ge dist(v,x) \Rightarrow
			dist(y_0,b)+dist(v,b)\ge dist(x,b)+dist(v,b)
			\Rightarrow $$
			$$ dist(y_0,b)\ge dist(x,b) \Rightarrow
			dist(y_0,b) = dist(x,b) \Rightarrow 
			dist(y_0,b) = dist(x_0,b) $$
			Como  $x_0$ é o vértice mais distante de $y_0$, temos
			que
			$$ dist(y_0,x_0)\ge dist(y_0,y) \Rightarrow
			dist(y_0,b) + dist(x_0,b)\ge dist(y_0,b) + dist(y,b)
			\Rightarrow $$
			$$ dist(x_0,b)\ge dist(y,b) \Rightarrow 
			dist(y_0,b)\ge dist(y,b) $$

			\bigskip

			Como $dist(y_0,b)\ge dist(y,b)$ para todos os casos
			e $dist(y,b)\ge dist(y_0,b)$ então
			$$ dist(y,b) =  dist(y_0,b) \Rightarrow
			dist(x,y) = dist(x_0,y_0) $$
			portanto, $x_0$,$y_0$-$path$ é um caminho máximo.

			\bigskip
			\bigskip

			\textbf{Caso 2.2:} Se $dist(x_0,a)> dist(x_0,b)$,
			então a árvore será como o ilustrado abaixo:

			\begin{tikzpicture}
				[scale=.8,auto=left,every node/.style={circle,
				fill=blue!20}]
				\node (x) at (0,4)  {x};
				\node (y) at (12,4) {y};
				\node (a) at (4,4)  {a};
				\node (b) at (8,4)  {b};
				\node (y0) at (1.5,6)  {$y_0$};
				\node (x0) at (10.5,2)  {$x_0$};
				  
				\node (n1) at (2,4) {...};
				\node (n2) at (6,4) {...};
				\node (n3) at (10,4) {...};
				\node (n4) at (9.25,3) {...};
				\node (n5) at (2.75,5) {...};

				\foreach \from/\to in {x/n1,n1/a,a/n2,n2/b,b/n3,
				n3/y, a/n5,n5/y0,b/n4,n4/x0}
				\draw (\from) -- (\to);
			\end{tikzpicture}

			E pode-se provar que $x_0$,$y_0$-$path$ é um caminho
			máximo de uma forma análoga ao Caso 2.1.


			

        \end{itemize}

    	\newpage



        
    \subsection{Cálculo do Diâmetro de uma Árvore}
    


\newpage
    
    
    

\end{document}