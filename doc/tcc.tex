\documentclass[a4paper,12pt]{article}

% Para usar a linguagem PT-BR.
\usepackage[utf8]{inputenc}
\usepackage[T1]{fontenc}
\usepackage[brazilian]{babel}
\selectlanguage{brazilian} 

\newcommand{\Oh}{\mathrm{O}}

\sloppy

\begin{document}
\begin{center}
   {\large \textbf{UNIVERSIDADE DE SÃO PAULO}} \\[1.4cm]
   
   {\large \textbf{INSTITUTO DE MATEMÁTICA E ESTATÍSTICA}}\\[4.2cm]
   
   {\Huge TRABALHO DE CONCLUSÃO }\\[0.3cm]
   {\Huge DE CURSO }\\[9cm]
   
   {\large { Aluna: Karina Suemi Awoki}}\\[0.3cm]
   
   {\large { Orientadora: Cristina Gomes Fernandes}}
   

\end{center}

\newpage
\section{Introdução}
..
\bigskip

\section{Definições}
    
    \subsection{Caminho máximo em árvores}

        Dada uma árvore $G = (V,A)$, dizemos que o caminho máximo de G é a sequência de 
        vértices $x_1$, $x_2$,... $x_n$ com $(x_i, x_{i+1}) \in A, \forall i \in$ [$1, n-1$],
        tal que esse caminho possua o maior número de vértices possíveis.

        \bigskip



        Calcular o caminho máximo em árvores é uma tarefa simples, computacionalmente
        falando. Primeiramente assumiremos que $\forall v \in V$, o vértice mais distante
        de $v$ é $x_1$ ou $x_n$, que são os extremos do caminho mais longo.

        Nota-se que se escolhermos qualquer vértice $v \in G$, e procurarmos pelo vértice mais
        distante dele, chegaremos em um dos extremos do caminho mais longo ($x_1$ ou $x_n$) e o
        chamaremos de $y$.
        E se fizermos o mesmo com $y$, procurar qual o vértice mais distante dele, chegaremos no
        outro extremo do caminho mais longo.
        

        Agora precisamos provar que $\forall v \in V$, o vértice mais distante
        de $v$ é $x_1$ ou $x_n$.
        Vamos chamar o conjunto de vértices do caminho máximo de $C$.
        \begin{itemize}
            \item Caso 1: Se $v \in C$, temos que o vértice mais distante de $v$ é $z$ com 
            distância $z'$ . Suponhamos que $z$ não seja $x_1$ e nem $x_n$, 

        \end{itemize}
        
    \subsection{Cálculo do Diâmetro de uma Árvore}
    
        
                                
    


\newpage
    
    
    

\end{document}