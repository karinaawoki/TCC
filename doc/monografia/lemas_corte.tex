\section {Lemas de cortes aproximados}

Em \cite{Schmidt15}, são apresentados 
algoritmos, que, dados uma floresta~$G$ e um 
inteiro~${0<m\le n}$, onde~$n$ é o número de vértices de~$G$, 
produzem cortes com algumas propriedades.
Os algoritmos em questão são usados como subrotinas no algoritmo
FST.

Abaixo reproduzimos os três resultados de \cite{Schmidt15} que 
atestam as propriedades dos cortes produzidos por estes algoritmos.
O primeiro deles é o mais simples e o algoritmo vinculado a ele
é usado no segundo.

\bigskip

\begin{lem}[]
\label{lema:simpleApproxCutTree}
	Para toda árvore~$T$ com~$n$ vértices e todo~${m \in [n]}$,
	existe um corte~$(B,W)$ em~$T$ tal 
	que~${\dfrac{m}{2} <|B| \le m}$ e~${e_T(B,W) \le \Delta(T)}$.
	Um corte que satisfaz esses requisitos pode ser computado em
	tempo~$O(n)$.
\end{lem}

\bigskip

Seja~$T$ uma árvore com~$n$ vértices e~$r$ um vértice de~$T$. 
Assumiremos que
a função {\sc número\_de\_descedentes}$(T$, $r)$ devolve um vetor
que associa cada vértice de~$T$ ao número de descendentes desse 
vértice na árvore~$T$, enraizada em~$r$. 
Note que essa função 
pode ser implementada usando-se uma busca em profundidade
a partir de~$r$,
resultando em um consumo de tempo~$O(n)$.


Segue um algoritmo que encontra um corte com as propriedades 
descritas no lema.

\bigskip

\begin{algorithm}[H]
\label{alg:simpleApproxCutTree}
	\SetKwInOut{Input}{entrada}
	\SetKwInOut{Output}{saída}

	\caption{Computa corte aproximado em uma árvore}
	\Input{árvore~${T =(V,E)}$ com~$n$ vértices, ~$m \in [n]$ 
	e~$r\in V$}
	\Output{$B\subseteq V$ com~$\dfrac{m}{2}<|B|\le m$ 
	e~$e_T(B,V\setminus B)\le \Delta(T)$}
	\lIf{$m =n$}{
		\Return $(V)$
	}
		$d \gets~$ {\sc número\_de\_descendentes}$(T$, $r)$\;
		$v \gets r$\;
		\lWhile{existe filho~$u$ de~$v$ com~$d[u] > m$ }{
			$v \gets u$
		}
		\eIf{existe filho~$u$ de~$v$ com~$d[u]>\dfrac{m}{2}$}
		{
			$B\gets V(T_u)$; \quad
			\tcp{$T_u$ é a sub-árvore de~$T$ enraizada em~$u$}
		}
		{
		$B \gets \emptyset$\;
		\For{cada filho~$u$ de~$v$}
		{
			\eIf{$|B|+|T_u| \le m$}{
				$B\gets B\cup V(T_u)$\;
			}
			{
				break\;
			}
		}
		}
	\Return $(B)$

\end{algorithm}	

\bigskip

%\subsection*{Análise do Algoritmo}
	
	\begin{proof}
	Suponha que~${T=(V,E)}$ e ${n=|V|}$.
	Se o algoritmo termina na linha~1, então o corte 
	devolvido satisfaz 
	%É fácil ver que se~$m=n$, então os valores~$B =V$ 
	%e~$W =\emptyset$ satisfazem 
	as condições do lema, dado 
	que~${e_G(V,\ \emptyset) = 0}$,
	e o consumo de tempo é claramente~$O(n)$. 
	%Isso pode ser computado em tempo~$O(n)$.

	Caso contrário, observe que o algoritmo termina de executar o laço
	da linha~4, pois a árvore é finita, e o faz em tempo~$O(n)$
	já que cada vértice é percorrido no máximo uma vez.

	Seja~$v$ o vértice em que o laço da linha~4 termina.
	Se o algoritmo executa a linha~6, o consumo de tempo é~$O(n)$
	e o corte~$(B,W)$ devolvido é tal que~${e_T(B,W) = 1}$
	e~${\dfrac{m}{2}<|B|\le m}$, pois o filho de~$v$ encontrado tem no 
	máximo~$m$ descendentes.
	Se isso não ocorrer, significa que o número de descendentes de 
	todos os filhos de~$v$ é no máximo metade de~$m$.
	Sabe-se que a união de dois conjuntos
	com no máximo~${\dfrac{m}{2}}$ vértices resulta em um conjunto
	com no máximo~$m$ vértices.
	Portanto, como~$v$ possui no mínimo~$m$ descendentes
	e cada filho de~$v$ possui no máximo~${\dfrac{m}{2}}$
	descendentes, 
	%eventualmente
	ao final do laço das linhas~9-15,~${\dfrac{m}{2} < |B| \le m}$. 
	%Se a linha~13 nunca for executada,~$B$
	%será composto por todos descendentes de~$v$, exceto o próprio~$v$.
	%Se a linha~13 for executada, 
	%No final do laço das linhas~9-15,~${|B|\le m}$
	%de forma que~$B$ tenha o maior valor possível,
	Desta forma, o algoritmo devolve um corte~$(B,W)$
	com~${\dfrac{m}{2} < |B| \le m}$ e~${e_T(B,W) \le \Delta(T)}$.

	O tempo consumido no laço das linhas~9-15 é~$O(n)$, pois
	os filhos de~$v$ e seus descendentes são percorridos no máximo
	uma vez nas iterações do laço e na união de conjuntos da linha~11,
	respectivamente.
	\end{proof}

	% Caso contrário, o algoritmo escolhe uma raiz arbitrária~$r$ 
	% para~$T$ e calcula o número de vértices das sub-árvores 
	% enraizadas em cada um dos vértices, que é o número de 
	% descendentes do vértice.
	% Isso serve para que saibamos a quantidade de vértices das 
	% sub-árvores sem que precisemos percorrer todos os vértices das 
	% sub-árvores a cada consulta.

	% Feito isso, seja~$v$ o vértice analisado no momento 
	% e~${V_1, V_2, \ldots, V_k}$ os conjuntos de vértices das
	% sub-árvores enraizadas 
	% nos~$k$ filhos de~$v$.
	% Se algum~$V_i$ satisfizer~${\dfrac{m}{2}<|V_i|\le m}$, podemos 
	% devolver a sub-árvore~$V_i$ como o conjunto~$B$, 
	% satisfazendo o Lema~\ref{lema:simpleApproxCutTree}, dado que o
	% corte determinado por~$V_i$ contém uma única aresta. 
	% Caso contrário, se~${|V_i|>m}$ para algum~$i$, 
	% e~$v$ passa a ser a raiz de~$V_i$.
	% Aplicamos novamente esse processo em~$v$.

	% Nota-se que esse procedimento irá parar em algum momento, dado 
	% que a árvore é finita, e ele parará quando encontrar uma 
	% sub-árvore que satisfaça o Lema~\ref{lema:simpleApproxCutTree}
	% (e essa sub-árvore definirá o conjunto~$B$) 
	% ou quando chegarmos numa situação em 
	% que~${|V_i|\le \dfrac{m}{2}}$ para todos os~$k$ filhos de~$v$.
	% Nesse último caso, sabemos 
	% que~${|V_1\cup V_2\cup \cdots \cup V_k|\ge m}$, pois
	% a sub-árvore enraizada em~$v$ possui mais que~$m$ vértices. 
	% Sabemos também que~${|V_i|\le \dfrac{m}{2}}$ para todos os~$k$ 
	% filhos de~$v$. 
	% Com isso, percorremos os~$V_i$ em ordem, e 
	% adicionamos~$V_i$ ao conjunto~$B$, parando antes 
	% que~${|B| >m}$. 
	% O conjunto~$B$ resultante é tal 
	% que~${e_T(B,V\setminus B)<\grau_T(v)\le \Delta(T)}$, 
	% o que satisfaz o lema.
	% Dado que~${|V_j|\le \dfrac{m}{2}}$ para todo~$j$, sabemos que 
	% existe um~$i$ tal 
	% que~${\dfrac{m}{2} <|V_1\cup V_2 \cup \cdots \cup V_i|\le m}$, 
	% pois a união das sub-árvores enraizadas em~$v$ possui pelo 
	% menos~$m$ vértices e, para todo~$\ell$ que 
	% satisfaça~${|V_1\cup V_2\cup \cdots\cup V_\ell|\le 
	% \dfrac{m}{2}}$, temos 
	% que~${|V_1\cup V_2\cup\cdots\cup V_\ell\cup V_{\ell+1}|\le m}$.


\bigskip
\bigskip
\bigskip

%%%%%%%%%%%%%%%%%%%%%%%%%%%%%%%%%%%%%%%%%%%%%%%%%%%%%%%%%%%%%%%%%%%
%%%%%%%%%%%%%%%%% LEMA 3 %%%%%%%%%%%%%%%%%%%%%%%%%%%%%%%%%%%%%%%%%%

%\subsection{Lema do corte aproximado simples para florestas}

O lema abaixo representa uma adaptação do algoritmo anterior
para florestas em vez de árvores.
Note que a adaptação não recebe uma raiz como parâmetro, mas 
apenas~$G$ e~$m$.

\begin{lem}[{\cite[Lema 2]{Schmidt15}}]
\label{lema:simpleApproxCutForest}
	Para toda floresta~$G$ com~$n$ vértices e todo~${m \in [n]}$,
	existe um corte~$(B,W)$ em~$G$ tal 
	que~${\dfrac{m}{2} <|B| \le m}$ e~${e_G(B,W) \le \Delta(G)}$.
	Um corte que satisfaz esses requisitos pode ser computado em
	tempo~$O(n)$.
\end{lem}

\bigskip

%Agora examinaremos como determinar um corte descrito pelo lema.
%Nota-se que essa é uma adaptação do algoritmo anterior
%para florestas em vez de árvores.

O algoritmo abaixo determina um corte como o descrito pelo lema.
Para simplificar, ele devolve apenas o conjunto~$B$, uma vez
que~$W$ é simplesmente~$V(G)\setminus B$.

\medskip
\medskip

\begin{algorithm}[H]
\label{alg:simpleApproxCutForest}
	\SetKwInOut{Input}{entrada}
	\SetKwInOut{Output}{saída}

	\caption{Computa corte aproximado simples em uma floresta}
	\Input{floresta~${G =(V,E)}$ com~$n$ vértices e~${m \in [n]}$}
	\Output{~$B\subseteq V$ tal que~${\dfrac{m}{2}\le|B|\le m}$ 
	e~${e_G(B,G\setminus B)\le \Delta(G)}$}
	Sejam~$V_1, V_2,\ldots, V_k$ os conjuntos de vértices das
	componentes de~$G$\;
	$B \gets \emptyset$\;
	\For{${i=1 \to k}$}{
		\If{$|B| + |V_i|\le m$}{
			$B \gets B\cup V_i$\;
		}	
		\ElseIf{$|B|<m$}{
			$r\gets $ vértice arbitrário de~$V_i$\;
			$B'\gets \mathbf{Algoritmo\ref{alg:simpleApproxCutTree}}(
			G[V_i], m-|B|, r)$\;

			$B \gets B\cup B'$\;

			break\;
		}	
	}
	\Return $(B)$\;

\end{algorithm}	

\bigskip

%\subsection*{Análise do Algoritmo}

	% Sejam~${T_1, T_2, \ldots,T_k}$ as árvores que compõe~$G$. 
	% Primeiro o algoritmo percorre as árvores determinando
	% o maior inteiro~$\ell$ tal 
	% que~${s=\displaystyle\sum_{i=1}^{\ell}|V(T_i)| \le m}$.
	% Colocamos~${V(T_1),V(T_2), \ldots,V(T_\ell)}$ no conjunto~$B$.
	% Caso~${|B|=m}$, temos que~${e_G(B,V\setminus B)=0}$, o que 
	% satisfaz o lema.
	% Caso contrário, ao encontrar um corte~$(B',W')$ em~$T_{\ell+1}$
	% que satisfaça~${\dfrac{m-s}{2}<|B'|\le m-s}$ 
	% com~${e_{T_{\ell+1}}(B',W') \le \Delta(T_{\ell+1})}$, e 
	% acrescentar~$B'$ ao conjunto~$B$, 
	% teremos~${\dfrac{m+s}{2}<|B| \le m}$ 
	% com~${e_G(B,W)\le\Delta(T_{\ell+1}) \le \Delta(G)}$.
	% Logo, depois de acrescentar~$B'$ em~$B$, teremos o 
	% corte~${(B,V\setminus B)}$ que satisfaz o lema.
	% O corte~$(B,W)$ é encontrado por meio do 
	% Algoritmo~\ref{alg:simpleApproxCutTree}.

	\begin{proof}
	Claro que~${|B|\le m}$ após cada execução da linha~5.
	Ademais, caso a linha~9 seja executada,~$B'$ é tal 
	que~${\dfrac{m-|B|}{2}< |B'|\le m-|B|}$
	pela descrição do Algoritmo~1.
	Portanto~${|B|\le m}$ após a linha~10 e, 
	como~${|B| + \dfrac{m-|B|}{2} = \dfrac{m+|B|}{2}}$, vale 
	que~${|B|>\dfrac{m}{2}}$ após a linha~10 também.
	Ou seja, na linha~14, temos que~${\dfrac{m}{2} < |B| \le m}$.
	Além disso,~${e_G(B,W)\le \Delta(G)}$ se a linha~10 é
	executada e~${e_G(B,W)=0}$ caso contrário.

	Note que, como a linha~1 pode ser feita usando uma busca em 
	profundidade e as demais linhas envolvem verificar o tamanho 
	das componentes (que já foi calculado na linha~1) e uma 
	chamada única ao Algoritmo~\ref{alg:simpleApproxCutTree}, então 
	essa rotina consome tempo~$O(n)$, dado que cada uma 
	das operações citadas são executadas em tempo~$O(n)$. 
\end{proof}

\bigskip
\bigskip
\bigskip


%%%%%%%%%%%%%%%%%%%%%%%%%%%%%%%%%%%%%%%%%%%%%%%%%%%%%%%%%%%%%%%%%%%
%%%%%%%%%%%%%%%%% LEMA 4 %%%%%%%%%%%%%%%%%%%%%%%%%%%%%%%%%%%%%%%%%%

%\subsection{Lema do corte aproximado}
Finalmente apresentamos o algoritmo que é usado repetidas 
vezes no algoritmo FST.
Ele utiliza o algoritmo anterior como subrotina.

\begin{lem}[{\cite[Lema 3]{Schmidt15}}]
\label{lema:approxCutForest}
	Para toda floresta~$G$ com~$n$ vértices, todo~${m \in [n]}$ e 
	todo~${c \in [0,1)}$, existe um corte~$(B,W)$ em~$G$ tal 
	que~${cm \le |B| \le m}$ 
	e~${e_G(B,W) \le \ceil[\Big]{ \dfrac{2c}{1-c}} \Delta(G)}$.
	Um corte que satisfaz esses requisitos pode ser computado em
	tempo~${O\Big(\ceil[\Big]{ \dfrac{2c}{1-c}} n+1\Big)}$.
\end{lem}


Segue o algoritmo que encontra um corte com
as propriedades do Lema~\ref{lema:approxCutForest}.
\bigskip
\begin{algorithm}[H]
\label{alg:approxCutForest}
	\SetKwInOut{Input}{entrada}
	\SetKwInOut{Output}{saída}

	\caption{Computa corte aproximado em uma floresta}
	\Input{floresta~${G =(V,E)}$ com~$n$ vértices,~${m \in [n]}$ e
	${c\in [0, 1)}$}
	\Output{corte~${(B,W)}$ tal que~${cm\le|B|\le m}$
	e~$e_G(B,W) \le \ceil[\Big]{ \dfrac{2c}{1-c}} \Delta(G)$}
	\lIf{$c=0$}{
		$\Return\ (\emptyset, V)$}
	\lIf{${c\le \dfrac{1}{2}}$}{
		$\Return\ \mathbf{Algoritmo\ref{alg:simpleApproxCutForest}}(
		G, m)$}
		$B \gets \emptyset$\;
		\While{$|B|<cm$}{
			$(B',W') \gets 
			\mathbf{Algoritmo\ref{alg:simpleApproxCutForest}}(
			G[V\setminus B], m-|B|)$ \;

			$B \gets B\cup B'$\;		
	}
	\Return $(B,V\setminus B)$\;

\end{algorithm}	
%\subsection*{Análise do Algoritmo}

\begin{proof}
	Se~${c=0}$, o Algoritmo~\ref{alg:approxCutForest} devolverá
	um corte onde~${B=\emptyset}$ e~${e_G(B,W) = 0}$. 
	Isso satisfaz as 
	propriedades do Lema~\ref{lema:approxCutForest}, dado 
	que~${cm=0=|B|}$.

	Se~${c \le \dfrac{1}{2}}$, o algoritmo 
	devolve o corte produzido pelo 
	Algoritmo~\ref{alg:simpleApproxCutForest}, ou seja,
	um corte~$(B',W')$ tal 
	que~${|B'|>\dfrac{m}{2}\ge cm}$ e~${e_G(B',W')\le \Delta(G)}$.

	Por outro lado, se~${c>\dfrac{1}{2}}$, então
	o conjunto~$B$ é inicializado com~$\emptyset$.
	O Algoritmo~\ref{alg:simpleApproxCutForest} é
	executado diversas vezes para encontrar a cada vez
	um corte~$(B',W')$ em~${G[V\setminus B]}$ 
	tal que~${\dfrac{m-|B|}{2}<|B'|\le m-|B|}$ 
	e~${e_{G[V\setminus B]}(B',W')\le\Delta(G[V\setminus B])}$, 
	acrescentando o 
	conjunto~$B'$ a~$B$ em cada uma das chamadas ao 
	Algoritmo~\ref{alg:simpleApproxCutForest}.
	Isso é feito até que~${|B|\ge cm}$.
	Note que em algum momento~${|B|\ge cm}$, pois a 
	cada vez que executamos o 
	Algoritmo~\ref{alg:simpleApproxCutForest}, acrescentamos pelo 
	menos um vértice em~$B$.
	Ademais,~${|B|\le m}$ durante todo o processo, já que o
	conjunto~$B'$ obtido 
	pelo Algoritmo~\ref{alg:simpleApproxCutForest} 
	é tal que~$|B'|\le m-|B|$. Logo, serão adicionados no 
	máximo~${m-|B|}$ vértices em~$B$ na linha~6.
	
	\bigskip
	
	Verificaremos agora se o consumo de tempo e a largura do corte
	devolvido pelo Algoritmo~\ref{alg:approxCutForest} satisfazem
	o que foi proposto no Lema~\ref{lema:approxCutForest}.
	
	Se~${c=0}$, a largura do corte devolvido é zero,
	dado que todos os vértices de~$G$ estão em~$W$.
	Portanto~${e_G(B,W)=0=\ceil[\Big]{\dfrac{2c}{1-c}} \Delta(G)}$.
	Já em relação ao tempo, neste caso o Algoritmo~\ref{alg:approxCutForest}
	executa apenas operações que levam tempo constante, logo ele
	leva tempo~$O(1)$. 
	Como~${\ceil[\Big]{\dfrac{2c}{1-c}} n+1 = 1}$,
	as propriedades do Lema~\ref{lema:approxCutForest} são
	satisfeitas.

	Se~${0<c\le \dfrac{1}{2}}$, executamos o
	Algoritmo~\ref{alg:simpleApproxCutForest} uma única vez, 
	portanto o tempo é~$O(n)$, que equivale 
	a~${O\Big(\ceil[\Big]{\dfrac{2c}{1-c}} n+1\Big)}$.
	Em relação à largura do corte, será devolvido o corte~$(B,W)$ do 
	Algoritmo~\ref{alg:simpleApproxCutForest} sem nenhuma 
	modificação, então sabe-se que~${e_G(B,W)\le \Delta(G)}$.
	Como~${0<c\le\dfrac{1}{2}}$, temos que~${\dfrac{2c}{1-c}>0}$, 
	que implica que~${\ceil[\Big]{ \dfrac{2c}{1-c}}\ge 1}$.
	Logo,~${e_G(B,W)\le\Delta(G)\le \ceil[\Big]{\dfrac{2c}{1-c}}
	\Delta(G)}$, 
	satisfazendo assim as propriedades do 
	Lema~\ref{lema:approxCutForest}.

	Caso contrário,~${c>\dfrac{1}{2}}$. Como em todas as vezes que
	executamos a linha~4 do Algoritmo~\ref{alg:approxCutForest}, 
	vale que~${|B|<cm}$, então também é válido que~${m-|B|>(1-c)m}$.
	Como~${m-|B|}$ é o valor do~$m$ que passamos para o 
	Algoritmo~\ref{alg:simpleApproxCutForest}, sabemos que o 
	conjunto $B'$ do corte devolvido terá mais 
	que~${\dfrac{m-|B|}{2} > \dfrac{(1-c)m}{2}}$ vértices.
	Portanto, executaremos o 
	Algoritmo~\ref{alg:simpleApproxCutForest} no 
	máximo~${\ceil[\bigg]{\dfrac{cm}{\frac{(1-c)m}{2}}} 
	=\ceil[\Big]{\dfrac{2c}{1-c}}}$ vezes, consumindo, no total,
	tempo~$O\Big(\ceil[\Big]{ \dfrac{2c}{1-c}} n\Big) = O\Big(\ceil[\Big]{ \dfrac{2c}{1-c}} n+1\Big)$.
	Usaremos essa mesma linha de pensamento para calcular a largura
	máxima do corte devolvido na linha~8. 
	Sabemos que o 
	Algoritmo~\ref{alg:simpleApproxCutForest} será 
	chamado no máximo~${\ceil[\Big]{\dfrac{2c}{1-c}}}$ 
	vezes e, para cada uma 
	das vezes, este devolve um corte cuja largura 
	máxima é~$\Delta(G)$, portanto a largura máxima do corte 
	devolvido é~${\ceil[\Big]{\dfrac{2c}{1-c}}\Delta(G)}$.
	\end{proof}




