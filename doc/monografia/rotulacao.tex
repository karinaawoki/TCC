\section {Rotulação dos vértices da árvore}
\label{sec:rotulacao}

	% Mais adiante, iremos nos referir a vértices com uma 
	% característica específica, e para isso precisamos atribuir 
	o algoritmo FST utiliza um 
	rótulo numérico~$x\in [n]$ distinto para cada um dos vértices 
	da árvore, onde~$n$ é o número de vértices da árvore~$T$. 
	Esses rótulos devem ter uma característica específica e por isso
	são calculados de uma maneira particular.
	Denotaremos o \textbf{rótulo} de um vértice~$v$ por~$\rot(v)$.

	Primeiramente, usaremos o método mencionado na 
	Seção~\ref{sec:caminhoMaximo} para calcular 
	um caminho~$P$, que é máximo em~$T$. 
	Para todo vértice~$v\in V(P)$, 
	% temos uma 
	% árvore~$T_v \in G - E(P)$ com~$v\in T_v$.
	seja~$T_v$ a árvore de~$T-E(P)$ que contém~$v$.

	Dessa forma, teremos sub-árvores enraizadas nos vértices do
	caminho máximo, como mostrado na figura abaixo, onde denotamos
	os vértices de~$P$ por~${v_0,v_1,\ldots,v_p}$.
	

 	%% fill=blue!30 -- núemro indica quão escuro é
	\begin{center} 
	\begin{figure}[h]
	\begin{tikzpicture}
		[scale=.7,auto=left,every node/.style={circle, draw=black,
				fill=blue!20}]
	\scalebox{0.95}{
		\draw [draw=yellow, fill=yellow!35, line width=2pt] (0,4.8) 
		ellipse (1.3cm and 2.4cm); %0
		\draw [draw=yellow, fill=yellow!35, line width=2pt] (3,4.4) 
		ellipse (1.3cm and 2.4cm); %1
		\draw [draw=yellow, fill=yellow!35, line width=2pt] (6,4.8) 
		ellipse (1.3cm and 2.4cm); %2
		\draw [draw=yellow, fill=yellow!35, line width=2pt] (9,4.4) 
		ellipse (1.3cm and 2.4cm); %3
 		\draw [draw=yellow, fill=yellow!35, line width=2pt] (12,4.8) 
 		ellipse (1.3cm and 2.4cm); %4
		\draw [draw=yellow, fill=yellow!35, line width=2pt] (18,4.8) 
		ellipse (1.3cm and 2.4cm); %5

		\node [text=blue!65,draw=none, fill=none] at 
		(-0.4,6.5) {$T_{v_0}$};
		\node [text=blue!65,draw=none, fill=none] at 
		(2.6,6.1) {$T_{v_1}$};
		\node [text=blue!65,draw=none, fill=none] at 
		(5.6,6.5) {$T_{v_2}$};
		\node [text=blue!65,draw=none, fill=none] at 
		(8.6,6.1) {$T_{v_3}$};
		\node [text=blue!65,draw=none, fill=none] at 
		(11.6,6.5) {$T_{v_4}$};
		\node [text=blue!65,draw=none, fill=none] at 
		(17.6,6.5) {$T_{v_p}$};

		\node [subtree] (y1) at (3,4.6)  {};
		\node [subtree, fill=red!30] (y2) at (6,5)  {};
		\node [subtree] (y3) at (9,4.6)  {};
		\node [subtree] (y4) at (12,5) {};

		\node (x0) at (0,5.4)  {$v_0$};
		\node (x1) at (3,5)    {$v_1$};
		\node (x2) at (6,5.4)  {$v_2$};
		\node (x3) at (9,5)    {$v_3$};
		\node (x4) at (12,5.4) {$v_4$};
		\node (x5) at (15,5)   {...};
		\node (x6) at (18,5.4) {$v_p$};


		\foreach \from/\to in {x0/x1,x1/x2,x2/x3,x3/x4,x4/x5,x5/x6}
		\draw (\from) -- (\to);
	}
	\end{tikzpicture} 
	\caption{Sub-árvores enraizadas no caminho máximo}
	\end{figure}
	\end{center}
	\bigskip
	Note que as árvores~$T_{v_0}$ e~$T_{v_p}$ possuem um único
	vértice cada uma. 
	Isso ocorre porque~$v_0$ e~$v_p$ são extremos
	do caminho máximo~$P$, e se houvesse algum vértice em uma
	dessas sub-árvores,
	%~$v_\ell$ faria parte do caminho máximo
	%e~$v_0$ ou~$v_p$ não seria um dos extremos, 
	o caminho~$P$ poderia ser estendido,
	o que levaria a uma contradição.

	\bigskip

	Cada vértice do caminho~$P$ recebe 
	um rótulo de forma
	%\begin{itemize}
		%\item~$\rot(v_i)>\rot(v_j)$ para todo~$i<j$ e
		que~${\rot(v_{i-1})<\rot(v') \le \rot(v_i)}$ para todo 
		vértice~$v'$ da sub-árvore~$T_{v_i}$ para~$i=1,2,\ldots,p$. 
	%\end{itemize}

	\bigskip
	\bigskip

	\subsection{Descrição do algoritmo de rotulação}
	Essa rotulação pode ser obtida facilmente 
	manipulando as listas de adjacências de ~$T$ e em seguida,
	executando uma busca em profundidade.
	Ambas as ações consomem tempo linear em~$n$, então
	pode-se rotular uma árvore com~$n$ vértices em tempo~$O(n)$.

	Para cada vértice~$v_i$ contido no caminho~$P$, com~$i>0$,
	percorremos a sua lista de adjacências e, quando encontrarmos o 
	vértice~$v_{i-1}$, o colocamos no início da lista.
	Dessa forma, ao aplicarmos uma busca em profundidade nessa 
	árvore usando o vértice~$v_p$ como raiz, a busca descerá primeiro
	pelos vértices do caminho máximo.
	Os rótulos são atribuidos em pós-ordem, ou seja, a busca
	atribuirá um rótulo
	a um vértice do caminho
	somente quando terminar de rotular todos os vértices da sub-árvore
	do vértice, 
	garantindo assim que para todo~$i\in \{1,\ldots, p\}$, temos 
	que~$\rot(v_{i-1})<\rot(j)\le\rot(v_i)$ para todo~$j$ 
	em~$V(T_{v_i})$.
	
	\bigskip
	\bigskip
	\bigskip

	\subsection{Mapeamento dos rótulos de vértices }
	\label{sec:map}
	Vamos definir agora as funções
	utilizadas pelo algoritmo FST
	que mapeiam um vértice em 
	outro, utilizando a rotulação acima como base.
	Chamaremos de~${\map^+_m}$ a função que devolve o~$m$-ésimo 
	próximo vértice na rotulação, considerando que os rótulos
	estão dispostos de forma circular
	e, de~${\map^-_m}$, a função inversa da primeira. 
	Ou seja,~${\map^-_m}$ devolve o~$m$-ésimo vértice anterior
	na rotulação.

	%Dessa forma, podemos ver que, para um 
	%vértice~$v$,~${\map^+_m(\rot(v)) =(\rot(v)+m)\mod n}$ 
	%e~${\map^-_m(\rot(v)) =(\rot(v)-m)\mod n}$.

	Para facilitar, identificaremos os vértices pelos seus 
	rótulos. 
	%Sendo assim, temos que, para um 
	%vértice~$v$, ~${\map^+_m(v) =(v+m)\mod n}$ 
	%e~${\map^-_m(v) =(v-m)\mod n}$ como uma simplificação das 
	%funções enunciadas anteriormente.

	Note que, para todo ${m\in [n]}$, caso~${n>2}$, se existem
	dois vértices~$v$ e~$v'$, ambos no caminho~$P$, 
	tais que~${v'= \map^+_m(v)}$, então o corte $(B,W)$ em~$T$ 
	com~${B =\{v+1, v+2,\ldots,~v+m\}}$ é tal que~${|B|=m}$
	e~${e_T(B,W)\le 2\le \Delta(T)}$.
	Isso pode ser visto na figura abaixo.
	\begin{center} \begin{figure}[h] \begin{tikzpicture}
		[scale=.7,auto=left,every node/.style={circle, draw=black,
				fill=blue!20}]
	\scalebox{1.02}{
		\draw [draw=yellow, fill=yellow!35, line width=2pt] (7.5,1.5) rectangle 
		(13.5, 6.5);

		\node [text=blue!65,draw=none, fill=none] at 
		(10.5,2.2) {\Large{$m$ vértices}};
		\node [draw=none, fill=none] at 
		(9,6.5) {\footnotesize{$\map^+_m$}};
		\node [draw=none, fill=none] at 
		(9,7.9) {\footnotesize{$\map^-_m$}};


		\node [subtree] (y1) at (3,4.6)  {};
		\node [subtree, fill=red!30] (y2) at (6,5)  {};
		\node [subtree] (y3) at (9,4.6)  {};
		\node [subtree] (y4) at (12,5) {};

		\node (x0) at (0,5.4)  {$v_0$};
		\node (x1) at (3,5)    {...};
		\node (x2) at (6,5.4)  {$v$};
		\node (x3) at (9,5)    {...};
		\node (x4) at (12,5.4) {$v'$};
		\node (x5) at (15,5)   {...};
		\node (x6) at (18,5.4) {$v_p$};


		\foreach \from/\to in {x0/x1,x1/x2,x3/x4,x5/x6}
		\draw (\from) -- (\to);

		\path[-, line width=1.5pt, draw=red]
		(x2) edge[right] (x3);
		\path[-, line width=1.5pt, draw=red]
		(x4) edge[right] (x5);

		\path[->, line width=1.5pt]
		(x2) edge[bend left=23 ] (x4);
		\path[->, line width=1.5pt]
		(x4) edge[bend right=45, looseness=1.7 ] (x2);
	}
	\end{tikzpicture}
	\caption{Mapeamento entre os vértices~$v$ e~$v'$ do caminho máximo
	induz um corte~$(B,W)$ com~$|B| = m$ e~$e_T(B,W) = 2$.} 
	\end{figure}
	\end{center}
	Caso~$n\le 2$, 
	é fácil ver que todos os vértices
	da árvore estarão no caminho máximo, portanto, para
	qualquer~$m\in[n]$, teremos um conjunto~$B$ tal
	que~$|B|=m$
	e~$e_T(B,W)\le |E(T)|=n-1=\Delta(T)\le 2$.

	%\subsubsection*{Algumas definições}


	\bigskip
	



%%%%%%%%%%%%%%%%%%%%%%%%%%%%%%%%%%%%%%%%%%%%%%%%%%%%%%%%%%%%%%%%%%%
%%%%%%%%%%%%%%%%%%%%%%%%%%%%%%%%%%%%%%%%%%%%%%%%%%%%%%%%%%%%%%%%%%%
%%%%%%%%%%%%%%%%%%%%%%%%%%%%%%%%%%%%%%%%%%%%%%%%%%%%%%%%%%%%%%%%%%%

	\subsection{Árvores de diâmetro grande}

	Mais adiante usaremos  	o lema abaixo na prova de 
	um dos teoremas.

	\begin{lem}[{\cite[Lema 5]{Schmidt15}}]
	\label{lema:caminhoLongo}
		Para toda árvore~$T$ com~$n$ 
		vértices e~${\diam^*(T)>\dfrac{1}{2}}$
		e para todo~${m\in[n]}$, existe um corte~$(B,W)$ em~$T$
		com~${|B|=m}$ e~${e_T(B,W)\le 2}$. Um corte que satisfaz esses
		requisitos pode ser computado em tempo~$O(n)$.
	\end{lem}

	\medskip
	\medskip

	\begin{proof}
	Considere um caminho máximo~$P$ em~$T$ e a rotulação dos vértices
	de~$T$ induzida por ele.
	Podemos ver que a função~$\map^+_m(x)$ é bijetora dado que, numa
	soma, elementos distintos do domínio tem imagens distintas e, 
	para cada elemento da imagem, temos um do domínio.

	Uma árvore com a propriedade ~${\diam^*(T)>\dfrac{1}{2}}$ é uma
	árvore onde cada caminho máximo contém mais que a metade dos 
	vértices da árvore, o que nos leva ao fato de que existem mais 
	vértices no caminho máximo~$P$ do que fora dele. 

	Como mostrado anteriormente, se um vértice~$v$ é mapeado em um
	vértice~$v'$ e ambos estão no caminho máximo~$P$, então existe um
	corte~$(B,W)$ em~$T$ com~${e_T(B,W)\le 2}$.
	Dado que~${\diam^*(T)>\dfrac{1}{2}}$, e usando o fato de 
	que~$\map^+_m(x)$ é uma função bijetora e que
	existem mais vértices no caminho máximo~$P$ do que fora dele,
	temos que pelo menos um vértice de~$P$ será mapeado em
	outro vértice de~$P$. 
	%Isso ocorre porque todos
	%os vértices mapeiam algum outro vértice, e não temos um 
	%número suficiente de vértices fora do caminho máximo para serem
	%mapeados por todos os vértices do caminho máximo, então algum
	%vértice do caminho máximo vai acabar mapeando outro do caminho
	%máximo, e
	Assim, obtemos um corte~$(B,W)$, com~$e_T(B,W)\le 2$,
	como descrito na subseção anterior.

	Um algoritmo que encontra um corte desse tipo, em árvores com 
	essa propriedade, percorre os vértices do caminho máximo~$P$
	e verifica se algum deles é mapeado em um vértice do caminho máximo.
	Isso pode ser feito em tempo~$O(n)$ se construirmos um vetor 
	binário que indica se cada um dos vértices de~$T$ está ou não no
	caminho máximo~$P$.
	\end{proof}
	%{\color{red}{-O algoritmo é bem simples. Não precisa colocar
	%código né?}}