\section {Algoritmo FST para bissecção}

	%\subsection{Teorema do corte exato}
	Nesta seção, finalmente apresentamos o algoritmo FST.
	Ele utiliza o Algoritmo~\ref{alg:dobraDiametro} 
	repetidas vezes além de manter informações derivadas
	do caminho mais longo e da rotulação inicial.

	\begin{lem}
	\label{lem:grauMaximo}
		% Para toda árvore~$T$ com~${n>2}$ vértices
		% e toda aresta~${e=\{v_1,v_2\}}$ em~$T$, 
		% tal que~${v_1,v_2\in V}$ e~${\grau_T(v_1)=\grau_T(v_2)=1}$,
		% temos que~${\Delta(T) = \Delta(T\cup e)}$.
		Para toda árvore~$T$ com~${n>2}$ vértices
		e todo par de vértices~$v_1$ e~$v_2$ em~$T$ 
		com~${\grau_T(v_1)=\grau_T(v_2)=1}$,
		temos que~${\Delta(T) = \Delta(T+ e)}$,
		onde~$e=\{v_1, v_2\}$.
	\end{lem}
	
	
	\begin{proof}
		Como~${n>2}$, temos que~${\Delta(T)\ge 2}$.
		Sabemos que ao adicionar~${e = \{v_1,v_2\}}$ em~$T$, 
		o grau de~$v_1$
		e~$v_2$ passará a ser 2, e os demais vértices da árvore
		irão manter os seus respectivos graus.
		Portanto,~${\grau_{T+e}(v_1)=\grau_{T+ e}(v_2)=2}$
		e, para todo vértice~$v$ tal que~$v\ne v_1$ e~$v\ne v_2$,
		temos que~${\grau_{T+ e}(v)=\grau_T(v)}$.
		Logo~$\Delta(T+e) = \Delta(T)$.
% 		Seja~$v$ um vértice de~$T$ tal 
% 		que~${\grau_T(v) = \Delta(T)\ge 2}$,
% 		dado que~${\grau_T(v_1)=\grau_T(v_2)=1}$
% 		sabe-se que~${v\ne v_1}$ e~${v\ne v_2}$, o que nos leva 
% 		a~${\grau_{T\cup e}(v)=\grau_T(v)=\Delta(T)}$.
% 		Dessa forma, temos que
% 		\begin{align}
% 			\Delta(T\cup e) &= \max\{\grau_{T\cup e}(v_1), 
% 			\grau_{T\cup e}(v)\}
%          		= \max\{2, \Delta(T)\}
% %				&= \grau_{T\cup e}(v) \nonumber\\
% %         		&= \grau_T(v) \nonumber\\
%          		= \Delta(T). \nonumber
% 		\end{align}		
	\end{proof}

	\bigskip

	Segue o teorema principal.

	\begin{teo}[Teorema do corte exato
	{\cite[Teorema 6]{Schmidt15}}]
	\label{teo:corteExato}
		Para todo~${i\in \mathbb{N^+_*}}$, toda árvore~$T$ com~$n$
		vértices e~${\diam^*(T)>\dfrac{1}{2^i}}$, e todo~${m\in[n]}$,
		existe um corte~$(B,W)$ em~$T$ com~$|B|=m$ 
		e~$e_T(B,W)\le 4\cdot 2^i\cdot \Delta(T)$.
		%Um corte que satisfaz esses requisitos pode ser computado
		%em tempo~${O\Big(\dfrac{n}{\diam^*(T)}\Big)}$.
	\end{teo}

	\medskip
	\medskip

	\begin{proof}
		Utilizaremos indução em~$i$ para provar o Teorema~\ref{teo:corteExato}.
		%a primeira parte do Teorema~\ref{teo:corteExato}.
		%(sem levar em conta o tempo de execução por enquanto).
		
		Primeiramente, temos que~${\diam^*(T)\le1}$, por definição. 
		Sendo assim,~${i\ge1}$, logo~${i=1}$ será a base 
		utilizada na indução.
		
		\textbf{Base:~${\mathbf {i=1}}$.}
		%Para~${n\le 2}$, sabe-se que~${|E(T)|=\Delta(T)}$ e, nesse
		%caso,~${e_T(B,W)\le |E(T)|=\Delta(T)\le 4\cdot 2^i\cdot 
		%\Delta(T)}$.
		Como~${\diam^*(T)>\dfrac{1}{2}}$,
		podemos utilizar o Lema~\ref{lema:caminhoLongo} que
		garante a existência de 
		um corte~$(B,W)$ 
		com~${e_T(B,W)\le 2}$.
		Note que, como~${\diam^*(T)>\dfrac{1}{2}}$,~$T$
		possui mais que um único vértice, logo~${\Delta(T)\ge 1}$
		e~${e_T(B,W)\le 4\cdot 2^i\cdot \Delta(T)}$.
		Portanto, a afirmação vale para~${i=1}$.


		\textbf{Passo:~${\mathbf{i\ge 2}}$.} 
		%Provaremos que a primeira parte o teorema
		%vale para~$i$, assumindo que ele é válido para~$i-1$.
		Para~${n\le 2}$, sabe-se que~${|E(T)|=\Delta(T)}$. 
		Portanto, para~${n\le 2}$, temos 
		que~${e_T(B,W)\le |E(T)|=\Delta(T)\le 2\cdot 2^i\cdot 
		\Delta(T)}$.
		Porém, se~${n>2}$, podemos aplicar o
		Teorema~\ref{teo:dobraDiametro}, que garante a existência 
		de um corte~$(B',W',S')$ do tipo~1 ou~2.

			Se existir um corte~$(B',W',S')$ do tipo 1, 
			teremos~${|B'|=m}$,~${S'=\emptyset}$
			e~${e_T(B',W')\le2\le 2\cdot 2^i\cdot \Delta(T)}$.
			Logo, o corte~$(B',W')$ satisfaz as propriedades
			do corte do teorema.

			Caso contrário, o corte~$(B',W',S')$ será do tipo 2, ou 
			seja,~${|B'|=m}$ ou~${|B'|<m\le |B'|+|S'|}$, 
			e~${e_T(B',W',S')\le \dfrac{2\cdot\Delta(T)}{\diam^*(T)}\le
			2\cdot2^i\cdot\Delta(T)}$.
			
			Se~${|B'|=m}$, então o corte~$(B',V\setminus B')$ satisfaz
			as propriedades do corte do
			teorema.
			Senão,~${|B'|<m\le |B'|+|S'|}$, que implica
			que~${0<m-|B'|\le|S'|}$,
			%Pelas propriedades do Teorema~\ref{teo:dobraDiametro}, 
			%sabemos que
			e~${\diam^*(T[S'])\ge 2\cdot \diam^*(T)>
			\dfrac{1}{2^{i-1}}}$.
			%Como~${0<m-|B'|\le|S'|}$ 
			%e~${\diam^*(T[S'])>\dfrac{1}{2^{i-1}}}$, 
			Assim, se~$T[S']$ for uma árvore,
			podemos aplicar a
			%Teorema~\ref{teo:corteExato} 
			hipótese de indução à~$T[S']$ 
			usando~${m-|B'|}$ como~$m$.
			Porém, %existem casos onde o conjunto~$S'$ retornado 
			%pelo Teorema~\ref{teo:dobraDiametro} é tal 
			é possível que~$T[S']$ 
			não seja conexo.
			De fato, examinando-se a prova do 
			Teorema~\ref{teo:dobraDiametro},~$T[S']$
			%pois, de acordo com o 
			%Algoritmo~\ref{alg:dobraDiametro},
			pode se 
			dividir em duas componentes, uma no começo e outra
			no fim do caminho máximo utilizado. 

			Isso pode ser visto na figura abaixo. 


			\begin{center} \begin{tikzpicture}
				[scale=.625,auto=left,
				every node/.style={circle, draw=black,
					fill=blue!20}]

				\draw [draw=blue!65,fill=yellow!20, line width=.8pt] (-1,1.5) 
				rectangle (7, 6.5);

				\draw [draw=none, fill=yellow!20, line width=.8pt] (11,4.7) 
				rectangle (19, 6.5);
				\draw [draw=none, fill=yellow!20, line width=.8pt] (13.5,1.5) 
				rectangle (19, 6.5);

				\node [text=blue!65,draw=none, fill=none] at 
				(0,2.5) {\Large{$S$}};
				\node [text=blue!65,draw=none, fill=none] at 
				(18,2.5) {\Large{$S$}};

				\node [subtree] (y1) at (3,4.6)  {};
				\node [subtree, fill=red!30] (y2) at (6,5)  {};
				\node [subtree] (y3) at (9,4.6)  {};
				\node [subtree] (y4) at (12,5) {};
				\node [subtree] (y5) at (15,4.6) {};

				\node (x0) at (0,5.4)  {$v_0$};
				\node (x1) at (3,5)    {...};
				\node (x2) at (6,5.4)  {$v_i$};
				\node (x3) at (9,5)    {...};
				\node (x4) at (12,5.4) {$v_j$};
				\node (x5) at (15,5)   {...};
				\node (x6) at (18,5.4) {$v_p$};

				\foreach \from/\to in {x0/x1,x1/x2,x2/x3,x3/x4,
				x4/x5,x5/x6}
				\draw (\from) -- (\to);

				% Define the points of a regular pentagon
				\path (13.5,4.7) coordinate (P0);
				\path (11,4.7) coordinate (P1);
				\path (11,6.5) coordinate (P2);
				\path (19,6.5) coordinate (P3);
				\path (19,1.5) coordinate (P4);
				\path (13.5,1.5) coordinate (P5);
				% Draw the edges of the pentagon
				\draw[draw=blue!65, line width=.8pt] (P0) -- (P1) -- (P2) -- (P3) 
				-- (P4) -- (P5) -- cycle;


			\end{tikzpicture} \end{center}
			Como o Teorema~\ref{teo:corteExato} se aplica a 
			árvores, precisamos tornar~$T[S']$ conexo de forma a não 
			alterar o seu grau máximo nem o seu diâmetro relativo.
			Sabemos que se adicionarmos uma aresta~$e$ em~$T$ que 
			liga o primeiro ao último vértice do caminho máximo~$P$
			usado na prova do Teorema~\ref{teo:dobraDiametro},
			então~$T[S']$ será conexo. 
			Diremos que~${T^* = T+ e}$.

			Observe que o Algoritmo~\ref{alg:dobraDiametro} já faz
			esse acerto em~$T[S']$ antes de devolvê-lo.
			No Algoritmo~5, temos que 
			retirar arestas da árvore para 
			desconectar os vértices de~$S'$ do resto da árvore
			e acrescentar a aresta~$e$, como descrito na prova do teorema.
			Poderíamos desconectar os vértices verificando todas as arestas que 
			ligam dois vértices de~$S'$ e colocar em~$T[S']$, mas podemos
			fazer isso no Algoritmo~\ref{alg:dobraDiametro} de forma 
			mais eficiente. 
			No Algoritmo~\ref{alg:dobraDiametro},
			analisamos os vértices de~$S'$ que se ligam a vértices 
			de~$V\setminus S'$. 
			Portanto podemos usar uma marcação para indicar que as 
			arestas que ligam~$S'$ e~${V\setminus S}$ estão inativas, 
			tornando $S'$ uma componente conexa isolada. 
			Obter a árvore~$T^*$ acrescentando a aresta~$e$ em~$T$, como descrito no
			Teorema~\ref{teo:corteExato}, também pode ser feito de 
			forma mais simples no Algoritmo~\ref{alg:dobraDiametro},
			da mesma forma que foi feito o procedimento citado acima.
			Desta forma, o Algoritmo~\ref{alg:dobraDiametro} retornará
			a árvore~$T^*[S']$ separada do resto de~$T$.

			Note também que em ambas as componentes de~$T[S']$, 
			temos que o caminho máximo de cada uma possui em um dos
			seus extremos o vértice~$v_0$ ou o vértice~$v_p$.
			Logo,~$T^*[S']$ será conexo e 
			%se adicionarmos uma aresta~$e=\{v_0,v_p\}$ em~$T$,
			%então~$T[S']$ será uma componente conexa e
			teremos um caminho máximo 
			em~$T^*[S']$ com o maior número possível de vértices.
			Além disso, de acordo com o Lema~\ref{lem:grauMaximo}, 
			temos que~${\Delta(T) = \Delta(T^*)}$, já que a aresta~$e$ 
			conecta os dois vértices extremos do caminho máximo
			e estes possuem grau um.
			Podemos, então, aplicar a indução à árvore~$T^*[S']$ 
			%para o Teorema~\ref{teo:corteExato}, 
			sabendo que não há aumento
			%sem nos preocupar com a modificação 
			do grau máximo e do diâmetro relativo.
			Ou seja,
			%Como assumimos na indução que o Teorema~\ref{teo:corteExato} vale
			%para~${i-1}$, então 
			existe um corte~$(B'',W'')$ em~$T^*[S']$
			tal que~${|B''|=m-|B'|}$ 
			e~${e_{T[S]}(B'',W'')\le 4\cdot 2^{i-1}\cdot
			\Delta(T^*[S'])\le 4\cdot 2^{i-1}\cdot\Delta(T^*)}$.

			Dessa forma, sabemos que existe um corte~$(B,W)$ em~$T$ tal
			que~${B=B'\cup B''}$ e~${W=V(T)\setminus B}$, sendo 
			que~${|B|=|B'| + m-|B'| = m}$ e
			\begin{align}
				e_T(B,W)&\le e_T(B',W',S') + e_{T[S]}(B'',W'') 
				\nonumber\\
				&\le 2\cdot2^i\cdot\Delta(T) + 4\cdot 2^{i-1}\cdot
				\Delta(T^*)\nonumber\\
				&= 2\cdot2^i\cdot\Delta(T) + 4\cdot 2^{i-1}\cdot
				\Delta(T)\nonumber\\
				&= 4\cdot 2^{i}\cdot\Delta(T), \nonumber
			\end{align}
			%provando assim que o Teorema~\ref{teo:corteExato} é
			%válido para~$i$ partindo do pressuposto de que ele vale
			%para~${i-1}$. Logo, temos que o 
			%Teorema~\ref{teo:corteExato} é válido para 
			%todo~${i\in \mathbb{N^+_*}}$.
			completando a prova por indução do teorema.
		%\end{itemize}
	\end{proof}

	\bigskip
	\bigskip

	\begin{coro}[{\cite[Corolário 7]{Schmidt15}}]
	\label{coro:corteExato}
		Para toda árvore~$T$ com~$n$ vértices e todo~${m\in[n]}$, existe
		um corte~$(B,W)$ em~$T$ com~${|B|=m}$ 
		e~${e_T(B,W)\le \dfrac{8}{\diam^*(T)}\Delta(T)}$.
		%Um corte que satisfaz esses requisitos pode ser computado em 
		%tempo~$O\Big(\dfrac{n}{\diam^*(G)}\Big)$.
	\end{coro}

	\begin{proof}
		%De acordo com o Teorema~\ref{teo:corteExato},
		Para toda árvore~$T$,
		como~$0<\diam^*(T)\le 1$,
		%existe um~$i$ tal que~${\diam^*(T)>\dfrac{1}{2^i}}$. 
		%Como~${\dfrac{1}{2^i}}$ é decrescente, então 
		existe um~$j$ tal
		que~${\dfrac{1}{2^j}}<\diam^*(T)\le \dfrac{1}{2^{j-1}}$,
		o que implica que~${2^j\le\dfrac{2}{\diam^*(T)}}$.

		Assim, o Teorema~\ref{teo:corteExato} garante a existência de um 
		corte~$(B,W)$ em~$T$ com~${|B|=m}$ 
		e~$e_T(B,W)\le 4\cdot 2^j\cdot\Delta(T)\le \dfrac{8}{\diam^*(T)} \Delta(T)$,
		e esse corte satisfaz as propriedades do 
		Corolário~\ref{coro:corteExato}. 
		%então podemos afirmar a existência de cortes desse tipo.
	\end{proof}

	\bigskip
	\bigskip
	\bigskip
	\bigskip
	\bigskip

	\subsection{Algoritmo FST}

		%Seja~${T=(V,E)}$ uma árvore com~$n$ vértices.
		% Assumimos que a função INSERE\_ARESTA$(T,v_1,v_2)$ irá 
		% inserir uma aresta em~$T$, ligando os 
		% vértices~$v_1$ e~$v_2$ em tempo constante. 
		O algoritmo FST faz uso das seguintes rotinas.
		A função {\sc caminho\_máximo}$(T)$ 
		recebe uma árvore~$T$ com~$n$ vértices e
		devolve um vetor contendo os vértices de um caminho 
		máximo em~$T$, de acordo com a ordem apresentada no caminho.
		Isso pode ser feito em tempo~$O(n)$, como descrito na 
		Seção~\ref{sec:caminhoMaximo}.

		A função {\sc rotula}$(T$, $c)$ recebe uma árvore~$T$ com~$n$ vértices 
		e um vetor~$c$
		contendo os vértices de um caminho em~$T$ e,
		devolve dois vetores: o primeiro contém a rotulação dos vértices 
		de~$T$
		descrita na Seção~\ref{sec:rotulacao}, e o segundo, o inverso
		da rotulação. 
		Essa função consome tempo~$O(n)$, como mostrado na 
		Seção~\ref{sec:rotulacao}.

		%Temos uma outra modificação no 
		%Algoritmo~\ref{alg:corteExato}   
		%Algoritmo~5   
		%em relação ao Teorema~\ref{teo:corteExato}.
		Em vez de calcularmos novamente o caminho máximo da nova
		árvore, podemos reutilizar o caminho calculado anteriormente,
		aproveitando somente a parte do caminho que está contida 
		em~$S'$. 
		Se usarmos essa parte no lugar de um caminho máximo,
		manteremos a propriedade da dobra do diâmetro como pode
		ser visto na equação~(\ref{eq:reaproveitaCaminho}) da 
		Seção~\ref{sec:dobraDiametro}. 
		Dessa forma, a rotulação continua basicamente a mesma,
		bastando apenas fazer um ajuste nos vértices para a esquerda.
		
		Para reutilizar o caminho da árvore antiga, utilizamos a 
		função {\sc recalcula\_caminho}$(T,S,c)$, que
		recebe como parâmetros uma árvore~$T$, um conjunto de vértices~$S\subset T$ 
		e um vetor~$c$, que contém os vértices de um caminho na árvore~$T$ ordenados
		de acordo com a sua posição neste caminho.
		Essa função devolve um vetor que contém os vértices de~$S\cap c$, na mesma 
		ordem que aparecem em~$c$. 
		Isso pode ser feito em tempo~$O(|c|)$ se 
		o conjunto~$S$ é dado por um vetor binário.
		%mantivermos um 
		%vetor binário indicando presença ou ausência dos vértices 
		%em~$V$.

		% A função {\sc recalcula\_caminho}$(V$, $c)$
		% recebe como parâmetros um vetor~$c$, que representa um 
		% caminho na árvore, e um conjunto de vértices~$V$. Essa função devolve
		% um vetor, ordenado de acordo com~$c$, contendo os 
		% vértices em~$c$ que estão no conjunto~$V$. 
		% Isso pode ser feito em tempo~$O(|c|)$ se mantivermos um 
		% vetor binário indicando presença ou ausência dos vértices 
		% em~$V$.

		% Também temos a função {\sc recalcula\_rotulação}$(V$, $r$, $v$, $c)$,
		% um conjunto de vértices~$V$ e três vetores, o primeiro 
		% que mapeia os vértices
		% até as suas rotulações, o segundo que mapeia uma rotulação
		% até o seu vertice e o terceiro representa os vértices de um
		% caminho. 
		% Essa função retorna um vetor 


		% Seja~$vet$ um vetor, temos que as funções PRIMEIRO$(vet)$ 
		% e ULTIMO$(vet)$ devolvem o primeiro e o último elementos 
		% do vetor, respectivamente em tempo constante.

		Segue o algoritmo FST que encontra um corte que satisfaz as 
		propriedades do Teorema~\ref{teo:corteExato}.
		
		\bigskip

		\begin{algorithm}[H]
		\label{alg:corteExato}
			\SetKwInOut{Input}{entrada}
			\SetKwInOut{Output}{saída}

			\caption{}
			\Input{árvore $T=(V,E)$ com $n$ vértices e $m\in[n]$}
			\Output{corte $(B,W)$ tal que $|B|=m$ 
			e~$e_T(B,W)\le \dfrac{8}{\diam^*(T)}\Delta(T)$}
			$B\gets \emptyset$\;
			$T'[S]\gets T$\;
			$r\gets$ vértice arbitrário de~$T$\;
			$P\gets$ {\sc caminho\_máximo}$(T)$\;
			$[\rot$, $\ind]\gets$ {\sc rotula}$(T$, $P)$\;
			\While{$|B|<m$}
			{
				%\tcp{$r$ receberá um vértice que pertence ao novo conjunto~$S'$}
				$[(B',W,S)$, $T'[S]$, $r]\gets
					\mathbf {Algoritmo\ref{alg:dobraDiametro}}(T'[S]$,
					$m-|B|$, $P$, $\rot$, $\ind$,~$r)$\;
				$B\gets B\cup B'$\;
				$P\gets$ {\sc recalcula\_caminho}$(T'$, $S$, $P)$\;
				$[\rot$, $\ind]\gets$ {\sc rotula}$(T'[S]$, $P)$
				% \If{$T^*[S']$ não é conexo}
				% {
				% 	INSERE\_ARESTA$(T[S']$, PRIMEIRO$(P)$, 
				% 	ULTIMO$(P))$\;
				% }
			}
			\Return $(B,V\setminus B)$\;

		\end{algorithm}	

		\bigskip

		\subsection*{Análise do Algoritmo}

		% Sabe-se que calcular um caminho máximo faz parte do
		% Teorema~\ref{teo:dobraDiametro} e deveria estar no
		% Algoritmo~\ref{alg:dobraDiametro}, mas, precisamos 
		%acessar 
		% esse caminho máximo para, possivelmente, acrescentar a 
		% aresta citada na prova do Teorema~\ref{teo:corteExato}, 
		% portanto, encontrar um caminho máximo fará parte do 
		% Algoritmo~\ref{alg:corteExato}.

		
		%, que, nesse caso, é a árvore~$T^*$ com
		%os vértices de~$S'$ isolados do resto de~$T^*$.

		% Como usaremos apenas a árvore de~$S'$ e~$T^*$ possui outros 
		% vértices além dos vértices de~$S'$,
		% precisamos mandar um vértice de~$S'$ para
		% o Algoritmo~\ref{alg:dobraDiametro},
		% para que ele saiba que a árvore tratada em questão é
		% somente a componente conexa que possui o vértice recebido.
		% Assim, o Algoritmo~\ref{alg:dobraDiametro} irá considerar
		% apenas a componente conexa de~$S'$ e ainda
		% poderá usar o vértice recebido como raiz.


		\bigskip

		Dado que, inicialmente,~$B=\emptyset$, sempre entraremos no laço das
		linhas~7-9.
		Em cada execução da linha~7, o corte~$(B',W,S)$
		terá as propriedades de um dos seguintes itens:
		\begin{enumerate}
			\item $|B'|=m-|B|$, $S=\emptyset$ e~$e_T(B',W)\le 2$ ou
			\item ${|B'|\le m-|B|\le |B'|+|S|}$, 
				com~${0<|S|\le\dfrac{n}{2},\
				e_T(B,W,S)\le \dfrac{2\cdot 
				\Delta(T)}{\diam^*(T)}}$, 
				e~${\diam^*(T[S])\ge 2\ \diam^*(T)}$.
		\end{enumerate} 
		Se o corte~$(B',W,S)$ 
		%tiver as propriedades do item~1,
		%ou mesmo se tiver as  propriedades do item~~$|B|$
		retornado for tal que~$|B'| = m-|B|$, teremos que~$|B|=m$
		ao final do laço.
		Portanto, na linha~11, será retornado um corte~$(B,W)$ 
		com~$|B| = m$ e~$e_T(B,W) \le \dfrac{2\cdot 
			\Delta(T)}{\diam^*(T)}\le \dfrac{8}{\diam^*(T)}\Delta(T)$.
		%também devolverá 
		%algum vértice de~$S'$ que será usado como raiz na 
		%próxima interação do Algoritmo~\ref{alg:dobraDiametro}.

		Caso o corte~$(B,W,S)$ devolvido seja tal que~$|B|<m$,
		então sabe-se que ele segue as propriedades do item~2, e
		portanto executaremos o laço das linhas~6-9.
		Se em alguma das iterações do laço o Algoritmo~4 retornar
		um corte~$(B',W,S)$ tal que~$|B'| = m-|B|$,
		na linha~7 o conjunto~$|B|=m$.

		Resta saber se ao final do laço das linhas~6-9 o conjunto~$B$ terá 
		exatamente~$m$ vértices.
		Caso contrário, sabe-se que~$(B',W,S)$ satisfaz as condições
		do item~2 e~${\diam^*(T[S])\ge 2\ \diam^*(T)}$.
		Assim, sabemos que o valor diâmetro relativo de~$T[S]$ sempre
		dobra, no mínimo.
		Assim, em algum momento~$\diam^*(T)>\dfrac{1}{2}$, fazendo
		com que a chamada ao Algoritmo~4 da linha~6 retorne um corte
		que condiz com o item~1, fazendo com que~$|B| = m$.
		Sabe-se que em nenhum momento~$|B|>m$, dado a chamada ao algoritmo~4
		da linha~6 sempre retorna um corte~$(B',W,S)$ de forma que~$|B'|\le m-|B|$,
		independente de qual itens esse corte satisfizer.
		Ao final do laço, o conjunto~$S$ terá no mínimo~$m-|B|$ vértices, 
		dado que~$m\le|B'|+|S|$.


		\bigskip

		Agora vamos analisar o consumo de tempo do algoritmo.
		Sabemos que o Algoritmo~5 
		%Sabemos que o Algoritmo~\ref{alg:corteExato} 
		irá executar
		o Algoritmo~\ref{alg:dobraDiametro} que consome 
		tempo ~${O\Big(\dfrac{n}{\diam^*(T)}\Big)}$, a função
		{\sc recalcula\_caminho} que consome tempo~$O(n)$, e a função
		{\sc rotula} que consome tempo~$O(n)$.
		Sendo~$i$ um inteiro tal que~$\diam^*(T)>\dfrac{1}{2^i}$, 
		temos que cada uma das operações citadas anteriormente
		será executada no máximo~$i$ vezes.

		Considere que~$\ell$ indica a interação do \textit{while}
		da linha 5,
		começando do~$i$ e indo até~1, e~$T_\ell$ e~$n_\ell$ 
		representam a árvore da interação~$\ell$ e
		o número de vértices dessa árvore, respectivamente.
		Dado que usamos o~$T^*[S']$ como a nova árvore
		e sabendo que~$|S'|\le \dfrac{n}{2}$,
		temos que o número de vértices de~$T_{\ell}$
		é maior ou igual ao 
		dobro do número de vértices de~$T_{\ell-1}$, logo
		\begin{align}
			 & n_{\ell} \ge 2\cdot n_{\ell-1} \ge 4\cdot n_{\ell-2} \nonumber\\
			\Rightarrow & n = 2^0\cdot n_i\ge 2^1\cdot n_{i-1} \ge\cdots\ge
			2^{i-j}n_j\ge\cdots\ge
			 2^{i-2}\cdot n_{2}\ge 2^{i-1}\cdot n_1. \nonumber 
		\end{align}
		Logo, para todo~$\ell\in[i]$, temos que~$n_\ell \le\dfrac{n} {2^{i-\ell}}$.
		Portanto temos que o tempo de execução do Algoritmo~5 é
		\begin{align}
			\displaystyle\sum_{\ell=1}^{i} \Big(O\Big(\dfrac{n_\ell}{\diam^*(T)}\Big)
			+ O(n_\ell) + O(n_\ell)\Big)
			&= \displaystyle\sum_{\ell=1}^{i} O\Big(\dfrac{n_\ell}{\diam^*(T)}\Big) \nonumber\\
			&= \displaystyle\sum_{\ell=1}^{i} O\Big(\dfrac{1}{2^{i-\ell}}
			\cdot\dfrac{n}{\diam^*(T)}\Big) \nonumber\\
			&= O\Big(\dfrac{n}{\diam^*(T)}\Big). \nonumber
		\end{align}

		Logo, o algoritmo leva tempo~$O\Big(\dfrac{n}{\diam^*(T)}\Big)$.

		\bigskip
		\bigskip
		\bigskip
