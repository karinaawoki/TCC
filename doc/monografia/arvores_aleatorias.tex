\section {Geração de árvores binárias aleatórias}

Nesta seção falaremos sobre o gerador de árvores binárias 
aleatórias, cuja definição foi tirada
da Wikipédia~\cite{rbt}. Tais tipos de árvores foram utilizadas nos 
testes do algoritmo FST.

Árvores binárias aleatórias são árvores com um número 
pré-definido de vértices, construídas aleatoriamente e 
onde cada vértice possui no máximo dois filhos.

O gerador que utilizamos recebe um inteiro~$n$ e cria um vetor
com uma permutação aleatória dos inteiros de~$0$ a~$n-1$.
Depois, constrói uma árvore binária de busca, inserindo 
os vértices na ordem dada pelo vetor.
Isso leva tempo~$O(n^2)$, dado que a inserção na árvore 
binária de busca pode levar tempo linear para cada um dos~$n$ 
vértices.

Usaremos a função {\sc insere}$(r, n)$, que criará
um vértice de valor~$n$ e fará a inserção desse na árvore
enraizada no vértice~$r$.

Segue abaixo o algoritmo do gerador.

\begin{algorithm}[H]
\label{alg:ABAgenerator}
	\SetKwInOut{Input}{entrada}
	\SetKwInOut{Output}{saída}

	\caption{Gerador de árvores binárias aleatórias}
	\Input{inteiro $n$}
	\Output{raiz de uma árvore binária gerada aleatoriamente}
	\tcp{Gera permutação aleatória}
	\For {$i = 0 \to n-1$}
	{
		$v[i] \gets i$\;
	}

	\For {$i = n-1 \to 1$}{
		$indice \gets$ {\sc random}$(i)$\;
		\tcp{{\sc random}$(i)$ devolve um inteiro
		aleatório em~$\{0,\ldots,i\}$}
		$v[indice] \leftrightarrow v[i]$\;
	}
	\tcp{Inserção dos elementos do vetor na árvore binária de
	busca}
	$raiz\gets$ {\sc null}\;
	\For{$i=0\to n-1$}
	{
		{\sc insere}$(raiz$, $v[i])$\;
	}
	\Return $raiz$\;

\end{algorithm}	

\bigskip
\bigskip

% Nota-se que numa árvore desse tipo, em grande parte dos 
% casos, a maioria dos vértices estará presente no caminho mais 
% longo. 
% O que favorece um pouco o algoritmo FST.