\section {Dobra do diâmetro relativo}
\label{sec:dobraDiametro}
	O teorema a seguir é o coração do algoritmo FST.
	Ele é chamado de \textit{teorema da dobra do diâmetro relativo}.	
	Nele utilizamos a seguinte notação para uma partição~$(X, Y, Z)$
	de~$V(T)$.
	Denotamos por~$e_T(X,Y,Z)$ o número de arestas de~$T$ cujos extremos
	estão em conjuntos distintos da partição~$(X,Y,Z)$.


		\begin{teo}[{\cite[Teorema 4]{Schmidt15}}]
		\label{teo:dobraDiametro}
			Para toda árvore~$T$ com~$n$ vértices e 
			todo~${m\in [n]}$,
			o conjunto de vértices de~$T$ pode ser particionado em 
			três partes~$B$,~$W$ e~$S$ tais que vale um dos 
			seguintes itens:
			\begin{enumerate}
				\item ${|B|=m}$,~${S=\emptyset}$, e~${e_T(B,W)\le 2}$, ou
				\item ${|B|\le m\le |B|+|S|}$, 
				com~${0<|S|\le\dfrac{n}{2}}$,
				${e_T(B,W,S)\le \dfrac{2\cdot 
				\Delta(T)}{\diam^*(T)}}$, 
				e~${\diam^*(T[S])\ge 2\ \diam^*(T)}$.
			\end{enumerate}
			%Uma partição~$(B,W,S)$ que satisfaz 1 ou 2 pode ser
			%computada em tempo~${O\Big(\dfrac{n}{\diam^*(T)}\Big)}$ 
		\end{teo}

	\bigskip
	\bigskip
	
	A prova deste teorema usa os seguintes conceitos.
	Para os vértices~${a,b,c\in V(T)}$, dizemos que~$b$
	\textbf{está entre}~$a$ e~$c$ se~${a=b}$ ou~${b=c}$ 
	ou se chegamos em~$b$ antes
	de~$c$ quando aumentamos o valor de~$a$  
	de um em um.
	Dados vértices~${x,z\in V(P)}$, dizemos que 
	se~${x+1\in T_z}$, então o
	vértice~$z$ \textbf{vem depois de~$x$ em~$P$} 
	e~$x$ \textbf{vem antes de~$z$ em~$P$}.


	Um vértice~${z\in V(P)}$ é \textbf{b-especial}
	se existe um vértice~${x\in T_z}$ tal 
	que~${\map^{-}_m(x)\in V(P)}$, e~$z$ é
	\textbf{f-especial} se existe um vértice~${x\in T_z}$
	tal que~${\map^{+}_m(x)\in V(P)}$.
	(O~$b$ vem de {\it backward} e o~$f$ de {\it forward}.)

	\bigskip
	\bigskip
		% Primeiramente, faremos uma série de definições para que
		% depois possamos explicar melhor como o teorema e o algoritmo
		% funcionam.

		% Usaremos o que foi explicado na seção~\ref{sec:caminhMaximo}
		% para encontrármos um caminho máximo. 
		% Depois, usaremos 

		\begin{proof}
		Considere uma árvore~$T$ com~${n}$ vértices e um 
		inteiro~${m\in[n]}$.
		Seja~$P$ um caminho máximo em~$T$ e, considere os vértices
		rotulados como descrito na Seção~\ref{sec:rotulacao}.
		Obteremos um corte~$(B,W,S)$ da seguinte forma:
		%\begin{itemize}
		\bigskip
		\bigskip

		\textbf{Caso 1:}
			Existe um vértice~${v\in V(P)}$ tal 
			que~${\map_m^+(v)\in V(P)}$. 
			
			A tripla~$(B,W,S)$ 
			com~${B =\{v+1, v+2,\ldots,~v+m\}}$,~${W=V(T)\setminus B}$ 
			e~${S=\emptyset}$ satisfaz as propriedades do
			primeiro caso do teorema,
			como explicado na Seção~\ref{sec:map}.

	\bigskip
	\bigskip

		\textbf{Caso 2:}
			% Caso contrário, sabemos que para todo vértice~${v\in V(P)}$
			% teremos que~${\map_m^+(v)\not\in V(P)}$.
			% dado que cada vértice de~$P$ mapeia vértices distintos que estão
			% fora de~$P$.
			Para todo vértice~${v\in V(P)}$,~${\map_m^+(v)\not\in V(P)}$.

			Como~$\map_m^+(\cdot)$ e~$\map_m^-(\cdot)$ são funções bijetoras, 
			existem~$|V(P)|$ ou mais vértices fora de~$P$.
			Isso implica que~${\diam^*(T)\le\dfrac{n}{2}}$, pois no 
			máximo~$\dfrac{n}{2}$ vértices estarão em~$P$.
			Nesse caso, teremos uma tripla~$(B,W,S)$ que satisfaz as
			propriedades do segundo caso do teorema.
			Isso será detalhado melhor a seguir.

			\bigskip
			
			% Para todo~${z\in V(P)}$, e sendo~$x_1$ e~$x_2$ os menores
			% vértices de~$T_z$ tal que~${\map^-_m(x_1)\in V(P)}$ 
			% e~${\map^+_m(x_2)\in V(P)}$,
			% usaremos as seguintes definições:
			% \begin{align}
			% 	z^b   = &\map^-_m(x_1) \nonumber \\
			% 	z^f   = &\map^+_m(x_2) \nonumber \\
			% 	P_z^b = &\map^-_m(V(T_z))\cap V(P) \nonumber\\
			% 	P_z^f = &\map^+_m(V(T_z))\cap V(P) \nonumber%\\
			% 	%H_z^f = &\map^+_m(V(T_))\cap V(P) \nonumber\\
			% 	%H_z^f = &\map^+_m(V(T_z))\cap V(P) \nonumber
			% \end{align}

			Sejam~${z\in V(P)}$ um vértice b-especial 
			e~$x$ o menor vértice de~$T_z$ tal 
			que~${\map^-_m(x)\in V(P)}$. 
			Sejam~${z^b = \map^-_m(x)}$,~${P_z^b = \map^-_m(V(T_z))\cap V(P)}$
			e~${H_z^f:= \cup\{V(T_u): u\in P^b_z,\ u\ne z^b \}}$.
			%${H_z^f =T_u}$ para todo 
			%vértice~${u\in P_z^b}$ com~${u\ne z^b}$.
			Veja a próxima figura com a representação da disposição destes 
			conjuntos de vértices.

	\begin{center} \begin{tikzpicture} 
		[scale=.7,auto=left,every node/.style={circle, draw=black,
				fill=white!70}]
	\scalebox{.9}{
		\draw [decorate,decoration={brace,amplitude=10pt,mirror,
		raise=4pt}, draw=brown, line width=1.5pt] 
		(8,5.7) -- (2,5.7);

		\draw [decorate,decoration={brace,amplitude=10pt,mirror,
		raise=4pt}, draw=Emerald, line width=1.5pt] 
		(4,2.7) -- (8.5,2.7);

		\draw [draw=yellow, fill=yellow!35, line width=2pt] (18.1,3.7) 
		ellipse (1.9cm and 3.7cm); 

		\node [text=blue!65,draw=none, fill=none] at 
		(17.7,6.8) {$T_z$};


		\node [scale=1.3,subtree,fill=red!50](y1) at (2.5,5) {};
		\node [subtree, fill=Emerald]      (y2) at (5,5)   {};
		\node [subtree, fill=Emerald]      (y3) at (7.5,5) {};
		\node [scale=1.3,subtree,fill=red!50](y4) at (10,5)  {};
		\node [subtree, fill=yellow]        (y5) at (12.5,5){};
		\node [subtree, fill=yellow]        (y6) at (15,5)  {};
		\node [scale=1.9, subtree,fill=CornflowerBlue] (y7) at (18.1,5)  {};

		\node (x0) at (0,5)  {...};
		\node [fill=brown](x1) at (2.5,5)    {};
		\node [fill=brown](x2) at (5,5)  {};
		\node [fill=brown](x3) at (7.5,5)    {};
		\node (x4) at (10,5) {};
		\node (x5) at (12.5,5)   {};
		\node (x6) at (15,5) {};
		\node (x7) at (18.1,5) {};
		\node (x8) at (20.6,5) {...};

		\node [draw=none, fill=none] at 
		(2,2) {\footnotesize{$\map^-_m(v)$}};
		\node [draw=none, fill=none](nv) at (2,2){};
		\node [draw=none, fill=none](nvt) at (2.5,3.6){};

		\node [draw=none, fill=none] at 
		(9.5,2) {\footnotesize{$\map^-_m(z)$}};
		\node [draw=none, fill=none](nz) at (9.5,2){};
		\node [draw=none, fill=none](nzt) at (10,3.6){};

		\node [draw=none, fill=none] at 
		(15.5,2.6) {\footnotesize{$x$}};
		\node [draw=none, fill=none](z) at (15.5,2.6){};
		\node [draw=none, fill=none](zt) at (18,2.6){};

		\node [draw=none, fill=none] at (2.6,5.7){$z^b$};
		\node [draw=none, fill=none] at (18.1,5.6){$z$};
		\node [draw=none, fill=none] at (15,5.6){$v$};
		\node [text=Brown,draw=none, fill=none] at (5,7){$P^b_z$};
		\node [text=Emerald,draw=none, fill=none] at (6.25,1.3){$H^b_z$};

		\foreach \from/\to in {x0/x1,x1/x2,x2/x3,x3/x4,x4/x5,x5/x6,x6/x7,x7/x8}
		\draw (\from) -- (\to);

		\path[->, line width=1pt]
		(nv) edge[left=23 ] (nvt);
		\path[->, line width=1pt]
		(z) edge[right=45] (zt);
		\path[->, line width=1pt]
		(nz) edge[right=45] (nzt);
	}
	\end{tikzpicture} \end{center}
		\bigskip
			De modo análogo, para um vértice~${z\in V(P)}$ tal 
			que~$z$ é f-especial,
			definimos~${z^f = \map^+_m(x)}$,
			onde~$x$ é o menor vértice de~$T_z$ tal 
			que~${\map^+_m(x)\in V(P)}$.
			Também
			definimos~${P_z^f = \map^+_m(V(T_z))\cap V(P)}$
			e~${H_z^f:= \cup\{V(T_u): u\in P^b_z,\ u\ne z^b \}}$.
			%${H_z^f =T_u}$ para todo 
			%vértice~${u\in P_z^f}$ com~${u\ne z^f}$.
			Veja a figura abaixo com a representação da disposição 
			destes conjuntos de vértices.
	\begin{center} \begin{tikzpicture} 
		[scale=.6,auto=left,every node/.style={circle, draw=black,
				fill=white!70}]
 	\scalebox{.94}{
		\draw [decorate,decoration={brace,amplitude=10pt,mirror,
		raise=4pt}, draw=brown, line width=1.5pt] 
		(23.1+-4.5,5.9) -- (23.1+-10.5,5.9);

		\draw [decorate,decoration={brace,amplitude=10pt,mirror,
		raise=4pt}, draw=Emerald, line width=1.5pt] 
		(23.1+-8.5,2.7) -- (23.1+-4,2.7);

		\draw [draw=yellow, fill=yellow!35, line width=2pt] (23.1+-18.1,3.) 
		ellipse (2.1cm and 4.5cm); 

		\node [text=blue!65,draw=none, fill=none] at 
		(23.1+-18.5,6.8) {$T_z$};


		

		\node [scale=1.3,subtree,fill=red!50](y1) at (23.1+-2.5,5) {};
		\node [subtree, fill=Emerald]      (y2) at (23.1+-5,5)   {};
		\node [subtree, fill=Emerald]      (y3) at (23.1+-7.5,5) {};
		\node [scale=1.3,subtree,fill=red!50](y4) at (23.1+-10,5)  {};
		\node [subtree, fill=yellow]        (y5) at (23.1+-12.5,5){};
		\node [subtree, fill=yellow]        (y6) at (23.1+-15,5)  {};
		\node [scale=1.9, subtree,fill=CornflowerBlue] (y7) at (23.1+-18.1,5)  {};

		\node (x0) at (23.1+0,5)  {...};
		\node (x1) at (23.1+-2.5,5)    {};
		\node [fill=brown](x2) at (23.1+-5,5)  {};
		\node [fill=brown](x3) at (23.1+-7.5,5)    {};
		\node [fill=brown](x4) at (23.1+-10,5) {};
		\node (x5) at (23.1+-12.5,5)   {};
		\node (x6) at (23.1+-15,5) {};
		\node (x7) at (23.1+-18.1,5) {};
		\node (x8) at (23.1+-20.6,5) {};
		\node (x9) at (23.1+-23.1,5) {...};

		\node [draw=none, fill=none] at 
		(23.1+-2,1) {\footnotesize{$\map^+_m(z)$}};
		\node [draw=none, fill=none](nv) at (23.1+-2,1){};
		\node [draw=none, fill=none](nvt) at (23.1+-2.5,3.6){};

		\node [draw=none, fill=none] at 
		(23.1+-9.5,1) {\footnotesize{$\map^+_m(v)$}};
		\node [draw=none, fill=none](nz) at (23.1+-9.5,1){};
		\node [draw=none, fill=none](nzt) at (23.1+-10,3.6){};

		\node [draw=none, fill=none] at 
		(23.1+-15.5,2) {\footnotesize{$x$}};
		\node [draw=none, fill=none](z) at (23.1+-15.5,2){};
		\node [draw=none, fill=none](zt) at (23.1+-18,2){};

		\node [draw=none, fill=none] at (23.1+-9.8,5.8){$z^f$};
		\node [draw=none, fill=none] at (23.1+-18.1,5.6){$z$};
		\node [draw=none, fill=none] at (23.1+-20.6,5.6){$v$};
		\node [text=Brown,draw=none, fill=none] at (23.1+-7.5,7.3){$P^f_z$};
		\node [text=Emerald,draw=none, fill=none] at (23.1+-6.35,1.1){$H^f_z$};


		\foreach \from/\to in {x0/x1,x1/x2,x2/x3,x3/x4,x4/x5,x5/x6,x6/x7,x7/x8,x8/x9}
		\draw (\from) -- (\to);

		\path[->, line width=1pt]
		(nv) edge[left=23 ] (nvt);
		\path[->, line width=1pt]
		(z) edge[right=45] (zt);
		\path[->, line width=1pt]
		(nz) edge[right=45] (nzt);
	}
	\end{tikzpicture}\end{center}
	Note que, no Caso 2, os vértices~$\map^-_m(v)$,~$\map^-_m(z)$,~$\map^+_m(v)$ 
	e~$\map^+_m(z)$ não estão em~$V(P)$, pois
	nenhum vértice de~$P$ é mapeado em outro vértice de~$P$.

	Para facilitar a descrição, adicionamos uma aresta ligando o
	primeiro ao último 
	vértice de~$P$. 
	Assim os elementos de~$P_z^b$ e~$P_z^f$ induzem 
	um caminho.
	Para valores diferentes de~$z$,
	os conjuntos~$P_z^b$ são disjuntos.
	O mesmo vale para os conjuntos~$P_z^f$.

	Usaremos a frente as seguintes propriedades:
	\begin{enumerate}
		% \item Para todo vértice~$z$ tal que~$z$ é b-especial, 
		% temos que o vértice~$z^b$ é f-especial.

		% Isso ocorre porque para o vértice~$v$, que vem antes 
		% de~$z$ em~$P$, temos que~${\map^-_m(v)\not\in V(P)}$, 
		% logo,~$\map^-_m(v)$ está em~$T_{z^b}$.
		% Sabemos que~$\map^-_m(v)$ não está na árvore a esquerda
		% de~$T_{z^b}$, pois se estivesse, o~$z^b$ seria um outro
		% vértice.

		% \item Para todo vértice~$z$ tal que~$z$ é f-especial, 
		% temos que o vértice~$z^f$ é b-especial.

		% Isso ocorre de forma análoga a anterior, dado que
		% para o vértice~$v$, que vem antes de~$z$ em~$P$, temos
		% que~${\map^+_m(v)\in T_{z^f}}$.

		\item Um vértice~${z\in V(P)}$ é b-especial se e somente 
		se~${z=u^f}$ para um vértice
		f-especial~${u\in V(P)}$.
		% tal que~$u$ é f-especial e~${z=u^f}$.
		%Sabemos que a volta é verdadeira usando a propriedade 2.
		\begin{proof}

		%A ida é verdadeira pois, 
		De fato, se~$z$ é b-especial,
		então existe pelo menos um vértice~${v\in V(P)}$ tal
		que~${\map^+_m(v)\in T_{z}}$.
		Seja~$v$ um tal vértice com o maior rótulo e~$u$ o vértice 
		que vem depois de~$v$ em~$P$. 
		Então~${\map^-_m(z)\in T_u}$ e o vértice a esquerda
		de~$z$ em~$P$ não é
		mapeado em~$u$, logo~$u$ é f-especial e~${z=u^f}$.
		%Portanto, é correto afirmar que~${z=u^f}$ para um 
		%vértice f-especial~${u\in V(P)}$ 
		%tal que~$u$ é f-especial e~${z=u^f}$
		%se o vértice~$z$ é b-especial.
		Por outro lado, se~$u$ é f-especial 
		e~$v$ é o vértice que vem antes 
		de~$u$ em~$P$, temos que~${\map^+_m(v)\not\in V(P)}$. 
		Logo,~$\map^+_m(v)$ está em~$T_{u^f}$.
		Isso se deve ao fato de que~$\map^+_m(v)$ não está na 
		árvore 
		do vértice que vem antes de
		%à esquerda
		%de~$T_{u^f}$, 
		$u^f$, pois, se estivesse,~$u^f$ seria um outro
		vértice.
		Logo,~${z=u^f}$ é b-especial. 
		%é correto afirmar que um vértice~${z\in V(P)}$ é b-especial
		%se~${z=u^f}$ para um vértice f-especial~${u\in V(P)}$.
		% tal que~$u$ é f-especial e. 

		\end{proof}

		\item Um vértice~${z\in V(P)}$ é f-especial se e somente 
		se~${z=u^b}$ para um vértice b-especial~${u\in V(P)}$. 
		%tal que~$u$ é b-especial e~${z=u^b}$.

		Podemos verificar que essa propriedade é verdadeira de forma
		semelhante à verificação da propriedade anterior.
	\end{enumerate}

	Seja~$B_{\esp}$ 
	o conjunto de todos os vértices b-especiais,~$F_{\esp}$ o
	conjunto de todos os vértices f-especiais e~$T'_z$ o conjunto de 
	vértices~${V(T_z)\setminus \{z\}}$. 
	Usando as propriedades 
	1 e 2 respectivamente, temos que
	\begin{align}
		%\displaystyle\sum_{u\in F_{\esp}}|T_{u^f}| = 
		%\displaystyle\sum_{z\in B_{\esp}}|T_{z}| %\nonumber \\
		%\ \ &\Rightarrow \
		\displaystyle\sum_{u\in F_{\esp}}|T_{u^f}'| = 
		\displaystyle\sum_{z\in B_{\esp}}|T_{z}'| \label{eq:4}\\
		%\displaystyle\sum_{u\in B_{\esp}}|T_{u^b}| = 
		%\displaystyle\sum_{z\in F_{\esp}}|T_{z}| %\nonumber
		%\ \ &\Rightarrow \
		\displaystyle\sum_{u\in B_{\esp}}|T_{u^b}'| = 
		\displaystyle\sum_{z\in F_{\esp}}|T_{z}'| \label{eq:3}.
	\end{align}

	Note que, para cada vértice b-especial~$z$, todos os vértices 
	de~$H^b_z$ e de~$P^b_z$ são mapeados em vértices de~$T_z$ 
	%e que nem todos os vértices de~$T_z$ são mapeados por um
	%vértices~$H^b_z\cup P_z^b$, 
	%dado que
	e~$\map^-_m(z)\not\in H^b_z\cup P_z^b$.
	O mesmo ocorre com cada vértice f-especial.
	Portanto, temos que
	\begin{align}
		|T_z| > |P^b_z| + |H^b_z| 
		\ &\Rightarrow \
		|T_z'| \ge |P^b_z| + |H^b_z| \label{eq:5}\\
		|T_z| > |P^f_z| + |H^f_z|
		\ &\Rightarrow \
		|T_z'| \ge |P^f_z| + |H^f_z| \label{eq:6}.
	\end{align}

	Como os conjuntos~$P^b_z$ são dois a dois disjuntos 
	e~$P^b_z\cup H^b_z\cup T'_{z^b}$
	nada mais é do que o conjunto dos vértices das sub-árvores enraizadas
	nos vértices de~$P^b_z$, temos 
	que os conjuntos~${P^b_z\cup H^b_z\cup T'_{z^b}}$ também são 
	dois a dois disjuntos.
	Como todos os vértices de~$P$ são mapeados em alguma sub-árvore,
	todo vértice de~$V(P)$ pertence a algum~$P^b_z$.
	Logo,
	todo vértice de~$V(T)$ pertence a 
	algum~${P^b_z\cup H^b_z\cup T'_{z^b}}$.
	O mesmo é válido para~${P^f_z\cup H^f_z\cup T'_{z^f}}$.
	Portanto, para~$d = \diam^*(T)$,
	\begin{align}
		(1-d)n = |V\setminus V(P)| &= 
		\displaystyle\sum_{z\in B_{\esp}}\Big(|T_{z^b}'|+|H_z^b|\Big)
		\label{eq:7}\\
		(1-d)n = |V\setminus V(P)| &= 
		\displaystyle\sum_{z\in F_{\esp}}\Big(|T_{z^f}'|+|H_z^f|\Big)
		\label{eq:9}\\
		dn = |V(P)| &= 
		\displaystyle\sum_{z\in B_{\esp}}|P_z^b|
		\label{eq:8}\\
		dn = |V(P)| &= 
		\displaystyle\sum_{z\in F_{\esp}}|P_z^f| 
		\label{eq:10}.
	\end{align}

	Usando as equações~(\ref{eq:7}),~(\ref{eq:9}),~(\ref{eq:8}) e~(\ref{eq:10})
	e, as equações~(\ref{eq:4}) e~(\ref{eq:3}) depois, temos que
	\begin{align}
		\dfrac{1-d}{d} &= 
		\dfrac{
			\displaystyle\sum_{z\in B_{\esp}}
			\Big(|T'_{z^b}|+|H_z^b|\Big) + 
			\displaystyle\sum_{z\in F_{\esp}}
			\Big(|T'_{z^f}|+|H_z^f|\Big)}
			{\displaystyle\sum_{z\in B_{\esp}}|P_z^b| +
			\displaystyle\sum_{z\in F_{\esp}}|P_z^f|}
			\nonumber\\
		&= \dfrac{
			\displaystyle\sum_{z\in B_{\esp}}
			\Big(|T'_{z}|+|H_z^b|\Big) + 
			\displaystyle\sum_{z\in F_{\esp}}
			\Big(|T'_{z}|+|H_z^f|\Big)}
			{\displaystyle\sum_{z\in B_{\esp}}|P_z^b| +
			\displaystyle\sum_{z\in F_{\esp}}|P_z^f|}.
			\nonumber
	\end{align}
			
			Logo
	\begin{align}
			\dfrac{1-d}{d}\displaystyle\sum_{z\in B_{\esp}}|P_z^b| &+ 
			\dfrac{1-d}{d}\displaystyle\sum_{z\in F_{\esp}}|P_z^f|
			= \displaystyle\sum_{z\in B_{\esp}}
			\Big(|T'_{z}|+|H_z^b|\Big) + 
			\displaystyle\sum_{z\in F_{\esp}}
			\Big(|T'_{z}|+|H_z^f|\Big),\nonumber
	\end{align}
	o que nos nos leva a dois casos:
	\begin{enumerate}[label=(\Roman*)]   %(\alph*)] 
		\item $\displaystyle\sum_{z\in B_{\esp}}
			\Big(|T'_{z}|+|H_z^b|\Big)\le
			\dfrac{1-d}{d}\displaystyle\sum_{z\in B_{\esp}}|P_z^b|$
			ou
		\item $\displaystyle\sum_{z\in B_{\esp}}
			\Big(|T'_{z}|+|H_z^b|\Big)>
			\dfrac{1-d}{d}\displaystyle\sum_{z\in B_{\esp}}|P_z^b|
			\ \Rightarrow \\
			\displaystyle\sum_{z\in F_{\esp}}
			\Big(|T'_{z}|+|H_z^f|\Big)<
			\dfrac{1-d}{d}\displaystyle\sum_{z\in F_{\esp}}|P_z^f|
			\ \Rightarrow \\
			\displaystyle\sum_{z\in F_{\esp}}
			\Big(|T'_{z}|+|H_z^f|\Big)\le
			\dfrac{1-d}{d}\displaystyle\sum_{z\in F_{\esp}}|P_z^f|$.
	\end{enumerate}

	Sabemos que, se (I) ocorrer, há um 
	vértice~${z\in B_{\esp}}$ tal que~${|T'_z|+|H^b_z|\le \dfrac{1-d}{d}|P^b_z|}$,
	pois se~${|T'_z|+|H^b_z| > \dfrac{1-d}{d}|P^b_z|}$ para
	todo~${z\in B_{\esp}}$,
	então (I)
	não teria ocorrido.
	O mesmo ocorre com (II). 
	Portanto, um dos casos a seguir sempre será válido:

	\begin{enumerate}[label=(\alph*)]
		\item existe um vértice~$z\in B_{\esp}$ 
			tal que~$|T'_{z}|+|H_z^b|\le
			\dfrac{1-d}{d}|P_z^b|$ ou
		\item existe um vértice~$z\in F_{\esp}$ 
			tal que~$|T'_{z}|+|H_z^f|\le
			\dfrac{1-d}{d}|P_z^f|.$
	\end{enumerate}

	\bigskip
	\bigskip
	
	\textbf{Caso (a):}
		Existe um vértice~$z\in B_{\esp}$ 
		tal que~${|T'_{z}|+|H_z^b|\le
		\dfrac{1-d}{d}|P_z^b|}$. 
		Disso,~${|H^b_z|\le \Big(\dfrac{1}{d}-1\Big)|P^b_z|-|T'_z|}$.
		Usando a segunda desigualdade em~(\ref{eq:5}), deduzimos
		que
		\begin{align}
			%|H^b_z|\le\Big(\dfrac{1}{2d}-1\Big)|P^b_z|
			%\Rightarrow
			|H^b_z|+|P^b_z|\le\dfrac{1}{2d}|P^b_z|. \nonumber
		\end{align}

		Construiremos a tripla~${(B,W,S)}$ 
		tomando~$S = H^b_z\cup P^b_z$.
		Sabemos que~$S\ne \emptyset$, pois~$z$ é b-especial,
		o que implica que~$P^b_z\ne \emptyset$. 
		Dessa forma, temos que
		\begin{align}
			|S| = |H^b_z|+|P^b_z|\le\dfrac{1}{2d}|P^b_z|
			&\Rightarrow
			%|S| \le\dfrac{1}{2d}|P^b_z| \le \dfrac{n}{2} \nonumber \\
			%&\Rightarrow
			%|S| \le\dfrac{1}{2d}|P^b_z|\le \dfrac{n}{2} 
			%\label{eq:metade} \\
			%&\Rightarrow
			2d\le \dfrac{1}{|S|}|P^b_z|  
			\label{eq:reaproveitaCaminho},
			%\nonumber
		\end{align}
		ou seja,~$\diam^*(T[S])\ge 2d$.

		Seja~${z^b_l\in P^b_z}$ um vértice tal 
		que~$\map^+_m(z^b_l)$ é o vértice de~$T_z$ com 
		maior rótulo possível, e~$j$ o vértice que vem
		antes de~$z$ em~$P$. Seja
		\begin{align}
			B_1 = \Big\{ v\in V(T):z^b_l<v\le j \Big\}
			\nonumber.
		\end{align}

		Agora podemos aplicar o Lema~\ref{lema:approxCutForest}
		à floresta~$T_z'$, com~${m-|B_1|}$ 
		e~${2-\dfrac{1}{1-d}}$
		no lugar de~$m$ e~$c$,
		respectivamente.
		Dessa forma, existe um corte~$(B_2,W_2)$ tal 
		que~${c\cdot(m-|B_1|)\le|B_2|\le m-|B_1|}$ e
		\begin{align}
		e_{T_z'}(B_2,W_2)&\le \ceil[\Big]{ \dfrac{2c}{1-c}} 
		\Delta(T_z') = \ceil[\Big]{ \dfrac{2}{d}-4} \Delta(T_z') 
		\nonumber\\
		&\le \ceil[\Big]{ \dfrac{2}{d}-4} \Delta(T) 
		\nonumber\\
		&\le \Big( \dfrac{2}{d}-3 \Big) \Delta(T).
		\label{eq:corte}
		\end{align}

		%~${\map^+_m(z^b_l)\in T'_z}$, sabemos que~$m\le|B_1| + |T'_z|$,
		%o que implica
		Como~${\map^+_m(z^b_l)\in T'_z}$ e todos
		os vértices entre~${\map^+_1(z^b_l)}$
		%$z^b_l+1$ 
		e~$j$ 
		estão em~$B_1$, 
		sabemos que~${m\le|B_1| + |T'_z|}$, que implica
		que~${m-|B_1|<|T_z|}$.  
		Logo,~${1\le m-|B_1|\le |T'_z|}$.
		Como~${d\in (0,\frac{1}{2}]}$, temos que~${c\in [0,1)}$.
		Portanto, os valores usados como~$m$ e~$c$ são 
		compatíveis com o que é pedido no 
		Lema~\ref{lema:approxCutForest}.
		A disposição dos vértices ficará como na figura a
		seguir.
	\begin{center} \begin{tikzpicture}[scale=.7,auto=left,
			every node/.style={circle, draw=black,
			fill=white!70}]
	\scalebox{.9}{
		\draw [draw=red!50, line width=2pt, fill=red!17](9,2.5) rectangle (16,6);

		\draw [draw=Emerald, line width=2pt, fill=Emerald!20]
		(8.7,1.4) -- (8.7,7.4) -- (1.8,7.4) -- (1.8,4.5) -- 
		(3.7,4.5) -- (3.7,1.4) -- cycle;

		\draw [draw=CornflowerBlue, line width=2pt, fill=CornflowerBlue!30]
		(16.3,1.4) -- (17.3,3.9) -- (18.3,1.4) -- cycle;

		\draw [decorate,decoration={brace,amplitude=10pt,mirror,
		raise=4pt}, draw=brown, line width=1.5pt] 
		(8,5.7) -- (2,5.7);

		\draw [decorate,decoration={brace,amplitude=10pt,mirror,
		raise=4pt}, draw=Emerald, line width=1.5pt] 
		(4,3.2) -- (8.5,3.2);
		


		\node [scale=1.3,subtree,fill=white](y1) at (2.5,5) {};
		\node [subtree, fill=Emerald]      (y2) at (5,5)   {};
		\node [subtree, fill=Emerald]      (y3) at (7.5,5) {};
		\node [subtree,fill=red!50]        (y4) at (10,5)  {};
		\node [subtree,fill=red!50]        (y5) at (12.5,5){};
		\node [subtree,fill=red!50]        (y6) at (15,5)  {};
		\node [scale=1.9, subtree,fill=CornflowerBlue!70] (y7) at (18.1,5)  {};

		\node (x0) at (0,5)  {...};
		\node [fill=brown](x1) at (2.5,5)    {};
		\node [fill=brown](x2) at (5,5)  {};
		\node [fill=brown](x3) at (7.5,5)    {};
		\node (x4) at (10,5) {};
		\node (x5) at (12.5,5)   {};
		\node (x6) at (15,5) {};
		\node (x7) at (18.1,5) {};
		\node (x8) at (20.6,5) {...};

		\node [draw=none, fill=none] at (7.6,5.7){$z^b_l$};
		\node [draw=none, fill=none] at (2.6,5.7){$z^b$};
		\node [draw=none, fill=none] at (18.1,5.6){$z$};
		\node [draw=none, fill=none] at (15,5.6){$j$};
		\node [text=Brown,draw=none, fill=none] at (5,7){$P^b_z$};
		\node [text=Emerald,draw=none, fill=none] at (6.25,1.8){$H^b_z$};

		\node [text=Emerald,draw=none, fill=none] at (6.2,0.8){${S}$};
		\node [text=red!50,draw=none, fill=none] at (12.5,1.9){${B_1}$};
		\node [text=CornflowerBlue,draw=none,fill=none] at 
		(17.3,0.8){${B_2}$};

		\foreach \from/\to in {x0/x1,x1/x2,x2/x3,x3/x4,x4/x5,x5/x6,x6/x7,x7/x8}
		\draw (\from) -- (\to);

		\draw [draw=Emerald, line width=2pt]
		(8.7,1.4) -- (8.7,7.4) -- (1.8,7.4) -- (1.8,4.5) -- 
		(3.7,4.5) -- (3.7,1.4) -- cycle;

		\draw [draw=CornflowerBlue, line width=2pt]
		(16.3,1.4) -- (17.3,3.9) -- (18.3,1.4) -- cycle;
	
	}
	\end{tikzpicture} \end{center}
		Temos, então, a tripla~$(B,W,S)$ com~${B = B_1\cup B_2}$ 
		e~${W = V(T)\setminus (B\cup S)}$.
		Dado que~${\map^+_m(z^b_l)\in T_z}$, temos 
		que~${m>|B_1|}$ e, como obtemos~$B_2$ através do 
		Lema~\ref{lema:approxCutForest} com~${m-|B_1|}$ 
		no lugar de~$m$, temos que~${|B_2|\le m-|B_1|}$, implicando
		em~${|B|\le m}$.
		De acordo com o Lema~\ref{lema:approxCutForest}, também temos 
		que~${|B_2|\ge c\cdot(m-|B_1|)}$ e, como estamos no Caso~2, 
		então~${d\le \dfrac{1}{2}}$ e~${|P^b_z|\le |S|}$, dado 
		que~$P^b_z\subseteq S$.
		Note que a 
		desigualdade~${|T'_z| + |H^b_z|\le \dfrac{1-d}{d}|P^b_z|}$
		implica que~$|T'_z|\dfrac{d}{1-d}\le |P^b_z|$.
		%cada vértice de~$P^b_z$ irá mapear em um vértice diferente de~$S$.
		Portanto,
		\begin{align}
			m-|B| &= m-|B_1|-|B_2| \nonumber\\
			&\le(m-|B_1|)-c\cdot (m-|B_1|) = (m-|B_1|)(1-c)
			\nonumber \\
			&\le|T_z'|(1-c) = |T_z'|\dfrac{d}{1-d}
			\nonumber 
			\le |P^b_z| \le |S|, \nonumber
		\end{align}
		o que implica que
		$m\le |B|+|S|$ e, portanto,
		$|B|\le m\le |B|+|S|$. 

		Agora mostraremos 
		que~${e_T(B,W,S)\le \dfrac{2\cdot \Delta(T)}{\diam^*(T)}}$.
		% Sabemos que os vértices do corte~$(B,W,S)$ são as arestas
		% que ligam~$z^b_l$ e~$f$ aos vértices que vêm depois deles 
		% em~$P$
		Note que o número de arestas na tripla~$(B,W,S)$ é, no 
		máximo,~${(\Delta(T)-1) + 1 + 1 + e_{T_z'}(B_2,W_2) +
		(\Delta(T)-2)}$.
		Essa soma representa a quantidade total máxima de arestas no corte,
		considerando
		as que conectam~$z^b$
		a~$W$, a que liga~$z^b_l$ a~$B_1$, a
		que conecta~$f$ e~$z$, as presentes no corte~$(B_2,W_2)$ e, 
		por fim, as arestas que ligam~$z$ a~$T_z'$ (talvez
		existam arestas ligando~$B_2$ a~$z$).
		Logo, usando a desigualdade~(\ref{eq:corte}), temos que
		\begin{align}
			e_T(B,W,S) &= (\Delta(T)-1) + 1 + 1 + e_{T_z'}(B_2,W_2) +
			(\Delta(T)-2) \nonumber\\
			&= 2\cdot\Delta(T) + e_{T_z'}(B_2,W_2) - 1 
			\nonumber\\
			&\le 2\cdot\Delta(T) + \Big( \dfrac{2}{d}-3 \Big) 
			\Delta(T) - 1 = \Big( \dfrac{2}{d}-1 \Big)\Delta(T)-1 
			\nonumber\\
			&< \dfrac{2\cdot \Delta(T)}{d}. \nonumber
		\end{align}

		Portanto, se a condição do Caso~(a) for satisfeita, então 
		existe uma tripla~$(B,W,S)$ que satisfaz as propriedades do
		Teorema~\ref{teo:dobraDiametro}.

	\bigskip
	\bigskip
	
	\textbf{Caso (b):}
	Existe um vértice~$z\in F_{\esp}$ 
	tal que~${|T'_{z}|+|H_z^f|\le
	\dfrac{1-d}{d}|P_z^f|}$. 
	Temos
	que~${|H^f_z|\le \Big(\dfrac{1}{d}-1\Big)|P^f_z|-|T'_z|}$.
	Usando a segunda desigualdade em~(\ref{eq:6}), temos 
	que~${|H^f_z|\le\Big(\dfrac{1}{2d}-1\Big)|P^f_z|}$, 
		o que implica que
		${|H^f_z|+|P^f_z|\le\dfrac{1}{2d}|P^f_z|}$.


	Tomaremos~${S = H^f_z \cup P^f_z}$ 
	e~${B_1 = \{v\in V(T): z<v<z^f\}}$. 
	Da mesma forma que no caso anterior, temos 
	que~${2d\le \diam^*(T[S])}$.

	Obtemos o corte~$(B_2,W_2)$ do 
	Lema~\ref{lema:approxCutForest} aplicado a~$T'_z$ 
	com~${m-|B_1|}$ e~${2-\dfrac{1}{1-d}}$ no lugar
	de~$m$ e~$c$, respectivamente. 
	Temos também que~${B = B_1 \cup B_2}$ 
	e~${W = V(T)\setminus (B\cup S)}$.
	Dessa forma, a disposição dos 
	vértices está ilustrada na figura a seguir.
	\begin{center} \begin{tikzpicture}
		[scale=.7,auto=left,every node/.style={circle, draw=black,
				fill=white!70}]
	\scalebox{0.9}{
		\draw [draw=red!50, line width=2pt, fill=red!17]
		(20.6+-8.83,2.5) -- (20.6+-8.83,4.57) -- (20.6+-10.93,4.57) -- 
		(20.6+-10.93,6) -- (20.6+-16,6) -- (20.6+-16,2.5) -- cycle;

		\draw [draw=Emerald, line width=2pt, fill=Emerald!20]
		(20.6+-8.7,1.4) -- (20.6+-8.7,4.7) -- (20.6+-10.8,4.7) -- 
		(20.6+-10.8,7.4) -- (20.6+-3.7,7.4) -- (20.6+-3.7,1.4) -- cycle;

		\draw [draw=CornflowerBlue, line width=2pt, fill=CornflowerBlue!30]
		(20.6+-16.3,1.4) -- (20.6+-17.3,3.9) -- (20.6+-18.3,1.4) -- cycle;

		\draw [decorate,decoration={brace,amplitude=10pt,mirror,
		raise=4pt}, draw=brown, line width=1.5pt] 
		(20.6+-4.5,5.7) -- (20.6+-10.5,5.7);

		\draw [decorate,decoration={brace,amplitude=10pt,mirror,
		raise=4pt}, draw=Emerald, line width=1.5pt] 
		(20.6+-8.5,3.2) -- (20.6+-4,3.2);
		


		\node [scale=1.3,subtree,fill=white](y1) at (20.6+-2.5,5) {};
		\node [subtree, fill=Emerald]      (y2) at  (20.6+-5,5)   {};
		\node [subtree, fill=Emerald]      (y3) at  (20.6+-7.5,5) {};
		\node [subtree,fill=red!50]        (y4) at  (20.6+-10,5)  {};
		\node [subtree,fill=red!50]        (y5) at  (20.6+-12.5,5){};
		\node [subtree,fill=red!50]        (y6) at  (20.6+-15,5)  {};
		\node [scale=1.9, subtree,fill=CornflowerBlue!70] (y7) at (20.6+-18.1,5)  {};

		\node (x0) at (20.6+0,5)  {...};
		\node (x1) at (20.6+-2.5,5)    {};
		\node [fill=brown](x2) at (20.6+-5,5)  {};
		\node [fill=brown](x3) at (20.6+-7.5,5)    {};
		\node [fill=brown](x4) at (20.6+-10,5) {};
		\node (x5) at (20.6+-12.5,5)   {};
		\node (x6) at (20.6+-15,5) {};
		\node (x7) at (20.6+-18.1,5) {};
		\node (x8) at (20.6+-20.6,5) {...};

		\node [draw=none, fill=none] at (20.6+-10,5.7){$z^f$};
		\node [draw=none, fill=none] at (20.6+-18.1,5.6){$z$};
		\node [text=Brown,draw=none, fill=none] at   (20.6+-7.5,7){$P^f_z$};
		\node [text=Emerald,draw=none, fill=none] at (20.6+-6.25,1.8){$H^f_z$};

		\node [text=Emerald,draw=none, fill=none] at(20.6+-6.2,0.8){${S}$};
		\node [text=red!50,draw=none, fill=none] at (20.6+-12.5,1.9){${B_1}$};
		\node [text=CornflowerBlue,draw=none,fill=none] at 
		(20.6+-17.3,0.8){${B_2}$};

		\foreach \from/\to in {x0/x1,x1/x2,x2/x3,x3/x4,x4/x5,x5/x6,x6/x7,x7/x8}
		\draw (\from) -- (\to);

		\draw [draw=red!50, line width=2pt]
		(20.6+-8.83,2.5) -- (20.6+-8.83,4.57) -- (20.6+-10.93,4.57) -- 
		(20.6+-10.93,6) -- (20.6+-16,6) -- (20.6+-16,2.5) -- cycle;

		\draw [draw=Emerald, line width=2pt]
		(20.6+-8.7,1.4) -- (20.6+-8.7,4.7) -- (20.6+-10.8,4.7) -- 
		(20.6+-10.8,7.4) -- (20.6+-3.7,7.4) -- (20.6+-3.7,1.4) -- cycle;

		\draw [draw=CornflowerBlue, line width=2pt]
		(20.6+-16.3,1.4) -- (20.6+-17.3,3.9) -- (20.6+-18.3,1.4) -- cycle;
	}
	\end{tikzpicture} \end{center}
	Assim como foi mostrado no Caso (a), temos 
	que~${e_T(B,W,S)<\dfrac{2\cdot \Delta(T)}{d}}$,
	satisfazendo assim as propriedades do Teorema~\ref{teo:dobraDiametro} 
	em todos os casos apresentados anteriormente.

\end{proof}
	


	\bigskip
	\bigskip
%%%%%%%%%%%%%%%%%%%%%%%%%%%%%%%%%%%%%%%%%%%%%%%%%%%%%%%%%%%%%%%%%%
%%%%%%%%%%%%%%%%%%%%%%%%%%%%%%%%%%%%%%%%%%%%%%%%%%%%%%%%%%%%%%%%%%
%%%%%%%%%%%%%%%%%%%%%%%%%%%%%%%%%%%%%%%%%%%%%%%%%%%%%%%%%%%%%%%%%%

	\subsection{Algoritmo da dobra do diâmetro}

		O algoritmo aqui apresentado ilustra
		a função {\sc separa\_conjunto}$(T,$ $S)$,
		%recebe uma árvore~$T$ e um conjunto~$S$ de vértices que
		%seja como o conjunto~$S$ especificado na seção 6.1 e
		%separa o conjunto~$S$ 
		%do restante da árvore.
		%Isso é feito removendo as arestas que estão no 
		%corte~$(S,V(T)\setminus S)$.
		que remove de~$T$ as arestas que estão no 
		corte~$(S, V(T)\setminus S)$, separando~$S$ do restante
		da árvore~$T$.
		Caso~$T[S]$ seja conexo,~$v_i$ e~$v_j$ são os vértices em~$S$ 
		%que estãolocalizados nos extremos.  
		de menor e maior rótulos, respectivamente.
		Caso contrário,~$v_i$ é o vértice de menor rótulo
		da componente de~$T[S]$ que contém o final do caminho máximo~$P$,
		e~$v_j$ é o vértice de maior rótulo da componente de~$T[S]$ que
		contém o início de~$P$.
		Sabemos que todos os vértices de~$S$ possuem
		rótulos consecutivos,
		%(considerando que o vértices de maior rótulo e 
		%de menor rótulo são consecutivos) 
		considerando que os vértices estão dispostos de forma circular.
		Sabemos também que~$v_i$ e~$v_j$ possuem ao todo~${\grau(v_i)}$
		arestas que os conectam a~$V(T)\setminus S$.
		(Veja a figura anterior.)
		Portanto, o que essa função faz é remover de~$T$ as arestas que 
		conectam~$v_i$ e~$v_j$ a~$V(T)\setminus S$ e, caso~$T[S]$ não seja
		conexo, inserir uma aresta em~$T$ que liga os vértices de menor e maior 
		rótulos em~$T$.
		Essa função devolverá um grafo~$T^*$, que é a árvore~$T$ modificada,
		onde~$T^*[S]$ será uma componente
		isolada do restante de~$T^*$. 
		%Chamaremos esse novo grafo de~$T'[S]$.
		Essa função consome tempo~$O(\Delta(T))$.

		Ao longo do Algoritmo~\ref{alg:dobraDiametro}, usaremos
		os conjuntos~$H_b^z$,~$H_f^z$,~$P_b^z$,~$P_f^z$ e~$T'_z$
		para diferentes vértices~$z$.
		Para calcular cada um destes conjuntos, basta fazer operações
		de soma ou subtração
		nos rótulos para cada~$z$.
		Como os conjuntos~${P^b_z\cup H^b_z\cup T'_{z^b}}$ são dois a dois disjuntos
		e os conjuntos~${P^f_z\cup H^f_z\cup T'_{z^f}}$ também são dois a dois disjuntos,
		então cada vértice da árvore será analisado no máximo duas vezes.
		Com isso, sabemos que a operação que calcula os vértices de cada um desses 
		conjuntos para todos os valores de~$z$ consome tempo~$O(n)$.

		Serão passados como parâmetros dois vetores,~$\rot$ e~$\ind$,
		que armazenam o rótulo de um vértice 
		%de acordo com o seu 
		na posição de seu índice
		e o índice de um vértice %de 
		na posição de seu rótulo, respectivamente.

		% Sabemos que a operação que calcula os vértices de cada um desses 
		% subconjuntos para todos os valores de~$z$ consome tempo~$O(2n)$,
		% dado que~${P^b_z\cup H^b_z\cup T'_{z^b}}$ é disjunto para todos os 
		% valores de~$z$ e~${P^b_z\cup H^b_z\cup T'_{z^b}}$ também é disjunto
		% para todos os valores de~$z$.
		% Portanto, cada vértice da árvore será analisado no máximo duas vezes.

		Segue o algoritmo que encontra um corte que satisfaz as propriedades do
		Teorema~\ref{teo:dobraDiametro}.

		\bigskip

	\begin{algorithm}[H]
	\label{alg:dobraDiametro}
		\SetKwInOut{Input}{entrada}
		\SetKwInOut{Output}{saída}

		\caption{}
		\Input{árvore~$T$ com~$n$ vértices,~$m\in[n]$, caminho máximo~$P$ em~$T$,
		vetor~$\rot$, vetor~$\ind$ e raiz~$r$ da árvore~$T$} 
		\Output{tripla~$(B,W,S)$ que satisfaz as propriedades do 
		Teorema~\ref{teo:dobraDiametro}, uma árvore~$T^*$ que corresponde
		a~$T[S]$ e um vértice~$s\in S$}
		\For{$i = 1 \to |V(P)|$}
		{
			\If{$\ind(\rot(i)+m)\in V(P)$}
			{
				\tcp{Caso 1}
				$B\gets \{\ind(\rot(i)+1),\ldots, \ind(\rot(i)+m)\}$\;
				$W\gets V(T)\setminus B$\;

				\Return $[(B,W,\emptyset)$, $\emptyset$, $null]$\;
			}
		}
		\tcp{Caso 2}
		$valor_B \gets \infty$\;
		\For{$z' \in B_{\esp}$}
		{
			\If{$\frac{|T'_{z'}|+|H_{z'}^b|}{|P^b_{z'}|} < valor_B$}
			{
			 	$valor_B \gets \frac{|T'_{z'}|+|H_{z'}^b|}{|P^b_{z'}|}$\;
			 	$z_B \gets z'$\; 
			}
		}
		$valor_F \gets \infty$\;
		\For{$z' \in F_{\esp}$}
		{
			\If{$\frac{|T'_{z'}|+|H_{z'}^f|}{|P^f_{z'}|} < valor_F$}
			{
			 	$valor_F \gets \frac{|T'_{z'}|+|H_{z'}^f|}{|P^f_{z'}|}$\;
			 	$z_F \gets z'$\; 
			}
		}
	\end{algorithm}	

	\LinesNumberedHidden
	\begin{algorithm}[H]
	\Numberline
		\If{$valor_B\le valor_F$}
		{
			\tcp{Caso 2(a)}
			\Numberline$z\gets z_B$\;
			\Numberline$z^b_\ell \gets$ último vértice de~$P_z^b$\;
			\Numberline$B_1 \gets \{\ind(\rot(z^b_l)+1),\ldots, \ind(\rot(z^b_l)+m)\}$\;
			\Numberline$(B_2,W_2) \gets \mathbf {Algoritmo\ref{alg:approxCutForest}}(T'_z,$ $|T'_z|,$ $m-|B_1|,$ $2-\frac{1}{1-\diam^*(T)})$\;
			\Numberline$B \gets B_1\cup B_2$\;
			\Numberline$W \gets V\setminus (B\cup S)$\; 
			\Numberline$S \gets P_z^b\cup H^b_z$\;
			\Numberline$T^*\gets $ {\sc separa\_conjunto}$(T$, $S)$\;
			\Numberline\Return $[(B,W,S)$, $T^*$, $z^b_\ell]$\;
		}
		\Numberline
		\Else
		{
			\tcp{Caso 2(b)}
			\Numberline$z\gets z_F$\;
			\Numberline$z^f \gets $ primeiro vértice de~$P_z^f$\;
			\Numberline$B_1 \gets \{\ind(\rot(z)+1),\ldots, \ind(\rot(z^f)-1)\}$\;
			\Numberline$(B_2,W_2) \gets \mathbf {Algoritmo\ref{alg:approxCutForest}}(T'_z,$ $|T'_z|,$ $m-|B_1|,$ $2-\frac{1}{1-\diam^*(T)})$\;
			\Numberline$B \gets B_1\cup B_2$\;
			\Numberline$W \gets V\setminus (B\cup S)$\; 
			\Numberline$S \gets P_z^f\cup H^f_z $\;
			\Numberline$T^*\gets $ {\sc separa\_conjunto}$(T$, $S)$\;
			\Numberline\Return $[(B,W,S)$, $T^*$, $z^f]$\;		
		}
	\end{algorithm}	
	\LinesNumbered
	\bigskip
	\bigskip
	\subsection*{Análise do algoritmo}
	Dado que o algoritmo recebe como entrada
	%Dado que já recebemos 
	o caminho máximo e a rotulação dos vértices
	da árvore induzida por esse caminho, precisamos apenas verificar em 
	qual dos casos a árvore~$T$ se encaixa
	e encontrar a partição~$(B,W,S)$.

	Primeiramente, verificamos se~$T$ se encaixa no Caso~1.
	Para isso é necessário percorrer todos os vértices do caminho máximo
	e verificar se algum deles é mapeado em um outro vértice do caminho máximo.
	Isso leva tempo linear no número de vértices no caminho máximo, que
	é~$O(n)$.

	Se a árvore~$T$ se encaixa no Caso~1, percorremos os vértices que
	serão devolvidos em tempo~$O(n)$.
	Caso contrário, precisamos encontrar os vértices pertencentes aos 
	conjuntos~$H_b^z$,~$H_f^z$,~$P_b^z$,~$P_f^z$ e~$T'_z$
	para cada~$z$,
	e isso é feito em tempo~$O(n)$, como mostrado anteriormente.

	% Feito isso,
	% para cada vértice~$z$ b-especial,
	% calculamos~${\alpha = \dfrac{|T'_{z}|+|H_{z}^b|}{|P^b_{z}|}}$
	% e, para cada vértice~$z'$ f-especial, 
	% calculamos~${\alpha = \dfrac{|T'_{z}|+|H_{z}^b|}{|P^b_{z}|}}$

	Feito isso, calculamos os 
	valores~${\dfrac{|T'_{z}|+|H_{z}^b|}{|P^b_{z}|}}$
	e~${\dfrac{|T'_{z}|+|H_{z}^b|}{|P^b_{z}|}}$
	para todos os vértices b-especiais e f-especiais, respectivamente.
	Isso leva tempo linear no número de vértices b-especiais e f-especiais,
	que é~$O(n)$.

	Depois disso, tanto no Caso 2(a) quanto no 2(b), realizamos uma chamada ao 
	Algoritmo~\ref{alg:approxCutForest}, que leva 
	tempo~$O\Big(\ceil[\Big]{ \dfrac{2c}{1-c}} n+1\Big)$,
	onde o valor passado como~$c$ é~$2-\dfrac{1}{1-d}$.
	Então
	temos que
	$$ \ceil[\Big]{\dfrac{2c}{1-c}} n+1 
	= \Big(\ceil[\Big]{\dfrac{2}{d}}-4 \Big)n+1 
	\le \ceil[\Big]{\dfrac{2}{d}}n.$$ 
	Logo, o
	Algoritmo~\ref{alg:approxCutForest} leva tempo~$O\Big(\dfrac{n}{d}\Big)$.

	Temos também a chamada a {\sc separa\_conjunto}$(T,$ $S)$, que 
	leva tempo~$O(\Delta(T))$, como mostrado nesta seção.
	
	Sendo assim, se a árvore se encaixa no Caso~1,
	o consumo de tempo do Algoritmo~4 é~$O(n)$.
	Caso contrário, serão executadas três operações de custo~$O(n)$
	e uma de custo~$O\Big(\dfrac{n}{d}\Big)$, como mostrado anteriormente. 
	%Então, podemos dizer que o
	
	Como~$d\le1$, temos que~$\dfrac{n}{d}\ge n$, o que implica que o
	consumo de tempo do Caso~2 é~$O\Big(\dfrac{n}{d}\Big)$, que 
	é o mesmo que~$O\Big(\dfrac{n}{\diam^*(T)}\Big)$.

	Portanto, é possível encontrar uma partição~$(B,W,S)$
	que satisfaça as condições do teorema em 
	tempo~$O\Big(\dfrac{n}{\diam^*(T)}\Big)$.




% Caso contrário, observe que o algoritmo termina 
% de executar o laço das linhas 7-8(?), pois a árvore
% é finita e o faz com tempo~$O(n)$ já que cada 
% vértice é percorrido no máximo uma vez.

% Seja~$v$ o vértice em que o laço das linhas 7-8(?)
% termina.
% Se o algoritmo executa a linha 11(?), o consumo de 
% tempo~$O(n)$ e o corte devolvido é tal 
% que~$\dfrac{m}{2}<|B|\le m$ e~$e_G(B,W) = 1$ 
% pois _____.
% Senão é porque todos os filhos~$u$ do vértice~$v$
% são tais que~$d[u]\le \dfrac{m}{2}$. 
% Claro que~$|B|\le m$ nesse caso.
% No início de cada iteração do laço das linhas 13-19(?)
% exceto a última, vale que~$|B|\le \dfrac{m}{2}$.
% Assim, após a execução da linha 15(?), vale
% que~$|B|\le m$, pois aumentamos~$|B|$ 
% de~$d[u]\le \dfrac{m}{2}$.
% Como~$d[v]\ge m$, eventualmente~$|B|\ge \dfrac{m}{2}$
% (de fato, note que como~$m\in[n]$ e~$m\neq n$,...)

	\bigskip
	\bigskip
	\bigskip
