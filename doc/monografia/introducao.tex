\section{Introdução}

O assunto abordado neste trabalho de conclusão de curso (TCC) se
enquadra na área de Otimização Combinatória e diz respeito ao 
\emph{Problema da Bissecção Mínima}. Para definir o problema 
precisamente, seguem primeiro algumas definições. 

Seja~${G=(V,E)}$ um grafo e~$B$ e~$W$ conjuntos de vértices de~$G$.
Dizemos que~$(B,W)$ é um \textbf{corte}
de~$G$ se~${B \cup W =V}$ e~${B\cap W =\emptyset}$.
Dizemos também que uma aresta~$e$ de~$G$ está no corte~$(B,W)$
se~$e$ tem uma ponta em~$B$ e outra em~$W$. 
Denotamos o número de arestas no corte~$(B,W)$ por \textbf{largura}
do corte ou por~$e_G(B,W)$.
O corte~$(B,W)$ é uma \textbf{bissecção} se~${|B| =|W|}$
quando~$|V|$ é par, ou se~${||B|-|W|| =1}$ quando~$|V|$ é ímpar.

O \emph{Problema da Bissecção Mínima} consiste em, dado um grafo, 
encontrar uma bissecção no grafo de largura mínima.

Sabe-se que o Problema da Bissecção Mínima é NP-difícil~\cite{GareyJS76} 
e a melhor aproximação conhecida para o caso geral do problema tem 
razão~$\Oh(\lg n)$~\cite{Racke08}, onde~$n$ é o número de vértices 
do grafo. 
Por outro lado, sabe-se que, para árvores e para grafos que  
se assemelham a árvores, há um algoritmo polinomial 
para encontrar uma bissecção mínima, 
proposto por Jansen, Karpinski, Lingas e 
Seidel~\cite{JansenKLS01}. 

Um melhor entendimento da estrutura 
dos grafos cuja bissecção mínima tem largura grande
pode ajudar no 
desenvolvimento de bons algoritmos de aproximação para
o problema ou ajudar na 
identificação de classes de grafos onde o problema torna-se 
especialmente difícil. 
Para tanto, certas questões têm sido estudadas em árvores, 
em grafos que têm uma estrutura semelhante às árvores, e também em 
grafos planares~\cite{FernandesST13,FernandesST15}, para os quais 
a complexidade do problema encontra-se em aberto. 

Vários outros resultados são conhecidos para o Problema da 
Bissecção Mínima, porém este TCC se 
concentrou no estudo do problema em árvores.


%%%%%%%%%%%%%%%%%%%%%%%%%%%%%%%%%%%%%%%%%%%%%%%%%%%%%%%%%%%%%%%%%%%
%%%%%%%%%%%%%%%%   OBJETIVOS  %%%%%%%%%%%%%%%%%%%%%%%%%%%%%%%%%%%%%
%%%%%%%%%%%%%%%%%%%%%%%%%%%%%%%%%%%%%%%%%%%%%%%%%%%%%%%%%%%%%%%%%%%

\subsection{Objetivos} 

O principal objetivo deste trabalho é o estudo, implementação e 
análise de um algoritmo recente, proposto por Fernandes, Schmidt e 
Taraz~\cite{FernandesST13}, para encontrar uma bissecção 
aproximadamente mínima em árvores de grau limitado. Denotaremos
esse algoritmo por FST. 

A vantagem do algoritmo FST sobre o algoritmo de Jansen et 
al.~\cite{JansenKLS01}, que encontra uma bissecção de largura 
mínima, é que este tem um consumo de tempo linear, enquanto que o 
segundo tem um consumo de tempo cúbico no número de vértices da 
árvore. 
Implementamos também o algoritmo de Jansen et al., para fins de 
comparação da largura das bissecções produzidas pelo algoritmo
FST. 

O algoritmo de Jansen et al.\ baseia-se em 
programação dinâmica. 
Já o algoritmo FST é mais refinado. 
Para atingir um consumo de tempo 
linear, ele utiliza técnicas bem conhecidas de percursos de árvore, 
como busca em profundidade, bem como um rastreamento de 
informações mais cuidadoso que permite que o processamento todo 
seja executado em tempo linear. 

Fizemos uma análise 
da performance do algoritmo FST em árvores binárias geradas 
aleatoriamente,
em árvores ternárias completas e,
em árvores de caminho longo.
Também obtivemos instâncias reais de um problema relacionado e as
adaptamos para 
utilizá-las no nosso estudo experimental.
 

\bigskip
\bigskip

%%%%%%%%%%%%%%%%%%%%%%%%%%%%%%%%%%%%%%%%%%%%%%%%%%%%%%%%%%%%%%%%%%%
%%%%%%%%%%%%%%%%%%%%%%%%%%  MÉTODO  %%%%%%%%%%%%%%%%%%%%%%%%%%%%%%%
%%%%%%%%%%%%%%%%%%%%%%%%%%%%%%%%%%%%%%%%%%%%%%%%%%%%%%%%%%%%%%%%%%%

\subsection{Métodos}

O estudo do algoritmo FST seguiu um roteiro de
como a implementação foi feita, conforme a proposta deste TCC. 
A linguagem escolhida para a implementação foi \texttt{C} e os 
algoritmos implementados foram colocados na página do gitHub
em \url{https://github.com/karinaawoki/TCC} e
\url{https://github.com/karinaawoki/jansen_arvores}. 
O roteiro consistiu em uma divisão da implementação do algoritmo 
em várias etapas, que permitiu um melhor acompanhamento do 
trabalho desenvolvido, e uma organização da forma de estudo. 
O algoritmo encontra-se descrito em um documento 
longo~\cite{Schmidt15} juntamente com a sua análise teórica. 
Este documento serviu como base para o entendimento de cada 
etapa, e de como a implementação de cada etapa deveria ser feita 
para que o consumo de tempo fosse de fato linear. 

Reuniões frequentes foram feitas junto à supervisora para 
apresentar as etapas já implementadas, bem como para tirar dúvidas 
sobre as próximas etapas ou detalhes da implementação. 
Em maio e junho, Tina Schmidt, uma das autoras do algoritmo que 
foi implementado, visitou o IME-USP, e a parte da 
implementação que estava pronta foi apresentada a ela, e 
algumas discussões foram feitas. 
Em especial, a Tina conseguiu
um conjunto de instâncias reais com Steffen Borgwardt, feitas por 
Kampeier~\cite{Kampmeier}, para usarmos 
em nossos experimentos.   

Em paralelo ao estudo e desenvolvimento da implementação, foram 
estudados também dois artigos da literatura: o artigo 
de Garey, 
Johnson e Stockmeyer~\cite{GareyJS76} que contém a prova de que o 
Problema da Bissecção Mínima é NP-difícil, e o artigo de 
Jansen et al.~\cite{JansenKLS01} que apresenta 
o algoritmo que resolve o problema em árvores. 
Após a implementação do algoritmo FST, foi feita 
a implementação do algoritmo de Jansen et al.\ e o estudo
experimental do algoritmo FST. 
