\documentclass[a4paper,12pt]{article}

% Para usar a linguagem PT-BR.
\usepackage[utf8]{inputenc}
\usepackage[T1]{fontenc}
\usepackage{tikz}
\usepackage[brazilian]{babel}
\usepackage[titlenumbered,ruled]{algorithm2e}
\usepackage{amsthm}


\selectlanguage{brazilian} 

\newtheorem{lem}{Lema}
\newtheorem{alg}{Descrevendo o algoritmo}

\newcommand{\Oh}{\mathrm{O}}

\sloppy

\begin{document}
	\section {Rotulação dos vértices da árvore} 7**

		Mais adiante, iremos nos referir a vértices com uma característica 
		específica e para isso precisamos atribuir um rótulo numérico $x\in [n]$ 
		distinto para cada um dos vértices da árvore, onde $n$ é o número de 
		vértices da árvore $T$.

		Primeiramente, usaremos o método mencionado na seção ___ para
		calcularmos o $x$-$y$-caminho, que é máximo em $T$.

		Temos que cada vértice contido no $x$-$y$-caminho receberá uma 
		rotulação de forma que ó vértice será representado por um número
		menor que 

		Temos que para todo par de vértices $x_1$ e $x_2$, se $x_1$ vem antes 
		de $x_2$ no $x$-$y$-caminho, então $x_1$ receberá um rótulo menor
		do que $x_2$.

	
	
\end{document}