\documentclass[a4paper,12pt]{article}

% Para usar a linguagem PT-BR.
\usepackage[utf8]{inputenc}
\usepackage[T1]{fontenc}
\usepackage[brazilian]{babel}
\selectlanguage{brazilian} 

\newcommand{\Oh}{\mathrm{O}}

\sloppy

\begin{document}
\begin{center}
   {\large \textbf{UNIVERSIDADE DE SÃO PAULO}} \\[1.4cm]
   
   {\large \textbf{INSTITUTO DE MATEMÁTICA E ESTATÍSTICA}}\\[4.2cm]
   
   {\Huge TRABALHO DE CONCLUSÃO }\\[0.3cm]
   {\Huge DE CURSO }\\[9cm]
   
   {\large { Aluna: Karina Suemi Awoki}}\\[0.3cm]
   
   {\large { Orientadora: Cristina Gomes Fernandes}}
   

\end{center}

\newpage
\section{Introdução}
..
\bigskip

\section{Definições}
    
    \subsection{Caminho máximo em árvores}

        Dada uma árvore $G = (V,A)$ e os seus caminhos máximos 
        $C_1, C_2,..., C_c$, dizemos que os vértices de um caminho
        máximo $C_k$ são representados pela sequência de vértices 
        $x_{1, k}$, $x_{2, k}$,... $x_{n, k}$ com 
        $\{x_{i, k}, x_{i+1, k}\} \in A, \forall i \in$ [$1, n-1$],
        $x_i\neq x_j \forall i, j \in [n]$ 
        com $i \neq j$, 
        e $k \in [c]$, 
        onde $c$ é o número de caminhos máximos existentes nessa
        árvore.
        E também deve ser satisfeita a condição de que cada 
        sequência possui o maior número de vértices possíveis 
        (portanto, todas essas sequências são do mesmo tamanho - 
        o máximo).

        \bigskip

        Encontrar um dos caminhos máximos em uma árvore é uma 
        tarefa simples, computacionalmente falando. Primeiramente
        assumiremos que $\forall v \in V$, o vértice mais distante
        de $v$ é $x_{1, k}$ ou $x_{n, k}$, que são os extremos do
        caminho mais longo $C_k$.

        Levando em consideração o que foi assumido anteriormente,
        nota-se que se escolhermos 
        qualquer vértice $v \in G$, e procurarmos pelo vértice mais
        distante dele, chegaremos em um dos extremos de um dos 
        caminhos mais longos ($x_{1, k}$ 
        ou $x_{n, k}$) - chamaremos esse vértice de $y$.
        E se fizermos o mesmo com $y$, procurarmos qual o vértice 
        mais distante dele, chegaremos no
        outro extremo do caminho mais longo.
        

        Agora precisamos provar que $\forall v \in V$, o vértice 
        mais distante
        de $v$ é $x_1$ ou $x_n$.
        Vamos chamar de $z$ o vértice mais distante de $v$.
        

        \begin{itemize}
            \item Caso 1: Se $\exists k \in [1, c]$ tal que 
            $v \in C_k$.
            Nesse caso é fácil ver que 

            $dist(x_{1, k}, x_{n, k})=
            dist(x_{1, k}, v)+dist(x_{n, k}, v)$, 
            já que $v$ é um vértice do caminho máximo $k$.
            Agora suponhamos que 

            $dist(z, v)>dist(x_{1, k}, v)$ e 
            $dist(z, v)>dist(x_{n, k}, v)$,
            
            então temos que 
            
            $dist(x_{1, k}, v)+dist(z, v)>
            dist(x_{1, k}, v)+dist(x_{n, k}, v)=
            dist(x_{1, k}, x_{n, k})$
            
            logo $x_{1, k}, x_{2, k},..., x_{n, k}$ não é um dos 
            caminhos máximos e então entramos numa contradição.

            Com isso temos que 
            $dist(z, v)>dist(x_{1, k}, v)$ e 
            $dist(z, v)>dist(x_{n, k}, v)$ não á válido, o que 
            implica em $dist(z, v)\le dist(x_{1, k}, v)$ ou 
            $dist(z, v)\le dist(x_{n, k}, v)$. 
            Portanto, como $z$ é o vértice mais distante, vale que
            
            $dist(z, v)=dist(x_{1, k}, v)$ ou 
            $dist(z, v)=dist(x_{n, k}, v)$,
            
            ou seja, $v$ é um dos vértices do caminho mais longo
            de $C_k$ ($x_{1, k}$ ou $x_{n, k}$).

            \item Caso 2: Se $\forall k \in [1, c], v \notin C_k$. 
            Diremos que a distância $d$ entre $v$ e $C_k$ é 
            min$\{dist(v, c'): c'\in C_k\}$.

            Agora suponhamos que $\exists k \in [1, c]$, tal que
            $dist(v, z)>dist(x_{1, k}, v)$ e 
            $dist(v, z)>dist(x_{n, k}, v)$.
            E dado que $d=$ min$\{dist(v, c'): c'\in C_k\}$, então
            $dist(v, x_{1, k})\ge d$ e 
            $dist(v, x_{n, k})\ge d$, 
            portanto, como $z$ é o vértice mais distante de $v$, 
            temos que
            $dist(v, z)\ge d$.

            É fácil ver que 
            $dist(x_{1, k}, x_{n, k})=
            dist(v, x_{1, k})+dist(v, x_{n, k})-2\times d$, 
            e isso implica em
            $dist(v, x_{1, k})+dist(v, z)-2\times d>
            dist(v, x_{1, k})+dist(v, x_{n, k})-2\times d=d$, 
            logo $d$ não é o maior caminho e entramos numa 
            contradição novamente, o que leva a afirmação
            "$\exists k \in [1, c]$, tal que
            $dist(v, z)>dist(x_{1, k}, v)$ e 
            $dist(v, z)>dist(x_{n, k}, v)$" a ser falsa.

            Consequentemente, $\forall k \in [1, c], 
            dist(v, z)\le dist(x_{1, k}, v)$ ou
            $dist(v, z)\le dist(x_{n, k}, v)$. 
            E como $z$ é o vértice mais distante de $v$, 
            
            $dist(v, z)=dist(x_{1, k}, v)$ ou 
            $dist(v, z)=dist(x_{n, k}, v)$,

            o que implica que $z$ é um dos vértices extremos de 
            $C_k$.






        \end{itemize}
        
    \subsection{Cálculo do Diâmetro de uma Árvore}
    


\newpage
    
    
    

\end{document}