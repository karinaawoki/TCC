\section {Algoritmo da dobra do diâmetro}

	\subsection{Teorema da dobra do diâmetro}
	
		\begin{teo}[{\cite[Teorema 4]{Schmidt15}}]
		\label{teo:dobraDiametro}
			Para toda árvore~$T$ com~$n\ge 3$ vértices e 
			todo~$m\in [n]$,
			o conjunto de vértices de~$T$ pode ser particionado em 
			três partes~$B$,~$W$ e~$S$ tal que vale um dos 
			seguintes itens:
			\begin{enumerate}
				\item ${|B|=m}$, ${S=\emptyset}$, e~${e_T(B,W)\le 2}$, ou
				\item ${|B|\le m\le |B|+|S|}$, 
				com~${0<|S|\le\dfrac{n}{2}}$,
				${e_T(B,W,S)\le \dfrac{2\cdot 
				\Delta(T)}{\diam^*(T)}}$, 
				e~${\diam^*(T[S])\ge 2\diam^*(T)}$.
			\end{enumerate}
			Uma partição~$(B,W,S)$ que satisfaz 1 ou 2 pode ser
			computada em tempo~${O\Big(\dfrac{n}{\diam^*(T)}\Big)}$ 
		\end{teo}

	\bigskip
	\bigskip

		% Primeiramente, faremos uma série de definições para que
		% depois possamos explicar melhor como o teorema e o algoritmo
		% funcionam.

		% Usaremos o que foi explicado na seção~\ref{sec:caminhMaximo}
		% para encontrármos um caminho máximo. 
		% Depois, usaremos 

		Primeiramente, temos uma árvore~$T$ com~${n\ge 3}$ vértices e um 
		inteiro~${m\in[n]}$.
		Seja~$P$ um caminho máximo de~$T$ e sabendo que os vértices
		estão rotulados como o descrito na seção~\ref{sec:rotulacao},
		teremos um corte~$(B,W,S)$ da seguinte forma:
		%\begin{itemize}
		\bigskip
		\bigskip
	
	\subsubsection*{Caso 1:}
			Se existe um vértice ~${v\in V(P)}$ tal 
			que~${\map_m^+(v)\in V(P)}$, então o 
			corte~$(B,W,S)$ 
			com~${S=\emptyset}$,~${B =\{v+1, v+2,\ldots,~v+m\}}$
			e~${W=V(T)\setminus B}$ satisfaz as propriedades do
			primeiro caso do Teorema~\ref{teo:dobraDiametro},
			como o explicado na seção~\ref{sec:map}.

	\bigskip
	\bigskip

		\subsubsection*{Caso 2:}
			Caso contrário, sabemos que para todo vértice~${v\in V(P)}$
			teremos que~${\map_m^+(v)\not\in V(P)}$.
			Isso implica que existem~$|V(P)|$ ou mais vértices fora de~$P$,
			dado que cada vértice de~$P$ mapeia vértices distintos que estão
			fora de~$P$.
			Isso nos leva a~$d\le\dfrac{n}{2}$, pois no 
			máximo~$\dfrac{n}{2}$ vértices estarão em~$P$.

			Nesse caso, teremos um corte~$(B,W,S)$ que satisfaz as
			propriedades do segundo caso do Teorema~\ref{teo:dobraDiametro}.
			Isso será detalhado melhor a seguir.

			\bigskip
			
			% Para todo~${z\in V(P)}$, e sendo~$x_1$ e~$x_2$ os menores
			% vértices de~$T_z$ tal que~${\map^-_m(x_1)\in V(P)}$ 
			% e~${\map^+_m(x_2)\in V(P)}$,
			% usaremos as seguintes definições:
			% \begin{align}
			% 	z^b   = &\map^-_m(x_1) \nonumber \\
			% 	z^f   = &\map^+_m(x_2) \nonumber \\
			% 	P_z^b = &\map^-_m(V(T_z))\cap V(P) \nonumber\\
			% 	P_z^f = &\map^+_m(V(T_z))\cap V(P) \nonumber%\\
			% 	%H_z^f = &\map^+_m(V(T_))\cap V(P) \nonumber\\
			% 	%H_z^f = &\map^+_m(V(T_z))\cap V(P) \nonumber
			% \end{align}

			Seja um vértice~${z\in V(P)}$ tal que~$z$ é b-especial e
			sendo~$x$ o menor vértice de~$T_z$ tal 
			que~${\map^-_m(x)\in V(P)}$, temos 
			que~${z^b = \map^-_m(x)}$.
			Temos também
			que~${P_z^b = \map^-_m(V(T_z))\cap V(P)}$
			e~${H_z^f =T_u}$ para todo 
			vértice~${u\in P_z^b}$ com~${u\ne z^b}$.

			Segue uma imagem com a representação da disposição dos 
			conjuntos de vértices.


	\begin{center} \begin{tikzpicture}
		[scale=.7,auto=left,every node/.style={circle, draw=black,
				fill=white!70}]

		\draw [decorate,decoration={brace,amplitude=10pt,mirror,
		raise=4pt}, draw=brown, line width=1.5pt] 
		(8,5.7) -- (2,5.7);

		\draw [decorate,decoration={brace,amplitude=10pt,mirror,
		raise=4pt}, draw=Emerald, line width=1.5pt] 
		(4,2.7) -- (8.5,2.7);

		\draw [draw=yellow, fill=yellow!20] (18.1,3.7) 
		ellipse (1.9cm and 3.7cm); 

		\node [text=blue!65,draw=none, fill=none] at 
		(17.7,6.8) {$T_z$};


		% \node [scale=1.3,subtree,fill=Green](y1) at (2.5,5) {};
		% \node [subtree, fill=Cyan]      (y2) at (5,5)   {};
		% \node [subtree, fill=Cyan]      (y3) at (7.5,5) {};
		% \node [scale=1.3,subtree,fill=Green](y4) at (10,5)  {};
		% \node [subtree, fill=Blue]        (y5) at (12.5,5){};
		% \node [subtree, fill=Blue]        (y6) at (15,5)  {};
		% \node [scale=1.9, subtree,fill=Gray] (y7) at (18.1,5)  {};


		\node [scale=1.3,subtree,fill=red!50](y1) at (2.5,5) {};
		\node [subtree, fill=Emerald]      (y2) at (5,5)   {};
		\node [subtree, fill=Emerald]      (y3) at (7.5,5) {};
		\node [scale=1.3,subtree,fill=red!50](y4) at (10,5)  {};
		\node [subtree, fill=yellow]        (y5) at (12.5,5){};
		\node [subtree, fill=yellow]        (y6) at (15,5)  {};
		\node [scale=1.9, subtree,fill=CornflowerBlue] (y7) at (18.1,5)  {};

		\node (x0) at (0,5)  {...};
		\node [fill=brown](x1) at (2.5,5)    {};
		\node [fill=brown](x2) at (5,5)  {};
		\node [fill=brown](x3) at (7.5,5)    {};
		\node (x4) at (10,5) {};
		\node (x5) at (12.5,5)   {};
		\node (x6) at (15,5) {};
		\node (x7) at (18.1,5) {};
		\node (x8) at (20.6,5) {...};

		\node [draw=none, fill=none] at 
		(2,2) {\footnotesize{$\map^-_m(v)$}};
		\node [draw=none, fill=none](nv) at (2,2){};
		\node [draw=none, fill=none](nvt) at (2.5,3.6){};

		\node [draw=none, fill=none] at 
		(9.5,2) {\footnotesize{$\map^-_m(z)$}};
		\node [draw=none, fill=none](nz) at (9.5,2){};
		\node [draw=none, fill=none](nzt) at (10,3.6){};

		\node [draw=none, fill=none] at 
		(15.5,2.6) {\footnotesize{$x$}};
		\node [draw=none, fill=none](z) at (15.5,2.6){};
		\node [draw=none, fill=none](zt) at (18,2.6){};

		\node [draw=none, fill=none] at (2.6,5.7){$z^b$};
		\node [draw=none, fill=none] at (18.1,5.6){$z$};
		\node [draw=none, fill=none] at (15,5.6){$v$};
		\node [text=Brown,draw=none, fill=none] at (5,7){$P^b_z$};
		\node [text=Emerald,draw=none, fill=none] at (6.25,1.3){$H^b_z$};

		\foreach \from/\to in {x0/x1,x1/x2,x2/x3,x3/x4,x4/x5,x5/x6,x6/x7,x7/x8}
		\draw (\from) -- (\to);

		\path[->, line width=1pt]
		(nv) edge[left=23 ] (nvt);
		\path[->, line width=1pt]
		(z) edge[right=45] (zt);
		\path[->, line width=1pt]
		(nz) edge[right=45] (nzt);

	\end{tikzpicture} \end{center}

		\bigskip

			De forma análoga, para um vértice~${z\in V(P)}$ tal 
			que~$z$ é f-especial,
			temos que~${z^f = \map^+_m(x)}$,
			sendo~$x$ o menor vértice de~$T_z$ tal 
			que~${\map^+_m(x)\in V(P)}$.
			Também temos
			que~${P_z^f = \map^+_m(V(T_z))\cap V(P)}$
			e~${H_z^f =T_u}$ para todo 
			vértice~${u\in P_z^f}$ com~${u\ne z^f}$.

			Segue abaixo outra imagem com a representação da disposição 
			dos conjuntos de vértices.


	\begin{center} \begin{tikzpicture}
		[scale=.6,auto=left,every node/.style={circle, draw=black,
				fill=white!70}]

		\draw [decorate,decoration={brace,amplitude=10pt,mirror,
		raise=4pt}, draw=brown, line width=1.5pt] 
		(-4.5,5.9) -- (-10.5,5.9);

		\draw [decorate,decoration={brace,amplitude=10pt,mirror,
		raise=4pt}, draw=Emerald, line width=1.5pt] 
		(-8.5,2.7) -- (-4,2.7);

		\draw [draw=yellow, fill=yellow!20] (-18.1,3.7) 
		ellipse (2cm and 4.5cm); 

		\node [text=blue!65,draw=none, fill=none] at 
		(-18.5,6.8) {$T_z$};


		

		\node [scale=1.3,subtree,fill=red!50](y1) at (-2.5,5) {};
		\node [subtree, fill=Emerald]      (y2) at (-5,5)   {};
		\node [subtree, fill=Emerald]      (y3) at (-7.5,5) {};
		\node [scale=1.3,subtree,fill=red!50](y4) at (-10,5)  {};
		\node [subtree, fill=yellow]        (y5) at (-12.5,5){};
		\node [subtree, fill=yellow]        (y6) at (-15,5)  {};
		\node [scale=1.9, subtree,fill=CornflowerBlue] (y7) at (-18.1,5)  {};

		\node (x0) at (0,5)  {...};
		\node (x1) at (-2.5,5)    {};
		\node [fill=brown](x2) at (-5,5)  {};
		\node [fill=brown](x3) at (-7.5,5)    {};
		\node [fill=brown](x4) at (-10,5) {};
		\node (x5) at (-12.5,5)   {};
		\node (x6) at (-15,5) {};
		\node (x7) at (-18.1,5) {};
		\node (x8) at (-20.6,5) {};
		\node (x9) at (-23.1,5) {...};

		\node [draw=none, fill=none] at 
		(-2,1) {\footnotesize{$\map^+_m(z)$}};
		\node [draw=none, fill=none](nv) at (-2,1){};
		\node [draw=none, fill=none](nvt) at (-2.5,3.6){};

		\node [draw=none, fill=none] at 
		(-9.5,1) {\footnotesize{$\map^+_m(v)$}};
		\node [draw=none, fill=none](nz) at (-9.5,1){};
		\node [draw=none, fill=none](nzt) at (-10,3.6){};

		\node [draw=none, fill=none] at 
		(-15.5,2) {\footnotesize{$x$}};
		\node [draw=none, fill=none](z) at (-15.5,2){};
		\node [draw=none, fill=none](zt) at (-18,2){};

		\node [draw=none, fill=none] at (-9.8,5.8){$z^f$};
		\node [draw=none, fill=none] at (-18.1,5.6){$z$};
		\node [draw=none, fill=none] at (-20.6,5.6){$v$};
		\node [text=Brown,draw=none, fill=none] at (-7.5,7.3){$P^f_z$};
		\node [text=Emerald,draw=none, fill=none] at (-6.35,1.1){$H^f_z$};


		\foreach \from/\to in {x0/x1,x1/x2,x2/x3,x3/x4,x4/x5,x5/x6,x6/x7,x7/x8,x8/x9}
		\draw (\from) -- (\to);

		\path[->, line width=1pt]
		(nv) edge[left=23 ] (nvt);
		\path[->, line width=1pt]
		(z) edge[right=45] (zt);
		\path[->, line width=1pt]
		(nz) edge[right=45] (nzt);

	\end{tikzpicture} \end{center}

	Note que os vértices~$\map^-_m(v)$,~$\map^-_m(z)$,~$\map^+_m(v)$ 
	e~$\map^+_m(z)$ não estão em~$V(P)$, pois estamos no caso 2, onde
	nenhum vértice de~$P$ é mapeado em outro vértice de~$P$.

	Considerando que existe uma aresta ligando o primeiro ao último 
	vértices de~$P$, vemos que os elementos de~$P_z^b$ aparecem 
	como um caminho conexo e que para valores diferentes de~$z$,
	temos que~$P_z^b$ é disjunto.
	O mesmo vale para~$P_z^f$.

	Usaremos futuramente as seguintes propriedades:
	\begin{enumerate}
		\item Para todo vértice~$z$ tal que~$z$ é b-especial, 
		temos que o vértice~$z^b$ é f-especial.

		Isso ocorre porque para o vértice~$v$, que vem antes 
		de~$z$ em~$P$, temos que~${\map^-_m(v)\not\in V(P)}$, 
		logo,~$\map^-_m(v)$ está em~$T_{z^b}$.
		Sabemos que~$\map^-_m(v)$ não está na árvore a esquerda
		de~$T{z^f}$, pois se estivesse, o~$z^f$ seria um outro
		vértice.

		\item Para todo vértice~$z$ tal que~$z$ é f-especial, 
		temos que o vértice~$z^f$ é b-especial.

		Isso ocorre de forma análoga a anterior, dado que
		para o vértice~$v$, que vem antes de~$z$ em~$P$, temos
		que~${\map^+_m(v)\in T_{z^f}}$.

		\item Um vértice~${z\in V(P)}$ é b-especial sse existe um
		vértice~${u\in V(P)}$ tal que~$u$ é f-especial e~${z=u^f}$.

		Sabemos que a volta é verdadeira usando a propriedade 2.
		
		A ida é verdadeira pois, dado que~$z$ é b-especial,
		temos que existe um ou mais vértices~${v_i\in V(P)}$ tal
		que~${\map^+_m(v_i)\in T_{z}}$.
		Seja~$v_k$ o vértice com o maior rótulo e~$u$ o vértice 
		que vem depois de~$v_k$ em~$P$. 
		Temos então que~${\map^-_m(z)\in T_u}$ e que alguns vértices
		de~$T_z$ mapeiam em~$v_k$, logo,~$u$ é f-especial e~${z=u^f}$.

		\item Um vértice~${z\in V(P)}$ é f-especial sse existe um
		vértice~${u\in V(P)}$ tal que~$u$ é b-especial e~${z=u^b}$.

		Podemos verificar que essa propriedade é verdadeira de forma
		semelhante a verificação da propriedade anterior.
	\end{enumerate}

	Sendo~$B$ um conjunto de todos os vértices b-especiais e~$F$ o
	conjunto de todos os vértices f-especiais e, usando as propriedades 
	3 e 4 respectivamente, temos que
	\begin{align}
		\displaystyle\sum_{u\in F}|T_{u^f}| = 
		\displaystyle\sum_{z\in B}|T_{z}| \nonumber \\
		\displaystyle\sum_{u\in B}|T_{u^b}| = 
		\displaystyle\sum_{z\in F}|T_{z}| \nonumber
	\end{align}

	Note que para todo vértice b-especial~$z$ todos os vértices 
	de~$H^b_z$ e de~$P^b_z$ mapeiam um vértice 
	de~$T_z$ e que nem todos os vértices de~$T_z$ são mapeados por
	vértices~$H^b_z\cup P_z^b$, 
	dado que~$\map^-_m(z)\not\in H^b_z\cup P_z^b$.
	O mesmo ocorre com os vértices f-especiais~$w$.
	Portanto temos que
	\begin{align}
		|T_z| &> |P^b_z| + |H^b_z| \\
		|T_w| &> |P^f_w| + |H^f_w|
	\end{align}

%%%%%%%%%%%%%%%%%%%%%%%%%%%%%%%%%%%%%%%%%%%%%%%%%%%%%%%%%%%%%%%%%%
%%%%%%%%%%%%%%%%%%%%%%%%%%%%%%%%%%%%%%%%%%%%%%%%%%%%%%%%%%%%%%%%%%
%%%%%%%%%%%%%%%%%%%%%%%%%%%%%%%%%%%%%%%%%%%%%%%%%%%%%%%%%%%%%%%%%%

	\subsection{Algoritmo da dobra do diâmetro}
	\begin{algorithm}[H]
	\label{alg:dobraDiametro}
		\SetKwInOut{Input}{input}
		\SetKwInOut{Output}{output}

		\caption{}
		\Input{árvore $T$ com $n$ vértices e ${m\in[n]}$}
		\Output{}

	\end{algorithm}	