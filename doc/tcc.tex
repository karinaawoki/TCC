\documentclass[a4paper,12pt]{article}

% Para usar a linguagem PT-BR.
\usepackage[utf8]{inputenc}
\usepackage[T1]{fontenc}
\usepackage{tikz}
\usepackage{xcolor}
\usepackage{array}
\usepackage[brazilian]{babel}
\usepackage[linesnumbered,ruled]{algorithm2e}
\usepackage{amsmath, amsthm, amssymb}
\usepackage{mathtools,booktabs}
\usepackage{colortbl}
\DeclarePairedDelimiter\ceil{\lceil}{\rceil}

\usetikzlibrary{shapes.geometric,arrows,fit,matrix,positioning}
\tikzset
{
    subtree/.style  = {isosceles triangle, draw=black, align=center, minimum height=0.5cm, minimum width=1cm, shape border rotate=90, anchor=north, fill=red!30}
}

\selectlanguage{brazilian} 

\newtheorem{lem}{Lema}
\newtheorem{prov}{Prova}
\newtheorem{alg}{Análise do algoritmo}

\newcommand{\Oh}{\mathrm{O}}
\newcommand\mycommfont[1]{\footnotesize\ttfamily\textcolor{blue}{#1}}
\SetCommentSty{mycommfont}


\sloppy

\begin{document}
 
\begin{center}
   {\large \textbf{UNIVERSIDADE DE SÃO PAULO}} \\[1.4cm]
   
   {\large \textbf{INSTITUTO DE MATEMÁTICA E ESTATÍSTICA}}\\[4.2cm]
   
   {\Huge TRABALHO DE CONCLUSÃO }\\[0.3cm]
   {\Huge DE CURSO }\\[9cm]
   
   {\large { Aluna: Karina Suemi Awoki}}\\[0.3cm]
   
   {\large { Orientadora: Cristina Gomes Fernandes}}
   

\end{center}

\newpage
\tableofcontents

\newpage

%%%%%%%%%%%%%%%%%%%%%%%%%%%%%%%%%%%%%%%%%%%%%%%%%%%%%%%%%%%%%%%%%%%
%%%%%%%%%%%%%%%%%%%  INTRODUÇÃO  %%%%%%%%%%%%%%%%%%%%%%%%%%%%%%%%%%
%%%%%%%%%%%%%%%%%%%%%%%%%%%%%%%%%%%%%%%%%%%%%%%%%%%%%%%%%%%%%%%%%%%

\section{Introdução}

O assunto abordado no Trabalho de Conclusão de Curso se
enquadra na área de Otimização Combinatória e diz respeito ao 
\emph{Problema da Bissecção Mínima}. Para definir o problema 
precisamente, seguem primeiro algumas definições. 

Seja~$G=(V,E)$ um grafo e~$B$ e~$W$ conjuntos de vértices de~$G$,
dizemos que~$(B,W)$ é um \textbf{corte}
de~$G$ se~$B \cup~W =~V(G)$ e~$B\cap~W =~\emptyset$.
Denotamos o número de arestas no corte~$(B,W)$ por \textbf{largura} do corte ou por~$e_G(B,W)$.
O corte~$(B,W)$ é uma \textbf{bissecção} se~$|B| =~|W|$
quando~$|V|$ é par, ou se~$||B|-~|W|| =~1$ quando~$|V|$ é ímpar.

O \emph{Problema da Bissecção Mínima} consiste em, dado um grafo, 
encontrar uma bissecção no grafo de largura mínima.

Sabe-se que o problema da bissecção é NP-difícil~\cite{GareyJS76} 
e a melhor aproximação conhecida para o caso geral do problema tem 
razão~$\Oh(\lg n)$~\cite{Racke08}, onde~$n$ é o número de vértices 
do grafo. 
Por outro lado, sabe-se que, para árvores e grafos que de uma 
certa maneira se assemelham a árvores, há um algoritmo polinomial 
de programação dinâmica para encontrar uma bissecção mínima, 
proposto por Jansen, Karpinski, Lingas e 
Seidel~\cite{JansenKLS01}. 

Fernandes, Schmidt e Taraz têm interesse em entender a estrutura 
dos grafos cuja bissecção mínima tem largura grande, de modo a 
desenvolver bons algoritmos de aproximação para o problema ou 
identificar melhor as classes de grafos onde o problema torna-se 
especialmente difícil. 
Para tanto, eles têm feito estudos de certas questões em árvores, 
em grafos que têm uma estrutura semelhante às árvores, e também em 
grafos planares~\cite{FernandesST13,FernandesST15}, para os quais 
a complexidade do problema encontra-se em aberto. 

Vários outros resultados são conhecidos para o Problema da 
Bissecção Mínima, porém este trabalho de conclusão de curso se 
concentrou no estudo do problema em árvores.

\newpage


%%%%%%%%%%%%%%%%%%%%%%%%%%%%%%%%%%%%%%%%%%%%%%%%%%%%%%%%%%%%%%%%%%%
%%%%%%%%%%%%%%%%   OBJETIVOS  %%%%%%%%%%%%%%%%%%%%%%%%%%%%%%%%%%%%%
%%%%%%%%%%%%%%%%%%%%%%%%%%%%%%%%%%%%%%%%%%%%%%%%%%%%%%%%%%%%%%%%%%%

\section{Objetivos} 

O principal objetivo deste trabalho foi o estudo, implementação e 
análise de um algoritmo recente, proposto por Fernandes, Schmidt e 
Taraz~\cite{FernandesST13}, para encontrar uma bissecção 
aproximadamente mínima em árvores de grau limitado, denotaremos
esse algoritmo por FST. 
A vantagem deste algoritmo sobre o algoritmo de Jansen et 
al.~\cite{JansenKLS01}, que encontra uma bissecção de largura 
mínima, é que este tem um consumo de tempo linear, enquanto que o 
segundo tem um consumo de tempo cúbico no número de vértices da 
árvore. 
Implementamos também o algoritmo de Jansen et al., para fins de 
comparação da largura das bissecções produzidas pelo algoritmo
FST. 

Como já mencionado, o algoritmo de Jansen et al.\ baseia-se em 
programação dinâmica. 
Já o algoritmo de Fernandes et al., para atingir um consumo 
linear, utiliza técnicas bem conhecidas de percursos de árvore, 
como busca em profundidade, bem como um rastreamento de 
informações mais cuidadoso que permite que o processamento todo 
seja executado em tempo linear. 
É um algoritmo mais refinado, dividido na implementação de uma 
série de etapas menores. 

Uma vez implementados os dois algoritmos, foi feita uma análise 
da sua performance em árvores binárias geradas aleatoriamente, e 
em árvores ternárias.
Obtivemos instâncias reais de um problema relacionado e as 
utilizamos também no estudo experimental do algoritmo implementado.
 

\newpage


%%%%%%%%%%%%%%%%%%%%%%%%%%%%%%%%%%%%%%%%%%%%%%%%%%%%%%%%%%%%%%%%%%%
%%%%%%%%%%%%%%%%%%%%%%%%%%  MÉTODO  %%%%%%%%%%%%%%%%%%%%%%%%%%%%%%%
%%%%%%%%%%%%%%%%%%%%%%%%%%%%%%%%%%%%%%%%%%%%%%%%%%%%%%%%%%%%%%%%%%%

\section{Método}

O estudo do algoritmo de Fernandes et al.\ seguiu um roteiro de
como a implementação foi feita, conforme a proposta do TCC. 
A linguagem escolhida para a implementação foi \texttt{C} e os 
algoritmos implementados foram colocados na página do gitHub. 
O roteiro consistiu em uma divisão da implementação do algoritmo 
em várias etapas, que permitiu um melhor acompanhamento do 
trabalho desenvolvido, e uma organização da forma de estudo. 
O algoritmo encontra-se descrito em um documento 
longo~\cite{Schmidt15} juntamente com a sua análise teórica. 
Este documento serviu como base para o entendimento de cada 
etapa, e de como a implementação de cada etapa deveria ser feita 
para que a implementação obtida fosse de fato linear. 

Reuniões frequentes foram feitas junto à supervisora para 
apresentar as etapas já implementadas, bem como para tirar dúvidas 
sobre as próximas etapas ou detalhes da implementação. 
Em maio e junho, Tina Schmidt, uma das autoras do algoritmo que 
foi implementado, visitou o IME, e a parte da 
implementação que estava pronta foi apresentada a ela, e 
algumas discussões foram feitas e em especial, a Tina conseguiu
um conjunto de instâncias reais com~\cite{Kampmeier} para usarmos 
em nossos experimentos.   

Em paralelo ao estudo e desenvolvimento da implementação, foram 
estudados também dois artigos da literatura. São eles o artigo 
de Garey, 
Johnson e Stockmeyer~\cite{GareyJS76}, que contém a prova de que o 
problema é NP-difícil, e o artigo de Jansen et al., que apresenta 
o algoritmo de programação dinâmica para o problema em árvores. 
Após a implementação do algoritmo de Fernandes et~al., foi feita 
a implementação do algoritmo de Jansen et~al.\ e a comparação dos 
dois. 

\newpage

%%%%%%%%%%%%%%%%%%%%%%%%%%%%%%%%%%%%%%%%%%%%%%%%%%%%%%%%%%%%%%%%%%%
%%%%%%%%%%%%%%%%%%%%%%%  DEFINIÇÕES  %%%%%%%%%%%%%%%%%%%%%%%%%%%%%%
%%%%%%%%%%%%%%%%%%%%%%%%%%%%%%%%%%%%%%%%%%%%%%%%%%%%%%%%%%%%%%%%%%%

\section{Definições}
	
	\subsection{Conjunto~$[n]$}
	Denotemos por~$\{1,2,\ldots~n\}$
	o conjunto~$[n]$ .

	\subsection{Conjunto de vértices e arestas}
	Para um grafo~$G$, denotamos seu
	conjunto de vértices por~$V(G)$ e seu
	conjunto de arestas por~$E(G)$.

	\subsection{Grau de vértices}
	O número de arestas que incidem em um determinado
	vértice~$v$ é chamado de 
	\textbf{grau} de~$v$ e será representado
	por~$grau(v)$. O 
	\textbf{grau máximo}, que é o grau de um vértice de
	maior grau em~$G$, será representado por~$\Delta(G)$.

	\subsection{Caminhos em árvores}
	Um \textbf{caminho} em uma árvore~$T$, de 
	comprimento~$p$, é uma sequência de 
	vértices~$v_0, v_1, \ldots,v_{p-1}, v_p$ 
	onde~$v_{k-1}, v_k$ é uma aresta para todo~$k =~1,2, \ldots, p$. 
	Como existe um único caminho que conecta quaisquer dois 
	vértices~$v_0$ e~$v_p$ em~$T$, chamaremos tal 
	caminho em~$T$ de~$v_0$,$v_p$-\textbf{caminho}.

	\subsection{Distâncias entre vértices}
	A \textbf{distância} entre dois vértices~$a$ e~$b$ 
	de~$T$ é representada por~$dist(a,b)$, 
	e é igual ao comprimento do~$a$,$b$-caminho em~$T$.

		

\newpage

%%%%%%%%%%%%%%%%%%%%%%%%%%%%%%%%%%%%%%%%%%%%%%%%%%%%%%%%%%%%%%%%%%%
%%%%%%%%%%%%%%%%%%%%%%%  SEÇÃO 5     %%%%%%%%%%%%%%%%%%%%%%%%%%%%%%
%%%%%%%%%%%%%%%%%%%%%%%%%%%%%%%%%%%%%%%%%%%%%%%%%%%%%%%%%%%%%%%%%%%


\section{Caminho máximo em árvores}\label{sec:caminhoMaximo}
	\subsection{Algoritmo que encontra um caminho máximo}

	Um \textbf{caminho máximo} em uma árvore~$T$ é um caminho 
	em~$T$ de comprimento máximo.

	Encontrar um caminho máximo em uma árvore~$T=(V,E)$ é uma tarefa
	computacionalmente simples que pode ser realizada em tempo~$O(n)$, 
	sendo~$n$ o número de vértices de~$T$, ou seja,~$n =~|V|$. 
	Primeiramente escolhe-se um vértice arbitrário~$v \in~V$ 
	e encontra-se um vértice~$y_0$ mais distante de~$v$.
	Depois, repetimos o mesmo processo a partir de~$y_0$, 
	encontrando um vértice~$x_0$ mais distante de~$y_0$. 
	Para encontrar um vértice mais distante de algum vértice dado, 
	basta usar uma busca em largura.  
	Feito isso, temos um caminho máximo em~$T$: o único
	caminho que liga~$x_0$ a~$y_0$.

	\bigskip

	\subsection{Prova de funcionalidade do algoritmo}

	\begin{lem}
	\label{lema:caminhoMax}
		O~$x_0$,$y_0$-caminho é um caminho máximo em~$T$.
	\end{lem}

	\bigskip

	\begin{proof}
		Suponhamos que exista um outro caminho mais longo que 
		o~$x_0$,$y_0$-caminho e sejam~$x$ e~$y$ os seus extremos.

		\begin{itemize}
	        \item \textbf{Caso 1:} O~$x$,$y$-caminho e 
	        o~$x_0$,$y_0$-caminho não possuem vértices em comum.

	        Existe um~$a$,$b$-caminho, de 
	        comprimento~$c \ge 1$, que conecta os dois caminhos, 
	        possuindo assim apenas o vértice~$a$ 
	        no ~$x$,$y$-caminho e apenas o vértice~$b$ 
	        no~$x_0$,$y_0$-caminho.

	        \begin{center} \begin{tikzpicture}
				[scale=.8,auto=left,every node/.style={circle, draw=black,
				fill=blue!20}]
				\node (x) at (0,5)  {$x$};
				\node (y) at (12,5) {$y$};
				\node (a) at (6,4)  {$a$};
				  
				\node (x0) at (0,0) {$x_0$};
				\node (y0) at (12,0) {$y_0$};
				\node (b)  at (6,1)  {$b$};

				\node (n1) at (3,4.5) {...}; %entre x e a 
				\node (n2) at (9,4.5) {...}; %entre a e y
				\node (n3) at (3,0.5) {...}; %entre x0 e b
				\node (n4) at (9,0.5) {...}; %entre y0 e b
				\node (n5) at (6,2.5) {...}; %entre a e b

				\foreach \from/\to in {x/n1,n1/a,a/n2,n2/y,x0/n3,
				n3/b,b/n4,n4/y0,a/n5,b/n5}
				\draw (\from) -- (\to);
			\end{tikzpicture} \end{center}


	        Portanto, temos 
	        que~$ dist(x,y) =~dist(x,a) +~dist(a,y)$ 
	        e~$ dist(x_0,y_0) =~dist(x_0,b) +~dist(b,y_0)$.

	        Como o~$x$,$y$-caminho é 
	        máximo,~$dist(x,y)\ge dist(x_0,y)$,
	        e isso implica que~$dist(x,a)\ge dist(x_0,b)+c$.

	        Como~$x_0$ é um vértice mais distante de~$y_0$, temos 
	        que~$dist(x_0,y_0)\ge~dist(x,y_0)$, que implica 
	        que~$dist(x_0,b)\ge~dist(x,a)+~c$.
	        Logo temos 
	        que~$dist(x,a)\ge~dist(x, a)+2c$,
	        uma contradição, visto que~$c\ge 1$.


			\bigskip
			\bigskip
			\bigskip


			\item \textbf{Caso 2:} O~$x$,$y$-caminho e 
			o~$x_0$,$y_0$-caminho possuem vértices em comum.

			A interseção entre esses dois caminhos é 
			um~$a$,$b$-caminho de comprimento~$c \ge 0$.
			Podemos assumir, sem perda de generalidade, 
			que~$dist(x,a) \le~dist(x,b)$ e 
			que~$dist(x_0,a) \le~dist(x_0,b)$, como na figura abaixo.
			(Do contrário troque~$a$ por~$b$ e possivelmente~$x$
			por~$y$.)

			\begin{center} \begin{tikzpicture}
				[scale=.8,auto=left,every node/.style={circle, draw=black,
				fill=blue!20}]
				\node (x) at (0,4)  {$x$};
				\node (y) at (12,4) {$y$};
				\node (a) at (4,4)  {$a$};
				\node (b) at (8,4)  {$b$};
				\node (x0) at (1.5,6)  {$x_0$};
				\node (y0) at (10.5,2)  {$y_0$};
				  
				\node (n1) at (2,4) {...};
				\node (n2) at (6,4) {...};
				\node (n3) at (10,4) {...};
				\node (n4) at (9.25,3) {...};
				\node (n5) at (2.75,5) {...};

				\foreach \from/\to in {x/n1,n1/a,a/n2,n2/b,b/n3,
				n3/y, a/n5,n5/x0,b/n4,n4/y0}
				\draw (\from) -- (\to);
			\end{tikzpicture} \end{center}


			Como~$x_0$ é um vértice mais distante de~$y_0$,
			temos que~$dist(x_0,a)\ge~dist(x,a)$.
			Por outro lado, como~$x$ é um vértice mais distante
			de~$y$, temos também que~$dist(x,a)\ge dist(x_0,a)$,
			e, portanto,~$dist(x_0,a) =~dist(x,a)$.
			Similarmente, como~$y$ é um vértice mais distante de~$x$, 
			vale que~$dist(y,b) \ge dist(y_0,b)$.

			Agora consideremos o vértice~$v$ que foi escolhido 
			arbitrariamente no início do algoritmo, e seja~$v'$
			o vértice mais próximo de~$v$ no~$x_0$,$y_0$-caminho.
			Se~$v'\ne b$, então, como~$y_0$ é um vértice mais 
			distante de~$v$,
			\begin{align}
				dist(v',y_0) &\ge dist(v',y) =~dist(v',b) + dist(b,y)\nonumber \\
				&\ge dist(v',b) + dist(b,y_0) \ge dist(v',y_0) \nonumber,
			\end{align} 
			o que implica que~$dist(v',y_0)~=~dist(v',y)$ assim
			como~$ dist(b,y)~=~dist(b,y_0)$. 
			Ou seja,~$dist(x,y) =~dist(x_0,y_0)$.
			Resta analisar o caso em que~$v'=b$ (é um caso especial,
			dado que pode-se ter arestas em comum entre~$v$,$v'$-caminho 
			e o~$y$,$b$-caminho).
			Como~$y$ é um vértice mais distante de~$x$, temos 
			que~$dist(v',y_0) \le~dist(v',y)$.
			Como~$x_0$ é um vértice mais distante de~$y_0$, temos
			que~$dist(x_0,v') \ge dist(y,v')$.
			
			Finalmente, como~$y_0$ é um vértice mais distante 
			de~$v$, vale que~$dist(y_0,v')\ge dist(x_0,v')$.

			Dessas três desigualdades, concluímos 
			que~$dist(y_0,v')~=~dist(y,v')~=~dist(x_0,v')~=~dist(x,v')$,
			onde a última igualdade vale pois já mostramos 
			que~$dist(x_0,a) =~dist(x,a)$.

			Assim sendo~$dist(x_0,y_0) =~dist(x,y)$, completando a prova.

		\end{itemize}
	\end{proof}

	\bigskip
	\bigskip


%%%%%%%%%%%%%%%%%%%%%%%%%%%%%%%%%%%%%%%%%%%%%%%%%%%%%%%%%%%%%%%%%%%
	\subsection{Diâmetro de um Grafo}

	Dizemos que o \textbf{diâmetro} de um grafo conexo~$G$ é o
	comprimento de um caminho máximo de~$G$. 
	Em outras palavras, o diâmetro pode ser definido como:
	$$ diam(G)=\max\{dist(x,y):x,y\in~V(G)\}.$$

	\bigskip

	Seja~$G$ um grafo não necessariamente conexo,
	o \textbf{diâmetro relativo} de~$G$
	 é a razão entre o número
	de vértices de um caminho máximo e o número total de vértices
	de~$G$. Este pode também ser representado por
	$$ diam^*(G) =~\dfrac{\displaystyle\sum_{
	G'~componente~conexa~de~G}^{}(diam(G')+1)}{|V(G)|}.$$
    
\newpage
%%%%%%%%%%%%%%%%%%%%%%%%%%%%%%%%%%%%%%%%%%%%%%%%%%%%%%%%%%%%%%%%%%%
%%%%%%%%%%%%%%%%%%%%%%%  SEÇÃO 6     %%%%%%%%%%%%%%%%%%%%%%%%%%%%%%
%%%%%%%%%%%%%%%%%%%%%%%%%%%%%%%%%%%%%%%%%%%%%%%%%%%%%%%%%%%%%%%%%%%
\section {Lemas de cortes aproximados}

Em \cite{Schmidt15}, são apresentados três lemas e seus
respectivos algoritmos, que, dada uma 
floresta~$G$ e um inteiro~$0<m\le~n$, onde~$n$ é o número de vértices 
de~$G$, produzem cortes com algumas propriedades.

Abaixo reproduzimos os três resultados de \cite{Schmidt15} que 
atestam as propriedades dos cortes produzidos por estes algoritmos:

\bigskip
\bigskip
\bigskip

\subsection{Lema do corte aproximado simples para árvores}
\begin{lem}[]
\label{lema:simpleApproxCutTree}
	Para toda árvore~$T$ com~$n$ vértices e todo~$m \in~[n]$,
	existe um corte~$(B,W)$ em~$T$ tal 
	que~$\dfrac{m}{2} <|B| \le~m$ e~$e_T(B,W) \le~\Delta(T)$.
	Um corte que satisfaz esses requisitos pode ser computado em
	tempo~$O(n)$.
\end{lem}

\bigskip

Seja~$T$ uma árvore e~$r$ um vértice de~$T$. 
Assumiremos que
a função~CALCULA\_NUMERO\_DE\_DESCENDENTES$(T,r)$ devolve um vetor
que associa cada vértice de~$T$ ao número de descendentes desse 
vértice na árvore~$T$, enraizada em~$r$. Nota-se que essa função pode
ser executada usando uma busca em profundidade e toma tempo~$O(n)$.


Segue o algoritmo que encontra um corte com as propriedades descritas
no lema.

\medskip

\begin{algorithm}[H]
\label{alg:simpleApproxCutTree}
	\SetKwInOut{Input}{input}
	\SetKwInOut{Output}{output}

	\caption{Computa corte aproximado em uma árvore}
	\Input{árvore~$T =~(V,E)$ com~$n$ vértices e~$m \in~[n]$}
	\Output{Um corte~$(B,W)$ tal que~$\dfrac{m}{2}<|B|\le~m$}
	$B \gets \emptyset$\;
	\eIf{$m =~n$}{
		$B \gets V$\;
	}
	{
		$r$ um vértice arbitrário de~$T$\;
		$d \gets~$ CALCULA\_NUMERO\_DE\_DESCENDENTES($T$,$r$)\;
		$v \gets r$\;
		\While{existe descendente~$u$ de~$v$ com~$d[u]\ge m$ }{
			$v \gets u$ \;
		}
		\eIf{existe descendente~$u$ de~$v$ com~$d[u]>\dfrac{m}{2}$}{
			$B\gets T_u$\;  
			\tcp{$T_u$ é a sub-árvore enraizada em~$u$}
		}
		{
		\For{cada filho~$u$ de~$v$}
		{
			\eIf{$|B+T_u| \le~m$}{
				$B\gets B\cup~T_u$\;
			}
			{
				break\;
			}
		}
		}
	}
	\Return $(B,V\setminus B)$

\end{algorithm}	

\bigskip
\bigskip
\bigskip

\textbf{Análise do Algoritmo}

	Suponha que~$T=(V,E)$.
	É fácil ver que se~$m=n$, então os valores~$B =~V$ e~$W =~\emptyset$ 
	satisfazem as condições do lema, dado que~$e_G(B,W)$ será~$0$. 
	Isso pode ser computado em tempo~$O(n)$.

	Caso contrário, é necessário escolher uma raiz arbitrária~$r$ para~$T$
	e calcular o número de vértices das sub-árvores enraizadas 
	em cada um dos nós, que é o número de descendentes do nó.
	Isso serve para que saibamos a quantidade de vértices das sub-árvores
	sem que precisemos percorrer todos os nós das sub-árvores a cada consulta.

	Feito isso, temos que~$v$ é o vértice analisado no momento 
	e~$V_1, V_2, \ldots, V_k$ são as sub-árvores enraizadas nos~$k$ filhos
	de~$v$.
	Se algum~$V_i$ satisfizer~$\dfrac{m}{2}<~|V_i|\le~m$, podemos retornar
	a sub-árvore~$|V_i|$ como sendo o conjunto~$B$, satisfazendo o Lema 
	\ref{lema:simpleApproxCutTree},
	dado que o corte será~$1$.
	Caso contrário, se~$|V_i|>~m$ para algum~$i$, assumimos que~$v$ 
	agora é a raiz de~$V_i$ e aplicamos novamente esse processo em~$v$.

	Nota-se que esse procedimento irá parar em algum momento, dado que a árvore 
	é finita, e ele parará quando encontrar
	uma sub-árvore que satisfaça o Lema 
	\ref{lema:simpleApproxCutTree}
	(e essa sub-árvore será o conjunto~$B$) 
	ou quando chegarmos numa situação em que  
	~$|V_i|\le~\dfrac{m}{2}$ para todos os~$k$ filhos de~$v$.
	Nesse último caso, sabemos que~$|V_1\cup~V_2\cup~\ldots \cup~V_k|\ge m$, pois
	a sub-árvore enraizada em~$v$ possui mais que~$m$ vértices. 
	Sabemos também que~$|V_i|\le~\dfrac{m}{2}$ para todos os~$k$ filhos de~$v$. 
	Com isso, percorremos os~$V_i$ em ordem, e 
	adicionamos~$V_i$ ao conjunto~$B$, parando antes que~$|B| >~m$. Isso nos 
	dará~$e_T(B,V\setminus B) <~grau(v)\le~\Delta(T)$, o que satisfaz o lema.
	Dado que~$|V_j|\le~\dfrac{m}{2}$ para todo~$j$,
	sabemos que existe um~$i$ tal 
	que~$\dfrac{m}{2} <~|V_1 \cup~V_2 \cup~\cdots \cup~V_i| \le~m$, pois
	a união das sub-árvores
	enraizadas em~$v$ possui mais que~$m$ vértices e, para todo~$\ell$ que 
	satisfaça~$|V_1\cup~V_2\cup~\cdots\cup~V_\ell|\le~\dfrac{m}{2}$, temos 
	que~$|V_1\cup~V_2\cup~\cdots~\cup~V_\ell\cup~V_{\ell+1}|\le~m$.


\bigskip
\bigskip
\bigskip

%%%%%%%%%%%%%%%%%%%%%%%%%%%%%%%%%%%%%%%%%%%%%%%%%%%%%%%%%%%%%%%%%%%
%%%%%%%%%%%%%%%%% LEMA 3 %%%%%%%%%%%%%%%%%%%%%%%%%%%%%%%%%%%%%%%%%%

\subsection{Lema do corte aproximado simples para florestas}

\begin{lem}[{\cite[Lemma 2]{Schmidt15}}]
\label{lema:simpleApproxCutForest}
	Para toda floresta~$G$ com~$n$ vértices e todo~$m \in~[n]$,
	existe um corte~$(B,W)$ em~$G$ tal 
	que~$\dfrac{m}{2} <|B| \le~m$ e~$e_G(B,W) \le~\Delta(G)$.
	Um corte que satisfaz esses requisitos pode ser computado em
	tempo~$O(n)$.
\end{lem}

\bigskip

Agora examinaremos como computar o corte descrito pelo lema.
Nota-se que essa é uma adaptação do algoritmo anterior, 
trocando árvore por floresta.

\medskip
\medskip

\begin{algorithm}[H]
\label{alg:simpleApproxCutForest}
	\SetKwInOut{Input}{input}
	\SetKwInOut{Output}{output}

	\caption{Computa corte aproximado simples em uma floresta}
	\Input{floresta~$G =~(V,E)$ com~$n$ vértices e~$m \in~[n]$}
	\Output{corte~$(B,W)$ tal que~$\dfrac{m}{2}\le|B|\le~m$}
	$B \gets \emptyset$\;
	Sejam~$V_1, V_2,\ldots, V_k$ os conjuntos de vértices das
	componentes conexas de~$G$\;
	\For{$i=1 \to k$}{
		\If{$|V_i| + |B|\le~m$}{
			~$B \gets B\cup~V_i$\;
		}	
		\ElseIf{$|B|<m$}{
			$(B',W')\gets Algoritmo\ref{alg:simpleApproxCutTree}(G[V_i], |V_i|, m-|B|)$\;
			\tcp {Devolve o corte~$(B',W')$ em~$G[V_i]$ com 
			$\dfrac{m-|B|}{2} <|B'| \le~m-|B|$ (usando o Lema \ref{lema:simpleApproxCutTree})}

			$B \gets B\cup~B'$\;

			break\;
		}	
	}
	\Return $(B,V\setminus B)$\;

\end{algorithm}	

\bigskip
\bigskip
\bigskip

\textbf{Análise do Algoritmo}

	Temos que~$T_1, T_2, \ldots,T_k$ são 
	as árvores de~$G$. 
	Primeiro percorremos as árvores na ordem que compõem. 
	Seja~$\ell$ o maior inteiro tal 
	que~$s=~\displaystyle\sum_{i=1}^{\ell}|V(T_i)| \le~m$.
	Colocamos~$T_1,T_2, \ldots,T_\ell$ no conjunto~$B$.
	Caso~$s=m$, temos~$|B|=m$ com~$e_G(B,V\setminus B)=0$, o que satisfaz o lema.
	Caso contrário, ao encontrar um corte~$(B',W')$ em~$T_{\ell+1}$ que
	satisfaça~$\dfrac{m-s}{2}<|B'|\le~m-s$ com~$e_{T_{\ell+1}}(B',W') \le~
	\Delta(T_{\ell+1})$, e acrescentar~$B'$ ao conjunto~$B$, 
	teremos~$\dfrac{m+s}{2}<|B| \le~m$ 
	com~$e_G(B,W)=\Delta(T_{\ell+1}) \le~\Delta(G)$.
	Logo, depois de acrescentar~$B'$ em~$B$, teremos o corte~$(B,V\setminus B)$
	que satisfaz o lema, e podemos fazer isso utilizando o 
	Lema~\ref{lema:simpleApproxCutTree}.
	
	Nota-se que, como a linha 2 pode ser feita usando uma
	busca em profundidade e as demais 
	linhas envolvem verificar o tamanho das componentes 
	(que já foi calculado na linha 2) e uma chamada ao
	Algoritmo~\ref{alg:simpleApproxCutTree}, então essa operação pode 
	ser feita em tempo~$O(n)$, dado que todas as operações
	citadas são executadas em tempo linear. 

\bigskip
\bigskip
\bigskip


%%%%%%%%%%%%%%%%%%%%%%%%%%%%%%%%%%%%%%%%%%%%%%%%%%%%%%%%%%%%%%%%%%%
%%%%%%%%%%%%%%%%% LEMA 4 %%%%%%%%%%%%%%%%%%%%%%%%%%%%%%%%%%%%%%%%%%

\subsection{Lema do corte aproximado}

\begin{lem}[{\cite[Lemma 3]{Schmidt15}}]
\label{lema:approxCutForest}
	Para toda floresta~$G$ com~$n$ vértices, todo~$m \in~[n]$
	e todo~$c \in~[0,1)$,
	existe um corte~$(B,W)$ em~$G$ tal 
	que~$cm \le~|B| \le~m$ 
	e~$e_G(B,W) \le~\ceil[\Big]{ \dfrac{2c}{1-c}} \Delta(G)$.
	Um corte que satisfaz esses requisitos pode ser computado em
	tempo~$O(\ceil[\Big]{ \dfrac{2c}{1-c}} n)$.
\end{lem}

\medskip
\medskip

\begin{algorithm}[H]
\label{alg:approxCutForest}
	\SetKwInOut{Input}{input}
	\SetKwInOut{Output}{output}

	\caption{Computa corte aproximado em uma floresta}
	\Input{floresta~$G =~(V,E)$ com~$n$ vértices,~$m \in~[n]$, 
	$c\in~(\dfrac{1}{2}, 1)$}
	\Output{Um corte~$(B,W)$ tal que~$cm\le|B|\le~m$}
	\eIf{$c\le~\dfrac{1}{2}$}{
		$(B,W) \gets$ Algoritmo~\ref{alg:simpleApproxCutForest}$(G, n, m)$ \;
		\tcp{Devolve o corte~$(B,W)$ em~$G$ com 
		$\dfrac{m}{2} <|B'| \le~m$ (usando o Lema \ref{lema:simpleApproxCutForest})}
		
		$B \gets B'$\;
	}
	{
		$B \gets \emptyset$\;
		\While{$|B|<cm$}{
			$(B',W') \gets$ 
			Algoritmo\ref{alg:simpleApproxCutForest}$(G[V\setminus B], 
			n-|B|, m-|B|)$ \;
			\tcp{Devolve o corte~$(B',W')$ em~$G[V\setminus B]$ com 
			$\dfrac{m-|B|}{2} <|B'| \le~m-|B|$ (usando o Lema \ref{lema:simpleApproxCutForest})}

			$B \gets B\cup~B'$\;		
		}
	}
	\Return $(B,V\setminus B)$

\end{algorithm}	

\bigskip
\bigskip
\bigskip

\textbf{Análise do Algoritmo}

	Sabemos que se~$c \le~\dfrac{1}{2}$, podemos aplicar o Algoritmo 
	\ref{alg:simpleApproxCutForest}, 
	dado que o resultado nos dará um conjunto~$B'$ tal 
	que~$|B'|>\dfrac{m}{2}\ge cm$ e~$e_G(B,W)\le~\Delta(G)$.

	Caso contrário, temos que~$B =~\emptyset$.
	Executaremos o 
	Algoritmo~\ref{alg:simpleApproxCutForest}, referente ao 
	Lema~\ref{lema:simpleApproxCutForest}
	diversas vezes para 
	encontrar 
	um corte~$(B',W')$ em~$G[V\setminus B]$ tal 
	que~$\dfrac{m-|B|}{2}<|B'|\le~m-|B|$,
	acrescentando o conjunto~$B'$ a~$B$ em cada uma das chamadas ao 
	Lema~\ref{lema:simpleApproxCutForest}.
	Fazemos isso até que~$|B|\ge cm$ seja satisfeito.
	Nota-se que em algum momento~$|B|\ge cm$ irá ocorrer, pois a cada vez
	que executamos o 
	Algoritmo~\ref{alg:simpleApproxCutForest}, 
	acrescentamos pelo menos um vértice em~$B$,
	e~$|B|>m$ nunca irá ocorrer, pois o conjunto~$B'$ obtido no 
	Lema~\ref{lema:simpleApproxCutForest} é
	tal que~$|B'|\le~m-|B|$. Logo, serão adicionados no 
	máximo~$m-|B|$ vértices em~$B$.

	Verificaremos agora se o consumo de tempo e a largura do 
	corte devolvido pelo Algoritmo~\ref{alg:approxCutForest} satisfazem
	o que foi proposto no Lema~\ref{lema:approxCutForest}.
	
	Caso~$c\le~\dfrac{1}{2}$, executamos o
	Algoritmo~\ref{alg:simpleApproxCutForest} uma única vez, portanto
	o tempo é~$O(n)$, que equivale a~$O(\ceil[\Big]{\dfrac{2c}{1-c}} n)$.
	Em relação a largura do corte,
	será devolvido o corte do Algoritmo~\ref{alg:simpleApproxCutForest}
	sem nenhuma modificação, então, sabe-se que~$e_G(B,W)\le~\Delta(G)$.
	Provaremos agora que~$\Delta(G) \le \ceil[\Big]{ \dfrac{2c}{1-c}}\Delta(G)$.
	Como~$0<c<1$, temos que~$0<1-c$ e~$0<2c$, implicando  
	em~$0<~\dfrac{2c}{1-c}$, assim sendo,~$1\le \ceil[\Big]{ \dfrac{2c}{1-c}}$.
	Portanto, 
	como~$\Delta(G)\ge 0$, então temos 
	que~$\Delta(G)\le~\ceil[\Big]{\dfrac{2c}{1-c}}\Delta(G)$, satisfazendo
	assim as propriedades do Lema~\ref{lema:approxCutForest}.

	Caso contrário,~$c>\dfrac{1}{2}$. Como em todas as vezes que
	executamos a linha 7 do Algoritmo~\ref{alg:approxCutForest}, vale
	que~$|B|<cm$, então também é válido que~$m-|B|>~(1-~c)m$.
	E como~$m-|B|$ é o valor do~$m$ que passaremos para o 
	Algoritmo~\ref{alg:simpleApproxCutForest}, sabemos que o conjunto $B$ 
	do corte devolvido terá mais que~$\dfrac{m-|B|}{2}$ vértices.
	Portanto, executaremos o Algoritmo~\ref{alg:simpleApproxCutForest} no 
	máximo~$\ceil[\Big]{\dfrac{cm}{\frac{(1-c)m}{2}}} 
	=~\ceil[\Big]{\dfrac{2c}{1-c}}$ vezes, dando, no total, um
	tempo~$O(\ceil[\Big]{ \dfrac{2c}{1-c}} n)$.
	Usaremos essa mesma linha de pensamento para calcular o largura
	máxima do corte devolvido. Sabemos que o 
	Algoritmo~\ref{alg:simpleApproxCutForest} será 
	chamado~$\ceil[\Big]{\dfrac{2c}{1-c}}$ vezes, e para cada uma das vezes
	ele devolve um corte cuja largura máxima é~$\Delta(G)$, portanto,
	a largura máxima do corte devolvido 
	é~$\ceil[\Big]{\dfrac{2c}{1-c}}\Delta(G)$.







%%%%%%%%%%%%%%%%%%%%%%%%%%%%%%%%%%%%%%%%%%%%%%%%%%%%%%%%%%%%%%%%%%%
%%%%%%%%%%%%%%%%%%%%%%%  SEÇÃO 7     %%%%%%%%%%%%%%%%%%%%%%%%%%%%%%
%%%%%%%%%%%%%%%%%%%%%%%%%%%%%%%%%%%%%%%%%%%%%%%%%%%%%%%%%%%%%%%%%%%
\newpage

\section {Rotulação dos vértices da árvore}

	Mais adiante, iremos nos referir a vértices com uma característica 
	específica, e para isso precisamos atribuir um rótulo numérico~$x\in~[n]$ 
	distinto para cada um dos vértices da árvore, onde~$n$ é o número de 
	vértices da árvore~$T$. 
	Denotaremos o \textbf{rótulo} de um vértice~$v$ por~$rot(v)$.

	Primeiramente, usaremos o método mencionado na seção
	\ref{sec:caminhoMaximo} para
	calcular o~$v_0$-$v_k$-caminho, de comprimento~$p$, que é máximo em~$T$. 
	Para todo 
	vértice~$v\in~V(v_0$-$v_k$-caminho$)$, teremos uma 
	árvore~$T_v \in~G\setminus E(v_0$-$v_k$-caminho$)$ de modo que~$v\in~T_v$.

	Dessa forma, teremos sub-árvores enraizadas nos vértices do
	caminho máximo, como o mostrado na figura abaixo.
	

 	%% fill=blue!30 -- núemro indica quão escuro é
	\begin{center} \begin{tikzpicture}
		[scale=.7,auto=left,every node/.style={circle, draw=black,
				fill=blue!20}]

		\draw [draw=yellow, fill=yellow!20] (0,4.8) ellipse (1.3cm and 2.4cm); %0
		\draw [draw=yellow, fill=yellow!20] (3,4.4) ellipse (1.3cm and 2.4cm); %1
		\draw [draw=yellow, fill=yellow!20] (6,4.8) ellipse (1.3cm and 2.4cm); %2
		\draw [draw=yellow, fill=yellow!20] (9,4.4) ellipse (1.3cm and 2.4cm); %3
 		\draw [draw=yellow, fill=yellow!20] (12,4.8) ellipse (1.3cm and 2.4cm); %4
		\draw [draw=yellow, fill=yellow!20] (18,4.8) ellipse (1.3cm and 2.4cm); %5

		\node [text=blue!65,draw=none, fill=none] at (-0.4,6.5) {$T_{v_0}$};
		\node [text=blue!65,draw=none, fill=none] at (2.6,6.1) {$T_{v_1}$};
		\node [text=blue!65,draw=none, fill=none] at (5.6,6.5) {$T_{v_2}$};
		\node [text=blue!65,draw=none, fill=none] at (8.6,6.1) {$T_{v_3}$};
		\node [text=blue!65,draw=none, fill=none] at (11.6,6.5) {$T_{v_4}$};
		\node [text=blue!65,draw=none, fill=none] at (17.6,6.5) {$T_{v_p}$};

		\node [subtree] (y1) at (3,4.6)  {};
		\node [subtree, fill=red!30] (y2) at (6,5)  {};
		\node [subtree] (y3) at (9,4.6)  {};
		\node [subtree] (y4) at (12,5) {};

		\node (x0) at (0,5.4)  {$v_0$};
		\node (x1) at (3,5)    {$v_1$};
		\node (x2) at (6,5.4)  {$v_2$};
		\node (x3) at (9,5)    {$v_3$};
		\node (x4) at (12,5.4) {$v_4$};
		\node (x5) at (15,5)   {...};
		\node (x6) at (18,5.4) {$v_p$};


		\foreach \from/\to in {x0/x1,x1/x2,x2/x3,x3/x4,x4/x5,x5/x6}
		\draw (\from) -- (\to);

	\end{tikzpicture} \end{center}

	Temos que cada vértice contido no~$v_0$-$v_p$-caminho receberá uma 
	rotulação de forma que:
	\begin{itemize}
		\item~$rot(v_i)>rot(v_j)$ para todo~$i<j$ e
		\item~$rot(v_{i-1})<~v' \le~rot(v_i)$ para todo vértice~$v'$ contido 
		na sub-árvore enraizada em~$v_i$. 
	\end{itemize}

	\bigskip
	\bigskip

	\subsection{Descrição do algoritmo de rotulação}
	Essa rotulação pode ser obtida facilmente 
	manipulando a lista de adjacência e em seguida,
	usando uma busca em profundidade.
	Ambas as ações são executadas em tempo linear, então
	pode-se rotular uma árvore em tempo~$O(n)$.

	Para cada vértice~$v_i$ contido no~$v_1$-$v_{p}$-caminho, percorremos
	a sua lista de adjacência e quando encontrarmos o 
	vértice~$v_{i-1}$, colocamos ele no início da lista.
	Dessa forma, quando aplicamos uma busca em profundidade
	nessa árvore, usando o~$v_p$ como raiz, a busca descerá primeiro
	pelos vértices do caminho máximo e atribuirá a rotulação
	somente quando terminar de ver todos os vértices filhos, garantindo
	assim que para todo~$r\in~\{1,\ldots, p\}$, temos 
	que~$rot(T_{v_i})\le~rot(v_i)$ e~$rot(v_i)>~rot(v_{i-1})$.
	
	\bigskip
	\bigskip

	\subsection{Mapeamento dos rótulos de vértices }
	Vamos definir agora as funções que mapeiam um vértice em 
	outro, utilizando a rotulação como base.
	Chamaremos de~$map^+_m$
	a função que devolve o~$m$-ésimo próximo vértice e,
	de~$map^-_m$ a função inversa da primeira, sendo assim
	ela devolve o~$m$-ésimo vértice anterior.
	Dessa forma, podemos ver que, seja~$v$ um 
	vértice, ~$map^+_m(rot(v)) =~rot(v)+~m$ 
	e~$map^-_m(rot(v)) =~rot(v)-~m$.

	Para facilitar, identificaremos os vértices pelos
	seus rótulos. Sendo assim, temos 
	que para um vértice~$v$, ~$map^+_m(v) =~v+~m$ 
	e~$map^-_m(v) =~v-~m$ como uma simplificação das funções
	enunciadas anteriormente.

	Nota-se que para todo $m\in~[n]$, caso~$n>2$, se existe 
	um~$v$ tal 
	que~$v\in~v_0$-$v_k$-caminho 
	e~$map^+_m(v)\in~v_0$-$v_k$-caminho, então 
	existe um corte $(B,W)\in~T$, tal 
	que~$B =~\{v+1, v+2,\ldots,~v+m\}$, com~$|B|=~m$
	e~$e_T(B,W)\le~2\le~\Delta(T)$.
	Isso pode ser visto na figura abaixo.

	\begin{center} \begin{tikzpicture}
		[scale=.7,auto=left,every node/.style={circle, draw=black,
				fill=blue!20}]

		\draw [draw=yellow, fill=yellow!20] (7.5,1.5) rectangle (13.5, 6.5);

		\node [text=blue!65,draw=none, fill=none] at (10.5,2.2) {\Large{$m$ vértices}};
		\node [draw=none, fill=none] at (9,6.5) {\footnotesize{$map^+_m$}};
		\node [draw=none, fill=none] at (9,7.9) {\footnotesize{$map^-_m$}};


		\node [subtree] (y1) at (3,4.6)  {};
		\node [subtree, fill=red!30] (y2) at (6,5)  {};
		\node [subtree] (y3) at (9,4.6)  {};
		\node [subtree] (y4) at (12,5) {};

		\node (x0) at (0,5.4)  {$v_0$};
		\node (x1) at (3,5)    {$v_1$};
		\node (x2) at (6,5.4)  {$v_2$};
		\node (x3) at (9,5)    {$v_3$};
		\node (x4) at (12,5.4) {$v_4$};
		\node (x5) at (15,5)   {...};
		\node (x6) at (18,5.4) {$v_p$};


		\foreach \from/\to in {x0/x1,x1/x2,x3/x4,x5/x6}
		\draw (\from) -- (\to);

		\path[-, line width=1.5pt, draw=red]
		(x2) edge[right] (x3);
		\path[-, line width=1.5pt, draw=red]
		(x4) edge[right] (x5);

		\path[->, line width=1.5pt]
		(x2) edge[bend left=23 ] (x4);
		\path[->, line width=1.5pt]
		(x4) edge[bend right=45, looseness=1.7 ] (x2);


	\end{tikzpicture} \end{center}
	
	Caso contrário, ~$n\le~2$. 
	É fácil ver que todos os vértices
	da árvore estarão no caminho máximo, portanto, para
	qualquer~$m\in[n]$, teremos um conjunto~$B$ tal
	que~$|B|=m$
	e~$e_T(B,W)\le~|E(T)|=~n-1=~\Delta(T)$.




%%%%%%%%%%%%%%%%%%%%%%%%%%%%%%%%%%%%%%%%%%%%%%%%%%%%%%%%%%%%%%%%%%%
%%%%%%%%%%%%%%%%%%%%%%%  SEÇÃO 8     %%%%%%%%%%%%%%%%%%%%%%%%%%%%%%
%%%%%%%%%%%%%%%%%%%%%%%%%%%%%%%%%%%%%%%%%%%%%%%%%%%%%%%%%%%%%%%%%%%
\newpage

\section {Algoritmo da dobra do diâmetro 8}


%%%%%%%%%%%%%%%%%%%%%%%%%%%%%%%%%%%%%%%%%%%%%%%%%%%%%%%%%%%%%%%%%%%
%%%%%%%%%%%%%%%%%%%%%%%  SEÇÃO 9     %%%%%%%%%%%%%%%%%%%%%%%%%%%%%%
%%%%%%%%%%%%%%%%%%%%%%%%%%%%%%%%%%%%%%%%%%%%%%%%%%%%%%%%%%%%%%%%%%%
\section {Algoritmo FST para bissecção 9}



%%%%%%%%%%%%%%%%%%%%%%%%%%%%%%%%%%%%%%%%%%%%%%%%%%%%%%%%%%%%%%%%%%%
%%%%%%%%%%%%%%%%%%%%%%%  SEÇÃO 10    %%%%%%%%%%%%%%%%%%%%%%%%%%%%%%
%%%%%%%%%%%%%%%%%%%%%%%%%%%%%%%%%%%%%%%%%%%%%%%%%%%%%%%%%%%%%%%%%%%
\section {Complexidade do problema da bissecção 10}



%%%%%%%%%%%%%%%%%%%%%%%%%%%%%%%%%%%%%%%%%%%%%%%%%%%%%%%%%%%%%%%%%%%
%%%%%%%%%%%%%%%%%%%%%%%  SEÇÃO 11    %%%%%%%%%%%%%%%%%%%%%%%%%%%%%%
%%%%%%%%%%%%%%%%%%%%%%%%%%%%%%%%%%%%%%%%%%%%%%%%%%%%%%%%%%%%%%%%%%%
\section {Algoritmo de Jansen 11}



%%%%%%%%%%%%%%%%%%%%%%%%%%%%%%%%%%%%%%%%%%%%%%%%%%%%%%%%%%%%%%%%%%%
%%%%%%%%%%%%%%%%%%%%%%%  SEÇÃO 12    %%%%%%%%%%%%%%%%%%%%%%%%%%%%%%
%%%%%%%%%%%%%%%%%%%%%%%%%%%%%%%%%%%%%%%%%%%%%%%%%%%%%%%%%%%%%%%%%%%
\newpage

\section {Geração de árvores binárias aleatórias}
	Nesta seção falaremos sobre o gerador de árvores binárias aleatórias, um dos tipos de árvores que foi utilizado nos
	testes do algoritmo FST.

	As árvores binárias aleatórias são árvores com um número
	pré-definido de vértices, que são construídas
	aleatoriamente e onde cada nó possui no máximo dois filhos.

	O gerador recebe um inteiro~$n$ e cria um vetor permutado
	aleatoriamente com valores de~$0$ a~$n-1$.
	Depois, constrói uma árvore binária de busca e vai
	inserindo os nós de acordo com a ordem dada pelo vetor.
	Isso leva tempo~$O(n^2)$, dado que a inserção na 
	árvore binária de busca leva tempo linear para cada um 
	dos~$n$ nós.

	Segue abaixo o algoritmo do gerador.
	\bigskip
	\bigskip

	\begin{algorithm}[H]
	\label{alg:insertion}
		\SetKwInOut{Input}{input}
		\SetKwInOut{Output}{output}

		\caption{Inserção do nó $n$ na árvore binária de busca}
		\Input{raiz $r$ da árvore e nó $n$}
		\Output{}
		\eIf {$n.chave>r.chave$}
		{
			\eIf{$r.right = NULL$}{
				$r.right\gets n$\; 
			}
			{
				Algoritmo \ref{alg:insertion}$(r.right,~n)$\;
			}
		}
		{
			\eIf{$r.left = NULL$}{
				$r.left\gets n$\; 
			}
			{
				Algoritmo \ref{alg:insertion}$(r.left, ~n)$\;
			}
		}

	\end{algorithm}	

	\bigskip

	\begin{algorithm}[H]
	\label{alg:ABAgenerator}
		\SetKwInOut{Input}{input}
		\SetKwInOut{Output}{output}

		\caption{Gerador de árvores binárias aleatórias}
		\Input{inteiro $n$}
		\Output{raiz da árvore binária aleatória gerada}
		\bigskip
		\tcp{Aleatoriza vetor}
		\For {$i = 0 \to n-1$}
		{
			$v[i] \gets i$\;
		}

		\For {$i = n-1 \to 1$}{
			$indice \gets$ RANDOM$(0\ldots i)$\;
			$v[indice] \leftrightarrow v[i]$\;
		}
		\bigskip
		\tcp{Inserção dos elementos do vetor na árvore binária de busca}
		$raiz.chave\gets v[0]$\;
		\For{$i=1\to n-1$}
		{
			new $no.chave\gets v[i]$\;
			Algoritmo~\ref{alg:insertion}$(raiz, no)$\;
		}
		\Return $raiz$\;

	\end{algorithm}	

	\bigskip
	\bigskip

	Nota-se que numa árvore desse tipo, em grande parte dos casos,  
	a maioria dos nós estará presente no caminho mais longo. 
	O que favorece um pouco o algoritmo FST.


%%%%%%%%%%%%%%%%%%%%%%%%%%%%%%%%%%%%%%%%%%%%%%%%%%%%%%%%%%%%%%%%%%%
%%%%%%%%%%%%%%%%%%%%%%%  SEÇÃO 13    %%%%%%%%%%%%%%%%%%%%%%%%%%%%%%
%%%%%%%%%%%%%%%%%%%%%%%%%%%%%%%%%%%%%%%%%%%%%%%%%%%%%%%%%%%%%%%%%%%
\section {Experimentos computacionais 13}

	\begin{table}[ht]
		\centering
		\caption {Random Binary Trees}
		\begin{tabular}{ c | >{\columncolor{blue!14}}c | >{\columncolor{red!14}}c | >{\columncolor{blue!14}}c | >{\columncolor{red!14}}c }
			\hline
			$|V(T)|$ & $e_T(B,W)$ - \textbf{FST} & $e_T(B,W)$ - \textbf{Jansen} & Tempo - \textbf{FST} & Tempo - \textbf{Jansen}  \\[10pt]
			\hline
			Jansen    & Noruega        & .955   & bla & blalaa \\ [6pt]
			FST       & Austr{\'a}lia  & .938   & jksdfk & auhsua\\[6pt]
			\hline                            
		 
		\end{tabular}
	\end{table}





%%%%%%%%%%%%%%%%%%%%%%%%%%%%%%%%%%%%%%%%%%%%%%%%%%%%%%%%%%%%%%%%%%%
%%%%%%%%%%%%%%%%%%%%%%%  SEÇÃO 14    %%%%%%%%%%%%%%%%%%%%%%%%%%%%%%
%%%%%%%%%%%%%%%%%%%%%%%%%%%%%%%%%%%%%%%%%%%%%%%%%%%%%%%%%%%%%%%%%%%
\section {Conclusões 14}



\newpage
\bibliographystyle{plain}
\bibliography{refs}
\end{document}
