\documentclass[a4paper,12pt]{article}
\usepackage[svgnames,dvipsnames,x11names]{xcolor} 
% Para usar a linguagem PT-BR.
\usepackage[utf8]{inputenc}
\usepackage[T1]{fontenc}
\usepackage{tikz}
\usepackage{array}
\usepackage[brazilian]{babel}
\usepackage[linesnumbered,ruled]{algorithm2e}
\usepackage{amsmath, amsthm, amssymb}
\usepackage{mathtools,booktabs}
\usepackage{colortbl}
\usepackage{multirow}
\usepackage{enumitem}
\usepackage{setspace}
\usepackage{etoolbox}
%\usepackage[hmargin=3.5cm,vmargin=3.7cm,bmargin=2cm]{geometry}

\DeclarePairedDelimiter\ceil{\lceil}{\rceil}

\usetikzlibrary{shapes.geometric,arrows,fit,matrix,positioning}
\tikzset
{
    subtree/.style  = {isosceles triangle, draw=black, 
    align=center, minimum height=0.5cm, minimum width=1cm, shape 
    border rotate=90, anchor=north, fill=red!30}
}
\usetikzlibrary{decorations.pathreplacing}

\selectlanguage{brazilian} 

\newtheorem{teo}{Teorema}
\newtheorem{lem}{Lema}
\newtheorem{prov}{Prova}
\newtheorem{alg}{Análise do algoritmo}
\newtheorem{prob}{Problema}
\newtheorem{coro}{Corolário}

\newcommand{\caminho}{\mathrm{-caminho}}
\newcommand{\dist}{\mathrm{dist}}
\newcommand{\diam}{\mathrm{diam}}
\newcommand{\grau}{\mathrm{grau}}
\newcommand{\rot}{\mathrm{rot}}
\newcommand{\map}{\mathrm{map}}
\newcommand{\Oh}{\mathrm{O}}
\newcommand\mycommfont[1]{\footnotesize\ttfamily\textcolor{blue}
{#1}}
\SetCommentSty{mycommfont}

\newcolumntype{P}[1]{>{\centering\arraybackslash}p{#1}}

\newcounter{algoline}
\newcommand\Numberline{\refstepcounter{algoline}\nlset{\thealgoline}}
\AtBeginEnvironment{algorithm}{\setcounter{algoline}{21}}


\sloppy

\onehalfspacing

\begin{document}
 
\begin{center}
   {\large \textbf{UNIVERSIDADE DE SÃO PAULO}} \\[1.4cm]
   
   {\large \textbf{INSTITUTO DE MATEMÁTICA E ESTATÍSTICA}}\\[4.2cm]
   
   {\Huge TRABALHO DE CONCLUSÃO }\\[0.3cm]
   {\Huge DE CURSO }\\[9cm]
   
   {\large { Aluna: Karina Suemi Awoki}}\\[0.3cm]
   
   {\large { Orientadora: Cristina Gomes Fernandes}}
   

\end{center}

\newpage


\begin{abstract}
	Este trabalho de conclusão de curso 
	consistiu no estudo e na implementação
	do algoritmo de FTS, que 
	é um algoritmo de aproximação para
	o problema da bissecção mínima aplicado
	a árvores.
	Este algoritmo 
	faz parte da tese
	de doutorado, ainda em preparação, de 
	Tina Janne Schmidt.

	Aqui encontram-se as provas dos lemas e teoremas que 
	serviram como bases para a criação do
	algoritmo de FST, 
	além de pseudocódigos, juntamente com a
	análise de complexidade de cada uma das 
	partes desse algoritmo.

	Além disso, foi estudado e implementado
	o algoritmo de Jansen~et~al., que
	é um algoritmo que encontra a solução ótima
	para o problema da bissecção mínima em árvores, 
	com a 
	finalidade de comparar soluções e 
	desempenho de tempo entre os dois algoritmos.
	Essa comparação foi feita através de testes 
	com tipos específicos de árvores, e alguns dos
	resultados estão presentes neste texto.

	% Ao final, é provado que a versão de decisão 
	% do problema da bissecção
	% mínima é NP-completa.

\end{abstract}

\newpage

\tableofcontents

\newpage

\part{Parte Objetiva}

%%%%%%%%%%%%%%%%%%%%%%%%%%%%%%%%%%%%%%%%%%%%%%%%%%%%%%%%%%%%%%%%%%%
%%%%%%%%%%%%%%%%%%%  INTRODUÇÃO  %%%%%%%%%%%%%%%%%%%%%%%%%%%%%%%%%%
%%%%%%%%%%%%%%%%%%%%%%%%%%%%%%%%%%%%%%%%%%%%%%%%%%%%%%%%%%%%%%%%%%%

\section{Introdução}

O assunto abordado neste trabalho de conclusão de curso (TCC) se
enquadra na área de Otimização Combinatória e diz respeito ao 
\emph{Problema da Bissecção Mínima}. Para definir o problema 
precisamente, seguem primeiro algumas definições. 

Seja~${G=(V,E)}$ um grafo e~$B$ e~$W$ conjuntos de vértices de~$G$.
Dizemos que~$(B,W)$ é um \textbf{corte}
de~$G$ se~${B \cup W =V}$ e~${B\cap W =\emptyset}$.
Dizemos também que uma aresta~$e$ de~$G$ está no corte~$(B,W)$
se~$e$ tem uma ponta em~$B$ e outra em~$W$. 
Denotamos o número de arestas no corte~$(B,W)$ por \textbf{largura}
do corte ou por~$e_G(B,W)$.
O corte~$(B,W)$ é uma \textbf{bissecção} se~${|B| =|W|}$
quando~$|V|$ é par, ou se~${||B|-|W|| =1}$ quando~$|V|$ é ímpar.

O \emph{Problema da Bissecção Mínima} consiste em, dado um grafo, 
encontrar uma bissecção no grafo de largura mínima.

Sabe-se que o Problema da Bissecção Mínima é NP-difícil~\cite{GareyJS76} 
e a melhor aproximação conhecida para o caso geral do problema tem 
razão~$\Oh(\lg n)$~\cite{Racke08}, onde~$n$ é o número de vértices 
do grafo. 
Por outro lado, sabe-se que, para árvores e para grafos que  
se assemelham a árvores, há um algoritmo polinomial 
para encontrar uma bissecção mínima, 
proposto por Jansen, Karpinski, Lingas e 
Seidel~\cite{JansenKLS01}. 

Um melhor entendimento da estrutura 
dos grafos cuja bissecção mínima tem largura grande
pode ajudar no 
desenvolvimento de bons algoritmos de aproximação para
o problema ou ajudar na 
identificação de classes de grafos onde o problema torna-se 
especialmente difícil. 
Para tanto, certas questões têm sido estudadas em árvores, 
em grafos que têm uma estrutura semelhante às árvores, e também em 
grafos planares~\cite{FernandesST13,FernandesST15}, para os quais 
a complexidade do problema encontra-se em aberto. 

Vários outros resultados são conhecidos para o Problema da 
Bissecção Mínima, porém este TCC se 
concentrou no estudo do problema em árvores.

\newpage


%%%%%%%%%%%%%%%%%%%%%%%%%%%%%%%%%%%%%%%%%%%%%%%%%%%%%%%%%%%%%%%%%%%
%%%%%%%%%%%%%%%%   OBJETIVOS  %%%%%%%%%%%%%%%%%%%%%%%%%%%%%%%%%%%%%
%%%%%%%%%%%%%%%%%%%%%%%%%%%%%%%%%%%%%%%%%%%%%%%%%%%%%%%%%%%%%%%%%%%

\subsection{Objetivos} 

O principal objetivo deste trabalho foi o estudo, implementação e 
análise de um algoritmo recente, proposto por Fernandes, Schmidt e 
Taraz~\cite{FernandesST13}, para encontrar uma bissecção 
aproximadamente mínima em árvores de grau limitado. Denotaremos
esse algoritmo por FST. 
A vantagem deste algoritmo sobre o algoritmo de Jansen et 
al.~\cite{JansenKLS01}, que encontra uma bissecção de largura 
mínima, é que este tem um consumo de tempo linear, enquanto que o 
segundo tem um consumo de tempo cúbico no número de vértices da 
árvore. 
Implementamos também o algoritmo de Jansen et al., para fins de 
comparação da largura das bissecções produzidas pelo algoritmo
FST. 

Como já mencionado, o algoritmo de Jansen et al.\ baseia-se em 
programação dinâmica. 
Já o algoritmo de Fernandes et al., para atingir um consumo 
linear, utiliza técnicas bem conhecidas de percursos de árvore, 
como busca em profundidade, bem como um rastreamento de 
informações mais cuidadoso que permite que o processamento todo 
seja executado em tempo linear. 
É um algoritmo mais refinado, dividido na implementação de uma 
série de etapas menores. 

Uma vez implementados os dois algoritmos, foi feita uma análise 
da sua performance em árvores binárias geradas aleatoriamente, e 
em árvores ternárias.
Obtivemos instâncias reais de um problema relacionado e as 
utilizamos também no estudo experimental do algoritmo implementado.
 

\bigskip
\bigskip

%%%%%%%%%%%%%%%%%%%%%%%%%%%%%%%%%%%%%%%%%%%%%%%%%%%%%%%%%%%%%%%%%%%
%%%%%%%%%%%%%%%%%%%%%%%%%%  MÉTODO  %%%%%%%%%%%%%%%%%%%%%%%%%%%%%%%
%%%%%%%%%%%%%%%%%%%%%%%%%%%%%%%%%%%%%%%%%%%%%%%%%%%%%%%%%%%%%%%%%%%

\subsection{Método}

O estudo do algoritmo de Fernandes et al.\ seguiu um roteiro de
como a implementação foi feita, conforme a proposta do TCC. 
A linguagem escolhida para a implementação foi \texttt{C} e os 
algoritmos implementados foram colocados na página do gitHub. 
O roteiro consistiu em uma divisão da implementação do algoritmo 
em várias etapas, que permitiu um melhor acompanhamento do 
trabalho desenvolvido, e uma organização da forma de estudo. 
O algoritmo encontra-se descrito em um documento 
longo~\cite{Schmidt15} juntamente com a sua análise teórica. 
Este documento serviu como base para o entendimento de cada 
etapa, e de como a implementação de cada etapa deveria ser feita 
para que a implementação obtida fosse de fato linear. 

Reuniões frequentes foram feitas junto à supervisora para 
apresentar as etapas já implementadas, bem como para tirar dúvidas 
sobre as próximas etapas ou detalhes da implementação. 
Em maio e junho, Tina Schmidt, uma das autoras do algoritmo que 
foi implementado, visitou o IME, e a parte da 
implementação que estava pronta foi apresentada a ela, e 
algumas discussões foram feitas e em especial, a Tina conseguiu
um conjunto de instâncias reais com Steffen Borgwardt, feitas por 
Kampeier~\cite{Kampmeier}, para usarmos 
em nossos experimentos.   

Em paralelo ao estudo e desenvolvimento da implementação, foram 
estudados também dois artigos da literatura. São eles o artigo 
de Garey, 
Johnson e Stockmeyer~\cite{GareyJS76}, que contém a prova de que o 
problema é NP-difícil, e o artigo de Jansen et al.~\cite{JansenKLS01}, 
que apresenta 
o algoritmo de programação dinâmica para o problema em árvores. 
Após a implementação do algoritmo de Fernandes et~al., foi feita 
a implementação do algoritmo de Jansen et al. e a comparação dos 
dois. 



%%%%%%%%%%%%%%%%%%%%%%%%%%%%%%%%%%%%%%%%%%%%%%%%%%%%%%%%%%%%%%%%%%%
%%%%%%%%%%%%%%%%%%%%%%%  DEFINIÇÕES  %%%%%%%%%%%%%%%%%%%%%%%%%%%%%%
%%%%%%%%%%%%%%%%%%%%%%%%%%%%%%%%%%%%%%%%%%%%%%%%%%%%%%%%%%%%%%%%%%%

\section{Definições}
	
%\begin{enumerate}
%	\item
	Para um grafo~$G$, denotamos seu
	conjunto de vértices por~$V(G)$ e seu
	conjunto de arestas por~$E(G)$.

\bigskip
%	\item
	O número de arestas que incidem em um 
	determinado vértice~$v$ de~$G$ é chamado de 
	\textbf{grau} de~$v$ em~$G$ e é representado
	por~$\grau_G(v)$. O 
	\textbf{grau máximo}, que é o grau de um vértice de
	maior grau em~$G$, é representado por~$\Delta(G)$.

\bigskip

	%\item
	Um \textbf{caminho} em~$G$ é uma sequência de 
	vértices~$v_0, v_1, \ldots,v_{p-1}, v_p$ 
	onde~$v_{k-1}, v_k$ é uma aresta de~$G$ para 
	todo~${k =1,2, \ldots, p}$. 
	O número~$p$ é o \textbf{comprimento} do caminho.

\bigskip

	%\item
	Um \textbf{caminho máximo} em~$G$ é um caminho 
	em~$G$ de comprimento máximo.
	O comprimento de um caminho máximo em~$G$
	é o \textbf{diâmetro} de~$G$, e é denotado por~$\diam(G)$.

	% O \textbf{diâmetro} de um grafo conexo~$G$ é o
	% comprimento de um caminho máximo em~$G$. 
	% Em outras palavras, o diâmetro pode ser definido como:
	% $$ \diam(G)=\max\{\dist(x,y):x,y\in V(G)\}.$$

	% \bigskip

	Seja~$G$ um grafo não necessariamente conexo.
	O \textbf{diâmetro relativo} de~$G$
	 é a razão entre o número
	de vértices de um caminho máximo de~$G$ e o número total de vértices
	de~$G$. Este pode também ser representado por
	$$ \diam^*(G) =\dfrac{\displaystyle\sum_{
	G'~componente~conexa~de~G}^{}(\diam(G')+1)}{|V(G)|}.$$

	Note que~${0<\diam^*(G)\le 1}$ para todo grafo~$G$ 
	com~${V(G)\ne \emptyset}$, pois há pelo menos um vértice 
	de~$G$ num caminho máximo de~$G$, e no máximo 
	todos os vértices de~$G$.

\bigskip

	%\item
	Em uma árvore~$T$, existe um único caminho que conecta 
	quaisquer dois vértices~$x$ e~$y$. 
	Chamaremos tal caminho em~$T$ de~$x,y$--\textbf{caminho}.

\bigskip

	%\item
	A \textbf{distância} entre dois vértices~$x$ e~$y$ de~$T$ é 
	representada por~$\dist(x,y)$, e é o comprimento 
	do~$x,y\caminho$ em~$T$.

\bigskip

	%\item
	Denotamos por~$[n]$
	o conjunto~$\{1,2,\ldots,n\}$.

%\end{enumerate}

%%%%%%%%%%%%%%%%%%%%%%%%%%%%%%%%%%%%%%%%%%%%%%%%%%%%%%%%%%%%%%%%%%%
%%%%%%%%%%%%%%%%%%%%%%%  SEÇÃO 5     %%%%%%%%%%%%%%%%%%%%%%%%%%%%%%
%%%%%%%%%%%%%%%%%%%%%%%%%%%%%%%%%%%%%%%%%%%%%%%%%%%%%%%%%%%%%%%%%%%


\section{Caminho máximo em árvores}
\label{sec:caminhoMaximo}
	\subsection{Algoritmo que encontra um caminho máximo}

	Encontrar um caminho máximo em uma árvore~$T=(V,E)$ é uma 
	tarefa computacionalmente simples que pode ser realizada em 
	tempo~$O(n)$, sendo~$n$ o número de vértices de~$T$, ou 
	seja,~${n =|V|}$. 
	Primeiramente escolhe-se um vértice arbitrário~$v \in V$ e 
	encontra-se um vértice~$y_0$ mais distante de~$v$.
	Depois, repetimos o mesmo processo a partir de~$y_0$, 
	encontrando um vértice~$x_0$ mais distante de~$y_0$. 
	Para encontrar um vértice mais distante de algum vértice dado, 
	basta usar uma busca em largura.  
	Feito isso, temos um caminho máximo em~$T$: o~$x_0,y_0\caminho$.

	\bigskip

	\subsection{Prova de correção do algoritmo}

	\begin{lem}
	\label{lema:caminhoMax}
		O~$x_0,y_0\caminho$ é um caminho máximo em~$T$.
	\end{lem}

	\bigskip

	\begin{proof}
		Suponhamos que exista um outro caminho mais longo que 
		o~$x_0,y_0\caminho$ e sejam~$x$ e~$y$ os seus extremos.

		\begin{itemize}
	        \item \textbf{Caso 1:} O~$x,y\caminho$ e 
	        o~$x_0,y_0\caminho$ não possuem vértices em comum.

	        Existe um~$a,b\caminho$, de 
	        comprimento~$c \ge 1$, que conecta os dois caminhos, 
	        com apenas o vértice~$a$ 
	        no ~$x,y\caminho$ e apenas o vértice~$b$ 
	        no~$x_0,y_0\caminho$.

	        \begin{center} \begin{tikzpicture}
				[scale=.8,auto=left,every node/.style={circle, 
				draw=black,
				fill=blue!20}]
				\node (x) at (0,5)  {$x$};
				\node (y) at (12,5) {$y$};
				\node (a) at (6,4)  {$a$};
				  
				\node (x0) at (0,0) {$x_0$};
				\node (y0) at (12,0) {$y_0$};
				\node (b)  at (6,1)  {$b$};

				\node (n1) at (3,4.5) {...}; %entre x e a 
				\node (n2) at (9,4.5) {...}; %entre a e y
				\node (n3) at (3,0.5) {...}; %entre x0 e b
				\node (n4) at (9,0.5) {...}; %entre y0 e b
				\node (n5) at (6,2.5) {...}; %entre a e b

				\foreach \from/\to in {x/n1,n1/a,a/n2,n2/y,x0/n3,
				n3/b,b/n4,n4/y0,a/n5,b/n5}
				\draw (\from) -- (\to);
			\end{tikzpicture} \end{center}


	        Como estamos em uma árvore,
	        que~${\dist(x,y) =\dist(x,a)+\dist(a,y)}$
	        e~${\dist(x_0,y_0) =\dist(x_0,b)+\dist(b,y_0)}$.

	        Como o~$x,y\caminho$ é 
	        máximo,~${\dist(x,y)\ge \dist(x_0,y)}$,
	        e isso implica que~${\dist(x,a)\ge \dist(x_0,b)+c}$.
 			
 			Temos que~${\dist(x_0,y_0)\ge \dist(x,y_0)}$,
	        pois~$x_0$ é um vértice mais distante de~$y_0$ e isso implica 
	        que~${\dist(x_0,b)\ge \dist(x,a)+c}$.
	        Logo temos 
	        que~${\dist(x,a)\ge \dist(x, a)+2c}$,
	        uma contradição, visto que~${c\ge 1}$.


			\bigskip
			\bigskip
			\bigskip


			\item \textbf{Caso 2:} O~$x,y\caminho$ e 
			o~$x_0,y_0\caminho$ possuem vértices em comum.

			A interseção entre esses dois caminhos é 
			um~$a,b\caminho$ de comprimento~$c \ge 0$.
			Podemos assumir, sem perda de generalidade, 
			que~${\dist(x,a) \le \dist(x,b)}$ e 
			que~${\dist(x_0,a) \le \dist(x_0,b)}$, como na figura 
			abaixo.
			(Do contrário troque~$a$ por~$b$ e possivelmente~$x$
			por~$y$.)

			\begin{center} \begin{tikzpicture}
				[scale=.8,auto=left,every node/.style={circle, 
				draw=black,
				fill=blue!20}]
				\node (x) at (0,4)  {$x$};
				\node (y) at (12,4) {$y$};
				\node (a) at (4,4)  {$a$};
				\node (b) at (8,4)  {$b$};
				\node (x0) at (1.5,6)  {$x_0$};
				\node (y0) at (10.5,2)  {$y_0$};
				  
				\node (n1) at (2,4) {...};
				\node (n2) at (6,4) {...};
				\node (n3) at (10,4) {...};
				\node (n4) at (9.25,3) {...};
				\node (n5) at (2.75,5) {...};

				\foreach \from/\to in {x/n1,n1/a,a/n2,n2/b,b/n3,
				n3/y, a/n5,n5/x0,b/n4,n4/y0}
				\draw (\from) -- (\to);
			\end{tikzpicture} \end{center}


			Temos que~${\dist(x_0,a)\ge \dist(x,a)}$,
			pois~$x_0$ é um vértice mais distante de~$y_0$.
			Por outro lado, como~$x$ é um vértice mais distante
			de~$y$, temos também que~${\dist(x,a)\ge \dist(x_0,a)}$,
			e, portanto,~${\dist(x_0,a) =\dist(x,a)}$.
			Similarmente, como~$y$ é um vértice mais distante 
			de~$x$, vale que~${\dist(y,b) \ge \dist(y_0,b)}$.

			Agora consideremos o vértice~$v$ que foi escolhido 
			arbitrariamente no início do algoritmo, e seja~$v'$
			o vértice mais próximo de~$v$ no~$x_0,y_0\caminho$.
			Se~$v'\ne b$, então, como~$y_0$ é um vértice mais 
			distante de~$v$,
			\begin{align}
				\dist(v',y_0) &\ge \dist(v',y) =\dist(v',b) + 
				\dist(b,y)\nonumber \\
				&\ge \dist(v',b) + \dist(b,y_0) \ge \dist(v',y_0) 
				\nonumber,
			\end{align} 
			o que implica que~${\dist(v',y_0)~=\dist(v',y)}$ 
			e~${\dist(b,y)~=\dist(b,y_0)}$. 
			Ou seja,~${\dist(x,y) =\dist(x_0,y_0)}$.
			Resta analisar o caso em que~$v'=b$ (é um caso 
			especial, dado que pode-se ter arestas em comum 
			entre o~$v,v'\caminho$ e o~$y,b\caminho$).
			Como~$y$ é um vértice mais distante de~$x$, temos 
			que~$\dist(v',y_0) \le \dist(v',y)$.
			Como~$x_0$ é um vértice mais distante de~$y_0$, temos
			que~$\dist(x_0,v') \ge \dist(y,v')$.
			Finalmente, como~$y_0$ é um vértice mais distante 
			de~$v$, vale que~$\dist(y_0,v')\ge \dist(x_0,v')$.
			Dessas três desigualdades, concluímos 
			que~${\dist(y_0,v')=\dist(y,v')=\dist(x_0,v')=\dist(x,v')}$,
			onde a última igualdade vale pois já mostramos 
			que~${\dist(x_0,a) =\dist(x,a)}$.

			Assim sendo,~${\dist(x_0,y_0) =\dist(x,y)}$, completando a 
			prova.
		\end{itemize}
	\end{proof}

	\bigskip
	\bigskip


%%%%%%%%%%%%%%%%%%%%%%%%%%%%%%%%%%%%%%%%%%%%%%%%%%%%%%%%%%%%%%%%%%%
	\subsection{Diâmetro de um grafo}
	\label{subsec:diametro}
	O \textbf{diâmetro} de um grafo conexo~$G$ é o
	comprimento de um caminho máximo em~$G$. 
	Em outras palavras, o diâmetro pode ser definido como:
	$$ \diam(G)=\max\{\dist(x,y):x,y\in V(G)\}.$$

	\bigskip

	Seja~$G$ um grafo não necessariamente conexo,
	o \textbf{diâmetro relativo} de~$G$
	 é a razão entre o número
	de vértices de um caminho máximo e o número total de vértices
	de~$G$. Este pode também ser representado por
	$$ \diam^*(G) =\dfrac{\displaystyle\sum_{
	G'~componente~conexa~de~G}^{}(\diam(G')+1)}{|V(G)|}.$$

	\medskip

	Nota-se que para todo grafo~$G$ com~${V(G)\ne \emptyset}$, temos
	que~${0<\diam^*(G)\le 1}$, pois haverá no mínimo um vértice 
	de~$G$ no caminho máximo, e no máximo, todos os vértices 
	de~$G$.
	Também representaremos o diâmetro relativo como~$d$.
    

%%%%%%%%%%%%%%%%%%%%%%%%%%%%%%%%%%%%%%%%%%%%%%%%%%%%%%%%%%%%%%%%%%%
%%%%%%%%%%%%%%%%%%%%%%%  SEÇÃO 6     %%%%%%%%%%%%%%%%%%%%%%%%%%%%%%
%%%%%%%%%%%%%%%%%%%%%%%%%%%%%%%%%%%%%%%%%%%%%%%%%%%%%%%%%%%%%%%%%%%

\section {Lemas de cortes aproximados}

Em \cite{Schmidt15}, são apresentados 
algoritmos, que, dados uma floresta~$G$ e um 
inteiro~$0<m\le n$, onde~$n$ é o número de vértices de~$G$, 
produzem cortes com algumas propriedades.
Os algoritmos em questão são usados como subrotinas no algoritmo
FST.

Abaixo reproduzimos os três resultados de \cite{Schmidt15} que 
atestam as propriedades dos cortes produzidos por estes algoritmos.
O primeiro deles é o mais simples e o algoritmo vinculado a ele
é usado no segundo.

\bigskip

\begin{lem}[]
\label{lema:simpleApproxCutTree}
	Para toda árvore~$T$ com~$n$ vértices e todo~$m \in [n]$,
	existe um corte~$(B,W)$ em~$T$ tal 
	que~${\dfrac{m}{2} <|B| \le m}$ e~${e_T(B,W) \le \Delta(T)}$.
	Um corte que satisfaz esses requisitos pode ser computado em
	tempo~$O(n)$.
\end{lem}

\bigskip

Seja~$T$ uma árvore com~$n$ vértices e~$r$ um vértice de~$T$. 
Assumiremos que
a função {\sc número\_de\_descedentes}$(T$, $r)$ devolve um vetor
que associa cada vértice de~$T$ ao número de descendentes desse 
vértice na árvore~$T$, enraizada em~$r$. Note que essa função 
pode ser implementada usando-se uma busca em profundidade,
resultando em um consumo de tempo~$O(n)$.


Segue um algoritmo que encontra um corte com as propriedades 
descritas no lema.

\medskip

\begin{algorithm}[H]
\label{alg:simpleApproxCutTree}
	\SetKwInOut{Input}{entrada}
	\SetKwInOut{Output}{saída}

	\caption{Computa corte aproximado em uma árvore}
	\Input{árvore~${T =(V,E)}$ com~$n$ vértices, ~$m \in [n]$ 
	e~$r\in V$}
	\Output{corte~$(B,W)$ em~$T$ com~$\dfrac{m}{2}<|B|\le m$ 
	e~$e_T(B,W)\le \Delta(T)$}
	$B \gets \emptyset$\;
	\lIf{$m =n$}{
		\Return $(V,\ \emptyset)$
	}
		$d \gets~$ {\sc número\_de\_descendentes}$(T$, $r)$\;
		$v \gets r$\;
		\lWhile{existe filho~$u$ de~$v$ com~$d[u] > m$ }{
			$v \gets u$
		}
		\eIf{existe filho~$u$ de~$v$ com~$d[u]>\dfrac{m}{2}$}
		{
			$B\gets V(T_u)$; \quad
			\tcp{$T_u$ é a sub-árvore de~$T$ enraizada em~$u$}
		}
		{
		\For{cada filho~$u$ de~$v$}
		{
			\eIf{$|B|+|T_u| \le m$}{
				$B\gets B\cup V(T_u)$\;
			}
			{
				break\;
			}
		}
		}
	\Return $(B,V\setminus B)$

\end{algorithm}	

\bigskip

%\subsection*{Análise do Algoritmo}

	Suponha que~$T=(V,E)$ e $n=|V|$.
	Se o algoritmo termina na linha 2, então o corte 
	devolvido satisfaz 
	%É fácil ver que se~$m=n$, então os valores~$B =V$ 
	%e~$W =\emptyset$ satisfazem 
	as condições do lema, dado 
	que~$e_G(V,\ \emptyset) = 0$,
	e o consumo de tempo é claramente~$O(n)$. 
	%Isso pode ser computado em tempo~$O(n)$.

	Caso contrário, o algoritmo escolhe uma raiz arbitrária~$r$ 
	para~$T$ e calcula o número de vértices das sub-árvores 
	enraizadas em cada um dos vértices, que é o número de 
	descendentes do vértice.
	Isso serve para que saibamos a quantidade de vértices das 
	sub-árvores sem que precisemos percorrer todos os vértices das 
	sub-árvores a cada consulta.

	Feito isso, seja~$v$ o vértice analisado no momento 
	e~${V_1, V_2, \ldots, V_k}$ os conjuntos de vértices das
	sub-árvores enraizadas 
	nos~$k$ filhos de~$v$.
	Se algum~$V_i$ satisfizer~${\dfrac{m}{2}<|V_i|\le m}$, podemos 
	devolver a sub-árvore~$V_i$ como o conjunto~$B$, 
	satisfazendo o Lema~\ref{lema:simpleApproxCutTree}, dado que o
	corte determinado por~$V_i$ contém uma única aresta. 
	Caso contrário, se~${|V_i|>m}$ para algum~$i$, 
	e~$v$ passa a ser a raiz de~$V_i$.
	Aplicamos novamente esse processo em~$v$.

	Nota-se que esse procedimento irá parar em algum momento, dado 
	que a árvore é finita, e ele parará quando encontrar uma 
	sub-árvore que satisfaça o Lema~\ref{lema:simpleApproxCutTree}
	(e essa sub-árvore definirá o conjunto~$B$) 
	ou quando chegarmos numa situação em 
	que~${|V_i|\le \dfrac{m}{2}}$ para todos os~$k$ filhos de~$v$.
	Nesse último caso, sabemos 
	que~${|V_1\cup V_2\cup \cdots \cup V_k|\ge m}$, pois
	a sub-árvore enraizada em~$v$ possui mais que~$m$ vértices. 
	Sabemos também que~${|V_i|\le \dfrac{m}{2}}$ para todos os~$k$ 
	filhos de~$v$. 
	Com isso, percorremos os~$V_i$ em ordem, e 
	adicionamos~$V_i$ ao conjunto~$B$, parando antes 
	que~${|B| >m}$. 
	O conjunto~$B$ resultante é tal 
	que~${e_T(B,V\setminus B)<\grau_T(v)\le \Delta(T)}$, 
	o que satisfaz o lema.
	Dado que~${|V_j|\le \dfrac{m}{2}}$ para todo~$j$, sabemos que 
	existe um~$i$ tal 
	que~${\dfrac{m}{2} <|V_1\cup V_2 \cup \cdots \cup V_i|\le m}$, 
	pois a união das sub-árvores enraizadas em~$v$ possui pelo 
	menos~$m$ vértices e, para todo~$\ell$ que 
	satisfaça~${|V_1\cup V_2\cup \cdots\cup V_\ell|\le 
	\dfrac{m}{2}}$, temos 
	que~${|V_1\cup V_2\cup\cdots\cup V_\ell\cup V_{\ell+1}|\le m}$.


\bigskip
\bigskip
\bigskip

%%%%%%%%%%%%%%%%%%%%%%%%%%%%%%%%%%%%%%%%%%%%%%%%%%%%%%%%%%%%%%%%%%%
%%%%%%%%%%%%%%%%% LEMA 3 %%%%%%%%%%%%%%%%%%%%%%%%%%%%%%%%%%%%%%%%%%

%\subsection{Lema do corte aproximado simples para florestas}

O lema abaixo representa uma adaptação do algoritmo anterior
para florestas em vez de árvores.
Note que a adaptação não recebe uma raiz como parâmetro, mas 
apenas~$G$ e~$m$.

\begin{lem}[{\cite[Lema 2]{Schmidt15}}]
\label{lema:simpleApproxCutForest}
	Para toda floresta~$G$ com~$n$ vértices e todo~${m \in [n]}$,
	existe um corte~$(B,W)$ em~$G$ tal 
	que~${\dfrac{m}{2} <|B| \le m}$ e~${e_G(B,W) \le \Delta(G)}$.
	Um corte que satisfaz esses requisitos pode ser computado em
	tempo~$O(n)$.
\end{lem}

\bigskip

Agora examinaremos como determinar um corte descrito pelo lema.
Nota-se que essa é uma adaptação do algoritmo anterior
para florestas em vez de árvores.

\medskip
\medskip

\begin{algorithm}[H]
\label{alg:simpleApproxCutForest}
	\SetKwInOut{Input}{entrada}
	\SetKwInOut{Output}{saída}

	\caption{Computa corte aproximado simples em uma floresta}
	\Input{floresta~${G =(V,E)}$ com~$n$ vértices e~$m \in [n]$}
	\Output{corte~$(B,W)$ tal que~${\dfrac{m}{2}\le|B|\le m}$}
	Sejam~$V_1, V_2,\ldots, V_k$ os conjuntos de vértices das
	componentes conexas de~$G$\;
	$B \gets \emptyset$\;
	\For{${i=1 \to k}$}{
		\If{$|B| + |V_i|\le m$}{
			$B \gets B\cup V_i$\;
		}	
		\ElseIf{$|B|<m$}{
			$(B',W')\gets \mathbf{Algoritmo\ref{alg:simpleApproxCutTree}}(
			G[V_i], |V_i|, m-|B|)$\;

			$B \gets B\cup B'$\;

			break\;
		}	
	}
	\Return $(B,V\setminus B)$\;

\end{algorithm}	

\bigskip

\subsection*{Análise do Algoritmo}

	Sejam~${T_1, T_2, \ldots,T_k}$ as árvores que compõe~$G$. 
	Primeiro o algoritmo percorre as árvores determinando
	o maior inteiro~$\ell$ tal 
	que~${s=\displaystyle\sum_{i=1}^{\ell}|V(T_i)| \le m}$.
	Colocamos~${V(T_1),V(T_2), \ldots,V(T_\ell)}$ no conjunto~$B$.
	Caso~${|B|=m}$, temos que~${e_G(B,V\setminus B)=0}$, o que 
	satisfaz o lema.
	Caso contrário, ao encontrar um corte~$(B',W')$ em~$T_{\ell+1}$
	que satisfaça~${\dfrac{m-s}{2}<|B'|\le m-s}$ 
	com~${e_{T_{\ell+1}}(B',W') \le \Delta(T_{\ell+1})}$, e 
	acrescentar~$B'$ ao conjunto~$B$, 
	teremos~${\dfrac{m+s}{2}<|B| \le m}$ 
	com~${e_G(B,W)\le\Delta(T_{\ell+1}) \le \Delta(G)}$.
	Logo, depois de acrescentar~$B'$ em~$B$, teremos o 
	corte~${(B,V\setminus B)}$ que satisfaz o lema.
	O corte~$(B,W)$ é encontrado por meio do 
	Algoritmo~\ref{alg:simpleApproxCutTree}.

	
	Nota-se que, como a linha~2 pode ser feita usando uma busca em 
	profundidade e as demais linhas envolvem verificar o tamanho 
	das componentes (que já foi calculado na linha~2) e uma 
	chamada única ao Algoritmo~\ref{alg:simpleApproxCutTree}, então 
	essa rotina consome tempo~$O(n)$, dado que cada uma 
	das operações citadas são executadas em tempo~$O(n)$. 

\bigskip
\bigskip
\bigskip


%%%%%%%%%%%%%%%%%%%%%%%%%%%%%%%%%%%%%%%%%%%%%%%%%%%%%%%%%%%%%%%%%%%
%%%%%%%%%%%%%%%%% LEMA 4 %%%%%%%%%%%%%%%%%%%%%%%%%%%%%%%%%%%%%%%%%%

%\subsection{Lema do corte aproximado}
Finalmente apresentamos o algoritmo que é usado repetidas 
vezes no algoritmo FST.
Ele utiliza o algoritmo anterior como subrotina.

\begin{lem}[{\cite[Lema 3]{Schmidt15}}]
\label{lema:approxCutForest}
	Para toda floresta~$G$ com~$n$ vértices, todo~${m \in [n]}$ e 
	todo~${c \in [0,1)}$, existe um corte~$(B,W)$ em~$G$ tal 
	que~${cm \le |B| \le m}$ 
	e~${e_G(B,W) \le \ceil[\Big]{ \dfrac{2c}{1-c}} \Delta(G)}$.
	Um corte que satisfaz esses requisitos pode ser computado em
	tempo~${O\Big(\ceil[\Big]{ \dfrac{2c}{1-c}} n+1\Big)}$.
\end{lem}


Segue o algoritmo que encontra um corte~$(B,W)$ seguindo
as propriedades do Lema~\ref{lema:approxCutForest}.

\begin{algorithm}[H]
\label{alg:approxCutForest}
	\SetKwInOut{Input}{entrada}
	\SetKwInOut{Output}{saída}

	\caption{Computa corte aproximado em uma floresta}
	\Input{floresta~${G =(V,E)}$ com~$n$ vértices,~${m \in [n]}$ e
	${c\in [0, 1)}$}
	\Output{corte~${(B,W)}$ tal que~${cm\le|B|\le m}$
	e~$e_G(B,W) \le \ceil[\Big]{ \dfrac{2c}{1-c}} \Delta(G)$}
	\lIf{$c=0$}{
		$\Return\ (\emptyset, V)$}
	\lIf{${c\le \dfrac{1}{2}}$}{
		$\Return\ \mathbf{Algoritmo\ref{alg:simpleApproxCutForest}}(
		G, m)$}
		$B \gets \emptyset$\;
		\While{$|B|<cm$}{
			$(B',W') \gets 
			\mathbf{Algoritmo\ref{alg:simpleApproxCutForest}}(
			G[V\setminus B], m-|B|)$ \;

			$B \gets B\cup B'$\;		
	}
	\Return $(B,V\setminus B)$\;

\end{algorithm}	
%\subsection*{Análise do Algoritmo}

	Se~$c=0$, o Algoritmo~\ref{alg:approxCutForest} devolverá
	um corte onde~$B=\emptyset$ e~$e_G(B,W) = 0$. 
	Isso irá satisfazer as 
	propriedades do Lema~\ref{lema:approxCutForest}, dado 
	que~$cm=0=|B|$.

	Caso contrário, se~$c \le \dfrac{1}{2}$, o algoritmo 
	devolve o corte produzido pelo 
	Algoritmo~\ref{alg:simpleApproxCutForest}, ou seja,
	um corte~$(B',W')$ tal 
	que~$|B'|>\dfrac{m}{2}\ge cm$ e~${e_G(B',W')\le \Delta(G)}$.

	Por outro lado, se~$c>\dfrac{1}{2}$, 
	o conjunto~$B$ é inicializado com $\emptyset$.
	O Algoritmo~\ref{alg:simpleApproxCutForest} é
	executado diversas vezes para encontrar a cada vez
	um corte~$(B',W')$ em~${G[V\setminus B]}$ 
	tal que~${\dfrac{m-|B|}{2}<|B'|\le m-|B|}$ 
	e~${e_{G[V\setminus B]}(B',W')\le\Delta(G[V\setminus B])}$, 
	acrescentando o 
	conjunto~$B'$ a~$B$ em cada uma das chamadas ao 
	Algoritmo~\ref{alg:simpleApproxCutForest}.
	Isso é feito até que~${|B|\ge cm}$.
	Note que em algum momento~$|B|\ge cm$, pois a 
	cada vez que executamos o 
	Algoritmo~\ref{alg:simpleApproxCutForest}, acrescentamos pelo 
	menos um vértice em~$B$, 
	e~${|B|\le m}$ durante todo o processo, já que o
	conjunto~$B'$ obtido 
	pelo Algoritmo~\ref{alg:simpleApproxCutForest} 
	é tal que~$|B'|\le m-|B|$. Logo, serão adicionados no 
	máximo~${m-|B|}$ vértices em~$B$.
	
	\bigskip
	
	Verificaremos agora se o consumo de tempo e a largura do corte
	devolvido pelo Algoritmo~\ref{alg:approxCutForest} satisfazem
	o que foi proposto no Lema~\ref{lema:approxCutForest}.
	
	Caso~${c=0}$, a largura do corte devolvido será zero,
	dado que todos os vértices de~$G$ estarão em~$W$.
	Portanto~${e_G(B,W)=0=\ceil[\Big]{\dfrac{2c}{1-c}} \Delta(G)}$.
	Já em relação ao tempo, neste caso o Algoritmo~\ref{alg:approxCutForest}
	executa apenas operações que levam tempo constante, logo ele
	leva tempo~$O(1)$. 
	Como~$\ceil[\Big]{\dfrac{2c}{1-c}} n+1 = 1$,
	as propriedades do Lema~\ref{lema:approxCutForest} são
	satisfeitas.

	Caso~$0<c\le \dfrac{1}{2}$, executamos o
	Algoritmo~\ref{alg:simpleApproxCutForest} uma única vez, 
	portanto o tempo é~$O(n)$, que equivale 
	a~${O\Big(\ceil[\Big]{\dfrac{2c}{1-c}} n+1\Big)}$.
	Em relação à largura do corte, será devolvido o corte~$(B,W)$ do 
	Algoritmo~\ref{alg:simpleApproxCutForest} sem nenhuma 
	modificação, então, sabe-se que~${e_G(B,W)\le \Delta(G)}$.
	Como~${0<c\le\dfrac{1}{2}}$, temos que~${\dfrac{2c}{1-c}>0}$, 
	que implica que~${\ceil[\Big]{ \dfrac{2c}{1-c}}\ge 1}$.
	Logo,~${e_G(B,W)\le\Delta(G)\le \ceil[\Big]{\dfrac{2c}{1-c}}
	\Delta(G)}$, 
	satisfazendo assim as propriedades do 
	Lema~\ref{lema:approxCutForest}.

	Caso contrário,~${c>\dfrac{1}{2}}$. Como em todas as vezes que
	executamos a linha~4 do Algoritmo~\ref{alg:approxCutForest}, 
	vale que~${|B|<cm}$, então também é válido que~${m-|B|>(1-c)m}$.
	Como~${m-|B|}$ é o valor do~$m$ que passamos para o 
	Algoritmo~\ref{alg:simpleApproxCutForest}, sabemos que o 
	conjunto $B'$ do corte devolvido terá mais 
	que~$\dfrac{m-|B|}{2}$ vértices.
	Portanto, executaremos o 
	Algoritmo~\ref{alg:simpleApproxCutForest} no 
	máximo~${\ceil[\bigg]{\dfrac{cm}{\frac{(1-c)m}{2}}} 
	=\ceil[\Big]{\dfrac{2c}{1-c}}}$ vezes, consumindo, no total,
	tempo~$O\Big(\ceil[\Big]{ \dfrac{2c}{1-c}} n\Big) = O\Big(\ceil[\Big]{ \dfrac{2c}{1-c}} n+1\Big)$.
	Usaremos essa mesma linha de pensamento para calcular a largura
	máxima do corte devolvido na linha~8. 
	Sabemos que o 
	Algoritmo~\ref{alg:simpleApproxCutForest} será 
	chamado no máximo~$\ceil[\Big]{\dfrac{2c}{1-c}}$ 
	vezes e, para cada uma 
	das vezes, este devolve um corte cuja largura 
	máxima é~$\Delta(G)$, portanto a largura máxima do corte 
	devolvido é~$\ceil[\Big]{\dfrac{2c}{1-c}}\Delta(G)$.



%%%%%%%%%%%%%%%%%%%%%%%%%%%%%%%%%%%%%%%%%%%%%%%%%%%%%%%%%%%%%%%%%%%
%%%%%%%%%%%%%%%%%%%%%%%  SEÇÃO 7     %%%%%%%%%%%%%%%%%%%%%%%%%%%%%%
%%%%%%%%%%%%%%%%%%%%%%%%%%%%%%%%%%%%%%%%%%%%%%%%%%%%%%%%%%%%%%%%%%%

\section {Rotulação dos vértices da árvore}
\label{sec:rotulacao}

	% Mais adiante, iremos nos referir a vértices com uma 
	% característica específica, e para isso precisamos atribuir 
	o algoritmo FST utiliza um 
	rótulo numérico~$x\in [n]$ distinto para cada um dos vértices 
	da árvore, onde~$n$ é o número de vértices da árvore~$T$. 
	Esses rótulos devem ter uma característica específica e por isso
	são calculados de uma maneira particular.
	Denotaremos o \textbf{rótulo} de um vértice~$v$ por~$\rot(v)$.

	Primeiramente, usaremos o método mencionado na 
	Seção~\ref{sec:caminhoMaximo} para calcular 
	um caminho~$P$, que é máximo em~$T$. 
	Para todo vértice~$v\in V(P)$, 
	% temos uma 
	% árvore~$T_v \in G - E(P)$ com~$v\in T_v$.
	seja~$T_v$ a árvore de~$T-E(P)$ que contém~$v$.

	Dessa forma, teremos sub-árvores enraizadas nos vértices do
	caminho máximo, como mostrado na figura abaixo, onde denotamos
	os vértices de~$P$ por~${v_0,v_1,\ldots,v_p}$.
	

 	%% fill=blue!30 -- núemro indica quão escuro é
	\begin{center} 
	\begin{figure}[h]
	\begin{tikzpicture}
		[scale=.7,auto=left,every node/.style={circle, draw=black,
				fill=blue!20}]
	\scalebox{0.95}{
		\draw [draw=yellow, fill=yellow!35, line width=2pt] (0,4.8) 
		ellipse (1.3cm and 2.4cm); %0
		\draw [draw=yellow, fill=yellow!35, line width=2pt] (3,4.4) 
		ellipse (1.3cm and 2.4cm); %1
		\draw [draw=yellow, fill=yellow!35, line width=2pt] (6,4.8) 
		ellipse (1.3cm and 2.4cm); %2
		\draw [draw=yellow, fill=yellow!35, line width=2pt] (9,4.4) 
		ellipse (1.3cm and 2.4cm); %3
 		\draw [draw=yellow, fill=yellow!35, line width=2pt] (12,4.8) 
 		ellipse (1.3cm and 2.4cm); %4
		\draw [draw=yellow, fill=yellow!35, line width=2pt] (18,4.8) 
		ellipse (1.3cm and 2.4cm); %5

		\node [text=blue!65,draw=none, fill=none] at 
		(-0.4,6.5) {$T_{v_0}$};
		\node [text=blue!65,draw=none, fill=none] at 
		(2.6,6.1) {$T_{v_1}$};
		\node [text=blue!65,draw=none, fill=none] at 
		(5.6,6.5) {$T_{v_2}$};
		\node [text=blue!65,draw=none, fill=none] at 
		(8.6,6.1) {$T_{v_3}$};
		\node [text=blue!65,draw=none, fill=none] at 
		(11.6,6.5) {$T_{v_4}$};
		\node [text=blue!65,draw=none, fill=none] at 
		(17.6,6.5) {$T_{v_p}$};

		\node [subtree] (y1) at (3,4.6)  {};
		\node [subtree, fill=red!30] (y2) at (6,5)  {};
		\node [subtree] (y3) at (9,4.6)  {};
		\node [subtree] (y4) at (12,5) {};

		\node (x0) at (0,5.4)  {$v_0$};
		\node (x1) at (3,5)    {$v_1$};
		\node (x2) at (6,5.4)  {$v_2$};
		\node (x3) at (9,5)    {$v_3$};
		\node (x4) at (12,5.4) {$v_4$};
		\node (x5) at (15,5)   {...};
		\node (x6) at (18,5.4) {$v_p$};


		\foreach \from/\to in {x0/x1,x1/x2,x2/x3,x3/x4,x4/x5,x5/x6}
		\draw (\from) -- (\to);
	}
	\end{tikzpicture} 
	\caption{Sub-árvores enraizadas no caminho máximo}
	\end{figure}
	\end{center}
	\bigskip
	Note que as árvores~$T_{v_0}$ e~$T_{v_p}$ possuem um único
	vértice cada uma. 
	Isso ocorre porque~$v_0$ e~$v_p$ são extremos
	do caminho máximo~$P$, e se houvesse algum vértice em uma
	dessas sub-árvores,
	%~$v_\ell$ faria parte do caminho máximo
	%e~$v_0$ ou~$v_p$ não seria um dos extremos, 
	o caminho~$P$ poderia ser estendido,
	o que levaria a uma contradição.

	\bigskip

	Cada vértice do caminho~$P$ recebe 
	um rótulo de forma
	%\begin{itemize}
		%\item~$\rot(v_i)>\rot(v_j)$ para todo~$i<j$ e
		que~${\rot(v_{i-1})<\rot(v') \le \rot(v_i)}$ para todo 
		vértice~$v'$ da sub-árvore~$T_{v_i}$ para~$i=1,2,\ldots,p$. 
	%\end{itemize}

	\bigskip
	\bigskip

	\subsection{Descrição do algoritmo de rotulação}
	Essa rotulação pode ser obtida facilmente 
	manipulando as listas de adjacências de ~$T$ e em seguida,
	executando uma busca em profundidade.
	Ambas as ações consomem tempo linear em~$n$, então
	pode-se rotular uma árvore com~$n$ vértices em tempo~$O(n)$.

	Para cada vértice~$v_i$ contido no caminho~$P$, com~$i>0$,
	percorremos a sua lista de adjacências e, quando encontrarmos o 
	vértice~$v_{i-1}$, o colocamos no início da lista.
	Dessa forma, ao aplicarmos uma busca em profundidade nessa 
	árvore usando o vértice~$v_p$ como raiz, a busca descerá primeiro
	pelos vértices do caminho máximo.
	Os rótulos são atribuídos em pós-ordem, ou seja, a busca
	atribuirá um rótulo
	a um vértice do caminho
	somente quando terminar de rotular todos os vértices da sub-árvore
	do vértice, 
	garantindo assim que para todo~$i\in \{1,\ldots, p\}$, temos 
	que~$\rot(v_{i-1})<\rot(j)\le\rot(v_i)$ para todo~$j$ 
	em~$V(T_{v_i})$.
	
	\bigskip
	\bigskip
	\bigskip

	\subsection{Mapeamento dos rótulos de vértices }
	\label{sec:map}
	Vamos definir agora as funções
	utilizadas pelo algoritmo FST
	que mapeiam um vértice em 
	outro, utilizando a rotulação acima como base.
	Chamaremos de~${\map^+_m}$ a função que devolve o~$m$-ésimo 
	próximo vértice na rotulação, considerando que os rótulos
	estão dispostos de forma circular
	e, de~${\map^-_m}$, a função inversa da primeira. 
	Ou seja,~${\map^-_m}$ devolve o~$m$-ésimo vértice anterior
	na rotulação.

	%Dessa forma, podemos ver que, para um 
	%vértice~$v$,~${\map^+_m(\rot(v)) =(\rot(v)+m)\mod n}$ 
	%e~${\map^-_m(\rot(v)) =(\rot(v)-m)\mod n}$.

	Para facilitar, identificaremos os vértices pelos seus 
	rótulos. 
	%Sendo assim, temos que, para um 
	%vértice~$v$, ~${\map^+_m(v) =(v+m)\mod n}$ 
	%e~${\map^-_m(v) =(v-m)\mod n}$ como uma simplificação das 
	%funções enunciadas anteriormente.

	Note que, para todo ${m\in [n]}$, caso~${n>2}$, se existem
	dois vértices~$v$ e~$v'$, ambos no caminho~$P$, 
	tais que~${v'= \map^+_m(v)}$, então o corte $(B,W)$ em~$T$ 
	com~${B =\{v+1, v+2,\ldots,~v+m\}}$ é tal que~${|B|=m}$
	e~${e_T(B,W)\le 2\le \Delta(T)}$.
	Isso pode ser visto na figura abaixo.
	\begin{center} \begin{figure}[h] \begin{tikzpicture}
		[scale=.7,auto=left,every node/.style={circle, draw=black,
				fill=blue!20}]
	\scalebox{1.02}{
		\draw [draw=yellow, fill=yellow!35, line width=2pt] (7.5,1.5) rectangle 
		(13.5, 6.5);

		\node [text=blue!65,draw=none, fill=none] at 
		(10.5,2.2) {\Large{$m$ vértices}};
		\node [draw=none, fill=none] at 
		(9,6.5) {\footnotesize{$\map^+_m$}};
		\node [draw=none, fill=none] at 
		(9,7.9) {\footnotesize{$\map^-_m$}};


		\node [subtree] (y1) at (3,4.6)  {};
		\node [subtree, fill=red!30] (y2) at (6,5)  {};
		\node [subtree] (y3) at (9,4.6)  {};
		\node [subtree] (y4) at (12,5) {};

		\node (x0) at (0,5.4)  {$v_0$};
		\node (x1) at (3,5)    {...};
		\node (x2) at (6,5.4)  {$v$};
		\node (x3) at (9,5)    {...};
		\node (x4) at (12,5.4) {$v'$};
		\node (x5) at (15,5)   {...};
		\node (x6) at (18,5.4) {$v_p$};


		\foreach \from/\to in {x0/x1,x1/x2,x3/x4,x5/x6}
		\draw (\from) -- (\to);

		\path[-, line width=1.5pt, draw=red]
		(x2) edge[right] (x3);
		\path[-, line width=1.5pt, draw=red]
		(x4) edge[right] (x5);

		\path[->, line width=1.5pt]
		(x2) edge[bend left=23 ] (x4);
		\path[->, line width=1.5pt]
		(x4) edge[bend right=45, looseness=1.7 ] (x2);
	}
	\end{tikzpicture}
	\caption{Mapeamento entre os vértices~$v$ e~$v'$ do caminho máximo
	induz um corte~$(B,W)$ com~$|B| = m$ e~$e_T(B,W) = 2$.} 
	\end{figure}
	\end{center}
	Caso~$n\le 2$, 
	é fácil ver que todos os vértices
	da árvore estarão no caminho máximo, portanto, para
	qualquer~$m\in[n]$, teremos um conjunto~$B$ tal
	que~$|B|=m$
	e~$e_T(B,W)\le |E(T)|=n-1=\Delta(T)\le 2$.

	%\subsubsection*{Algumas definições}


	\bigskip
	



%%%%%%%%%%%%%%%%%%%%%%%%%%%%%%%%%%%%%%%%%%%%%%%%%%%%%%%%%%%%%%%%%%%
%%%%%%%%%%%%%%%%%%%%%%%%%%%%%%%%%%%%%%%%%%%%%%%%%%%%%%%%%%%%%%%%%%%
%%%%%%%%%%%%%%%%%%%%%%%%%%%%%%%%%%%%%%%%%%%%%%%%%%%%%%%%%%%%%%%%%%%

	\subsection{Árvores de diâmetro grande}

	Mais adiante usaremos  	o lema abaixo na prova de 
	um dos teoremas.

	\begin{lem}[{\cite[Lema 5]{Schmidt15}}]
	\label{lema:caminhoLongo}
		Para toda árvore~$T$ com~$n$ 
		vértices e~${\diam^*(T)>\dfrac{1}{2}}$
		e para todo~${m\in[n]}$, existe um corte~$(B,W)$ em~$T$
		com~${|B|=m}$ e~${e_T(B,W)\le 2}$. Um corte que satisfaz esses
		requisitos pode ser computado em tempo~$O(n)$.
	\end{lem}

	\medskip
	\medskip

	\begin{proof}
	Considere um caminho máximo~$P$ em~$T$ e a rotulação dos vértices
	de~$T$ induzida por ele.
	Podemos ver que a função~$\map^+_m(x)$ é bijetora dado que, numa
	soma, elementos distintos do domínio tem imagens distintas e, 
	para cada elemento da imagem, temos um do domínio.

	Uma árvore com a propriedade ~${\diam^*(T)>\dfrac{1}{2}}$ é uma
	árvore onde cada caminho máximo contém mais que a metade dos 
	vértices da árvore, o que nos leva ao fato de que existem mais 
	vértices no caminho máximo~$P$ do que fora dele. 

	Como mostrado anteriormente, se um vértice~$v$ é mapeado em um
	vértice~$v'$ e ambos estão no caminho máximo~$P$, então existe um
	corte~$(B,W)$ em~$T$ com~${e_T(B,W)\le 2}$.
	Dado que~${\diam^*(T)>\dfrac{1}{2}}$, e usando o fato de 
	que~$\map^+_m(x)$ é uma função bijetora e que
	existem mais vértices no caminho máximo~$P$ do que fora dele,
	temos que pelo menos um vértice de~$P$ será mapeado em
	outro vértice de~$P$. 
	%Isso ocorre porque todos
	%os vértices mapeiam algum outro vértice, e não temos um 
	%número suficiente de vértices fora do caminho máximo para serem
	%mapeados por todos os vértices do caminho máximo, então algum
	%vértice do caminho máximo vai acabar mapeando outro do caminho
	%máximo, e
	Assim, obtemos um corte~$(B,W)$, com~$e_T(B,W)\le 2$,
	como descrito na subseção anterior.

	Um algoritmo que encontra um corte desse tipo, em árvores com 
	essa propriedade, percorre os vértices do caminho máximo~$P$
	e verifica se algum deles é mapeado em um vértice do caminho máximo.
	Isso pode ser feito em tempo~$O(n)$ se construirmos um vetor 
	binário que indica se cada um dos vértices de~$T$ está ou não no
	caminho máximo~$P$.
	\end{proof}
	%{\color{red}{-O algoritmo é bem simples. Não precisa colocar
	%código né?}}


%%%%%%%%%%%%%%%%%%%%%%%%%%%%%%%%%%%%%%%%%%%%%%%%%%%%%%%%%%%%%%%%%%%
%%%%%%%%%%%%%%%%%%%%%%%  SEÇÃO 8     %%%%%%%%%%%%%%%%%%%%%%%%%%%%%%
%%%%%%%%%%%%%%%%%%%%%%%%%%%%%%%%%%%%%%%%%%%%%%%%%%%%%%%%%%%%%%%%%%%

\section {Algoritmo da dobra do diâmetro}
\label{sec:dobraDiametro}
	\subsection{Teorema da dobra do diâmetro}
	
		\begin{teo}[{\cite[Teorema 4]{Schmidt15}}]
		\label{teo:dobraDiametro}
			Para toda árvore~$T$ com~$n\ge 3$ vértices e 
			todo~$m\in [n]$,
			o conjunto de vértices de~$T$ pode ser particionado em 
			três partes~$B$,~$W$ e~$S$ tal que vale um dos 
			seguintes itens:
			\begin{enumerate}
				\item ${|B|=m}$, ${S=\emptyset}$, e~${e_T(B,W)\le 2}$, ou
				\item ${|B|\le m\le |B|+|S|}$, 
				com~${0<|S|\le\dfrac{n}{2}}$,
				${e_T(B,W,S)\le \dfrac{2\cdot 
				\Delta(T)}{\diam^*(T)}}$, 
				e~${\diam^*(T[S])\ge 2\diam^*(T)}$.
			\end{enumerate}
			%Uma partição~$(B,W,S)$ que satisfaz 1 ou 2 pode ser
			%computada em tempo~${O\Big(\dfrac{n}{\diam^*(T)}\Big)}$ 
		\end{teo}

	\bigskip
	\bigskip

		% Primeiramente, faremos uma série de definições para que
		% depois possamos explicar melhor como o teorema e o algoritmo
		% funcionam.

		% Usaremos o que foi explicado na seção~\ref{sec:caminhMaximo}
		% para encontrármos um caminho máximo. 
		% Depois, usaremos 

		Primeiramente, temos uma árvore~$T$ com~${n\ge 3}$ vértices e um 
		inteiro~${m\in[n]}$.
		Seja~$P$ um caminho máximo de~$T$ e sabendo que os vértices
		estão rotulados como o descrito na seção~\ref{sec:rotulacao},
		teremos um corte~$(B,W,S)$ da seguinte forma:
		%\begin{itemize}
		\bigskip
		\bigskip
	
	\subsubsection*{Caso 1:}
			Se existe um vértice ~${v\in V(P)}$ tal 
			que~${\map_m^+(v)\in V(P)}$, então o 
			corte~$(B,W,S)$ 
			com~${S=\emptyset}$,~${B =\{v+1, v+2,\ldots,~v+m\}}$
			e~${W=V(T)\setminus B}$ satisfaz as propriedades do
			primeiro caso do Teorema~\ref{teo:dobraDiametro},
			como o explicado na seção~\ref{sec:map}.

	\bigskip
	\bigskip

		\subsubsection*{Caso 2:}
			Caso contrário, sabemos que para todo vértice~${v\in V(P)}$
			teremos que~${\map_m^+(v)\not\in V(P)}$.
			Isso implica que existem~$|V(P)|$ ou mais vértices fora de~$P$,
			dado que cada vértice de~$P$ mapeia vértices distintos que estão
			fora de~$P$.
			Isso implica que~$\diam^*(T)\le\dfrac{n}{2}$, pois no 
			máximo~$\dfrac{n}{2}$ vértices estarão em~$P$.

			Nesse caso, teremos um corte~$(B,W,S)$ que satisfaz as
			propriedades do segundo caso do Teorema~\ref{teo:dobraDiametro}.
			Isso será detalhado melhor a seguir.

			\bigskip
			
			% Para todo~${z\in V(P)}$, e sendo~$x_1$ e~$x_2$ os menores
			% vértices de~$T_z$ tal que~${\map^-_m(x_1)\in V(P)}$ 
			% e~${\map^+_m(x_2)\in V(P)}$,
			% usaremos as seguintes definições:
			% \begin{align}
			% 	z^b   = &\map^-_m(x_1) \nonumber \\
			% 	z^f   = &\map^+_m(x_2) \nonumber \\
			% 	P_z^b = &\map^-_m(V(T_z))\cap V(P) \nonumber\\
			% 	P_z^f = &\map^+_m(V(T_z))\cap V(P) \nonumber%\\
			% 	%H_z^f = &\map^+_m(V(T_))\cap V(P) \nonumber\\
			% 	%H_z^f = &\map^+_m(V(T_z))\cap V(P) \nonumber
			% \end{align}

			Seja um vértice~${z\in V(P)}$ tal que~$z$ é b-especial e
			sendo~$x$ o menor vértice de~$T_z$ tal 
			que~${\map^-_m(x)\in V(P)}$, temos 
			que~${z^b = \map^-_m(x)}$.
			Temos também
			que~${P_z^b = \map^-_m(V(T_z))\cap V(P)}$
			e~${H_z^f =T_u}$ para todo 
			vértice~${u\in P_z^b}$ com~${u\ne z^b}$.

			Segue uma imagem com a representação da disposição dos 
			conjuntos de vértices.


	\begin{center} \begin{tikzpicture}
		[scale=.7,auto=left,every node/.style={circle, draw=black,
				fill=white!70}]

		\draw [decorate,decoration={brace,amplitude=10pt,mirror,
		raise=4pt}, draw=brown, line width=1.5pt] 
		(8,5.7) -- (2,5.7);

		\draw [decorate,decoration={brace,amplitude=10pt,mirror,
		raise=4pt}, draw=Emerald, line width=1.5pt] 
		(4,2.7) -- (8.5,2.7);

		\draw [draw=yellow, fill=yellow!35, line width=2pt] (18.1,3.7) 
		ellipse (1.9cm and 3.7cm); 

		\node [text=blue!65,draw=none, fill=none] at 
		(17.7,6.8) {$T_z$};


		% \node [scale=1.3,subtree,fill=Green](y1) at (2.5,5) {};
		% \node [subtree, fill=Cyan]      (y2) at (5,5)   {};
		% \node [subtree, fill=Cyan]      (y3) at (7.5,5) {};
		% \node [scale=1.3,subtree,fill=Green](y4) at (10,5)  {};
		% \node [subtree, fill=Blue]        (y5) at (12.5,5){};
		% \node [subtree, fill=Blue]        (y6) at (15,5)  {};
		% \node [scale=1.9, subtree,fill=Gray] (y7) at (18.1,5)  {};


		\node [scale=1.3,subtree,fill=red!50](y1) at (2.5,5) {};
		\node [subtree, fill=Emerald]      (y2) at (5,5)   {};
		\node [subtree, fill=Emerald]      (y3) at (7.5,5) {};
		\node [scale=1.3,subtree,fill=red!50](y4) at (10,5)  {};
		\node [subtree, fill=yellow]        (y5) at (12.5,5){};
		\node [subtree, fill=yellow]        (y6) at (15,5)  {};
		\node [scale=1.9, subtree,fill=CornflowerBlue] (y7) at (18.1,5)  {};

		\node (x0) at (0,5)  {...};
		\node [fill=brown](x1) at (2.5,5)    {};
		\node [fill=brown](x2) at (5,5)  {};
		\node [fill=brown](x3) at (7.5,5)    {};
		\node (x4) at (10,5) {};
		\node (x5) at (12.5,5)   {};
		\node (x6) at (15,5) {};
		\node (x7) at (18.1,5) {};
		\node (x8) at (20.6,5) {...};

		\node [draw=none, fill=none] at 
		(2,2) {\footnotesize{$\map^-_m(v)$}};
		\node [draw=none, fill=none](nv) at (2,2){};
		\node [draw=none, fill=none](nvt) at (2.5,3.6){};

		\node [draw=none, fill=none] at 
		(9.5,2) {\footnotesize{$\map^-_m(z)$}};
		\node [draw=none, fill=none](nz) at (9.5,2){};
		\node [draw=none, fill=none](nzt) at (10,3.6){};

		\node [draw=none, fill=none] at 
		(15.5,2.6) {\footnotesize{$x$}};
		\node [draw=none, fill=none](z) at (15.5,2.6){};
		\node [draw=none, fill=none](zt) at (18,2.6){};

		\node [draw=none, fill=none] at (2.6,5.7){$z^b$};
		\node [draw=none, fill=none] at (18.1,5.6){$z$};
		\node [draw=none, fill=none] at (15,5.6){$v$};
		\node [text=Brown,draw=none, fill=none] at (5,7){$P^b_z$};
		\node [text=Emerald,draw=none, fill=none] at (6.25,1.3){$H^b_z$};

		\foreach \from/\to in {x0/x1,x1/x2,x2/x3,x3/x4,x4/x5,x5/x6,x6/x7,x7/x8}
		\draw (\from) -- (\to);

		\path[->, line width=1pt]
		(nv) edge[left=23 ] (nvt);
		\path[->, line width=1pt]
		(z) edge[right=45] (zt);
		\path[->, line width=1pt]
		(nz) edge[right=45] (nzt);

	\end{tikzpicture} \end{center}

		\bigskip

			De forma análoga, para um vértice~${z\in V(P)}$ tal 
			que~$z$ é f-especial,
			temos que~${z^f = \map^+_m(x)}$,
			sendo~$x$ o menor vértice de~$T_z$ tal 
			que~${\map^+_m(x)\in V(P)}$.
			Também temos
			que~${P_z^f = \map^+_m(V(T_z))\cap V(P)}$
			e~${H_z^f =T_u}$ para todo 
			vértice~${u\in P_z^f}$ com~${u\ne z^f}$.

			Segue abaixo outra imagem com a representação da disposição 
			dos conjuntos de vértices.


	\begin{center} \begin{tikzpicture}
		[scale=.6,auto=left,every node/.style={circle, draw=black,
				fill=white!70}]

		\draw [decorate,decoration={brace,amplitude=10pt,mirror,
		raise=4pt}, draw=brown, line width=1.5pt] 
		(-4.5,5.9) -- (-10.5,5.9);

		\draw [decorate,decoration={brace,amplitude=10pt,mirror,
		raise=4pt}, draw=Emerald, line width=1.5pt] 
		(-8.5,2.7) -- (-4,2.7);

		\draw [draw=yellow, fill=yellow!35, line width=2pt] (-18.1,3.) 
		ellipse (2.1cm and 4.5cm); 

		\node [text=blue!65,draw=none, fill=none] at 
		(-18.5,6.8) {$T_z$};


		

		\node [scale=1.3,subtree,fill=red!50](y1) at (-2.5,5) {};
		\node [subtree, fill=Emerald]      (y2) at (-5,5)   {};
		\node [subtree, fill=Emerald]      (y3) at (-7.5,5) {};
		\node [scale=1.3,subtree,fill=red!50](y4) at (-10,5)  {};
		\node [subtree, fill=yellow]        (y5) at (-12.5,5){};
		\node [subtree, fill=yellow]        (y6) at (-15,5)  {};
		\node [scale=1.9, subtree,fill=CornflowerBlue] (y7) at (-18.1,5)  {};

		\node (x0) at (0,5)  {...};
		\node (x1) at (-2.5,5)    {};
		\node [fill=brown](x2) at (-5,5)  {};
		\node [fill=brown](x3) at (-7.5,5)    {};
		\node [fill=brown](x4) at (-10,5) {};
		\node (x5) at (-12.5,5)   {};
		\node (x6) at (-15,5) {};
		\node (x7) at (-18.1,5) {};
		\node (x8) at (-20.6,5) {};
		\node (x9) at (-23.1,5) {...};

		\node [draw=none, fill=none] at 
		(-2,1) {\footnotesize{$\map^+_m(z)$}};
		\node [draw=none, fill=none](nv) at (-2,1){};
		\node [draw=none, fill=none](nvt) at (-2.5,3.6){};

		\node [draw=none, fill=none] at 
		(-9.5,1) {\footnotesize{$\map^+_m(v)$}};
		\node [draw=none, fill=none](nz) at (-9.5,1){};
		\node [draw=none, fill=none](nzt) at (-10,3.6){};

		\node [draw=none, fill=none] at 
		(-15.5,2) {\footnotesize{$x$}};
		\node [draw=none, fill=none](z) at (-15.5,2){};
		\node [draw=none, fill=none](zt) at (-18,2){};

		\node [draw=none, fill=none] at (-9.8,5.8){$z^f$};
		\node [draw=none, fill=none] at (-18.1,5.6){$z$};
		\node [draw=none, fill=none] at (-20.6,5.6){$v$};
		\node [text=Brown,draw=none, fill=none] at (-7.5,7.3){$P^f_z$};
		\node [text=Emerald,draw=none, fill=none] at (-6.35,1.1){$H^f_z$};


		\foreach \from/\to in {x0/x1,x1/x2,x2/x3,x3/x4,x4/x5,x5/x6,x6/x7,x7/x8,x8/x9}
		\draw (\from) -- (\to);

		\path[->, line width=1pt]
		(nv) edge[left=23 ] (nvt);
		\path[->, line width=1pt]
		(z) edge[right=45] (zt);
		\path[->, line width=1pt]
		(nz) edge[right=45] (nzt);

	\end{tikzpicture} \end{center}

	Note que os vértices~$\map^-_m(v)$,~$\map^-_m(z)$,~$\map^+_m(v)$ 
	e~$\map^+_m(z)$ não estão em~$V(P)$, pois estamos no caso 2, onde
	nenhum vértice de~$P$ é mapeado em outro vértice de~$P$.

	Considerando que existe uma aresta ligando o primeiro ao último 
	vértices de~$P$, vemos que os elementos de~$P_z^b$ aparecem 
	como um caminho conexo e que para valores diferentes de~$z$,
	temos que~$P_z^b$ é disjunto.
	O mesmo vale para~$P_z^f$.

	Usaremos futuramente as seguintes propriedades:
	\begin{enumerate}
		\item Para todo vértice~$z$ tal que~$z$ é b-especial, 
		temos que o vértice~$z^b$ é f-especial.

		Isso ocorre porque para o vértice~$v$, que vem antes 
		de~$z$ em~$P$, temos que~${\map^-_m(v)\not\in V(P)}$, 
		logo,~$\map^-_m(v)$ está em~$T_{z^b}$.
		Sabemos que~$\map^-_m(v)$ não está na árvore a esquerda
		de~$T{z^f}$, pois se estivesse, o~$z^f$ seria um outro
		vértice.

		\item Para todo vértice~$z$ tal que~$z$ é f-especial, 
		temos que o vértice~$z^f$ é b-especial.

		Isso ocorre de forma análoga a anterior, dado que
		para o vértice~$v$, que vem antes de~$z$ em~$P$, temos
		que~${\map^+_m(v)\in T_{z^f}}$.

		\item Um vértice~${z\in V(P)}$ é b-especial sse existe um
		vértice~${u\in V(P)}$ tal que~$u$ é f-especial e~${z=u^f}$.

		Sabemos que a volta é verdadeira usando a propriedade 2.
		
		A ida é verdadeira pois, dado que~$z$ é b-especial,
		temos que existe um ou mais vértices~${v_i\in V(P)}$ tal
		que~${\map^+_m(v_i)\in T_{z}}$.
		Seja~$v_k$ o vértice com o maior rótulo e~$u$ o vértice 
		que vem depois de~$v_k$ em~$P$. 
		Temos então que~${\map^-_m(z)\in T_u}$ e que alguns vértices
		de~$T_z$ mapeiam em~$v_k$, logo,~$u$ é f-especial e~${z=u^f}$.

		\item Um vértice~${z\in V(P)}$ é f-especial sse existe um
		vértice~${u\in V(P)}$ tal que~$u$ é b-especial e~${z=u^b}$.

		Podemos verificar que essa propriedade é verdadeira de forma
		semelhante a verificação da propriedade anterior.
	\end{enumerate}

	Seja~$T'_z$ o conjunto de 
	vértices~${V(T_z)\setminus \{z\}}$,~$B_{esp}$ 
	um conjunto de todos os vértices b-especiais e~$F_{esp}$ o
	conjunto de todos os vértices f-especiais, usando as propriedades 
	3 e 4 respectivamente, temos que
	\begin{align}
		\displaystyle\sum_{u\in F_{esp}}|T_{u^f}| = 
		\displaystyle\sum_{z\in B_{esp}}|T_{z}| %\nonumber \\
		&\Rightarrow
		\displaystyle\sum_{u\in F_{esp}}|T_{u^f}'| = 
		\displaystyle\sum_{z\in B_{esp}}|T_{z}'| \label{eq:4}\\
		\displaystyle\sum_{u\in B_{esp}}|T_{u^b}| = 
		\displaystyle\sum_{z\in F_{esp}}|T_{z}| %\nonumber
		&\Rightarrow
		\displaystyle\sum_{u\in B_{esp}}|T_{u^b}'| = 
		\displaystyle\sum_{z\in F_{esp}}|T_{z}'| \label{eq:3}.
	\end{align}

	Note que para todo vértice b-especial~$z$ todos os vértices 
	de~$H^b_z$ e de~$P^b_z$ mapeiam um vértice 
	de~$T_z$ e que nem todos os vértices de~$T_z$ são mapeados por
	vértices~$H^b_z\cup P_z^b$, 
	dado que~$\map^-_m(z)\not\in H^b_z\cup P_z^b$.
	O mesmo ocorre com os vértices f-especiais~$w$.
	Portanto temos que
	\begin{align}
		|T_z| > |P^b_z| + |H^b_z| 
		&\Rightarrow
		|T_z'| \ge |P^b_z| + |H^b_z| \label{eq:5}\\
		|T_w| > |P^f_w| + |H^f_w|
		&\Rightarrow
		|T_w'| \ge |P^f_w| + |H^f_w| \label{eq:6}.
	\end{align}

	Como~$P^b_z$ é disjunto e~$P^b_z\cup H^b_z\cup T'_{z^b}$
	nada mais é do que o conjunto de sub-árvores enraizadas
	nos vértices de~$P^b_z$, então temos 
	que~${P^b_z\cup H^b_z\cup T'_{z^b}}$ também é disjunto.
	Como todos os vértices de~$P$ são mapeados em alguma sub-árvore,
	então todo vértice de~$V(P)$ pertence a algum~$P^b_z$, logo,
	todo vértice de~$V(T)$ pertence a 
	algum~${P^b_z\cup H^b_z\cup T'_{z^b}}$.
	O mesmo é válido para~${P^f_z\cup H^f_z\cup T'_{z^f}}$.
	Portanto
	\begin{align}
		(1-d)n = |V\setminus V(P)| &= 
		\displaystyle\sum_{z\in B_{esp}}\Big(|T_{z^b}'|+|H_z^b|\Big)
		\label{eq:7}\\
		(1-d)n = |V\setminus V(P)| &= 
		\displaystyle\sum_{z\in F_{esp}}\Big(|T_{z^f}'|+|H_z^f|\Big)
		\label{eq:9}\\
		dn = |V(P)| &= 
		\displaystyle\sum_{z\in B_{esp}}|P_z^b|
		\label{eq:8}\\
		dn = |V(P)| &= 
		\displaystyle\sum_{z\in F_{esp}}|P_z^f| 
		\label{eq:10}.
	\end{align}

	Usando as equações~\ref{eq:7},~\ref{eq:9},~\ref{eq:8} e~\ref{eq:10}
	e, as equações~\ref{eq:4} e~\ref{eq:3} depois, temos que
	\begin{align}
		\dfrac{1-d}{d} &= 
		\dfrac{
			\displaystyle\sum_{z\in B_{esp}}
			\Big(|T'_{z^b}|+|H_z^b|\Big) + 
			\displaystyle\sum_{z\in F_{esp}}
			\Big(|T'_{z^f}|+|H_z^f|\Big)}
			{\displaystyle\sum_{z\in B_{esp}}|P_z^b| +
			\displaystyle\sum_{z\in F_{esp}}|P_z^f|}
			\nonumber\\
		&= \dfrac{
			\displaystyle\sum_{z\in B_{esp}}
			\Big(|T'_{z}|+|H_z^b|\Big) + 
			\displaystyle\sum_{z\in F_{esp}}
			\Big(|T'_{z}|+|H_z^f|\Big)}
			{\displaystyle\sum_{z\in B_{esp}}|P_z^b| +
			\displaystyle\sum_{z\in F_{esp}}|P_z^f|}
			\nonumber \\
			\Rightarrow	
			\dfrac{1-d}{d}\displaystyle\sum_{z\in B_{esp}}|P_z^b| &+ 
			\dfrac{1-d}{d}\displaystyle\sum_{z\in F_{esp}}|P_z^f|
			= \displaystyle\sum_{z\in B_{esp}}
			\Big(|T'_{z}|+|H_z^b|\Big) + 
			\displaystyle\sum_{z\in F_{esp}}
			\Big(|T'_{z}|+|H_z^f|\Big)\nonumber
	\end{align}
	Isso nos leva a dois casos
	\begin{enumerate}[label=(\alph*)]
		\item $\displaystyle\sum_{z\in B_{esp}}
			\Big(|T'_{z}|+|H_z^b|\Big)\le
			\dfrac{1-d}{d}\displaystyle\sum_{z\in B_{esp}}|P_z^b|$
			ou
		\item $\displaystyle\sum_{z\in B_{esp}}
			\Big(|T'_{z}|+|H_z^b|\Big)>
			\dfrac{1-d}{d}\displaystyle\sum_{z\in B_{esp}}|P_z^b|
			\Rightarrow \\
			\displaystyle\sum_{z\in F_{esp}}
			\Big(|T'_{z}|+|H_z^f|\Big)<
			\dfrac{1-d}{d}\displaystyle\sum_{z\in F_{esp}}|P_z^f|
			\Rightarrow \\
			\displaystyle\sum_{z\in F_{esp}}
			\Big(|T'_{z}|+|H_z^f|\Big)\le
			\dfrac{1-d}{d}\displaystyle\sum_{z\in F_{esp}}|P_z^f|$.
	\end{enumerate}

	Sabemos que se o caso (a) for verdadeiro, teremos um 
	vértice~$z\in B_{esp}$ tal que~${|T'_z|+|H^b_z|\le |P^b_z|}$,
	pois se todos os vértices em~$B_{esp}$ 
	satisfizerem~${|T'_z|+|H^b_z| > |P^b_z|}$, então o caso (a)
	não será verdadeiro.
	O mesmo ocorre com o caso (b). 
	Portanto temos um dos casos
	a seguir sempre será válido.

	\begin{enumerate}[label=(\alph*)]
		\item existe um vértice~$z\in B_{esp}$ 
			tal que $|T'_{z}|+|H_z^b|\le
			\dfrac{1-d}{d}|P_z^b|$ ou
		\item existe um vértice~$z\in F_{esp}$ 
			tal que $|T'_{z}|+|H_z^f|\le
			\dfrac{1-d}{d}|P_z^f|$
	\end{enumerate}

	\bigskip
	\bigskip
	
	\subsubsection*{Caso (a) seja verdadeiro:}
		Existe um vértice~$z\in B_{esp}$ 
		tal que ${|T'_{z}|+|H_z^b|\le
		\dfrac{1-d}{d}|P_z^b|}$, logo, temos
		que~${|H^b_z|\le \Big(\dfrac{1}{d}-1\Big)|P^b_z|-|T'_z|}$.
		Usando a equação~\ref{eq:5}, temos 
		que
		\begin{align}
			|H^b_z|\le\Big(\dfrac{1}{2d}-1\Big)|P^b_z|
			\Rightarrow
			|H^b_z|+|P^b_z|\le\dfrac{1}{2d}|P^b_z| \nonumber
		\end{align}

		Construiremos o corte~$(B,W,S)$ de forma 
		que~$S = |H^b_z|+|P^b_z|$.
		Sabemos que~$S\ne \emptyset$, pois~$z$ é b-especial,
		o que implica em~$P^b_z\ne \emptyset$. 
		Dessa forma, temos que

		\begin{align}
			|S| = |H^b_z|+|P^b_z|\le\dfrac{1}{2d}|P^b_z|
			&\Rightarrow
			|S| \le\dfrac{1}{2d}|P^b_z| \nonumber \\
			&\Rightarrow
			|S| \le\dfrac{1}{2d}|P^b_z|\le \dfrac{n}{2} 
			\label{eq:metade} \\
			&\Rightarrow
			2d\le \dfrac{1}{|S|}|P^b_z|  
			\label{eq:reaproveitaCaminho}\\
			&\Rightarrow
			2d\le \dfrac{1}{|S|}|P^b_z|\le \diam^*(T[S]) 
			\nonumber
		\end{align}


		Seja o vértice~${z^b_l\in P^b_z}$ tal 
		que~$\map^+_m(z^b_l)$ é o vértice de~$T_z$ com 
		maior rótulo possível, e~$j$ é o vértice que vem
		antes de~$z$ em~$P$. Assumiremos que
		\begin{align}
			B_1 = \Big\{ v\in V(T):z^b_l<v\le j \Big\}
			\nonumber.
		\end{align}

		Agora podemos usar o Lema~\ref{lema:approxCutForest}
		na flotesta~$T_z'$, usando~${m-|B_1|}$ 
		e~${2-\dfrac{1}{1-d}}$ como os valores de~$m$ e~$c$,
		respectivamente.
		Dessa forma, teremos um corte~$(B_2,W_2)$ tal 
		que~${cm\le|B_2|\le m}$ e
		\begin{align}
		e_{T_z'}(B_2,W_2)&\le \ceil[\Big]{ \dfrac{2c}{1-c}} 
		\Delta(T_z') = \ceil[\Big]{ \dfrac{2}{d}-4} \Delta(T_z') 
		\nonumber\\
		&\le \ceil[\Big]{ \dfrac{2}{d}-4} \Delta(T) 
		\nonumber\\
		&\le \Big( \dfrac{2}{d}-3 \Big) \Delta(T).
		\label{eq:corte}
		\end{align}
		Como temos que~${\map^+_m(z^b_l)\in T_z}$ e que todos
		os vértices entre~$z^b_l+1$ e~$j$ estão em~$B_1$,
		então~${m-|B_1|<|T_z|}$. Logo,~${1\le m-|B_1|\le |T'_z|}$.
		Como~$d\in (0,\frac{1}{2}]$, então, temos 
		que~${c\in [0,1)}$.
		Portanto, os valores passados como~$m$ e~$c$ são 
		compatíveis com o que é pedido no 
		Lema~\ref{lema:approxCutForest}.
		A disposição dos vértices ficará como o representado a
		seguir.



		\begin{center} \begin{tikzpicture}
		[scale=.7,auto=left,every node/.style={circle, draw=black,
				fill=white!70}]

		\draw [draw=red!50, line width=2pt, fill=red!17](9,2.5) rectangle (16,6);

		\draw [draw=Emerald, line width=2pt, fill=Emerald!20]
		(8.7,1.4) -- (8.7,7.4) -- (1.8,7.4) -- (1.8,4.5) -- 
		(3.7,4.5) -- (3.7,1.4) -- cycle;

		\draw [draw=CornflowerBlue, line width=2pt, fill=CornflowerBlue!30]
		(16.3,1.4) -- (17.3,3.9) -- (18.3,1.4) -- cycle;

		\draw [decorate,decoration={brace,amplitude=10pt,mirror,
		raise=4pt}, draw=brown, line width=1.5pt] 
		(8,5.7) -- (2,5.7);

		\draw [decorate,decoration={brace,amplitude=10pt,mirror,
		raise=4pt}, draw=Emerald, line width=1.5pt] 
		(4,3.2) -- (8.5,3.2);
		


		\node [scale=1.3,subtree,fill=white](y1) at (2.5,5) {};
		\node [subtree, fill=Emerald]      (y2) at (5,5)   {};
		\node [subtree, fill=Emerald]      (y3) at (7.5,5) {};
		\node [subtree,fill=red!50]        (y4) at (10,5)  {};
		\node [subtree,fill=red!50]        (y5) at (12.5,5){};
		\node [subtree,fill=red!50]        (y6) at (15,5)  {};
		\node [scale=1.9, subtree,fill=CornflowerBlue!70] (y7) at (18.1,5)  {};

		\node (x0) at (0,5)  {...};
		\node [fill=brown](x1) at (2.5,5)    {};
		\node [fill=brown](x2) at (5,5)  {};
		\node [fill=brown](x3) at (7.5,5)    {};
		\node (x4) at (10,5) {};
		\node (x5) at (12.5,5)   {};
		\node (x6) at (15,5) {};
		\node (x7) at (18.1,5) {};
		\node (x8) at (20.6,5) {...};

		\node [draw=none, fill=none] at (7.6,5.7){$z^b_l$};
		\node [draw=none, fill=none] at (2.6,5.7){$z^b$};
		\node [draw=none, fill=none] at (18.1,5.6){$z$};
		\node [draw=none, fill=none] at (15,5.6){$f$};
		\node [text=Brown,draw=none, fill=none] at (5,7){$P^b_z$};
		\node [text=Emerald,draw=none, fill=none] at (6.25,1.8){$H^b_z$};

		\node [text=Emerald,draw=none, fill=none] at (6.2,0.8){$\mathbf{S}$};
		\node [text=red!50,draw=none, fill=none] at (12.5,1.9){$\mathbf{B_1}$};
		\node [text=CornflowerBlue,draw=none,fill=none] at 
		(17.3,0.8){$\mathbf{B_2}$};

		\foreach \from/\to in {x0/x1,x1/x2,x2/x3,x3/x4,x4/x5,x5/x6,x6/x7,x7/x8}
		\draw (\from) -- (\to);

		\draw [draw=Emerald, line width=2pt]
		(8.7,1.4) -- (8.7,7.4) -- (1.8,7.4) -- (1.8,4.5) -- 
		(3.7,4.5) -- (3.7,1.4) -- cycle;

		\draw [draw=CornflowerBlue, line width=2pt]
		(16.3,1.4) -- (17.3,3.9) -- (18.3,1.4) -- cycle;

	\end{tikzpicture} \end{center}


		Temos, então, o corte~$(B,W,S)$ com~${B = B_1\cup B_2}$ 
		e~${W = V(T)\setminus (B\cup S)}$.
		Dado que~${\map^+_m(z^b_l)\in T_z}$, temos 
		que~${m<|B_1|}$ e, como obtemos~$B_2$ através do 
		Lema~\ref{lema:approxCutForest} usando~$m-|B_1|$ 
		como~$m$, então teremos que~${B_2\le m-|B_1|}$, implicando
		em~${m\ge|B|}$.
		De acordo com o Lema~\ref{lema:approxCutForest} também temos 
		que~${|B_2|\ge c\cdot(m-|B_1|)}$ e, como estamos no caso~2, 
		então~$d\le \dfrac{1}{2}$ e~$|P^b_z|\le |S|$, dado que
		cada vértice de~$P^b_z$ irá mapear em um vértice diferente 
		de~$S$.
		Portanto
		\begin{align}
			m-|B| &= m-|B_1|-|B_2| \nonumber\\
			&\le(m-|B_1|)-c\cdot (m-|B_1|) = (m-|B_1|)(1-c)
			\nonumber \\
			&\le|T_z'|(1-c) = |T_z'|\dfrac{d}{1-d} \le
			|T_z'|\dfrac{\frac{1}{2}}{1-\frac{1}{2}} = |T_z'|
			\nonumber \\
			&\le |P^b_z| \le |S| \nonumber \\ 
			\nonumber\\
			\Rightarrow m\le |&B|+|S| \Rightarrow |B|\le 
			m\le |B|+|S|. \nonumber
		\end{align}

		Agora mostraremos 
		que~${e_T(B,W,S)\le \dfrac{2\cdot \Delta(T)}{\diam^*(T)}}$.
		$\\$
		% Sabemos que os vértices do corte~$(B,W,S)$ são as arestas
		% que ligam~$z^b_l$ e~$f$ aos vértices que vêm depois deles 
		% em~$P$
		Temos que o número de arestas no corte~$(B,W,S)$ é, no 
		máximo,~${(\Delta(T)-1) + 1 + 1 + e_{T_z'}(B_2,W_2) +
		(\Delta(T)-2)}$.
		Essa soma representa a quantidade total de arestas, sendo elas
		as que conectam~$z^b$
		a~$W$, a que liga~$z^b_l$ a~$B_1$, a
		que conecta~$f$ e~$z$, as presentes no corte~$(B_2,W_2)$ e, 
		por fim, as arestas que ligam~$z$ a~$T_z'$ (talvez
		existam arestas ligando~$B_2$ a~$z$).
		Logo, usando a equação~\ref{eq:corte}, temos que
		\begin{align}
			e_T(B,W,S) &= (\Delta(T)-1) + 1 + 1 + e_{T_z'}(B_2,W_2) +
			(\Delta(T)-2) \nonumber\\
			&= 2\cdot\Delta(T) + e_{T_z'}(B_2,W_2) - 1 
			\nonumber\\
			&\le 2\cdot\Delta(T) + \Big( \dfrac{2}{d}-3 \Big) 
			\Delta(T) - 1 = \Big( \dfrac{2}{d}-1 \Big)\Delta(T)-1 
			\nonumber\\
			&< \dfrac{2\cdot \Delta(T)}{d}. \nonumber
		\end{align}

		Portanto, se as condições do caso~(a) forem satisfeitas, então 
		existe um corte~$(B,W,S)$ que satisfaz as propriedades do
		Teorema~\ref{teo:dobraDiametro}.

	\bigskip
	\bigskip
	
	\subsubsection*{Caso (b) seja verdadeiro:}

	Existe um vértice~$z\in F_{esp}$ 
	tal que ${|T'_{z}|+|H_z^f|\le
	\dfrac{1-d}{d}|P_z^f|}$, logo, temos
	que~${|H^f_z|\le \Big(\dfrac{1}{d}-1\Big)|P^f_z|-|T'_z|}$.
	Usando a equação~\ref{eq:6}, temos 
	que
	\begin{align}
		|H^f_z|\le\Big(\dfrac{1}{2d}-1\Big)|P^f_z|
		\Rightarrow
		|H^f_z|+|P^f_z|\le\dfrac{1}{2d}|P^f_z| \nonumber
	\end{align}


	Assumiremos que~${S = H^f_z \cup P^f_z}$ 
	e~${B_1 = \{v\in V(T): z<v<z^f\}}$. 
	Da mesma forma que no caso anterior, teremos 
	que~${2d\le \diam^*(T[S])}$.

	Obteremos o corte~$(B_2,W_2)$ usando o 
	Lema~\ref{lema:approxCutForest} em~$T'_z$ 
	com~${m-|B|}$ e~${2-\dfrac{1}{1-d}}$ sendo os valores
	de~$m$ e~$c$, respectivamente. 
	Temos também que~${B = B_1 \cup B_2}$ 
	e~${W = V(T)\setminus (B\cup S)}$.
	Dessa forma, a disposição dos 
	vértices está ilustrada na imagem a seguir.

	\begin{center} \begin{tikzpicture}
		[scale=.7,auto=left,every node/.style={circle, draw=black,
				fill=white!70}]

		\draw [draw=red!50, line width=2pt, fill=red!17]
		(-8.83,2.5) -- (-8.83,4.57) -- (-10.93,4.57) -- (-10.93,6) -- 
		(-16,6) -- (-16,2.5) -- cycle;

		\draw [draw=Emerald, line width=2pt, fill=Emerald!20]
		(-8.7,1.4) -- (-8.7,4.7) -- (-10.8,4.7) -- (-10.8,7.4) -- 
		(-3.7,7.4) -- (-3.7,1.4) -- cycle;

		\draw [draw=CornflowerBlue, line width=2pt, fill=CornflowerBlue!30]
		(-16.3,1.4) -- (-17.3,3.9) -- (-18.3,1.4) -- cycle;

		\draw [decorate,decoration={brace,amplitude=10pt,mirror,
		raise=4pt}, draw=brown, line width=1.5pt] 
		(-4.5,5.7) -- (-10.5,5.7);

		\draw [decorate,decoration={brace,amplitude=10pt,mirror,
		raise=4pt}, draw=Emerald, line width=1.5pt] 
		(-8.5,3.2) -- (-4,3.2);
		


		\node [scale=1.3,subtree,fill=white](y1) at (-2.5,5) {};
		\node [subtree, fill=Emerald]      (y2) at (-5,5)   {};
		\node [subtree, fill=Emerald]      (y3) at (-7.5,5) {};
		\node [subtree,fill=red!50]        (y4) at (-10,5)  {};
		\node [subtree,fill=red!50]        (y5) at (-12.5,5){};
		\node [subtree,fill=red!50]        (y6) at (-15,5)  {};
		\node [scale=1.9, subtree,fill=CornflowerBlue!70] (y7) at (-18.1,5)  {};

		\node (x0) at (0,5)  {...};
		\node (x1) at (-2.5,5)    {};
		\node [fill=brown](x2) at (-5,5)  {};
		\node [fill=brown](x3) at (-7.5,5)    {};
		\node [fill=brown](x4) at (-10,5) {};
		\node (x5) at (-12.5,5)   {};
		\node (x6) at (-15,5) {};
		\node (x7) at (-18.1,5) {};
		\node (x8) at (-20.6,5) {...};

		\node [draw=none, fill=none] at (-10,5.7){$z^f$};
		\node [draw=none, fill=none] at (-18.1,5.6){$z$};
		\node [text=Brown,draw=none, fill=none] at (-7.5,7){$P^f_z$};
		\node [text=Emerald,draw=none, fill=none] at (-6.25,1.8){$H^f_z$};

		\node [text=Emerald,draw=none, fill=none] at (-6.2,0.8){$\mathbf{S}$};
		\node [text=red!50,draw=none, fill=none] at (-12.5,1.9){$\mathbf{B_1}$};
		\node [text=CornflowerBlue,draw=none,fill=none] at 
		(-17.3,0.8){$\mathbf{B_2}$};

		\foreach \from/\to in {x0/x1,x1/x2,x2/x3,x3/x4,x4/x5,x5/x6,x6/x7,x7/x8}
		\draw (\from) -- (\to);

		\draw [draw=red!50, line width=2pt]
		(-8.83,2.5) -- (-8.83,4.57) -- (-10.93,4.57) -- (-10.93,6) -- 
		(-16,6) -- (-16,2.5) -- cycle;

		\draw [draw=Emerald, line width=2pt]
		(-8.7,1.4) -- (-8.7,4.7) -- (-10.8,4.7) -- (-10.8,7.4) -- 
		(-3.7,7.4) -- (-3.7,1.4) -- cycle;

		\draw [draw=CornflowerBlue, line width=2pt]
		(-16.3,1.4) -- (-17.3,3.9) -- (-18.3,1.4) -- cycle;

	\end{tikzpicture} \end{center}
	Assim como foi mostrado no caso 2.(a), temos 
	que~${e_T(B,W,S)<\dfrac{2\cdot \Delta(T)}{d}}$.
	Satisfazendo assim as propriedades do Teorema~\ref{teo:dobraDiametro} em todos os casos apresentados anteriormente.

	


	\bigskip
	\bigskip
	\bigskip
%%%%%%%%%%%%%%%%%%%%%%%%%%%%%%%%%%%%%%%%%%%%%%%%%%%%%%%%%%%%%%%%%%
%%%%%%%%%%%%%%%%%%%%%%%%%%%%%%%%%%%%%%%%%%%%%%%%%%%%%%%%%%%%%%%%%%
%%%%%%%%%%%%%%%%%%%%%%%%%%%%%%%%%%%%%%%%%%%%%%%%%%%%%%%%%%%%%%%%%%

	\subsection{Algoritmo da dobra do diâmetro}

		Assumimos que a função {\sc separa\_conjunto}$(T,$ $S)$
		separa o conjunto~$S$ do restante da árvore.
		Isso é feito apagando as arestas que estão no 
		corte~$(S,V(T)\setminus S)$.
		Sejam~$v_i$ e~$v_j$ primeiro e último vértices em~$S$, respectivamente.
		Sabemos que todos os vértices de~$S$ possuem
		rótulos consecutivos (considerando que o vértices de maior rótulo e 
		de menor rótulo são consecutivos) e que~$v_i$ e~$v_j$ possuem~${(\grau(v_i)-1)}$
		e~$1$ arestas que os conectam a~$V(T)\setminus S$, respectivamente.
		Portanto, o que essa função faz é apagar de~$T$ as arestas que 
		conectam~$v_i$ e~$v_j$ a~$V(T)\setminus S$ e, caso~$S$ não seja
		conexo, inserir uma aresta que liga os vértices de menor e maior rótulos em~$T$.
		Essa função devolverá um grafo~$T$ modificado, onde~$S$ será uma componente
		isolada do restante da árvore. 
		%Chamaremos esse novo grafo de~$T'[S]$.
		Essa função consome tempo~$O(\grau(v_i))$.

	\begin{algorithm}[H]
	\label{alg:dobraDiametro}
		\SetKwInOut{Input}{input}
		\SetKwInOut{Output}{output}

		\caption{}
		\Input{árvore~$T$ com~$n$ vértices,~$m\in[n]$, caminho máximo~$P$,
		vetor~$rotulo$, vetor~$indice$ e raíz da árvore~$r$} 
		\Output{corte~$(B,W,S)$ que satisfaz as propriedades do 
		Teorema~\ref{teo:dobraDiametro}, a árvore~$T'[S]$ e um vértice~$r\in S$}
		\For{$i = 1 \to |V(P)|$}
		{
			\If{$rotulo(i)\in V(P)$}
			{
				\tcp{caso 1}
				$B\gets indice(rotulo(i)+1)\cup\ldots\cup indice(rotulo(i)+m)$\;
				$W\gets V(T)\setminus B$\;

				\Return $[(B,W,\emptyset),$ $\emptyset,$ $null]$\;
			}
		}
		\tcp{caso 2}
		$bvalor \gets$ infinito \;
		$fvalor \gets$ infinito\;
		\For{$z' \in B_{esp}$}
		{
			\If{$\frac{|T'_{z'}|+|H_{z'}^b|}{|P^b_{z'}|} < bvalor$}
			{
			 	$bvalor \gets \frac{|T'_{z'}|+|H_{z'}^b|}{|P^b_{z'}|}$\;
			 	$z \gets z'$\; 
			}
		}
		\For{$z' \in F_{esp}$}
		{
			\If{$\frac{|T'_{z'}|+|H_{z'}^f|}{|P^f_{z'}|} < fvalor$}
			{
			 	$fvalor \gets \frac{|T'_{z'}|+|H_{z'}^f|}{|P^f_{z'}|}$\;
			 	$z \gets z'$\; 
			}
		}
	\end{algorithm}	

	\LinesNumberedHidden
	\begin{algorithm}[H]
	\Numberline
		\If{$bvalor\le fvalor$}
		{
			\tcp{caso 2.a)}
			$z^b_\ell \gets $ último vértice de~$P_z^b$\;
			$B_1 \gets indice(rotulo(z^b_l)+1)\cup\cdots\cup indice(rotulo(z^b_l)+m)$\;
			$B_2 \gets \mathbf {Algoritmo\ref{lema:approxCutForest}}(T'_z,$ $|T'_z|,$ $m-|B_1|,$ $2-\frac{1}{1-d})$\;
			$B \gets B_1\cup B_2$\;
			$W \gets V\setminus (B\cup S)$\; 
			$S = P_z^b\cup H^b_z $\;
			\Return $[(B,W,S),$ $z^b_\ell]$\;
		}
		\Else
		{
			\tcp{caso 2.b)}
			\Numberline$z^f \gets $ primeiro vértice de~$P_z^f$\;
			\Numberline$B_1 \gets indice(rotulo(z)+1)\cup\cdots\cup indice(rotulo(z^f)-1)$\;
			\Numberline$B_2 \gets \mathbf {Algoritmo\ref{lema:approxCutForest}}(T'_z,$ $|T'_z|,$ $m-|B_1|,$ $2-\frac{1}{1-d})$\;
			\Numberline$B \gets B_1\cup B_2$\;
			\Numberline$W \gets V\setminus (B\cup S)$\; 
			\Numberline$S = P_z^f\cup H^f_z $\;
			\Numberline\Return $[(B,W,S),$ $z^f]$\;		

		}

	\end{algorithm}	
	\LinesNumbered

%%%%%%%%%%%%%%%%%%%%%%%%%%%%%%%%%%%%%%%%%%%%%%%%%%%%%%%%%%%%%%%%%%%
%%%%%%%%%%%%%%%%%%%%%%%  SEÇÃO 9     %%%%%%%%%%%%%%%%%%%%%%%%%%%%%%
%%%%%%%%%%%%%%%%%%%%%%%%%%%%%%%%%%%%%%%%%%%%%%%%%%%%%%%%%%%%%%%%%%%

\section {Algoritmo FST para bissecção}

	%\subsection{Teorema do corte exato}
	Nesta seção, finalmente apresentamos o algoritmo FST.
	Ele utiliza o Algoritmo~\ref{alg:dobraDiametro} 
	repetidas vezes além de manter informações derivadas
	do caminho mais longo e da rotulação inicial.

	\begin{lem}
	\label{lem:grauMaximo}
		% Para toda árvore~$T$ com~${n>2}$ vértices
		% e toda aresta~${e=\{v_1,v_2\}}$ em~$T$, 
		% tal que~${v_1,v_2\in V}$ e~${\grau_T(v_1)=\grau_T(v_2)=1}$,
		% temos que~${\Delta(T) = \Delta(T\cup e)}$.
		Para toda árvore~$T$ com~${n>2}$ vértices
		e todo par de vértices~$v_1$ e~$v_2$ em~$T$ 
		com~${\grau_T(v_1)=\grau_T(v_2)=1}$,
		temos que~${\Delta(T) = \Delta(T+ e)}$,
		onde~$e=\{v_1, v_2\}$.
	\end{lem}
	
	
	\begin{proof}
		Como~${n>2}$, temos que~${\Delta(T)\ge 2}$.
		Sabemos que ao adicionar~${e = \{v_1,v_2\}}$ em~$T$, 
		o grau de~$v_1$
		e~$v_2$ passará a ser 2, e os demais vértices da árvore
		irão manter os seus respectivos graus.
		Portanto,~${\grau_{T+e}(v_1)=\grau_{T+ e}(v_2)=2}$
		e, para todo vértice~$v$ tal que~$v\ne v_1$ e~$v\ne v_2$,
		temos que~${\grau_{T+ e}(v)=\grau_T(v)}$.
		Logo~$\Delta(T+e) = \Delta(T)$.
% 		Seja~$v$ um vértice de~$T$ tal 
% 		que~${\grau_T(v) = \Delta(T)\ge 2}$,
% 		dado que~${\grau_T(v_1)=\grau_T(v_2)=1}$
% 		sabe-se que~${v\ne v_1}$ e~${v\ne v_2}$, o que nos leva 
% 		a~${\grau_{T\cup e}(v)=\grau_T(v)=\Delta(T)}$.
% 		Dessa forma, temos que
% 		\begin{align}
% 			\Delta(T\cup e) &= \max\{\grau_{T\cup e}(v_1), 
% 			\grau_{T\cup e}(v)\}
%          		= \max\{2, \Delta(T)\}
% %				&= \grau_{T\cup e}(v) \nonumber\\
% %         		&= \grau_T(v) \nonumber\\
%          		= \Delta(T). \nonumber
% 		\end{align}		
	\end{proof}

	\bigskip

	Segue o teorema principal.

	\begin{teo}[Teorema do corte exato
	{\cite[Teorema 6]{Schmidt15}}]
	\label{teo:corteExato}
		Para todo~${i\in \mathbb{N^+_*}}$, toda árvore~$T$ com~$n$
		vértices e~${\diam^*(T)>\dfrac{1}{2^i}}$, e todo~${m\in[n]}$,
		existe um corte~$(B,W)$ em~$T$ com~$|B|=m$ 
		e~$e_T(B,W)\le 4\cdot 2^i\cdot \Delta(T)$.
		%Um corte que satisfaz esses requisitos pode ser computado
		%em tempo~${O\Big(\dfrac{n}{\diam^*(T)}\Big)}$.
	\end{teo}

	\medskip
	\medskip

	\begin{proof}
		Utilizaremos indução em~$i$ para provar o Teorema~\ref{teo:corteExato}.
		%a primeira parte do Teorema~\ref{teo:corteExato}.
		%(sem levar em conta o tempo de execução por enquanto).
		
		Primeiramente, temos que~${\diam^*(T)\le1}$, como mostrado 
		na Seção~\ref{subsec:diametro}. 
		Sendo assim, sabe-se que~${i\ge1}$, logo~${i=1}$ será a base 
		utilizada na indução.
		
		\textbf{Caso base,~${\mathbf {i=1}}$.}
		%Para~${n\le 2}$, sabe-se que~${|E(T)|=\Delta(T)}$ e, nesse
		%caso,~${e_T(B,W)\le |E(T)|=\Delta(T)\le 4\cdot 2^i\cdot 
		%\Delta(T)}$.
		Como~${\diam^*(T)>\dfrac{1}{2}}$, logo, 
		podemos utilizar o Lema~\ref{lema:caminhoLongo} e obter 
		um corte~$(B,W)$ 
		com~${e_T(B,W)\le 2 < 4\cdot 2^i\cdot \Delta(T)}$.
		Portanto, o teorema é válido para~${i=1}$.


		\textbf{Caso~${\mathbf{i\ge 2}}$.} Provaremos que 
		%a primeira parte 
		o teorema
		vale para~$i$, assumindo que ele é válido para~$i-1$.
		Para~${n\le 2}$, sabe-se que~${|E(T)|=\Delta(T)}$. Portanto, 
		para~${n\le 2}$, temos 
		que~${e_T(B,W)\le |E(T)|=\Delta(T)\le 2\cdot 2^i\cdot 
		\Delta(T)}$.
		Porém, se~${n>2}$, podemos utilizar o 
		Algoritmo~\ref{alg:dobraDiametro}, que nos devolverá um 
		corte~$(B',W',S')$ do tipo~1 ou~2.

			Se o corte~$(B',W',S')$ for do tipo 1, 
			teremos~${|B'|=m}$,~${S'=\emptyset}$
			e~${e_T(B',W')\le2\le 2\cdot 2^i\cdot \Delta(T)}$.
			Logo, temos que o corte~$(B',W')$ satisfaz as propriedades
			do corte do teorema.

			Caso contrário, o corte~$(B',W',S')$ será do tipo 2, ou 
			seja,~${|B'|=m}$ ou~${|B'|<m\le |B'|+|S'|}$, 
			e~${e_T(B',W',S')\le \dfrac{2\cdot\Delta(T)}{\diam^*(T)}\le
			2\cdot2^i\cdot\Delta(T)}$.
			
			Se~${|B'|=m}$, temos então que o corte~$(B',V\setminus B')$ satisfaz
			as propriedades do corte do
			teorema.
			Senão,~${|B'|<m\le |B'|+|S'|}$, que implica
			que~${0<m-|B'|\le|S'|}$. 
			Pelas propriedades do Teorema~\ref{teo:dobraDiametro}, 
			sabemos que~${\diam^*(T[S'])\ge 2\cdot \diam^*(T)>
			\dfrac{1}{2^{i-1}}}$.
			Como~${0<m-|B'|\le|S'|}$ 
			e~${\diam^*(T[S'])>\dfrac{1}{2^{i-1}}}$, podemos aplicar o 
			Teorema~\ref{teo:corteExato} em~$T[S']$ 
			usando~${m-|B'|}$ como~$m$ e~$i-1$ como~$i$.
			
			Porém, existem casos onde o conjunto~$S'$ retornado 
			pelo Teorema~\ref{teo:dobraDiametro} é tal que~$T[S']$ 
			não é conexo, 
			pois, de acordo com o 
			Algoritmo~\ref{alg:dobraDiametro},~$T[S']$ pode se 
			dividir em duas componentes, uma no começo e outra
			no fim do caminho máximo. 

			Isso pode ser visto na figura abaixo. 


			\begin{center} \begin{tikzpicture}
				[scale=.625,auto=left,
				every node/.style={circle, draw=black,
					fill=blue!20}]

				\draw [draw=blue!65,fill=yellow!20, line width=.8pt] (-1,1.5) 
				rectangle (7, 6.5);

				\draw [draw=none, fill=yellow!20, line width=.8pt] (11,4.7) 
				rectangle (19, 6.5);
				\draw [draw=none, fill=yellow!20, line width=.8pt] (13.5,1.5) 
				rectangle (19, 6.5);

				\node [text=blue!65,draw=none, fill=none] at 
				(0,2.5) {\Large{$S$}};
				\node [text=blue!65,draw=none, fill=none] at 
				(18,2.5) {\Large{$S$}};

				\node [subtree] (y1) at (3,4.6)  {};
				\node [subtree, fill=red!30] (y2) at (6,5)  {};
				\node [subtree] (y3) at (9,4.6)  {};
				\node [subtree] (y4) at (12,5) {};
				\node [subtree] (y5) at (15,4.6) {};

				\node (x0) at (0,5.4)  {$v_0$};
				\node (x1) at (3,5)    {...};
				\node (x2) at (6,5.4)  {$v_i$};
				\node (x3) at (9,5)    {...};
				\node (x4) at (12,5.4) {$v_j$};
				\node (x5) at (15,5)   {...};
				\node (x6) at (18,5.4) {$v_p$};

				\foreach \from/\to in {x0/x1,x1/x2,x2/x3,x3/x4,
				x4/x5,x5/x6}
				\draw (\from) -- (\to);

				% Define the points of a regular pentagon
				\path (13.5,4.7) coordinate (P0);
				\path (11,4.7) coordinate (P1);
				\path (11,6.5) coordinate (P2);
				\path (19,6.5) coordinate (P3);
				\path (19,1.5) coordinate (P4);
				\path (13.5,1.5) coordinate (P5);
				% Draw the edges of the pentagon
				\draw[draw=blue!65, line width=.8pt] (P0) -- (P1) -- (P2) -- (P3) 
				-- (P4) -- (P5) -- cycle;


			\end{tikzpicture} \end{center}
			Como o Teorema~\ref{teo:corteExato} se aplica a 
			árvores, precisamos tornar~$S'$ conexo de forma a não 
			alterar o seu grau máximo.
			Sabemos que se adicionarmos uma aresta~$e$ em~$T$, que 
			liga o primeiro ao último vértice do caminho máximo,
			então~$S'$ será conexo. 
			Diremos que~${T^* = T\cup e}$.

			Note também que em ambas as componentes de~$T[S']$, 
			temos que o caminho máximo de cada uma possui em um dos
			seus extremos o vértice~$v_0$ ou o vértice~$v_p$.
			Logo,~$T^*[S']$ será conexo e 
			%se adicionarmos uma aresta~$e=\{v_0,v_p\}$ em~$T$,
			%então~$T[S']$ será uma componente conexa e
			teremos um caminho máximo 
			em~$T^*[S']$ com o maior número possível de vértices.
			Além disso, de acordo com o Lema~\ref{lem:grauMaximo}, 
			temos que~${\Delta(T) = \Delta(T^*)}$, já que a aresta~$e$ 
			conecta os dois vértices extremos do caminho máximo
			e estes possuem grau um.
			Podemos, então, aplicar a indução à árvore~$T^*[S']$ 
			%para o Teorema~\ref{teo:corteExato}, 
			sabendo que não há aumento
			%sem nos preocupar com a modificação 
			do grau máximo e do diâmetro relativo.
			Ou seja,
			%Como assumimos na indução que o Teorema~\ref{teo:corteExato} vale
			%para~${i-1}$, então 
			existe um corte~$(B'',W'')$ em~$T^*[S']$
			tal que~${|B''|=m-|B'|}$ 
			e~${e_{T[S]}(B'',W'')\le 4\cdot 2^{i-1}\cdot
			\Delta(T^*[S'])\le 4\cdot 2^{i-1}\cdot\Delta(T^*)}$.

			Dessa forma, sabemos que existe um corte~$(B,W)$ em~$T$ tal
			que~${B=B'\cup B''}$ e~${W=V(T)\setminus B}$, sendo 
			que~${|B|=|B'| + m-|B'| = m}$ e
			\begin{align}
				e_T(B,W)&\le e_T(B',W',S') + e_{T[S]}(B'',W'') 
				\nonumber\\
				&\le 2\cdot2^i\cdot\Delta(T) + 4\cdot 2^{i-1}\cdot
				\Delta(T^*)\nonumber\\
				&= 2\cdot2^i\cdot\Delta(T) + 4\cdot 2^{i-1}\cdot
				\Delta(T)\nonumber\\
				&= 4\cdot 2^{i}\cdot\Delta(T), \nonumber
			\end{align}
			%provando assim que o Teorema~\ref{teo:corteExato} é
			%válido para~$i$ partindo do pressuposto de que ele vale
			%para~${i-1}$. Logo, temos que o 
			%Teorema~\ref{teo:corteExato} é válido para 
			%todo~${i\in \mathbb{N^+_*}}$.
			completando a prova por indução do teorema.
		%\end{itemize}
	\end{proof}

	\bigskip
	\bigskip

	\begin{coro}[{\cite[Corolário 7]{Schmidt15}}]
	\label{coro:corteExato}
		Para toda árvore~$T$ com~$n$ vértices e todo~${m\in[n]}$, existe
		um corte~$(B,W)$ em~$T$ com~${|B|=m}$ 
		e~${e_T(B,W)\le \dfrac{8}{\diam^*(T)}\Delta(T)}$.
		%Um corte que satisfaz esses requisitos pode ser computado em 
		%tempo~$O\Big(\dfrac{n}{\diam^*(G)}\Big)$.
	\end{coro}

	\begin{proof}
		%De acordo com o Teorema~\ref{teo:corteExato},
		Para toda árvore~$T$,
		como~$0<\diam^*(T)\le 1$,
		%existe um~$i$ tal que~${\diam^*(T)>\dfrac{1}{2^i}}$. 
		%Como~${\dfrac{1}{2^i}}$ é decrescente, então 
		existe um~$j$ tal
		que~${\dfrac{1}{2^j}}<\diam^*(T)\le \dfrac{1}{2^{j-1}}$,
		o que implica que~${2^j\le\dfrac{2}{\diam^*(T)}}$.

		Assim, o Teorema~\ref{teo:corteExato} garante a existência de um 
		corte~$(B,W)$ em~$T$ com~${|B|=m}$ 
		e~$e_T(B,W)\le 4\cdot 2^j\cdot\Delta(T)\le \dfrac{8}{\diam^*(T)} \Delta(T)$,
		e esse corte satisfaz as propriedades do 
		Corolário~\ref{coro:corteExato}. 
		%então podemos afirmar a existência de cortes desse tipo.
	\end{proof}

	\bigskip
	\bigskip
	\bigskip
	\bigskip
	\bigskip

	\subsection{Algoritmo FST}

		%Seja~${T=(V,E)}$ uma árvore com~$n$ vértices.
		% Assumimos que a função INSERE\_ARESTA$(T,v_1,v_2)$ irá 
		% inserir uma aresta em~$T$, ligando os 
		% vértices~$v_1$ e~$v_2$ em tempo constante. 
		O algoritmo FST faz uso das seguintes rotinas.
		A função {\sc caminho\_máximo}$(T)$ 
		recebe~$T$ com~$n$ vértices e
		devolve um vetor contendo os vértices de um caminho 
		máximo em~$T$, de acordo com a ordem apresentada no caminho.
		Isso pode ser feito em tempo~$O(n)$, como descrito na 
		Seção~\ref{sec:caminhoMaximo}.

		A função {\sc rotula}$(T$, $c)$ recebe uma árvore~$T$ com~$n$ vértices 
		e um vetor~$c$
		contendo os vértices de um caminho em~$T$ e,
		devolve dois vetores: o primeiro contém a rotulação dos vértices 
		de~$T$
		descrita na Seção~\ref{sec:rotulacao}, e o segundo, o inverso
		da rotulação. 
		Essa função consome tempo~$O(n)$, como mostrado na 
		Seção~\ref{sec:rotulacao}.

		A função {\sc recalcula\_caminho}$(S$, $c)$
		recebe como parâmetros um conjunto de vértices~$S$ de uma árvore~$T[S]$ 
		e um vetor~$c$, com os vértices de um caminho na árvore~$T$.
		Essa função devolve um vetor que contém os vértices de~$S\cap c$, na mesma 
		ordem que aparecem em~$c$. 
		Isso pode ser feito em tempo~$O(|c|)$ se 
		o conjunto~$S$ é dado por um vetor binário.
		%mantivermos um 
		%vetor binário indicando presença ou ausência dos vértices 
		%em~$V$.

		% A função {\sc recalcula\_caminho}$(V$, $c)$
		% recebe como parâmetros um vetor~$c$, que representa um 
		% caminho na árvore, e um conjunto de vértices~$V$. Essa função devolve
		% um vetor, ordenado de acordo com~$c$, contendo os 
		% vértices em~$c$ que estão no conjunto~$V$. 
		% Isso pode ser feito em tempo~$O(|c|)$ se mantivermos um 
		% vetor binário indicando presença ou ausência dos vértices 
		% em~$V$.

		% Também temos a função {\sc recalcula\_rotulação}$(V$, $r$, $v$, $c)$,
		% um conjunto de vértices~$V$ e três vetores, o primeiro 
		% que mapeia os vértices
		% até as suas rotulações, o segundo que mapeia uma rotulação
		% até o seu vertice e o terceiro representa os vértices de um
		% caminho. 
		% Essa função retorna um vetor 


		% Seja~$vet$ um vetor, temos que as funções PRIMEIRO$(vet)$ 
		% e ULTIMO$(vet)$ devolvem o primeiro e o último elementos 
		% do vetor, respectivamente em tempo constante.

		Segue o algoritmo FST que encontra um corte que satisfaz as 
		propriedades do Teorema~\ref{teo:corteExato}.
		
		\bigskip

		\begin{algorithm}[H]
		\label{alg:corteExato}
			\SetKwInOut{Input}{input}
			\SetKwInOut{Output}{output}

			\caption{Computa corte exato em uma árvore}
			\Input{árvore $T=(V,E)$ com $n$ vértices e $m\in[n]$}
			\Output{corte $(B,W)$ tal que $|B|=m$ 
			e~$e_T(B,W)\le \dfrac{8}{\diam^*(T)}\Delta(T)$}
			$B\gets \emptyset$\;
			$S'\gets V$\;
			$r\gets$ um vértice arbitrário de~$T$\;
			$P\gets$ {\sc caminho\_máximo}$(T)$\;
			$[rot$, $ind]\gets$ {\sc rotula}$(T$, $P)$\;
			$[(B',W',S')$, $T^*[S']$, $r]\gets
				\mathbf {Algoritmo\ref{alg:dobraDiametro}}(T$,
				$|V|$, $m$, $P$, $rot$, $ind$,~$r)$\;
			\While{$|B|<m$}
			{
				\tcp{$r$ receberá um vértice que pertence ao novo conjunto~$S'$}
				$[(B',W',S')$, $T^*[S']$, $r]\gets
					\mathbf {Algoritmo\ref{alg:dobraDiametro}}(T^*[S']$,
					$|S'|$, $m-|B|$, $P$, $rot$, $ind$,~$r)$\;
				$B\gets B\cup B'$\;
				$P\gets$ {\sc recalcula\_caminho}$(S'$, $P)$\;
				$[rot$, $ind]\gets$ {\sc rotula}$(T^*[S']$, $P)$
				% \If{$T[S']$ não é conexo}
				% {
				% 	INSERE\_ARESTA$(T[S']$, PRIMEIRO$(P)$, 
				% 	ULTIMO$(P))$\;
				% }
			}
			\Return $(B,V\setminus B)$\;

		\end{algorithm}	

		\bigskip

		\subsection*{Análise do Algoritmo}

		% Sabe-se que calcular um caminho máximo faz parte do
		% Teorema~\ref{teo:dobraDiametro} e deveria estar no
		% Algoritmo~\ref{alg:dobraDiametro}, mas, precisamos 
		%acessar 
		% esse caminho máximo para, possivelmente, acrescentar a 
		% aresta citada na prova do Teorema~\ref{teo:corteExato}, 
		% portanto, encontrar um caminho máximo fará parte do 
		% Algoritmo~\ref{alg:corteExato}.

		No algoritmo, temos que 
		retirar arestas de~$G$ para 
		desconectar os vértices de~$S'$ do resto da árvore
		e acrescentar a aresta~$e$, como descrito no teorema.
		Poderíamos desconectar os vértices verificando todas as arestas que 
		ligam dois vértices de~$S'$ e colocar em~$T[S']$, mas podemos
		fazer isso no Algoritmo~\ref{alg:dobraDiametro} de forma 
		mais eficiente. 
		Sabemos que, no Algoritmo~\ref{alg:dobraDiametro},
		analisamos os vértices de~$S'$ que se ligam a vértices 
		de~$V\setminus S'$. 
		Portanto, podemos usar uma marcação para indicar que as 
		arestas que ligam~$S'$ e~${V\setminus S}$ estão inativas, 
		tornando $S'$ uma componente conexa isolada. 
		Obter a árvore~$T^*$ acrescentando a aresta~$e$ em~$T$, como descrito no
		Teorema~\ref{teo:corteExato}, também pode ser feito de 
		forma mais simples no Algoritmo~\ref{alg:dobraDiametro},
		da mesma forma que foi feito o procedimento citado acima.
		Desta forma, o Algoritmo~\ref{alg:dobraDiametro} retornará
		a componente~$T^*[S']$, que, nesse caso, é a árvore~$T^*$ com
		os vértices de~$S'$ isolados do resto de~$T^*$.

		Como usaremos apenas a árvore de~$S'$ e~$T^*[S']$ possui outros 
		vértices além dos vértices de~$S'$,
		então, precisamos mandar um vértice de~$S'$ para
		o Algoritmo~\ref{alg:dobraDiametro},
		para que ele saiba que a árvore tratada em questão é
		somente a componente conexa que possui o vértice recebido.
		Assim, o Algoritmo~\ref{alg:dobraDiametro} irá considerar
		apenas a componente conexa de~$S'$ e ainda
		poderá usar o vértice recebido como raiz.

		%também devolverá 
		%algum vértice de~$S'$ que será usado como raiz na 
		%próxima interação do Algoritmo~\ref{alg:dobraDiametro}.

		Temos uma outra modificação no 
		%Algoritmo~\ref{alg:corteExato}   
		Algoritmo~5   
		em relação ao Teorema~\ref{teo:corteExato}.
		Em vez de calcularmos novamente o caminho máximo da nova
		árvore, podemos reutilizar o caminho calculado anteriormente,
		aproveitando somente a parte do caminho que está contida 
		em~$S'$. 
		Se usarmos essa parte como caminho máximo,
		manteremos a propriedade da dobra do diâmetro como pode
		ser visto na equação~\ref{eq:reaproveitaCaminho} da 
		Seção~\ref{sec:dobraDiametro}. 
		Dessa forma, a rotulação continua basicamente a mesma,
		basta apenas "deslocar" os vértices para a esquerda.

		\bigskip

		Agora vamos analisar o consumo de tempo do algoritmo.
		Sabemos que o Algoritmo~5 
		%Sabemos que o Algoritmo~\ref{alg:corteExato} 
		irá executar
		o Algoritmo~\ref{alg:dobraDiametro} que consome 
		tempo ~${O\Big(\dfrac{n}{\diam^*(T)}\Big)}$, a função
		{\sc recalcula\_caminho} que consome tempo~$O(n)$, e a função
		{\sc rotula} que consome tempo~$O(n)$.
		Sendo~$i$ um inteiro tal que~$\diam^*(T)>\dfrac{1}{2^i}$, 
		temos que cada uma das operações citadas anteriormente
		irá ser executada no máximo~$i$ vezes.

		Considere que~$\ell$ representa a interação,
		começando do~$i$ e indo até~1, e~$T_\ell$ e~$n_\ell$ 
		representam a árvore da interação~$\ell$ e
		o número de vértices dessa árvore, respectivamente.
		Dado que usamos o~$T^*[S']$ como a nova árvore
		e sabendo que~$|S'|\le \dfrac{n}{2}$,
		temos que o número de vértices de~$T_{\ell}$
		é maior ou igual ao 
		dobro do número de vértices de~$T_{\ell-1}$, logo
		\begin{align}
			 & n_{\ell} \ge 2\cdot n_{\ell-1} \ge 4\cdot n_{\ell-2} \nonumber\\
			\Rightarrow & n = 2^0\cdot n_i\ge 2^1\cdot n_{i-1} \ge\cdots\ge
			2^{i-j}n_j\ge\cdots\ge
			 2^{i-2}\cdot n_{2}\ge 2^{i-1}\cdot n_1. \nonumber 
		\end{align}
		Logo, para todo~$\ell\in[i]$, temos que~$n_\ell \le\dfrac{n} {2^{i-\ell}}$.
		Portanto temos que o tempo de execução é
		\begin{align}
			\displaystyle\sum_{\ell=1}^{i} \Big(O\Big(\dfrac{n_\ell}{\diam^*(T)}\Big)
			+ O(n_\ell) + O(n_\ell)\Big)
			&= \displaystyle\sum_{\ell=1}^{i} O\Big(\dfrac{n_\ell}{\diam^*(T)}\Big) \nonumber\\
			&\le \displaystyle\sum_{\ell=1}^{i} O\Big(\dfrac{1}{2^{i-\ell}}
			\cdot\dfrac{n}{\diam^*(T)}\Big) \nonumber\\
			&\le O\Big(\dfrac{n}{\diam^*(T)}\Big). \nonumber
		\end{align}

		Logo, o algoritmo leva tempo~$O\Big(\dfrac{n}{\diam^*(T)}\Big)$.


%%%%%%%%%%%%%%%%%%%%%%%%%%%%%%%%%%%%%%%%%%%%%%%%%%%%%%%%%%%%%%%%%%%
%%%%%%%%%%%%%%%%%%%%%%%  SEÇÃO 10    %%%%%%%%%%%%%%%%%%%%%%%%%%%%%%
%%%%%%%%%%%%%%%%%%%%%%%%%%%%%%%%%%%%%%%%%%%%%%%%%%%%%%%%%%%%%%%%%%%

\section {Complexidade do problema da bissecção}
	Nesta seção veremos duas reduções que provam que o problema 
	da bissecção mínima é NP-Completo,
	usando o fato de
	que a versão de decisão do
	problema {\sc max2sat} é NP-completa.

	Primeiramente, vamos definir alguns problemas para facilitar
	a compreensão das reduções.

	\medskip

	\begin{prob}[{\sc max2sat}($C,k$) -- Satisfatibilidade máxima com no máximo 
	dois literais por cláusula{~\cite{GareyJS76}}]
		Dado um conjunto~$C$ com~$p$ cláusulas distintas na forma 
		disjuntiva, com no máximo
		dois literais cada, e um inteiro~$k\le p$,
		decidir se existem valores para cada uma das variáveis, de forma
		que~$k$ ou mais cláusulas sejam verdadeiras.

		%~$C_i=(x_1 x_2)$,
	\end{prob}

	\medskip

	\begin{prob}[MaxCut($G,W$) -- Corte Máximo]
		Dado um grafo~${G}$
		%onde cada uma das arestas tem peso 1, 
		e um inteiro positivo~${W}$, decidir se existe um
		corte~$(T,F)$ em $G$ tal 
		que~${e_G(T,F)\ge W}$.
		
	\end{prob}

	

	% \begin{prob}[Corte mínimo em subconjuntos de mesmo tamanho 
	% {~\cite{GareyJS76}}]
	% 	Dado um grafo~$G(V,E)$, onde cada uma das arestas tem 
	% 	peso 1, e um inteiro positivo~$W\le|E|$, encontrar um
	% 	conjunto de vértices~$S\subseteq V$ tal 
	% 	que~$e_G(S,V\setminus~S)\ge W$.
		
	% \end{prob}

	\bigskip
	\bigskip
	\bigskip

	\subsection{Redução do \textbf{\textsc {max2sat}} para MaxCut}

	\begin{teo}
		A instância~$(G,W)$ de MaxCut tem solução sse
		a instância~$(C,k)$ de {\sc max2sat} tem solução.
	\end{teo}

		Dada uma instância~$(C,k)$ do problema {\sc max2sat}, 
		precisamos construir uma instância~$(G,W)$
		do problema MaxCut.
		Para isso vamos definir alguns conceitos referentes
		ao problema {\sc max2sat}:
		\begin{itemize}
			\item $p = $ número de cláusulas
			\item $n = $ número de variáveis
		\end{itemize}

		Primeiramente, vamos definir essa redução para depois
		explicar o porquê ela funciona.
		A redução consiste em criar um grafo~$G=(V,E)$, de 
		acordo com uma entrada para o problema {\sc max2sat}.
		Segue o conjunto de vértices e arestas de~$G$, 
		respectivamente.
		
		\begin{align}
			V =  &\{x_i: 1\le i\le n\} \ \cup \nonumber\\
				&\{\overline{x_i}: 1\le i\le n\} \ \cup\nonumber\\ 
				&\{T_j: 0\le j\le 3p\} \ \cup \nonumber\\
				&\{F_j: 0\le j\le 3p\} \ \cup \nonumber\\
				&\{t_{ij}: 1\le i\le n, 0\le j\le 3p\} 
					\ \cup \nonumber\\
				&\{f_{ij}: 1\le i\le n, 0\le j\le 3p\}\nonumber
		\end{align}

		\begin{align}
			E_1=&\{ \{T_i,F_j\}: 0\le i\le 3p, 0\le j\le 3p\}\ \cup 
					\nonumber\\
				&\{\{t_{ij}, f_{ij}\}: 1\le i\le n, 
					0\le j\le 3p\}\ \cup \nonumber \\
				&\{\{x_i, f_{ij}\}: 1\le i\le n, 
					0\le j\le 3p\} \ \cup \nonumber \\
				&\{\{\overline{x_i}, t_{ij}\}: 1\le 
					i\le n, 0\le j\le 3p\} \nonumber
		\end{align}

		% \begin{align}
		% 	E_1=&\textcolor{blue}{\{ \{T_i,F_j\}: 0\le i\le 
		% 		3p, 0\le j\le 3p\}} &\cup \nonumber\\
		% 		&\textcolor{green!70}{\{\{t_{ij}, f_{ij}\}: 1\le i\le n, 
		% 			0\le j\le 3p\}} &\cup \nonumber \\
		% 		&\textcolor{brown}{\{\{x_i, f_{ij}\}: 1\le i\le n, 
		% 			0\le j\le 3p\}} &\cup \nonumber \\
		% 		&\textcolor{violet}{\{\{\overline{x_i}, t_{ij}\}: 1\le 
		% 			i\le n, 0\le j\le 3p\}} \nonumber \\
		% \end{align}

		E para cada uma das~$p$ cláusulas da forma~$C_i=(a_i\lor b_i)$, 
		temos também as arestas a seguir.

		\begin{align}
			E_2=&\{ \{a_i,b_i\}: 1\le i\le p, a_i\ne b_i \}\ \cup 
					\nonumber\\
				&\{\{a_i, F_{2\cdot i-1}\}: 1\le i\le p\}\ \cup \nonumber \\
				&\{\{b_i, F_{2\cdot i}\}: 1\le i\le p\} \nonumber
		\end{align}

		Dessa forma, temos o grafo~${G = (V,E)}$, com~${E=E_1\cup E_2}$.
		Segue abaixo um exemplo de como ficaria~$G$ sendo gerado por
		uma cláusula do tipo~${(x_1\lor \overline{x_2})}$.
		As arestas de~$E_2$ estão representadas pela cor 
		\textcolor{red}{\textbf {vermelha}}.

		\bigskip
		\bigskip

		\begin{center} \begin{tikzpicture}
			[scale=.7,auto=left,every node/.style={circle, 
			draw=black,
			fill=blue!20}]
			
			\node (t0) at (4, 13)  {$T_0$};
			\node (t1) at (7, 13)  {$T_1$};
			\node (t2) at (10,13)  {$T_2$};
			\node (t3) at (13,13)  {$T_3$};

			\node (f0) at (4, 10)  {$F_0$};
			\node (f1) at (7, 10)  {$F_1$};
			\node (f2) at (10,10)  {$F_2$};
			\node (f3) at (13,10)  {$F_3$};




			\node (t10) at (2.5,6)  {$t_{10}$};
			\node (t11) at (2.5,4)  {$t_{11}$};
			\node (t12) at (2.5,2)  {$t_{12}$};
			\node (t13) at (2.5,0)  {$t_{13}$};	

			\node (f10) at (4.5,6)  {$f_{10}$};
			\node (f11) at (4.5,4)  {$f_{11}$};
			\node (f12) at (4.5,2)  {$f_{12}$};
			\node (f13) at (4.5,0)  {$f_{13}$};
			  
			\node (f20) at (14.5,6)  {$f_{20}$};
			\node (f21) at (14.5,4)  {$f_{21}$};
			\node (f22) at (14.5,2)  {$f_{22}$};
			\node (f23) at (14.5,0)  {$f_{23}$};

			\node (t20) at (12.5,6)  {$t_{20}$};
			\node (t21) at (12.5,4)  {$t_{21}$};
			\node (t22) at (12.5,2)  {$t_{22}$};
			\node (t23) at (12.5,0)  {$t_{23}$};

			
			


			\node (x1n) at (0,3)  {$\overline{x_1}$};
			\node (x1)  at (7,3)  {$x_1$};
			\node (x2n)  at (10,3)  {$\overline{x_2}$};
			\node (x2) at (17,3)  {$x_2$};


			\foreach \from/\to in {t0/f0,t0/f1,t0/f2,t0/f3,
			t1/f0,t1/f1,t1/f2,t1/f3,
			t3/f0,t3/f1,t3/f2,t3/f3,
			t2/f0,t2/f1,t2/f2,t2/f3}%%%%
			\draw[] (\from) -- (\to);

			\foreach \from/\to in {t10/f10,t11/f11,t12/f12,t13/f13,
			t20/f20,t21/f21,t22/f22,t23/f23}%%%%
			\draw[] (\from) -- (\to);

			\foreach \from/\to in {t10/x1n,t11/x1n,t12/x1n,t13/x1n,
			t20/x2n,t21/x2n,t22/x2n,t23/x2n}
			\draw[] (\from) -- (\to);

			\foreach \from/\to in {x1/f10,x1/f11,x1/f12,x1/f13,
			x2/f20,x2/f21,x2/f22,x2/f23}
			\draw[] (\from) -- (\to);

			\foreach \from/\to in {x1/x2n,x1/f1,x2n/f2}
			\draw[draw=red!60, line width=2.5pt] (\from) -- (\to);
		\end{tikzpicture} \end{center}

		\bigskip
		\bigskip

		Se encontrarmos um corte~$(T,F)$ em~$G$ tal 
		que~${e_G(T, F)\ge |A_1| + 2\cdot k}$, 
		então teremos que, para todo vértice do tipo~$x_i$,
		se~${x_i\in T}$, a variável que ele representa será
		verdadeira, já se~${x_i\in F}$, a variável que ele representa
		será falsa. Tendo que as variáveis possuem os valores conforme
		o que foi dito anteriormente, $k$ ou mais cláusulas serão 
		verdadeiras.

		\bigskip
		\bigskip
		\bigskip
		\bigskip
		\bigskip

		\textbf{Explicação:}

		\bigskip

		Mostraremos agora que para todas as entradas do problema
		{\sc max2sat}, existe uma solução da sua redução para o problema
		MaxCut.
		Para isso, vamos supor que já temos um conjunto de valores
		para as variáveis do problema {\sc max2sat} de forma que~$k$ ou mais
		cláusulas sejam verdadeiras.

		Sabemos que existem no máximo~$2p$ arestas do 
		tipo~${\{x_i,F_j\}}$ ou ${\{\overline{x_i},F_j\}}$. Dado 
		que~${2p<3p+1}$,
		que as funções~${f(i)=2\cdot i - 1}$ e~${g(i)=2\cdot i}$ têm 
		imagem ímpar e par, respectivamente, e ambas são injetoras, 
		portanto, temos que cada~$F_i$ se liga a apenas um vértice
		do tipo~$x_i$ ou~${\overline{x_i}}$.

		Definiremos agora os vértices contidos nos 
		subconjuntos~$T$ e~$F$ de~$G$.
		Suponhamos que para todo~${i\in[0,3p]}$,~${T_i\in T}$ 
		e~${F_i\in F}$ e
		para todo~${j\in[1,n]}$ e ${k\in[0,3p]}$, ~$x_j$ pertence ao 
		mesmo subconjunto que~$t_{jk}$ e ambos não pertencem ao mesmo 
		subconjunto que~$\overline{x_j}$ e que~$f_{jk}$.
		Assim sendo, temos que todas as arestas de~$E_1$ estão no 
		corte~$(T,F)$.
		Suponhamos também que se a variável~$x_i$ é verdadeira,
		então, o vértice~$x_i$ está em $T$, e se mantermos as 
		propriedades que foram citadas anteriormente, isso não afetará o 
		fato de todas as arestas de~$E_1$ estarem no corte~$(T,F)$.
		Portanto, assumiremos inicialmente que essa é a disposição dos 
		vértices nos subconjuntos~$T$ e~$F$.

		Temos que, em uma cláusula~${C_i=(a_i,b_i)}$, ou
		apenas um dos literais é satisfeito, ambos são ou nenhum
		deles é.
		\begin{enumerate}
			\item Se apenas um dos literais é satisfeito, temos que,
			das arestas formadas pela cláusula~$C_i$, duas
			estão no corte~$(T,F)$ e uma está fora.
			\item Se os dois literais são satisfeitos, também teremos
			duas arestas no corte~$(T,F)$ e uma fora.
			\item Já no caso de nenhum literal ser satisfeito, nenhuma
			dessas arestas da cláusula estará no corte~$(T,F)$.
		\end{enumerate}

		Podemos ver isso de forma mais clara na imagem a seguir.

		\begin{center} \begin{tikzpicture}
			[scale=.24,auto=left,every node/.style={circle, 
			draw=black,
			fill=blue!20}]
			


			\node (t0_) at (4 +20,13)  {};
			\node (t1_) at (7 +20,13)  {};
			\node (t2_) at (10+20,13)  {};
			\node (t3_) at (13+20,13)  {};

			\node [fill=violet!50](f0_) at (4 +20,10)  {};
			\node [fill=violet!50](f1_) at (7 +20,10)  {};
			\node [fill=violet!50](f2_) at (10+20,10)  {};
			\node [fill=violet!50](f3_) at (13+20,10)  {};

			\node (t10_) at (2.5+20,6)  {};
			\node (t11_) at (2.5+20,4)  {};
			\node (t12_) at (2.5+20,2)  {};
			\node (t13_) at (2.5+20,0)  {};	

			\node [fill=violet!50](f10_) at (4.5+20,6)  {};
			\node [fill=violet!50](f11_) at (4.5+20,4)  {};
			\node [fill=violet!50](f12_) at (4.5+20,2)  {};
			\node [fill=violet!50](f13_) at (4.5+20,0)  {};
			  
			\node (f20_) at (14.5+20,6)  {};
			\node (f21_) at (14.5+20,4)  {};
			\node (f22_) at (14.5+20,2)  {};
			\node (f23_) at (14.5+20,0)  {};

			\node [fill=violet!50](t20_) at (12.5+20,6)  {};
			\node [fill=violet!50](t21_) at (12.5+20,4)  {};
			\node [fill=violet!50](t22_) at (12.5+20,2)  {};
			\node [fill=violet!50](t23_) at (12.5+20,0)  {};
			
			\node [fill=violet!50](x1n_) at (0 +20, 3)  {};
			\node (x1_)  at (7 +20, 3)  {};
			\node (x2n_) at (10+20, 3) {};
			\node [fill=violet!50](x2_)  at (17+20, 3) {};






			\node (t0__) at (4 +40,13)  {};
			\node (t1__) at (7 +40,13)  {};
			\node (t2__) at (10+40,13)  {};
			\node (t3__) at (13+40,13)  {};

			\node [fill=violet!50](f0__) at (4 +40,10)  {};
			\node [fill=violet!50](f1__) at (7 +40,10)  {};
			\node [fill=violet!50](f2__) at (10+40,10)  {};
			\node [fill=violet!50](f3__) at (13+40,10)  {};

			\node [fill=violet!50](t10__) at (2.5+40,6)  {};
			\node [fill=violet!50](t11__) at (2.5+40,4)  {};
			\node [fill=violet!50](t12__) at (2.5+40,2)  {};
			\node [fill=violet!50](t13__) at (2.5+40,0)  {};	

			\node (f10__) at (4.5+40,6)  {};
			\node (f11__) at (4.5+40,4)  {};
			\node (f12__) at (4.5+40,2)  {};
			\node (f13__) at (4.5+40,0)  {};
			  
			\node [fill=violet!50](f20__) at (14.5+40,6)  {};
			\node [fill=violet!50](f21__) at (14.5+40,4)  {};
			\node [fill=violet!50](f22__) at (14.5+40,2)  {};
			\node [fill=violet!50](f23__) at (14.5+40,0)  {};

			\node (t20__) at (12.5+40,6)  {};
			\node (t21__) at (12.5+40,4)  {};
			\node (t22__) at (12.5+40,2)  {};
			\node (t23__) at (12.5+40,0)  {};
			
			\node (x1n__) at (0 +40, 3)  {};
			\node [fill=violet!50](x1__)  at (7 +40, 3)  {};
			\node [fill=violet!50](x2n__) at (10+40, 3) {};
			\node (x2__)  at (17+40, 3) {};






			\node (t0) at (4 ,13)  {};
			\node (t1) at (7 ,13)  {};
			\node (t2) at (10,13)  {};
			\node (t3) at (13,13)  {};

			\node [fill=violet!50](f0) at (4 ,10)  {};
			\node [fill=violet!50](f1) at (7 ,10)  {};
			\node [fill=violet!50](f2) at (10,10)  {};
			\node [fill=violet!50](f3) at (13,10)  {};



			\node [fill=violet!50](t10) at (2.5,6)  {};
			\node [fill=violet!50](t11) at (2.5,4)  {};
			\node [fill=violet!50](t12) at (2.5,2)  {};
			\node [fill=violet!50](t13) at (2.5,0)  {};	

			\node (f10) at (4.5,6)  {};
			\node (f11) at (4.5,4)  {};
			\node (f12) at (4.5,2)  {};
			\node (f13) at (4.5,0)  {};
			  
			\node (f20) at (14.5,6)  {};
			\node (f21) at (14.5,4)  {};
			\node (f22) at (14.5,2)  {};
			\node (f23) at (14.5,0)  {};

			\node [fill=violet!50](t20) at (12.5,6)  {};
			\node [fill=violet!50](t21) at (12.5,4)  {};
			\node [fill=violet!50](t22) at (12.5,2)  {};
			\node [fill=violet!50](t23) at (12.5,0)  {};


			\node (x1n) at (0,3)  {};
			\node [fill=violet!50](x1)  at (7,3)  {};
			\node (x2n) at (10,3) {};
			\node [fill=violet!50](x2)  at (17,3) {};



			\node (l1) at (1,-7)  {};
			\node [fill=violet!50](l2)  at (1,-5)  {};
			\node [draw=none, fill=none] at 
				(7,-5) {\small{vértice de $F$}};
			\node [draw=none, fill=none] at 
				(7,-7) {\small{vértice de $T$}};
			\node [draw=none, fill=none](a1) at (35,-5) {};
			\node [draw=none, fill=none](a1_) at (39,-5) {};
			\node [draw=none, fill=none](a2) at (35,-7) {};
			\node [draw=none, fill=none](a2_) at (39,-7) {};
			\node [draw=none, fill=none] at 
				(45,-5) {\small{aresta $\in(T,F)$}};
			\node [draw=none, fill=none] at 
				(45,-7) {\small{aresta $\not\in(T,F)$}};
			\foreach \from/\to in {t0_/f0_,t0_/f1_,t0_/f2_,t0_/f3_,
			t1_/f0_,t1_/f1_,t1_/f2_,t1_/f3_,
			t3_/f0_,t3_/f1_,t3_/f2_,t3_/f3_,
			t2_/f0_,t2_/f1_,t2_/f2_,t2_/f3_}%%%%
			\draw[] (\from) -- (\to);

			\foreach \from/\to in 
			{t10_/f10_,t11_/f11_,t12_/f12_,t13_/f13_,
			t20_/f20_,t21_/f21_,t22_/f22_,t23_/f23_}%%%%
			\draw[] (\from) -- (\to);

			\foreach \from/\to in 
			{t10_/x1n_,t11_/x1n_,t12_/x1n_,t13_/x1n_,
			t20_/x2n_,t21_/x2n_,t22_/x2n_,t23_/x2n_}
			\draw[] (\from) -- (\to);

			\foreach \from/\to in 
			{x1_/f10_,x1_/f11_,x1_/f12_,x1_/f13_,
			x2_/f20_,x2_/f21_,x2_/f22_,x2_/f23_,x1_/f1_,x2n_/f2_}
			\draw[] (\from) -- (\to);

			\foreach \from/\to in {x1_/x2n_}
			\draw[draw=red!60, line width=2.7pt] (\from) -- (\to);


			

			\foreach \from/\to in {t0__/f0__,t0__/f1__,t0__/f2__,t0__/f3__,
			t1__/f0__,t1__/f1__,t1__/f2__,t1__/f3__,
			t3__/f0__,t3__/f1__,t3__/f2__,t3__/f3__,
			t2__/f0__,t2__/f1__,t2__/f2__,t2__/f3__}%%%%
			\draw[] (\from) -- (\to);

			\foreach \from/\to in 
			{t10__/f10__,t11__/f11__,t12__/f12__,t13__/f13__,
			t20__/f20__,t21__/f21__,t22__/f22__,t23__/f23__}%%%%
			\draw[] (\from) -- (\to);

			\foreach \from/\to in 
			{t10__/x1n__,t11__/x1n__,t12__/x1n__,t13__/x1n__,
			t20__/x2n__,t21__/x2n__,t22__/x2n__,t23__/x2n__}
			\draw[] (\from) -- (\to);

			\foreach \from/\to in 
			{x1__/f10__,x1__/f11__,x1__/f12__,x1__/f13__,
			x2__/f20__,x2__/f21__,x2__/f22__,x2__/f23__,a1/a1_}
			\draw[] (\from) -- (\to);

			\foreach \from/\to in {x1__/x2n__,x1__/f1__,x2n__/f2__,a2/a2_}
			\draw[draw=red!60, line width=2.7pt] (\from) -- (\to);




			\foreach \from/\to in {t0/f0,t0/f1,t0/f2,t0/f3,
			t1/f0,t1/f1,t1/f2,t1/f3,
			t3/f0,t3/f1,t3/f2,t3/f3,
			t2/f0,t2/f1,t2/f2,t2/f3}%%%%
			\draw[] (\from) -- (\to);

			\foreach \from/\to in {t10/f10,t11/f11,t12/f12,t13/f13,
			t20/f20,t21/f21,t22/f22,t23/f23}%%%%
			\draw[] (\from) -- (\to);

			\foreach \from/\to in {t10/x1n,t11/x1n,t12/x1n,t13/x1n,
			t20/x2n,t21/x2n,t22/x2n,t23/x2n}
			\draw[] (\from) -- (\to);

			\foreach \from/\to in {x1/f10,x1/f11,x1/f12,x1/f13,
			x2/f20,x2/f21,x2/f22,x2/f23,x2n/f2,x1/x2n}
			\draw[] (\from) -- (\to);

			\foreach \from/\to in {x1/f1}
			\draw[draw=red!60, line width=2.7pt] (\from) -- (\to);
		\end{tikzpicture} \end{center}

		Sabe-se que temos~$2k$ ou mais arestas de~$E_1$ que estão no
		corte~$(T,F)$, pois, para cada uma das~$k$ cláusulas
		verdadeiras, teremos duas arestas de~$E_1$ no corte, como
		o mostrado anteriormente.

		Se houvesse alguma aresta duplicada (duas arestas que ligam
		os mesmos vértices), teríamos um problema na contagem de
		arestas dentro e fora do corte.
		Isso poderia ocorrer no caso de duas cláusulas serem 
		idênticas,~${C_i=C_j=(a_i,b_i)}$.
		Isso faria com que houvesse uma aresta duplicada entre
		os vértices~$a_1$ e~$b_i$, mas isso não acontece, dado que
		no enunciado é assumido que as cláusulas são distintas.
		Também não há o problema de dois vértices estarem em cláusulas
		diferentes, dado que para cada cláusula, os vértices se ligam
		a um~$F_i$ diferente.
		E mesmo se repetirmos as variáveis na mesma cláusula, cada
		elemento da cláusula se ligará a um~$F_i$ diferente.

		\bigskip

		Mostramos que existe uma solução, em MaxCut, 
		para todas as entradas do
		problema {\sc max2sat} reduzidas para MaxCut, de forma que
		a solução do MaxCut leva a uma solução de {\sc max2sat}.
		Agora mostraremos que qualquer solução obtida em 
		uma redução para MaxCut levará a uma solução de {\sc max2sat}.


\subsection{Redução MaxCut para Bissecção}

% \begin{center} \begin{tikzpicture}
% 	[scale=.7,auto=left,every node/.style={circle, 
% 	draw=black,
% 	fill=blue!20}]

% 	\node (an1) at (-3 -7, 1) {$v_1$};
% 	\node (an2) at (-5 -7, 2) {$v_2$};
% 	\node (an3) at (-3 -7, 10) {$v_3$};
% 	\node (an4) at (-5 -7, 9) {$v_4$};
% 	\node (an5) at (-6 -7, 6) {$v_5$};

% 	%legenda
% 	\node [draw=none, fill=none](a1)  at (-5  ,-0.5) {};
% 	\node [draw=none, fill=none](a1_) at (-6.4,-0.5) {};
% 	\node [draw=none, fill=none](a2)  at (-5  ,-1.5) {};
% 	\node [draw=none, fill=none](a2_) at (-6.4,-1.5) {};

% 	\node [draw=none, fill=none] at 
% 		(-3,-0.5) {\small{aresta $\in(T,F)$}};
% 	\node [draw=none, fill=none] at 
% 		(-3,-1.5) {\small{aresta $\not\in(T,F)$}};
	

% 	\foreach \from/\to in {an1/an2,an1/an3,an1/an4,an2/an4,an2/an5,
% 	a1/a1_}
% 	\draw[draw=red!60, line width=2.7pt] (\from) -- (\to);

% 	\foreach \from/\to in {an1/an5,an2/an3,an3/an4,an3/an5,an4/an5,
% 	a2/a2_}
% 	\draw[draw=CornflowerBlue,line width=1.3pt] (\from) -- (\to);

% \end{tikzpicture} \end{center}


\begin{center} \begin{tikzpicture}
	[scale=.6,auto=left,every node/.style={circle, 
	draw=black,
	fill=blue!20}]

	\draw [rotate around={60:(-13,1.5)}, draw=YellowOrange, fill=YellowOrange!25,line width=3pt] (-13,1.5) 
		ellipse (1.3cm and 2.4cm);
	\draw [rotate around={143:(-13.5,8)}, draw=NavyBlue, fill=NavyBlue!20,line width=3pt] (-13.5,8) 
		ellipse (2cm and 3.8cm);
	



	\draw [rotate around={60:(-4,1.5)}, draw=YellowOrange, fill=YellowOrange!25,line width=3pt] (-4,1.5) 
		ellipse (1.3cm and 2.4cm);
	\draw [rotate around={143:(-4.5,8)}, draw=NavyBlue, fill=NavyBlue!20,line width=3pt] (-4.5,8) 
		ellipse (2cm and 3.8cm);




	\draw [rotate around={-60:(4,1.5)}, draw=red!80, fill=red!20,line width=3pt] (4,1.5) 
		ellipse (1.3cm and 2.4cm);
	\draw [rotate around={-143:(4.5,8)}, draw=Emerald, fill=Emerald!25,line width=3pt] (4.5,8) 
		ellipse (2cm and 3.8cm);
	

	\node [text=YellowOrange,draw=none, fill=none] at (-14.4,0.1){$\mathbf{V_1}$};
	\node [text=NavyBlue,draw=none, fill=none] at (-15.8,9.3){$\mathbf{V_2}$};

	\node [text=YellowOrange,draw=none, fill=none] at (-5.5,0.1){$\mathbf{V_1}$};
	\node [text=NavyBlue,draw=none, fill=none] at (-6.8,9.3){$\mathbf{V_2}$};


	\node [text=red!80,draw=none, fill=none] at (5.4,0.1){$\mathbf{U_2}$};
	\node [text=Emerald,draw=none, fill=none] at (6.8,9.3){$\mathbf{U_1}$};

	\node (an1) at (-12, 1) {$v_1$};
	\node (an2) at (-14, 2) {$v_2$};
	\node (an3) at (-12, 10) {$v_3$};
	\node (an4) at (-14, 9) {$v_4$};
	\node (an5) at (-15, 6) {$v_5$};

	\node (v1) at (-3, 1)  {$v_1$};
	\node (v2) at (-5, 2)  {$v_2$};
	\node (v3) at (-3, 10) {$v_3$};
	\node (v4) at (-5, 9)  {$v_4$};
	\node (v5) at (-6, 6)  {$v_5$};

	\node (u1) at (3, 1)  {$u_1$};
	\node (u2) at (5, 2)  {$u_2$};
	\node (u3) at (3, 10) {$u_3$};
	\node (u4) at (5, 9)  {$u_4$};
	\node (u5) at (6, 6)  {$u_5$};

	%legenda
	\node [draw=none, fill=none](a1)  at (2  ,-1.5) {};
	\node [draw=none, fill=none](a1_) at (0.4,-1.5) {};
	\node [draw=none, fill=none](a2)  at (2  ,-2.5) {};
	\node [draw=none, fill=none](a2_) at (0.4,-2.5) {};

	\node [draw=none, fill=none] at 
		(3.5,-1.5) {\small{aresta $\in A$}};
	\node [draw=none, fill=none] at 
		(3.5,-2.5) {\small{aresta $\not\in A'$}};
	

	\foreach \from/\to in {v1/v4,v1/v2,v1/v3,v2/v4,v2/v5,
	a1/a1_,
	an1/an4,an1/an2,an1/an3,an2/an4,an2/an5}
	\draw[draw=red!60, line width=2.7pt] (\from) -- (\to);

	\foreach \from/\to in {v1/v5,v2/v3,v3/v4,v3/v5,v4/v5,
	u1/u2,u1/u3,u1/u4,u1/u5,u2/u3,u2/u4,u2/u5,u3/u4,u3/u5,u4/u5,
	v1/u1, v1/u2, v1/u3, v1/u4, v1/u5,
	v2/u1, v2/u2, v2/u3, v2/u4, v2/u5,
	v3/u1, v3/u2, v3/u3, v3/u4, v3/u5,
	v4/u1, v4/u2, v4/u3, v4/u4, v4/u5,
	v5/u1, v5/u2, v5/u3, v5/u4, v5/u5,
	a2/a2_}
	%an1/an5,an2/an3,an3/an4,an3/an5,an4/an5}
	\draw[draw=CornflowerBlue,line width=1.3pt] (\from) -- (\to);

	\draw [draw=black!70, line width=2pt]
		(-11. ,5.4) -- 
		(-11. ,5.8) --
		(-9.5 ,5.8) --
		(-9.5 ,6.3) --
		(-8.  ,5.6) --%%%
		(-9.5 ,4.9) --
		(-9.5 ,5.4) --
		cycle;

	

\end{tikzpicture} \end{center}

%\section {Complexidade do problema da bissecção - arrumando}

%%%%%%%%%%%%%%%%%%%%%%%%%%%%%%%%%%%%%%%%%%%%%%%%%%%%%%%%%%%%%%%%%%%
%%%%%%%%%%%%%%%%%%%%%%%  SEÇÃO 11    %%%%%%%%%%%%%%%%%%%%%%%%%%%%%%
%%%%%%%%%%%%%%%%%%%%%%%%%%%%%%%%%%%%%%%%%%%%%%%%%%%%%%%%%%%%%%%%%%%
\section {Algoritmo de Jansen et al.}

O algoritmo de Jansen~et~al. é um algoritmo de 
programação dinâmica que sempre encontra
a solução ótima para o problema da bissecção
mínima aplicado a grafos planares.
Seu consumo de tempo é~$O(n^3)$.

Como o foco deste trabalho é o problema da 
bissecção mínima aplicado a árvores, e dado
que árvores são um tipo específico de grafos
planares, então, nesta seção, analisaremos a 
fórmula de recorrência específica para encontrar
a solução desse problema em árvores. 

\bigskip
\bigskip

\subsection*{Fórmula de recorrência}

Seja~$T$ uma árvore com~$n$ vértices. 
Para todo~${v\in V(T)}$, todo~${m\in[n]}$
e sendo~$T_v$ a sub-árvore que contém~$v$
e todos os seus descendentes,
temos que a função~$b(v,w)$ representa o valor
de~$e_{T_v}(B,W)$, onde~${m=|B|}$,~${v\in B}$
e~$(B,W)$ é o corte de largura mínima em~$T_v$.
O mesmo vale para a função~$w(v,m)$, mudando somente
o fato de que, neste caso,~${v\in W}$.
Ou seja,~$w(v,w)$ representa o valor
de~$e_{T_v}(B,W)$, onde~${w=|B|}$,~${v\in W}$
e~$(B,W)$ é o corte de largura mínima em~$T_v$.
Temos também que as soluções inviáveis recebem~$\infty$
como valor da largura do corte.

Sendo assim, para todo vértice~$v$, é válido que:
\begin{itemize}
	\item Se~$v$ é uma folha de~$T$, então
	
	\medskip

	$ b(v,m)~ = 
	\begin{cases}
		0, &m=1 \\
		\infty, &m\ne 1
	\end{cases}$

	\medskip

	$w(v,m) = 
	\begin{cases}
		0, &m=0 \\
		\infty, &m\ne 0,
	\end{cases}$

dado que, como~$v$ é uma folha,
então os únicos
cortes possíveis em~$T_v$ são~${(B,W) = (\{v\},\emptyset)}$ 
e~${(B,W) = (\emptyset, \{v\})}$.

A função~$b(v,m)$ possui a propriedade~${v\in B}$,
então, só temos soluções válidas quando~${m=1}$.
Já a função~$w(v,m)$ tem como propriedade~${v\in W}$,
logo, o único valor viável para~$m$ é~$0$.
Ambas as soluções não possuem arestas no corte, dado que
a árvore~$T_v$
possui um único vértice.

\item Caso contrário, o vértice~$v$ não é uma folha. 
Portanto, temos que~$u_1,u_2,\ldots,u_k$ são
os vértices filhos de~$v$ e 
que~$T_{u_i}$ é a sub-árvore que contém~$u_i$ e 
todos os seus descendentes.
Temos também que~${T_{v^i} = T_{u_i}\cup \{v\}\cup \{v,u_i\}} $,
como mostrado na figura abaixo.
	\begin{center} \begin{tikzpicture}
		[scale=.48,auto=left,every node/.style={circle, draw=black,
				fill=white!70}]

		\draw [draw=Emerald, fill=Emerald!25,line width=3pt]  (8.8-14,4) 
		ellipse (1.8cm and 2.7cm);
		\draw [draw=NavyBlue, fill=NavyBlue!20,line width=3pt](0.2-14,4) 
		ellipse (1.8cm and 2.7cm);
		
		\node [text=NavyBlue,draw=none, fill=none] at 
		(0.2-14,0.5) {$\mathbf{T_{u_1}}$};
		\node [text=Emerald,draw=none, fill=none] at 
		(8.8-14,0.5) {$\mathbf{T_{u_k}}$};

		\draw [draw=YellowOrange,fill=red!25,line width=3pt] (4.5,6.) 
		ellipse (1.1cm and 3.6cm);
		\draw [rotate around={-138:(7,5.8)}, draw=Emerald, fill=Emerald!25,line width=3pt] (7,5.8) 
		ellipse (2.cm and 5.8cm);
		\draw [rotate around={138:(2,5.8)}, draw=NavyBlue, fill=NavyBlue!20,line width=3pt] (2,5.8) 
		ellipse (2.cm and 5.8cm);

		\node [text=NavyBlue,draw=none, fill=none] at 
		(2.3-3,0.5) {$\mathbf{T_{v^1}}$};
		\node [text=Emerald,draw=none, fill=none] at 
		(6.7+3,0.5) {$\mathbf{T_{v^k}}$};

		\draw [draw=red!60,fill=none,line width=3pt] (4.5,6.) 
		ellipse (1.1cm and 3.6cm);
		\draw [rotate around={138:(2,5.8)}, draw=NavyBlue, fill=none,line width=3pt] (2,5.8) 
		ellipse (2.0cm and 5.8cm);
		\draw [rotate around={-138:(7,5.8)}, draw=Emerald, fill=none,line width=3pt] (7,5.8) 
		ellipse (2.0cm and 5.8cm);

		\node [scale=0.9,subtree,fill=NavyBlue!50](y1) at (0.2 , 5) {};
		\node [scale=0.9, subtree,fill=Emerald!50](y7) at (8.8 ,5)  {};

		\node (main) at (4.5, 8){$v$};

		\node (x0) at (0.2 ,5)  {$u_1$};
		\node (x1) at (4.5 ,4) {$...$};
		\node (x3) at (8.8 ,5) {$u_k$};


		\node [scale=0.9,subtree,fill=NavyBlue!50](y1) at (0.2-14 , 5) {};
		\node [scale=0.9, subtree,fill=Emerald!50](y7) at (8.8-14 ,5)  {};

		\node (main_) at (4.5-14, 8){$v$};

		\node (x0_) at (0.2-14 ,5)  {$u_1$};
		\node (x1_) at (4.5-14 ,4) {$...$};
		\node (x3_) at (8.8-14 ,5) {$u_k$};


		\foreach \from/\to in {main/x0, main/x1, %main/x2, 
		main/x3,
		main_/x0_, main_/x1_, %main/x2, 
		main_/x3_}
		\draw (\from) -- (\to);

	\end{tikzpicture} \end{center}
	Para facilitar o entendimento das funções~$b(v,m)$ 
	e~$w(v,m)$, vamos quebrá-las usando mais dois pares 
	de fórmulas.

	\medskip

	O primeiro par é~$\tilde{b}_i(v,w)$ 
	e~$\tilde{w}_i(v,w)$. 
	Essas funções recebem um 
	vértice~$v$ e um inteiro~${m\in [n]}$ como parâmetros 
	e devolvem a largura do corte mínimo~$(B,W)$ 
	em~$T_{v^i}$ tal que~${m = |B|}$. 
	Em~$\tilde{b}_i(v,m)$ temos que~${v\in B}$ e
	em~$\tilde{w}_i(v,m)$ temos que~${v\in W}$.
	
	Essas funções podem ser representadas por
	\begin{align*}
		\tilde{b}_i(v,m) &= \min \{ b(u_i,m-1),\ w(u_i, m-1)+1 \} \nonumber \\
		\tilde{w}_i(v,m) &= \min \{ b(u_i,m)+1,\ w(u_i, m) \}. \nonumber
	\end{align*}
	A função~$\tilde{b}_i(v,m)$ assume que~${v\in B}$, 
	logo, para obter um conjunto~$B$ com~$m$ vértices
	é necessário que o corte~$(B',W')$ em~$T_{u_i}$ 
	seja tal que~${|B'|=m-1}$,
	desta forma, teremos que~$B = B'\cup \{v\}$ e~$W = W'$,
	implicando em~$|B| = m$. 
	Caso~${u_i\in W}$,
	sabemos que a aresta~${\{ v,u_i \}}$ estará
	no corte~$(B,W)$ (dado que~${v\in B}$), e por isso 
	acrescentamos~$1$ ao valor de~${w(u_i, m-1)}$.
	Caso contrário,~$v$ e~$u_i$ pertencerão ao
	subconjunto~$B$ e nenhuma nova aresta
	será adiciona ao corte.

	Já na função~$\tilde{w}_i(v,m)$, temos que~${v\in W}$.
	Para que tenhamos um subconjunto~$B$ com~$m$ vértices,
	precisamos encontrar 
	um corte~$(B',W')$ em~$T_{u_i}$ 
	tal que~${|B'|=m}$.
	Sendo assim, temos que~${B = B'}$ e~${W = W'\cup \{v\}}$.
	Se~${u_i\in B}$, sabemos que a aresta~$\{ v,u_i \}$
	estará no corte~$(B,W)$ e, caso contrário,
	os dois vértices estarão em~$W$ e nenhuma aresta será
	adicionada.
	
	\bigskip

	O segundo par de funções é~$\tilde{\tilde{b}}_i(v,m)$
	e~$\tilde{\tilde{w}}_i(v,m)$.
	Essas funções recebem um vértice~$v$ e um~${m\in [n]}$ e
	devolvem a largura de um corte mínimo na 
	árvore~${T_{v^1}\cup T_{v^2}\cup \cdots \cup T_{v^i}}$
	de forma que~${v\in B}$ na primeira função
	e~${v\in W}$ na segunda.
	A condição inicial de ambas é quando~${i=1}$, e nesse caso
	temos que 
	\begin{align*}
		\tilde{\tilde{b}}_1(v,m) =  \tilde{b}_1(v,m)\ \ \text{ e }\ \
		\tilde{\tilde{w}}_1(v,m) =  \tilde{w}_1(v,m), \nonumber
	\end{align*}
	pois a largura do corte devolvido será a referente 
	somente a árvore~$T_{v^1}$.

	Seja~$(B_i,W_i)$ um corte de largura máxima em~$T_{v^i}$ 
	com~${|B_i| = m_i}$,
	%Sendo~$m_i = |B_i|$ em~$T_{v^i}$,
	temos que as fórmulas gerais dessas funções irão verificar todas as 
	combinações possíveis de valores de~$m_1,m_2,\ldots,m_i$,
	%para cada uma das sub-árvores 
	%contidas em, 
	de forma
	%a união dos subconjuntos~$B$ de cada uma das 
	%sub-árvores~$ T_{v^1},T_{v^2},\ldots,T_{v^i} $
	que~${|B_1\cup B_2\cup\cdots\cup B_i| = m}$.
	%seja igual a~$m$.
	%que a soma desses valores seja igual a~$m$. ???
	
	Por exemplo, para calcular o valor 
	de~$\tilde{\tilde{w}}_i(v,m)$ com~${m = 2}$ e~${i=3}$,
	verificaremos todas as possíveis combinações de~$m_1,m_2$ e~$m_3$
	%para cada uma das sub-árvores~$T_{v^1}$,~$T_{v^2}$ e~$T_{v^3}$
	de forma 
	%que a união dos subconjuntos~$B$ referentes as
	%árvores~$T_{v^1}$,~$T_{v^2}$ e~$T_{v^3}$
	%tenha exatamente~$m=2$ vértices.
	que~${|B_1\cup B_2\cup B_3| = m = 2}$.
	Segue abaixo uma tabela com as possíveis combinações
	de~~$m_1,m_2$ e~$m_3$ referentes a este exemplo.
	\begin{table}[h]
		\centering
		\begin{tabular}{| c | c | c | c | c|}
		%\begin{tabular}{c | >{\columncolor{blue!14}}c | >{\columncolor{red!14}}c | >{\columncolor{blue!14}}c | >{\columncolor{red!14}}c }
			\specialrule{1.7pt}{1pt}{1pt}
%			& $T_{v^1}$ & $T_{v^2}$ & $T_{v^1}$ & $T_{v^1}\cup T_{v^2}\cup T_{v^3}$  \\[3pt]
			Combinação& $|B_1|$ & $|B_2|$ & $|B_3|$ & $|B_1\cup B_2\cup B_3|$  \\[3pt]

			\specialrule{1.7pt}{1pt}{1pt}
			  	1ª  & 0  & 0  & 2  & 2 \\ [3pt]
				2ª  & 0  & 2  & 0  & 2 \\ [3pt]
			 	3ª  & 2  & 0  & 0  & 2 \\ [3pt]
				4ª  & 1  & 1  & 0  & 2 \\ [3pt]
				5ª  & 1  & 0  & 1  & 2 \\ [3pt]
				6ª  & 0  & 1  & 1  & 2 \\ [3pt]
			\specialrule{1.7pt}{1pt}{1pt}
		 
		\end{tabular}
	\end{table}

	Portanto, será devolvido o número de arestas no 
	corte~$(B,W)$ em~${T_{v^1}\cup T_{v^2}\cup\cdots\cup T_{v^i}}$  
	de forma que~${B=B_1\cup B_2\cup \cdots\cup B_i}$ 
	e que a combinação de valores 
	de~$m_1,m_2,\ldots,m_i$ 
	leve ao número mínimo de arestas no corte.

	Seja~$(B_{i-1}^*,W_{i-1}^*)$ o corte de lárgura mínima
	em~${T_{v^1}\cup T_{v^2}\cup \cdots \cup T_{v^{i-1}}}$
	com~${|B|_{i-1}^* = m_{i-1}^*}$.
	Como temos que, para cada valor 
	de~$m$, as funções~$\tilde{\tilde{b}}_{i-1}(v,m)$
	e~$\tilde{\tilde{w}}_{i-1}(v,m)$ representam
	a largura do corte~$(B_{i-1}^*,W_{i-1}^*)$ 
	com~${v\in B_{i-1}^*}$
	e~${v\in W_{i-1}^*}$, respectivamente, 
	%mínimo na árvore~${T_{v^1}\cup T_{v^2}\cup \cdots \cup T_{v^{i-1}}}$,
	logo, sabemos que esse corte de largura mínima possui
	a melhor combinação dos valores de~$m_1,m_2,\ldots,m_{i-1}$.
	Com isso, ao invés de testar todas as combinações
	de~$m_1,m_2,\ldots,m_{i-1}$
	para cada sub-árvore em~$\{ T_{v^1}, T_{v^2},\ldots, T_{v^i} \}$,
	testaremos apenas combinações de valores de~$m_i$ e~$m_{i-1}^*$
	referentes a~$T_{v^i}$ e 
	a~$ {T_{v^1}\cup T_{v^2}\cup\cdots\cup T_{v^{i-1}}}$, respectivamente.
	Isso irá gerar um corte~$(B,W)$ 
	em~${T_{v^1}\cup T_{v^2}\cup\ldots\cup T_{v^i}}$
	tal que~${B = B_{i-1}^* \cup B_i}$ 
	e~${W = W_{i-1}^* \cup W_i}$,
	onde~$m_i$ e~$m_{i-1}^*$ possuem a melhor combinação de valores.
	Sendo assim, as fórmulas gerais de~$\tilde{\tilde{b}}_{i}(v,m)$
	e~$\tilde{\tilde{w}}_{i}(v,m)$ são
	%Testaremos~$m=0$ em~$T_{v^1}$ com~$m=2$ em~$T_{v^2}$,~$m=1$ 
	%em~$T_{v^1}$ com~$m=1$ em~$T_{v^2}$ 
	%e~$m=2$ em~$T_{v^1}$ com~$m=0$ em~$T_{v^2}$.
	\begin{align*}
		\tilde{\tilde{b}}_{i}(v,m) &= 
			\min_{0\le\ell\le m} \{ \tilde{\tilde{b}}_{i-1}(v,\ell) + 
			\tilde{b}_i(v, m-\ell +1) \} \nonumber \\
		\tilde{\tilde{w}}_{i}(v,m) &= 
			\min_{0\le\ell\le m} \{ \tilde{\tilde{w}}_{i-1}(v,\ell) + 
			\tilde{w}_i(v, m-\ell) \}. \nonumber
	\end{align*}
	Pode-se ver que em~$\tilde{\tilde{b}}_{i}(v,m)$
	temos que~${m_i = m-\ell+1}$ 
	e que~${m_{i-1}^* = \ell}$
	e, como~${B = B_{i-1}^* \cup B_i}$ e~${v\in B}$, 
	então,~${|B| = m_i + m_{i-1}^* -1 = m}$.


	Na função~$\tilde{\tilde{w}}_{i}(v,m)$, 
	temos que~${m_i = m-\ell}$ e que~${m_{i-1}^* = \ell}$
	e, como~${B = B_{i-1}^* \cup B_i}$ e~${v\in W}$, 
	então,~${|B| = m_i + m_{i-1}^*= m}$.

	\bigskip

	Logo, sendo~$k$ o número de filhos de~$v$,
	as funções~$b(v,m)$ e~$w(v,m)$ são da forma
	\begin{align*}
		b(v,m) = \tilde{\tilde{b}}_k(v,m)\ \ \text{ e }\ \
		w(v,m) = \tilde{\tilde{w}}_k(v,m). \nonumber
	\end{align*}

	Seja~$r$ a raíz de~$T$, temos que
	a largura do corte mínimo~$(B,W)$ em~$T$, onde~${|B|=m}$, 
	é~$\min\{b(r,m), w(r,m)\}$.
\end{itemize}

%%%%%%%%%%%%%%%%%%%%%%%%%%%%%%%%%%%%%%%%%%%%%%%%%%%%%%%%%%%%%%%%%%%
%%%%%%%%%%%%%%%%%%%%%%%  SEÇÃO 12    %%%%%%%%%%%%%%%%%%%%%%%%%%%%%%
%%%%%%%%%%%%%%%%%%%%%%%%%%%%%%%%%%%%%%%%%%%%%%%%%%%%%%%%%%%%%%%%%%%

\section {Geração de árvores binárias aleatórias}

Nesta seção falaremos sobre o gerador de árvores binárias 
aleatórias, cuja definição foi tirada
da Wikipédia~\cite{rbt}. Tais tipos de árvores foram utilizadas nos 
testes do algoritmo FST.

Árvores binárias aleatórias são árvores com um número 
pré-definido de vértices, construídas aleatoriamente e 
onde cada vértice possui no máximo dois filhos.

O gerador que utilizamos recebe um inteiro~$n$ e cria um vetor
com uma permutação aleatória dos inteiros de~$0$ a~$n-1$.
Depois, constrói uma árvore binária de busca, inserindo 
os vértices na ordem dada pelo vetor.
Isso leva tempo~$O(n^2)$, dado que a inserção na árvore 
binária de busca pode levar tempo linear para cada um dos~$n$ 
vértices.

Usaremos a função {\sc insere}$(r, n)$, que criará
um vértice de valor~$n$ e fará a inserção desse na árvore
enraizada no vértice~$r$.

Segue abaixo o algoritmo do gerador.

\begin{algorithm}[H]
\label{alg:ABAgenerator}
	\SetKwInOut{Input}{entrada}
	\SetKwInOut{Output}{saída}

	\caption{Gerador de árvores binárias aleatórias}
	\Input{inteiro $n$}
	\Output{raiz de uma árvore binária gerada aleatoriamente}
	\tcp{Gera permutação aleatória}
	\For {$i = 0 \to n-1$}
	{
		$v[i] \gets i$\;
	}

	\For {$i = n-1 \to 1$}{
		$indice \gets$ {\sc random}$(i)$\;
		\tcp{{\sc random}$(i)$ devolve um inteiro
		aleatório em~$\{0,\ldots,i\}$}
		$v[indice] \leftrightarrow v[i]$\;
	}
	\tcp{Inserção dos elementos do vetor na árvore binária de
	busca}
	$raiz\gets$ {\sc null}\;
	\For{$i=0\to n-1$}
	{
		{\sc insere}$(raiz$, $v[i])$\;
	}
	\Return $raiz$\;

\end{algorithm}	

\bigskip
\bigskip

% Nota-se que numa árvore desse tipo, em grande parte dos 
% casos, a maioria dos vértices estará presente no caminho mais 
% longo. 
% O que favorece um pouco o algoritmo FST.


%%%%%%%%%%%%%%%%%%%%%%%%%%%%%%%%%%%%%%%%%%%%%%%%%%%%%%%%%%%%%%%%%%%
%%%%%%%%%%%%%%%%%%%%%%%  SEÇÃO 13    %%%%%%%%%%%%%%%%%%%%%%%%%%%%%%
%%%%%%%%%%%%%%%%%%%%%%%%%%%%%%%%%%%%%%%%%%%%%%%%%%%%%%%%%%%%%%%%%%%
%\newpage
\section {Experimentos computacionais}

	\subsection{Árvores Binárias Aleatórias}
	% \begin{table}[htbf]
	% 	\centering
	% 	\caption {Random Binary Trees}
	% 	\begin{tabular}{ c|c | >{\columncolor{blue!14}}c | >{\columncolor{red!14}}c | >{\columncolor{blue!14}}c | >{\columncolor{red!14}}c }
	% 		\specialrule{1.7pt}{1pt}{1pt}
	% 		$|V(T)|$ & seed & $e_T(B,W)$ - \textbf{FST} & $e_T(B,W)$ - \textbf{Jansen} & Tempo - \textbf{FST} & Tempo - \textbf{Jansen}  \\[10pt]

	% 		\specialrule{1.7pt}{1pt}{1pt}
	% 		\multirow{10}*{100} & 1  & 2   & 2  & 0.000146  & 0.002095 \\[3pt]\cmidrule{2-6} 
	% 	                        & 2  & 2   & 2  & 0.000187  & 0.002130 \\[3pt]\cmidrule{2-6}
	% 		                    & 3  & 2   & 2  & 0.000158  & 0.002440 \\[3pt]\cmidrule{2-6}
	% 		                    & 4  & 2   & 2  & 0.000142  & 0.001993 \\[3pt]\cmidrule{2-6}
	% 		                    & 5  & 3   & 2  & 0.000188  & 0.000967 \\[3pt]\cmidrule{2-6}
	% 		                    & 6  & 3   & 2  & 0.000185  & 0.001003 \\[3pt]\cmidrule{2-6}
	% 		                    & 7  & 2   & 2  & 0.000131  & 0.001005 \\[3pt]\cmidrule{2-6}
	% 		                    & 8  & 5   & 2  & 0.000183  & 0.000884 \\[3pt]\cmidrule{2-6}
	% 		                    & 9  & 2   & 2  & 0.000176  & 0.001037 \\[3pt]\cmidrule{2-6}
	% 		                    & 10 & 2   & 2  & 0.000130  & 0.001030 \\[3pt]

	% 		\specialrule{1.7pt}{1pt}{1pt}
		 
	% 	\end{tabular}
	% \end{table}




		\begin{table}[h]
		\centering
		\begin{tabular}{| c | >{\columncolor{blue!14}}P{1.8cm} | >{\columncolor{red!14}}P{2.6cm} | P{2.6cm} | >{\columncolor{blue!14}}P{2.5cm} | >{\columncolor{red!14}}P{2.6cm} |}
		%\begin{tabular}{c | >{\columncolor{blue!14}}c | >{\columncolor{red!14}}c | >{\columncolor{blue!14}}c | >{\columncolor{red!14}}c }
			\specialrule{1.7pt}{1pt}{1pt}
			$|V(T)|$ & $e_T(B,W)$ \textbf{FST} (pior~caso) & $e_T(B,W)$ \textbf{Jansen~et~al.} (pior~caso) & nº de vezes que encontrou o ótimo & Tempo médio \textbf{FST} (segundos) & Tempo médio \textbf{Jansen~et~al.}   (segundos) \\[10pt]

			\specialrule{1.7pt}{1pt}{1pt}

			  	100  & 5  & 2  & 7/10  & 0.000162  &   0.001458 \\ [3pt]
				200  & 7  & 2  & 9/10  & 0.000200  &   0.007780 \\ [3pt]
				400  & 8  & 3  & 6/10  & 0.000395  &   0.051422 \\ [3pt]
				800  & 9  & 3  & 3/10  & 0.000816  &   0.400407 \\ [3pt]
				1600 & 8  & 3  & 2/10  & 0.001218  &   3.136476 \\ [3pt]
				3200 & 9  & 3  & 4/10  & 0.002107  &  25.733941 \\ [3pt]
				6400 & 11 & 3  & 0/10  & 0.004222  & 202.527711 \\ [3pt]
			   12800 & 11 & 3  & 0/10  & 0.009259  &1343.374207 \\ [3pt]

			\specialrule{1.7pt}{1pt}{1pt}
		 
		\end{tabular}
	\end{table}

		\subsection{Árvores Ternárias}
		Árvores ternárias são árvores onde todos os nós, exceto as folhas, possuem três filhos.

		\begin{table}[h]
		\centering
		\begin{tabular}{| c | c | >{\columncolor{blue!14}}P{2.1cm} | >{\columncolor{red!14}}P{2.6cm} | >{\columncolor{blue!14}}P{2.6cm} | >{\columncolor{red!14}}P{2.6cm} | }
		%\begin{tabular}{c | >{\columncolor{blue!14}}c | >{\columncolor{red!14}}c | >{\columncolor{blue!14}}c | >{\columncolor{red!14}}c }
			\specialrule{1.7pt}{1pt}{1pt}
			Altura & $|V(T)|$ & $e_T(B,W)$ \textbf{FST} & $e_T(B,W)$ \textbf{Jansen~et~al.} & Tempo \textbf{FST} (segundos) & Tempo \textbf{Jansen~el~al.} (segundos)  \\[10pt]

			\specialrule{1.7pt}{1pt}{1pt}

			  	1 & 4    & 2  & 2  & 0.000091  &   0.000009 \\ [3pt]
				2 & 13   & 8  & 3  & 0.000177  &   0.000033 \\ [3pt]
				3 & 40   & 10 & 4  & 0.000173  &   0.000336 \\ [3pt]
				4 & 121  & 11 & 4  & 0.000450  &   0.006730 \\ [3pt]
				5 & 364  & 10 & 5  & 0.000568  &   0.071686 \\ [3pt]
				6 & 1093 & 11 & 6  & 0.001301  &   1.655212 \\ [3pt]
				7 & 3280 & 13 & 7  & 0.004268  &  44.435073 \\ [3pt]
				8 & 9841 & 14 & 8  & 0.011514  &1193.373235 \\ [3pt]

			\specialrule{1.7pt}{1pt}{1pt}
		 
		\end{tabular}
	\end{table}

	\bigskip
	\bigskip
	\bigskip



		\subsection{Árvores de Caminho Longo}
		O caminho máximo dessas árvores possui mais que
		a metade dos vértices da árvore. 
		Portanto, temos que seu diâmetro relativo é maior que meio.

		Como o visto na seção~\ref{sec:dobraDiametro}, todas as árvores desse tipo, caem
		no caso~1, onde o corte da bissecção mínima é sempre igual ou menor que dois.

		\begin{table}[htbf]
		\centering
		\begin{tabular}{| c | >{\columncolor{blue!14}}P{2.1cm} | >{\columncolor{red!14}}P{2.6cm} | >{\columncolor{blue!14}}P{2.6cm} | >{\columncolor{red!14}}P{2.6cm} |}
		%\begin{tabular}{c | >{\columncolor{blue!14}}c | >{\columncolor{red!14}}c | >{\columncolor{blue!14}}c | >{\columncolor{red!14}}c }
			\specialrule{1.7pt}{1pt}{1pt}
			$|V(T)|$ & $e_T(B,W)$ \textbf{FST} & $e_T(B,W)$ \textbf{Jansen~et~al.} & Tempo \textbf{FST} (segundos) & Tempo \textbf{Jansen~et~al.}   (segundos) \\[10pt]

			\specialrule{1.7pt}{1pt}{1pt}

			  	100  & 2  &  1  & 0.000080  &   0.000663 \\ [3.2pt] 
				200  & 2  &  1  & 0.000114  &   0.004626 \\ [3.2pt]
				400  & 2  &  1  & 0.000181  &   0.037905 \\ [3.2pt]
				800  & 2  &  1  & 0.000327  &   0.250858 \\ [3.2pt]
				1600 & 2  &  2  & 0.00 584  &   2.136476 \\ [3.2pt]
				3200 & 2  &  2  & 0.001118  &  17.872311 \\ [3.2pt]
				6400 & 2  &  2  & 0.002568  & 130.088130 \\ [3.2pt]
			   12800 & 2  &  2  & 0.006036  &1339.149309 \\ [3.2pt]

			\specialrule{1.7pt}{1pt}{1pt}
		 
		\end{tabular}
	\end{table}

	\bigskip
	\bigskip
	\bigskip
	\bigskip
	\bigskip

	Para gerar as árvores ternárias e as árvores de caminho longo, 
	foram utilizados os geradores feitos por Kampmeier~\cite{Kampmeier}.



%%%%%%%%%%%%%%%%%%%%%%%%%%%%%%%%%%%%%%%%%%%%%%%%%%%%%%%%%%%%%%%%%%%
%%%%%%%%%%%%%%%%%%%%%%%  SEÇÃO 14    %%%%%%%%%%%%%%%%%%%%%%%%%%%%%%
%%%%%%%%%%%%%%%%%%%%%%%%%%%%%%%%%%%%%%%%%%%%%%%%%%%%%%%%%%%%%%%%%%%
\newpage

\section {Resultados}
Conforme os resultados dos experimentos computacionais
 apresentados anteriormente, observa-se
que o algoritmo de FST possui grande vantagem sobre o algoritmo
de Jansen et al. em relação à complexidade de tempo e, em alguns 
dos casos consegue se equiparar a ele no quesito optimalidade de
solução.

Vemos que ele consegue ser ótimo ou quase ótimo quando aplicado
a árvores que possuem um caminho máximo longo. 
Isso se deve pelo
fato de que árvores de caminho longo possuem grandes chances
de terem mais vértices no caminho máximo do que fora dele, e
árvores desse tipo acabam caindo no primeiro caso do Algoritmo da 
dobra do diâmetro, que irá cortar no máximo duas arestas da árvore.

Infelizmente, existem casos onde o algoritmo de FST não se dá
tão bem em relação à optimalidade.
Como pode ser visto nos resultados em árvores ternárias mostrados 
anteriormente,
o algoritmo de FST devolve cortes cujas larguras são relativamente
grandes comparadas as larguras devolvidas pelo algoritmo de Jansen
et al.. 
Isso ocorre devido à pequena quantidade de vértices presentes 
no caminho máximo em uma árvore ternária, 
implicando numa grande chance de uma árvore desse tipo cair 
no segundo caso
do Algoritmo da dobra do diâmetro, que irá gerar um corte
cuja largura é
limitada superiormente por um valor alto.
%com um alto valor de limite superior.

\newpage
\bibliographystyle{plain}
\bibliography{refs}

\newpage

\part{Parte Subjetiva}
\newpage
%\section{Desafios e frustrações}


\section{Disciplinas relevantes para o TCC}
Em geral, todas as disciplinas tiveram um papel importante
para a minha formação. Mas, as que contribuíram
diretamente para esse trabalho de conclusão de curso
 foram as seguintes:

\begin{itemize}
	\item \textbf{Introdução à Computação} e 
	\textbf{Princípios de Desenvolvimento de Algoritmos}

		% que foram disciplinas fundamentais que contribuiram 
		% para a formação de uma base sólida em relação
		% a lógica de programação.

		que foram disciplinas fundamentais que me deram  
		 uma base sólida em relação
		à lógica de programação e desenvolvimento de algoritmos.

	\item \textbf{Estrutura de dados}

		onde foram vistos conceitos importantes usados 
		nesse trabalho e também foi
		onde tive meu primeiro contato com grafos,
		em um exercício programa.
		
	\item \textbf{Análise de Algoritmos}

		onde foram estudados conceitos de classes de complexidade computacional,
		análise de tempo de algoritmos e programação dinâmica, que foram
		utilizados nesse trabalho.

	\item \textbf{Algoritmos em Grafos}
	
		onde foram vistos conceitos importantes e algoritmos 
		base que foram muito utilizados nesse trabalho, 
		além de ter sido uma matéria fundamental que despertou o meu interesse 
		e curiosidade em relação à esse assunto.
\end{itemize}

\newpage

\section{Agradecimentos}
Gostaria de agradecer, primeiramente, a minha família
que me apoiou em todas as minhas decisões, sempre
me incentivando a seguir em frente com meus objetivos e sonhos,
e esteve ao meu lado em todos os momentos de alegrias e dificuldades.

Agradeço também as boas amizades que fiz no IME, em especial
ao Victor, Mateus, Lucas, Luciana, Ruan, Renato, Jonath, Bruno, Massao, 
Helena, Ludmila, Vinicius, Gabriel e Gervásio, que sempre 
estiveram presentes, compartilhando
alegrias, conquistas, preocupações, pressões e desafios do curso
e fizeram com que as longas viagens diárias da zona leste pro Butantã 
valessem muito a pena.

Sou grata a Tina Schmidt, que me permitiu estudar a sua tese de doutorado e
com quem tive a oportunidade de discutir problemas relacionados ao assunto.

Por fim, também sou muitíssimo grata a todos os funcionários do IME e professores que
me deram aula e aceitaram dividir um pouco do seu conhecimento.
Em especial aos professores Mandel, Cris, Yoshiharu, Coelho, Daniel e Nami, que são excelentes
professores e deram aulas que me fizeram sentir que realmente 
escolhi o curso certo para mim.
Principalmente a professora Cris, que foi uma excelente orientadora,
sempre muito presente, paciente e atenciosa, me auxiliando em tudo
o que foi preciso para o desenvolvimento desse trabalho.

\end{document}