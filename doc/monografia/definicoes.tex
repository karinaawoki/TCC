\section{Definições}
	
%\begin{enumerate}
%	\item
	Para um grafo~$G$, denotamos seu
	conjunto de vértices por~$V(G)$ e seu
	conjunto de arestas por~$E(G)$.

\bigskip
%	\item
	O número de arestas que incidem em um 
	determinado vértice~$v$ de~$G$ é chamado de 
	\textbf{grau} de~$v$ em~$G$ e é representado
	por~$\grau_G(v)$. O 
	\textbf{grau máximo}, que é o grau de um vértice de
	maior grau em~$G$, é representado por~$\Delta(G)$.

\bigskip

	%\item
	Um \textbf{caminho} em~$G$ é uma sequência de 
	vértices~$v_0, v_1, \ldots,v_{p-1}, v_p$ 
	onde~$v_{k-1}, v_k$ é uma aresta de~$G$ para 
	todo~${k =1,2, \ldots, p}$. 
	O número~$p$ é o \textbf{comprimento} do caminho.

\bigskip

	%\item
	Um \textbf{caminho máximo} em~$G$ é um caminho 
	em~$G$ de comprimento máximo.
	O comprimento de um caminho máximo em~$G$
	é o \textbf{diâmetro} de~$G$, e é denotado por~$\diam(G)$.

	% O \textbf{diâmetro} de um grafo conexo~$G$ é o
	% comprimento de um caminho máximo em~$G$. 
	% Em outras palavras, o diâmetro pode ser definido como:
	% $$ \diam(G)=\max\{\dist(x,y):x,y\in V(G)\}.$$

	% \bigskip

	Seja~$G$ um grafo não necessariamente conexo.
	O \textbf{diâmetro relativo} de~$G$
	 é a razão entre o número
	de vértices de um caminho máximo de~$G$ e o número total de vértices
	de~$G$. Este pode também ser representado por
	$$ \diam^*(G) =\dfrac{\displaystyle\sum_{
	G'~componente~conexa~de~G}^{}(\diam(G')+1)}{|V(G)|}.$$

	Note que~${0<\diam^*(G)\le 1}$ para todo grafo~$G$ 
	com~${V(G)\ne \emptyset}$, pois há pelo menos um vértice 
	de~$G$ num caminho máximo de~$G$, e no máximo 
	todos os vértices de~$G$ num caminho máximo de~$G$.

\bigskip

	%\item
	Em uma árvore~$T$, existe um único caminho que conecta 
	quaisquer dois vértices~$x$ e~$y$. 
	Chamaremos tal caminho em~$T$ de~$x,y$--\textbf{caminho}.

\bigskip

	%\item
	A \textbf{distância} entre dois vértices~$x$ e~$y$ de~$T$ é 
	representada por~$\dist(x,y)$, e é o comprimento 
	do~$x,y\caminho$ em~$T$.

\bigskip

	%\item
	Denotamos por~$[n]$
	o conjunto~$\{1,2,\ldots,n\}$.

%\end{enumerate}

\bigskip
\bigskip
\bigskip
