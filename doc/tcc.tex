\documentclass[a4paper,12pt]{article}

% Para usar a linguagem PT-BR.
\usepackage[utf8]{inputenc}
\usepackage[T1]{fontenc}
\usepackage[brazilian]{babel}
\usepackage[all, 2cell]{xy} \UseAllTwocells \SilentMatrices
\selectlanguage{brazilian} 

\newcommand{\Oh}{\mathrm{O}}

\sloppy

\begin{document}
\begin{center}
   {\large \textbf{UNIVERSIDADE DE SÃO PAULO}} \\[1.4cm]
   
   {\large \textbf{INSTITUTO DE MATEMÁTICA E ESTATÍSTICA}}\\[4.2cm]
   
   {\Huge TRABALHO DE CONCLUSÃO }\\[0.3cm]
   {\Huge DE CURSO }\\[9cm]
   
   {\large { Aluna: Karina Suemi Awoki}}\\[0.3cm]
   
   {\large { Orientadora: Cristina Gomes Fernandes}}
   

\end{center}

\newpage
\section{Introdução}
..
\bigskip

\section{Definições}
    
    \subsection{Caminho máximo em árvores}

    	Um \textbf{caminho} em uma árvore $T=(V,E)$, de 
    	comprimento $p$, é uma sequência de vértices 
    	$v_0, v_1, ...,v_{p-1}, v_p$ 
    	onde {$v_{k-1}, v_k$} é uma aresta para todo 
    	$k = 1,2,..., p$. 
    	Como existe um único caminho que conecta dois vértices,
    	então chamaremos um caminho $v_0, v_1, ...,v_{p-1}, v_p$ de
    	$v_0$,$v_p$-$path$.

    	Portanto, um \textbf{caminho máximo} em uma árvore $T$ é
    	um caminho de $T$ cujo comprimento é máximo.

    	Encontrar um caminho máximo em uma árvore $T$ é uma tarefa
    	computacionalmente simples que pode ser realizada em tempo 
    	$O(n)$. 
    	Primeiramente, é necessário escolher um vértice aleatório 
    	$v \in V$ e encontrar $y_0$,
    	que é o vértice mais distante de $v$.
    	Depois, repetimos o mesmo processo em $y_0$, encontrar
    	$x_0$ que é o vértice mais distante de $y_0$. 
    	Para encontrar o vértice mais distante de algum vértice, 
    	basta usar uma busca em largura.  
    	Feito isso, temos um caminho máximo de $T$, que é o único
    	caminho que liga $x_0$ a $y_0$.


    	Agora provaremos que, de fato, $x_0$,$y_0$-$path$ é um
    	caminho máximo.
    	Suponhamos que $x$,$y$-$path$ seja um caminho máximo e:

    	\begin{itemize}
            \item Caso 1: Os caminhos $x$,$y$-$path$ e 
            $x_0$,$y_0$-$path$ não possuem vértices em comum.
         

            %Logo, sabemos que existe um caminho $a$,$b$-$path$ de 
            %forma que $a \in x$,$y$-$path$, 
            %$b \in x_0$,$y_0$-$path$ e os demais 
            %vértices de $a$,$b$-$path$ não estão em nenhum dos 
            %dois caminhos. Nota-se que $a$,$b$-$path$ é uma
            %conexão entre os dois caminhos.

            Logo, sabemos que existe um caminho $a$,$b$-$path$, 
            de comprimento $c$,
            que funciona como uma conexão entre os dois caminhos,
            possuindo assim apenas o vértice $a$ em 
            $x$,$y$-$path$ e apenas o vértice $b$ em
            $x_0$,$y_0$-$path$ $b$.

            Portanto temos,
            $$ dist(x,y) = dist(x,a) + dist(a,y) $$ e
            $$ dist(x_0,y_0) = dist(x_0,b) + dist(b,y_0).$$

            Como $x$,$y$-$path$ é um caminho máximo e  temos que 
            $$dist(x,y)\ge dist(x_0,y_0) \Rightarrow 
            dist(x,a) + dist(a,y)\ge dist(x_0,b) + dist(b,y_0)$$ 
            e como


        \end{itemize}




    	\newpage



        Dada uma árvore $T = (V,A)$ e os seus caminhos máximos 
        $C_1, C_2,..., C_c$, dizemos que os vértices de um caminho
        máximo $C_k$ são representados pela sequência de vértices 
        $x_{1, k}$, $x_{2, k}$,... $x_{n, k}$ com 
        $\{x_{i, k}, x_{i+1, k}\} \in A, \forall i \in$ [$1, n-1$],
        $x_i\neq x_j \forall i, j \in [n]$ 
        com $i \neq j$, 
        e $k \in [c]$, 
        onde $c$ é o número de caminhos máximos existentes nessa
        árvore.
        E também deve ser satisfeita a condição de que cada 
        sequência possui o maior número de vértices possíveis 
        (portanto, todas essas sequências são do mesmo tamanho - 
        o máximo).

        \bigskip

        Encontrar um dos caminhos máximos em uma árvore é uma 
        tarefa simples, computacionalmente falando. Primeiramente
        assumiremos que $\forall v \in V$, o vértice mais distante
        de $v$ é $x_{1, k}$ ou $x_{n, k}$, que são os extremos do
        caminho mais longo $C_k$.

        Levando em consideração o que foi assumido anteriormente,
        nota-se que se escolhermos 
        qualquer vértice $v \in T$, e procurarmos pelo vértice mais
        distante dele, chegaremos em um dos extremos de um dos 
        caminhos mais longos ($x_{1, k}$ 
        ou $x_{n, k}$) - chamaremos esse vértice de $y$.
        E se fizermos o mesmo com $y$, procurarmos qual o vértice 
        mais distante dele, chegaremos no
        outro extremo do caminho mais longo.
        

        Agora precisamos provar que $\forall v \in V$, o vértice 
        mais distante
        de $v$ é $x_1$ ou $x_n$.
        Vamos chamar de $z$ o vértice mais distante de $v$.
        

        \begin{itemize}
            \item Caso 1: Se $\exists k \in [1, c]$ tal que 
            $v \in C_k$.
            Nesse caso é fácil ver que 

            $dist(x_{1, k}, x_{n, k})=
            dist(x_{1, k}, v)+dist(x_{n, k}, v)$, 
            já que $v$ é um vértice do caminho máximo $k$.
            Agora suponhamos que 

            $$dist(z, v)>dist(x_{1, k}, v)$$ e 
            $dist(z, v)>dist(x_{n, k}, v)$,
            
            então temos que 
            
            $$dist(x_{1, k}, v)+dist(z, v)>
            dist(x_{1, k}, v)+dist(x_{n, k}, v)=
            dist(x_{1, k}, x_{n, k})$$
            
            logo $x_{1, k}, x_{2, k},..., x_{n, k}$ não é um dos 
            caminhos máximos e então entramos numa contradição.

            Com isso temos que 
            $dist(z, v)>dist(x_{1, k}, v)$ e 
            $dist(z, v)>dist(x_{n, k}, v)$ não á válido, o que 
            implica em $dist(z, v)\le dist(x_{1, k}, v)$ ou 
            $dist(z, v)\le dist(x_{n, k}, v)$. 
            Portanto, como $z$ é o vértice mais distante, vale que
            
            $dist(z, v)=dist(x_{1, k}, v)$ ou 
            $dist(z, v)=dist(x_{n, k}, v)$,
            
            ou seja, $v$ é um dos vértices do caminho mais longo
            de $C_k$ ($x_{1, k}$ ou $x_{n, k}$).

            \item Caso 2: Se $\forall k \in [1, c], v \notin C_k$. 
            Diremos que a distância $d$ entre $v$ e $C_k$ é 
            min$\{dist(v, c'): c'\in C_k\}$.

            Agora suponhamos que $\exists k \in [1, c]$, tal que
            $dist(v, z)>dist(x_{1, k}, v)$ e 
            $dist(v, z)>dist(x_{n, k}, v)$.
            E dado que $d=$ min$\{dist(v, c'): c'\in C_k\}$, então
            $dist(v, x_{1, k})\ge d$ e 
            $dist(v, x_{n, k})\ge d$, 
            portanto, como $z$ é o vértice mais distante de $v$, 
            temos que
            $dist(v, z)\ge d$.

            É fácil ver que 
            $dist(x_{1, k}, x_{n, k})=
            dist(v, x_{1, k})+dist(v, x_{n, k})-2\times d$, 
            e isso implica em
            $dist(v, x_{1, k})+dist(v, z)-2\times d>
            dist(v, x_{1, k})+dist(v, x_{n, k})-2\times d=d$, 
            logo $d$ não é o maior caminho e entramos numa 
            contradição novamente, o que leva a afirmação
            "$\exists k \in [1, c]$, tal que
            $dist(v, z)>dist(x_{1, k}, v)$ e 
            $dist(v, z)>dist(x_{n, k}, v)$" a ser falsa.

            Consequentemente, $\forall k \in [1, c], 
            dist(v, z)\le dist(x_{1, k}, v)$ ou
            $dist(v, z)\le dist(x_{n, k}, v)$. 
            E como $z$ é o vértice mais distante de $v$, 
            
            $dist(v, z)=dist(x_{1, k}, v)$ ou 
            $dist(v, z)=dist(x_{n, k}, v)$,

            o que implica que $z$ é um dos vértices extremos de 
            $C_k$.






        \end{itemize}
        
    \subsection{Cálculo do Diâmetro de uma Árvore}
    


\newpage
    
    
    

\end{document}