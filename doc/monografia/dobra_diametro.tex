\section {Algoritmo da dobra do diâmetro}
\label{sec:dobraDiametro}
	\subsection{Teorema da dobra do diâmetro}
	
		\begin{teo}[{\cite[Teorema 4]{Schmidt15}}]
		\label{teo:dobraDiametro}
			Para toda árvore~$T$ com~$n\ge 3$ vértices e 
			todo~$m\in [n]$,
			o conjunto de vértices de~$T$ pode ser particionado em 
			três partes~$B$,~$W$ e~$S$ tal que vale um dos 
			seguintes itens:
			\begin{enumerate}
				\item ${|B|=m}$, ${S=\emptyset}$, e~${e_T(B,W)\le 2}$, ou
				\item ${|B|\le m\le |B|+|S|}$, 
				com~${0<|S|\le\dfrac{n}{2}}$,
				${e_T(B,W,S)\le \dfrac{2\cdot 
				\Delta(T)}{\diam^*(T)}}$, 
				e~${\diam^*(T[S])\ge 2\diam^*(T)}$.
			\end{enumerate}
			Uma partição~$(B,W,S)$ que satisfaz 1 ou 2 pode ser
			computada em tempo~${O\Big(\dfrac{n}{\diam^*(T)}\Big)}$ 
		\end{teo}

	\bigskip
	\bigskip

		% Primeiramente, faremos uma série de definições para que
		% depois possamos explicar melhor como o teorema e o algoritmo
		% funcionam.

		% Usaremos o que foi explicado na seção~\ref{sec:caminhMaximo}
		% para encontrármos um caminho máximo. 
		% Depois, usaremos 

		Primeiramente, temos uma árvore~$T$ com~${n\ge 3}$ vértices e um 
		inteiro~${m\in[n]}$.
		Seja~$P$ um caminho máximo de~$T$ e sabendo que os vértices
		estão rotulados como o descrito na seção~\ref{sec:rotulacao},
		teremos um corte~$(B,W,S)$ da seguinte forma:
		%\begin{itemize}
		\bigskip
		\bigskip
	
	\subsubsection*{Caso 1:}
			Se existe um vértice ~${v\in V(P)}$ tal 
			que~${\map_m^+(v)\in V(P)}$, então o 
			corte~$(B,W,S)$ 
			com~${S=\emptyset}$,~${B =\{v+1, v+2,\ldots,~v+m\}}$
			e~${W=V(T)\setminus B}$ satisfaz as propriedades do
			primeiro caso do Teorema~\ref{teo:dobraDiametro},
			como o explicado na seção~\ref{sec:map}.

	\bigskip
	\bigskip

		\subsubsection*{Caso 2:}
			Caso contrário, sabemos que para todo vértice~${v\in V(P)}$
			teremos que~${\map_m^+(v)\not\in V(P)}$.
			Isso implica que existem~$|V(P)|$ ou mais vértices fora de~$P$,
			dado que cada vértice de~$P$ mapeia vértices distintos que estão
			fora de~$P$.
			Isso nos leva a~$d\le\dfrac{n}{2}$, pois no 
			máximo~$\dfrac{n}{2}$ vértices estarão em~$P$.

			Nesse caso, teremos um corte~$(B,W,S)$ que satisfaz as
			propriedades do segundo caso do Teorema~\ref{teo:dobraDiametro}.
			Isso será detalhado melhor a seguir.

			\bigskip
			
			% Para todo~${z\in V(P)}$, e sendo~$x_1$ e~$x_2$ os menores
			% vértices de~$T_z$ tal que~${\map^-_m(x_1)\in V(P)}$ 
			% e~${\map^+_m(x_2)\in V(P)}$,
			% usaremos as seguintes definições:
			% \begin{align}
			% 	z^b   = &\map^-_m(x_1) \nonumber \\
			% 	z^f   = &\map^+_m(x_2) \nonumber \\
			% 	P_z^b = &\map^-_m(V(T_z))\cap V(P) \nonumber\\
			% 	P_z^f = &\map^+_m(V(T_z))\cap V(P) \nonumber%\\
			% 	%H_z^f = &\map^+_m(V(T_))\cap V(P) \nonumber\\
			% 	%H_z^f = &\map^+_m(V(T_z))\cap V(P) \nonumber
			% \end{align}

			Seja um vértice~${z\in V(P)}$ tal que~$z$ é b-especial e
			sendo~$x$ o menor vértice de~$T_z$ tal 
			que~${\map^-_m(x)\in V(P)}$, temos 
			que~${z^b = \map^-_m(x)}$.
			Temos também
			que~${P_z^b = \map^-_m(V(T_z))\cap V(P)}$
			e~${H_z^f =T_u}$ para todo 
			vértice~${u\in P_z^b}$ com~${u\ne z^b}$.

			Segue uma imagem com a representação da disposição dos 
			conjuntos de vértices.


	\begin{center} \begin{tikzpicture}
		[scale=.7,auto=left,every node/.style={circle, draw=black,
				fill=white!70}]

		\draw [decorate,decoration={brace,amplitude=10pt,mirror,
		raise=4pt}, draw=brown, line width=1.5pt] 
		(8,5.7) -- (2,5.7);

		\draw [decorate,decoration={brace,amplitude=10pt,mirror,
		raise=4pt}, draw=Emerald, line width=1.5pt] 
		(4,2.7) -- (8.5,2.7);

		\draw [draw=yellow, fill=yellow!35, line width=2pt] (18.1,3.7) 
		ellipse (1.9cm and 3.7cm); 

		\node [text=blue!65,draw=none, fill=none] at 
		(17.7,6.8) {$T_z$};


		% \node [scale=1.3,subtree,fill=Green](y1) at (2.5,5) {};
		% \node [subtree, fill=Cyan]      (y2) at (5,5)   {};
		% \node [subtree, fill=Cyan]      (y3) at (7.5,5) {};
		% \node [scale=1.3,subtree,fill=Green](y4) at (10,5)  {};
		% \node [subtree, fill=Blue]        (y5) at (12.5,5){};
		% \node [subtree, fill=Blue]        (y6) at (15,5)  {};
		% \node [scale=1.9, subtree,fill=Gray] (y7) at (18.1,5)  {};


		\node [scale=1.3,subtree,fill=red!50](y1) at (2.5,5) {};
		\node [subtree, fill=Emerald]      (y2) at (5,5)   {};
		\node [subtree, fill=Emerald]      (y3) at (7.5,5) {};
		\node [scale=1.3,subtree,fill=red!50](y4) at (10,5)  {};
		\node [subtree, fill=yellow]        (y5) at (12.5,5){};
		\node [subtree, fill=yellow]        (y6) at (15,5)  {};
		\node [scale=1.9, subtree,fill=CornflowerBlue] (y7) at (18.1,5)  {};

		\node (x0) at (0,5)  {...};
		\node [fill=brown](x1) at (2.5,5)    {};
		\node [fill=brown](x2) at (5,5)  {};
		\node [fill=brown](x3) at (7.5,5)    {};
		\node (x4) at (10,5) {};
		\node (x5) at (12.5,5)   {};
		\node (x6) at (15,5) {};
		\node (x7) at (18.1,5) {};
		\node (x8) at (20.6,5) {...};

		\node [draw=none, fill=none] at 
		(2,2) {\footnotesize{$\map^-_m(v)$}};
		\node [draw=none, fill=none](nv) at (2,2){};
		\node [draw=none, fill=none](nvt) at (2.5,3.6){};

		\node [draw=none, fill=none] at 
		(9.5,2) {\footnotesize{$\map^-_m(z)$}};
		\node [draw=none, fill=none](nz) at (9.5,2){};
		\node [draw=none, fill=none](nzt) at (10,3.6){};

		\node [draw=none, fill=none] at 
		(15.5,2.6) {\footnotesize{$x$}};
		\node [draw=none, fill=none](z) at (15.5,2.6){};
		\node [draw=none, fill=none](zt) at (18,2.6){};

		\node [draw=none, fill=none] at (2.6,5.7){$z^b$};
		\node [draw=none, fill=none] at (18.1,5.6){$z$};
		\node [draw=none, fill=none] at (15,5.6){$v$};
		\node [text=Brown,draw=none, fill=none] at (5,7){$P^b_z$};
		\node [text=Emerald,draw=none, fill=none] at (6.25,1.3){$H^b_z$};

		\foreach \from/\to in {x0/x1,x1/x2,x2/x3,x3/x4,x4/x5,x5/x6,x6/x7,x7/x8}
		\draw (\from) -- (\to);

		\path[->, line width=1pt]
		(nv) edge[left=23 ] (nvt);
		\path[->, line width=1pt]
		(z) edge[right=45] (zt);
		\path[->, line width=1pt]
		(nz) edge[right=45] (nzt);

	\end{tikzpicture} \end{center}

		\bigskip

			De forma análoga, para um vértice~${z\in V(P)}$ tal 
			que~$z$ é f-especial,
			temos que~${z^f = \map^+_m(x)}$,
			sendo~$x$ o menor vértice de~$T_z$ tal 
			que~${\map^+_m(x)\in V(P)}$.
			Também temos
			que~${P_z^f = \map^+_m(V(T_z))\cap V(P)}$
			e~${H_z^f =T_u}$ para todo 
			vértice~${u\in P_z^f}$ com~${u\ne z^f}$.

			Segue abaixo outra imagem com a representação da disposição 
			dos conjuntos de vértices.


	\begin{center} \begin{tikzpicture}
		[scale=.6,auto=left,every node/.style={circle, draw=black,
				fill=white!70}]

		\draw [decorate,decoration={brace,amplitude=10pt,mirror,
		raise=4pt}, draw=brown, line width=1.5pt] 
		(-4.5,5.9) -- (-10.5,5.9);

		\draw [decorate,decoration={brace,amplitude=10pt,mirror,
		raise=4pt}, draw=Emerald, line width=1.5pt] 
		(-8.5,2.7) -- (-4,2.7);

		\draw [draw=yellow, fill=yellow!35, line width=2pt] (-18.1,3.) 
		ellipse (2.1cm and 4.5cm); 

		\node [text=blue!65,draw=none, fill=none] at 
		(-18.5,6.8) {$T_z$};


		

		\node [scale=1.3,subtree,fill=red!50](y1) at (-2.5,5) {};
		\node [subtree, fill=Emerald]      (y2) at (-5,5)   {};
		\node [subtree, fill=Emerald]      (y3) at (-7.5,5) {};
		\node [scale=1.3,subtree,fill=red!50](y4) at (-10,5)  {};
		\node [subtree, fill=yellow]        (y5) at (-12.5,5){};
		\node [subtree, fill=yellow]        (y6) at (-15,5)  {};
		\node [scale=1.9, subtree,fill=CornflowerBlue] (y7) at (-18.1,5)  {};

		\node (x0) at (0,5)  {...};
		\node (x1) at (-2.5,5)    {};
		\node [fill=brown](x2) at (-5,5)  {};
		\node [fill=brown](x3) at (-7.5,5)    {};
		\node [fill=brown](x4) at (-10,5) {};
		\node (x5) at (-12.5,5)   {};
		\node (x6) at (-15,5) {};
		\node (x7) at (-18.1,5) {};
		\node (x8) at (-20.6,5) {};
		\node (x9) at (-23.1,5) {...};

		\node [draw=none, fill=none] at 
		(-2,1) {\footnotesize{$\map^+_m(z)$}};
		\node [draw=none, fill=none](nv) at (-2,1){};
		\node [draw=none, fill=none](nvt) at (-2.5,3.6){};

		\node [draw=none, fill=none] at 
		(-9.5,1) {\footnotesize{$\map^+_m(v)$}};
		\node [draw=none, fill=none](nz) at (-9.5,1){};
		\node [draw=none, fill=none](nzt) at (-10,3.6){};

		\node [draw=none, fill=none] at 
		(-15.5,2) {\footnotesize{$x$}};
		\node [draw=none, fill=none](z) at (-15.5,2){};
		\node [draw=none, fill=none](zt) at (-18,2){};

		\node [draw=none, fill=none] at (-9.8,5.8){$z^f$};
		\node [draw=none, fill=none] at (-18.1,5.6){$z$};
		\node [draw=none, fill=none] at (-20.6,5.6){$v$};
		\node [text=Brown,draw=none, fill=none] at (-7.5,7.3){$P^f_z$};
		\node [text=Emerald,draw=none, fill=none] at (-6.35,1.1){$H^f_z$};


		\foreach \from/\to in {x0/x1,x1/x2,x2/x3,x3/x4,x4/x5,x5/x6,x6/x7,x7/x8,x8/x9}
		\draw (\from) -- (\to);

		\path[->, line width=1pt]
		(nv) edge[left=23 ] (nvt);
		\path[->, line width=1pt]
		(z) edge[right=45] (zt);
		\path[->, line width=1pt]
		(nz) edge[right=45] (nzt);

	\end{tikzpicture} \end{center}

	Note que os vértices~$\map^-_m(v)$,~$\map^-_m(z)$,~$\map^+_m(v)$ 
	e~$\map^+_m(z)$ não estão em~$V(P)$, pois estamos no caso 2, onde
	nenhum vértice de~$P$ é mapeado em outro vértice de~$P$.

	Considerando que existe uma aresta ligando o primeiro ao último 
	vértices de~$P$, vemos que os elementos de~$P_z^b$ aparecem 
	como um caminho conexo e que para valores diferentes de~$z$,
	temos que~$P_z^b$ é disjunto.
	O mesmo vale para~$P_z^f$.

	Usaremos futuramente as seguintes propriedades:
	\begin{enumerate}
		\item Para todo vértice~$z$ tal que~$z$ é b-especial, 
		temos que o vértice~$z^b$ é f-especial.

		Isso ocorre porque para o vértice~$v$, que vem antes 
		de~$z$ em~$P$, temos que~${\map^-_m(v)\not\in V(P)}$, 
		logo,~$\map^-_m(v)$ está em~$T_{z^b}$.
		Sabemos que~$\map^-_m(v)$ não está na árvore a esquerda
		de~$T{z^f}$, pois se estivesse, o~$z^f$ seria um outro
		vértice.

		\item Para todo vértice~$z$ tal que~$z$ é f-especial, 
		temos que o vértice~$z^f$ é b-especial.

		Isso ocorre de forma análoga a anterior, dado que
		para o vértice~$v$, que vem antes de~$z$ em~$P$, temos
		que~${\map^+_m(v)\in T_{z^f}}$.

		\item Um vértice~${z\in V(P)}$ é b-especial sse existe um
		vértice~${u\in V(P)}$ tal que~$u$ é f-especial e~${z=u^f}$.

		Sabemos que a volta é verdadeira usando a propriedade 2.
		
		A ida é verdadeira pois, dado que~$z$ é b-especial,
		temos que existe um ou mais vértices~${v_i\in V(P)}$ tal
		que~${\map^+_m(v_i)\in T_{z}}$.
		Seja~$v_k$ o vértice com o maior rótulo e~$u$ o vértice 
		que vem depois de~$v_k$ em~$P$. 
		Temos então que~${\map^-_m(z)\in T_u}$ e que alguns vértices
		de~$T_z$ mapeiam em~$v_k$, logo,~$u$ é f-especial e~${z=u^f}$.

		\item Um vértice~${z\in V(P)}$ é f-especial sse existe um
		vértice~${u\in V(P)}$ tal que~$u$ é b-especial e~${z=u^b}$.

		Podemos verificar que essa propriedade é verdadeira de forma
		semelhante a verificação da propriedade anterior.
	\end{enumerate}

	Seja~$T'_z$ o conjunto de 
	vértices~${V(T_z)\setminus \{z\}}$,~$B_{esp}$ 
	um conjunto de todos os vértices b-especiais e~$F_{esp}$ o
	conjunto de todos os vértices f-especiais, usando as propriedades 
	3 e 4 respectivamente, temos que
	\begin{align}
		\displaystyle\sum_{u\in F_{esp}}|T_{u^f}| = 
		\displaystyle\sum_{z\in B_{esp}}|T_{z}| %\nonumber \\
		&\Rightarrow
		\displaystyle\sum_{u\in F_{esp}}|T_{u^f}'| = 
		\displaystyle\sum_{z\in B_{esp}}|T_{z}'| \label{eq:4}\\
		\displaystyle\sum_{u\in B_{esp}}|T_{u^b}| = 
		\displaystyle\sum_{z\in F_{esp}}|T_{z}| %\nonumber
		&\Rightarrow
		\displaystyle\sum_{u\in B_{esp}}|T_{u^b}'| = 
		\displaystyle\sum_{z\in F_{esp}}|T_{z}'| \label{eq:3}.
	\end{align}

	Note que para todo vértice b-especial~$z$ todos os vértices 
	de~$H^b_z$ e de~$P^b_z$ mapeiam um vértice 
	de~$T_z$ e que nem todos os vértices de~$T_z$ são mapeados por
	vértices~$H^b_z\cup P_z^b$, 
	dado que~$\map^-_m(z)\not\in H^b_z\cup P_z^b$.
	O mesmo ocorre com os vértices f-especiais~$w$.
	Portanto temos que
	\begin{align}
		|T_z| > |P^b_z| + |H^b_z| 
		&\Rightarrow
		|T_z'| \ge |P^b_z| + |H^b_z| \label{eq:5}\\
		|T_w| > |P^f_w| + |H^f_w|
		&\Rightarrow
		|T_w'| \ge |P^f_w| + |H^f_w| \label{eq:6}.
	\end{align}

	Como~$P^b_z$ é disjunto e~$P^b_z\cup H^b_z\cup T'_{z^b}$
	nada mais é do que o conjunto de sub-árvores enraizadas
	nos vértices de~$P^b_z$, então temos 
	que~${P^b_z\cup H^b_z\cup T'_{z^b}}$ também é disjunto.
	Como todos os vértices de~$P$ são mapeados em alguma sub-árvore,
	então todo vértice de~$V(P)$ pertence a algum~$P^b_z$, logo,
	todo vértice de~$V(T)$ pertence a 
	algum~${P^b_z\cup H^b_z\cup T'_{z^b}}$.
	O mesmo é válido para~${P^f_z\cup H^f_z\cup T'_{z^f}}$.
	Portanto
	\begin{align}
		(1-d)n = |V\setminus V(P)| &= 
		\displaystyle\sum_{z\in B_{esp}}\Big(|T_{z^b}'|+|H_z^b|\Big)
		\label{eq:7}\\
		(1-d)n = |V\setminus V(P)| &= 
		\displaystyle\sum_{z\in F_{esp}}\Big(|T_{z^f}'|+|H_z^f|\Big)
		\label{eq:9}\\
		dn = |V(P)| &= 
		\displaystyle\sum_{z\in B_{esp}}|P_z^b|
		\label{eq:8}\\
		dn = |V(P)| &= 
		\displaystyle\sum_{z\in F_{esp}}|P_z^f| 
		\label{eq:10}.
	\end{align}

	Usando as equações~\ref{eq:7},~\ref{eq:9},~\ref{eq:8} e~\ref{eq:10}
	e, as equações~\ref{eq:4} e~\ref{eq:3} depois, temos que
	\begin{align}
		\dfrac{1-d}{d} &= 
		\dfrac{
			\displaystyle\sum_{z\in B_{esp}}
			\Big(|T'_{z^b}|+|H_z^b|\Big) + 
			\displaystyle\sum_{z\in F_{esp}}
			\Big(|T'_{z^f}|+|H_z^f|\Big)}
			{\displaystyle\sum_{z\in B_{esp}}|P_z^b| +
			\displaystyle\sum_{z\in F_{esp}}|P_z^f|}
			\nonumber\\
		&= \dfrac{
			\displaystyle\sum_{z\in B_{esp}}
			\Big(|T'_{z}|+|H_z^b|\Big) + 
			\displaystyle\sum_{z\in F_{esp}}
			\Big(|T'_{z}|+|H_z^f|\Big)}
			{\displaystyle\sum_{z\in B_{esp}}|P_z^b| +
			\displaystyle\sum_{z\in F_{esp}}|P_z^f|}
			\nonumber \\
			\Rightarrow	
			\dfrac{1-d}{d}\displaystyle\sum_{z\in B_{esp}}|P_z^b| &+ 
			\dfrac{1-d}{d}\displaystyle\sum_{z\in F_{esp}}|P_z^f|
			= \displaystyle\sum_{z\in B_{esp}}
			\Big(|T'_{z}|+|H_z^b|\Big) + 
			\displaystyle\sum_{z\in F_{esp}}
			\Big(|T'_{z}|+|H_z^f|\Big)\nonumber
	\end{align}

	\newpage

	Isso nos leva a dois casos
	\begin{enumerate}[label=(\alph*)]
		\item $\displaystyle\sum_{z\in B_{esp}}
			\Big(|T'_{z}|+|H_z^b|\Big)\le
			\dfrac{1-d}{d}\displaystyle\sum_{z\in B_{esp}}|P_z^b|$
			ou
		\item $\displaystyle\sum_{z\in B_{esp}}
			\Big(|T'_{z}|+|H_z^b|\Big)>
			\dfrac{1-d}{d}\displaystyle\sum_{z\in B_{esp}}|P_z^b|
			\Rightarrow \\
			\displaystyle\sum_{z\in F_{esp}}
			\Big(|T'_{z}|+|H_z^f|\Big)<
			\dfrac{1-d}{d}\displaystyle\sum_{z\in F_{esp}}|P_z^f|
			\Rightarrow \\
			\displaystyle\sum_{z\in F_{esp}}
			\Big(|T'_{z}|+|H_z^f|\Big)\le
			\dfrac{1-d}{d}\displaystyle\sum_{z\in F_{esp}}|P_z^f|$.
	\end{enumerate}

	Sabemos que se o caso (a) for verdadeiro, teremos um 
	vértice~$z\in B_{esp}$ tal que~${|T'_z|+|H^b_z|\le |P^b_z|}$,
	pois se todos os vértices em~$B_{esp}$ 
	satisfizerem~${|T'_z|+|H^b_z| > |P^b_z|}$, então o caso (a)
	não será verdadeiro.
	O mesmo ocorre com o caso (b). 
	Portanto temos um dos casos
	a seguir sempre será válido.

	\begin{enumerate}[label=(\alph*)]
		\item existe um vértice~$z\in B_{esp}$ 
			tal que $|T'_{z}|+|H_z^b|\le
			\dfrac{1-d}{d}|P_z^b|$ ou
		\item existe um vértice~$z\in F_{esp}$ 
			tal que $|T'_{z}|+|H_z^f|\le
			\dfrac{1-d}{d}|P_z^f|$
	\end{enumerate}

	\bigskip
	\bigskip
	
	\subsubsection*{Caso (a) seja verdadeiro:}
		Existe um vértice~$z\in B_{esp}$ 
		tal que ${|T'_{z}|+|H_z^b|\le
		\dfrac{1-d}{d}|P_z^b|}$, logo, temos
		que~${|H^b_z|\le \Big(\dfrac{1}{d}-1\Big)|P^b_z|-|T'_z|}$.
		Usando a equação~\ref{eq:5}, temos 
		que
		\begin{align}
			|H^b_z|\le\Big(\dfrac{1}{2d}-1\Big)|P^b_z|
			\Rightarrow
			|H^b_z|+|P^b_z|\le\dfrac{1}{2d}|P^b_z| \nonumber
		\end{align}

		Construiremos o corte~$(B,W,S)$ de forma 
		que~$S = |H^b_z|+|P^b_z|$.
		Sabemos que~$S\ne \emptyset$, pois~$z$ é b-especial,
		o que implica em~$P^b_z\ne \emptyset$. 
		Dessa forma, temos que

		\begin{align}
			|S| = |H^b_z|+|P^b_z|\le\dfrac{1}{2d}|P^b_z|
			&\Rightarrow
			|S| \le\dfrac{1}{2d}|P^b_z| \nonumber \\
			&\Rightarrow
			|S| \le\dfrac{1}{2d}|P^b_z|\le \dfrac{n}{2} 
			\label{eq:metade} \\
			&\Rightarrow
			2d\le \dfrac{1}{|S|}|P^b_z|  
			\label{eq:reaproveitaCaminho}\\
			&\Rightarrow
			2d\le \dfrac{1}{|S|}|P^b_z|\le \diam^*(T[S]) 
			\nonumber
		\end{align}


		Seja o vértice~${z^b_l\in P^b_z}$ tal 
		que~$\map^+_m(z^b_l)$ é o vértice de~$T_z$ com 
		maior rótulo possível, e~$j$ é o vértice que vem
		antes de~$z$ em~$P$. Assumiremos que
		\begin{align}
			B_1 = \Big\{ v\in V(T):z^b_l<v\le j \Big\}
			\nonumber.
		\end{align}

		Agora podemos usar o Lema~\ref{lema:approxCutForest}
		na componente conexa~$T_z'$, usando~${m-|B_1|}$ 
		e~${2-\dfrac{1}{1-d}}$ como os valores de~$m$ e~$c$,
		respectivamente.
		Dessa forma, teremos um corte~$(B_2,W_2)$ tal 
		que~${cm\le|B_2|\le m}$ e
		\begin{align}
		e_{T_z'}(B_2,W_2)&\le \ceil[\Big]{ \dfrac{2c}{1-c}} 
		\Delta(T_z') = \ceil[\Big]{ \dfrac{2}{d}-4} \Delta(T_z') 
		\nonumber\\
		&\le \ceil[\Big]{ \dfrac{2}{d}-4} \Delta(T) 
		\nonumber\\
		&\le \Big( \dfrac{2}{d}-3 \Big) \Delta(T).
		\label{eq:corte}
		\end{align}
		Como temos que~${\map^+_m(z^b_l)\in T_z}$ e que todos
		os vértices entre~$z^b_l+1$ e~$j$ estão em~$B_1$,
		então~${m-|B_1|<|T_z|}$. Logo,~${1\le m-|B_1|\le |T'_z|}$.
		Como~$d\in (0,\frac{1}{2}]$, então, temos 
		que~${c\in [0,1)}$.
		Portanto, os valores passados como~$m$ e~$c$ são 
		compatíveis com o que é pedido no 
		Lema~\ref{lema:approxCutForest}.
		A disposição dos vértices ficará como o representado a
		seguir.



		\begin{center} \begin{tikzpicture}
		[scale=.7,auto=left,every node/.style={circle, draw=black,
				fill=white!70}]

		\draw [draw=red!50, line width=2pt, fill=red!17](9,2.5) rectangle (16,6);

		\draw [draw=Emerald, line width=2pt, fill=Emerald!20]
		(8.7,1.4) -- (8.7,7.4) -- (1.8,7.4) -- (1.8,4.5) -- 
		(3.7,4.5) -- (3.7,1.4) -- cycle;

		\draw [draw=CornflowerBlue, line width=2pt, fill=CornflowerBlue!30]
		(16.3,1.4) -- (17.3,3.9) -- (18.3,1.4) -- cycle;

		\draw [decorate,decoration={brace,amplitude=10pt,mirror,
		raise=4pt}, draw=brown, line width=1.5pt] 
		(8,5.7) -- (2,5.7);

		\draw [decorate,decoration={brace,amplitude=10pt,mirror,
		raise=4pt}, draw=Emerald, line width=1.5pt] 
		(4,3.2) -- (8.5,3.2);
		


		\node [scale=1.3,subtree,fill=white](y1) at (2.5,5) {};
		\node [subtree, fill=Emerald]      (y2) at (5,5)   {};
		\node [subtree, fill=Emerald]      (y3) at (7.5,5) {};
		\node [subtree,fill=red!50]        (y4) at (10,5)  {};
		\node [subtree,fill=red!50]        (y5) at (12.5,5){};
		\node [subtree,fill=red!50]        (y6) at (15,5)  {};
		\node [scale=1.9, subtree,fill=CornflowerBlue!70] (y7) at (18.1,5)  {};

		\node (x0) at (0,5)  {...};
		\node [fill=brown](x1) at (2.5,5)    {};
		\node [fill=brown](x2) at (5,5)  {};
		\node [fill=brown](x3) at (7.5,5)    {};
		\node (x4) at (10,5) {};
		\node (x5) at (12.5,5)   {};
		\node (x6) at (15,5) {};
		\node (x7) at (18.1,5) {};
		\node (x8) at (20.6,5) {...};

		\node [draw=none, fill=none] at (7.6,5.7){$z^b_l$};
		\node [draw=none, fill=none] at (2.6,5.7){$z^b$};
		\node [draw=none, fill=none] at (18.1,5.6){$z$};
		\node [draw=none, fill=none] at (15,5.6){$f$};
		\node [text=Brown,draw=none, fill=none] at (5,7){$P^b_z$};
		\node [text=Emerald,draw=none, fill=none] at (6.25,1.8){$H^b_z$};

		\node [text=Emerald,draw=none, fill=none] at (6.2,0.8){$\mathbf{S}$};
		\node [text=red!50,draw=none, fill=none] at (12.5,1.9){$\mathbf{B_1}$};
		\node [text=CornflowerBlue,draw=none,fill=none] at 
		(17.3,0.8){$\mathbf{B_2}$};

		\foreach \from/\to in {x0/x1,x1/x2,x2/x3,x3/x4,x4/x5,x5/x6,x6/x7,x7/x8}
		\draw (\from) -- (\to);

		\draw [draw=Emerald, line width=2pt]
		(8.7,1.4) -- (8.7,7.4) -- (1.8,7.4) -- (1.8,4.5) -- 
		(3.7,4.5) -- (3.7,1.4) -- cycle;

		\draw [draw=CornflowerBlue, line width=2pt]
		(16.3,1.4) -- (17.3,3.9) -- (18.3,1.4) -- cycle;

	\end{tikzpicture} \end{center}


		Temos, então, o corte~$(B,W,S)$ com~${B = B_1\cup B_2}$ 
		e~${W = V(T)\setminus (B\cup S)}$.
		Dado que~${\map^+_m(z^b_l)\in T_z}$, temos 
		que~${m<|B_1|}$ e, como obtemos~$B_2$ através do 
		Lema~\ref{lema:approxCutForest} usando~$m-|B_1|$ 
		como~$m$, então teremos que~${B_2\le m-|B_1|}$, implicando
		em~${m\ge|B|}$.
		De acordo com o Lema~\ref{lema:approxCutForest} também temos 
		que~${|B_2|\ge c\cdot(m-|B_1|)}$ e, como estamos no caso~2, 
		então~$d\le \dfrac{1}{2}$ e~$|P^b_z|\le |S|$, dado que
		cada vértice de~$P^b_z$ irá mapear em um vértice diferente 
		de~$S$.
		Portanto
		\begin{align}
			m-|B| &= m-|B_1|-|B_2| \nonumber\\
			&\le(m-|B_1|)-c\cdot (m-|B_1|) = (m-|B_1|)(1-c)
			\nonumber \\
			&\le|T_z'|(1-c) = |T_z'|\dfrac{d}{1-d} \le
			|T_z'|\dfrac{\frac{1}{2}}{1-\frac{1}{2}} = |T_z'|
			\nonumber \\
			&\le |P^b_z| \le |S| \nonumber \\ 
			\nonumber\\
			\Rightarrow m\le |&B|+|S| \Rightarrow |B|\le 
			m\le |B|+|S|. \nonumber
		\end{align}

		Agora mostraremos 
		que~${e_T(B,W,S)\le \dfrac{2\cdot \Delta(T)}{\diam^*(T)}}$.
		$\\$
		% Sabemos que os vértices do corte~$(B,W,S)$ são as arestas
		% que ligam~$z^b_l$ e~$f$ aos vértices que vêm depois deles 
		% em~$P$
		Temos que o número de arestas no corte~$(B,W,S)$ é, no 
		máximo,~${(\Delta(T)-1) + 1 + 1 + e_{T_z'}(B_2,W_2) +
		(\Delta(T)-2)}$.
		Essa soma representa a quantidade total de arestas, sendo elas
		as que conectam~$z^b$
		a~$W$, a que liga~$z^b_l$ a~$B_1$, a
		que conecta~$f$ e~$z$, as presentes no corte~$(B_2,W_2)$ e, 
		por fim, as arestas que ligam~$z$ a~$T_z'$ (talvez
		existam arestas ligando~$B_2$ a~$z$).
		Logo, usando a equação~\ref{eq:corte}, temos que
		\begin{align}
			e_T(B,W,S) &= (\Delta(T)-1) + 1 + 1 + e_{T_z'}(B_2,W_2) +
			(\Delta(T)-2) \nonumber\\
			&= 2\cdot\Delta(T) + e_{T_z'}(B_2,W_2) - 1 
			\nonumber\\
			&\le 2\cdot\Delta(T) + \Big( \dfrac{2}{d}-3 \Big) 
			\Delta(T) - 1 = \Big( \dfrac{2}{d}-1 \Big)\Delta(T)-1 
			\nonumber\\
			&< \dfrac{2\cdot \Delta(T)}{d}. \nonumber
		\end{align}

		Portanto, se as condições do caso~(a) forem satisfeitas, então 
		existe um corte~$(B,W,S)$ que satisfaz as propriedades do
		Teorema~\ref{teo:dobraDiametro}.

	\bigskip
	\bigskip
	\newpage
	
	\subsubsection*{Caso (b) seja verdadeiro:}

	\begin{center} \begin{tikzpicture}
		[scale=.7,auto=left,every node/.style={circle, draw=black,
				fill=white!70}]

		\draw [draw=red!50, line width=2pt, fill=red!17]
		(-8.83,2.5) -- (-8.83,4.57) -- (-10.93,4.57) -- (-10.93,6) -- 
		(-16,6) -- (-16,2.5) -- cycle;

		\draw [draw=Emerald, line width=2pt, fill=Emerald!20]
		(-8.7,1.4) -- (-8.7,4.7) -- (-10.8,4.7) -- (-10.8,7.4) -- 
		(-3.7,7.4) -- (-3.7,1.4) -- cycle;

		\draw [draw=CornflowerBlue, line width=2pt, fill=CornflowerBlue!30]
		(-16.3,1.4) -- (-17.3,3.9) -- (-18.3,1.4) -- cycle;

		\draw [decorate,decoration={brace,amplitude=10pt,mirror,
		raise=4pt}, draw=brown, line width=1.5pt] 
		(-4.5,5.7) -- (-10.5,5.7);

		\draw [decorate,decoration={brace,amplitude=10pt,mirror,
		raise=4pt}, draw=Emerald, line width=1.5pt] 
		(-8.5,3.2) -- (-4,3.2);
		


		\node [scale=1.3,subtree,fill=white](y1) at (-2.5,5) {};
		\node [subtree, fill=Emerald]      (y2) at (-5,5)   {};
		\node [subtree, fill=Emerald]      (y3) at (-7.5,5) {};
		\node [subtree,fill=red!50]        (y4) at (-10,5)  {};
		\node [subtree,fill=red!50]        (y5) at (-12.5,5){};
		\node [subtree,fill=red!50]        (y6) at (-15,5)  {};
		\node [scale=1.9, subtree,fill=CornflowerBlue!70] (y7) at (-18.1,5)  {};

		\node (x0) at (0,5)  {...};
		\node (x1) at (-2.5,5)    {};
		\node [fill=brown](x2) at (-5,5)  {};
		\node [fill=brown](x3) at (-7.5,5)    {};
		\node [fill=brown](x4) at (-10,5) {};
		\node (x5) at (-12.5,5)   {};
		\node (x6) at (-15,5) {};
		\node (x7) at (-18.1,5) {};
		\node (x8) at (-20.6,5) {...};

		\node [draw=none, fill=none] at (-10,5.7){$z^f$};
		\node [draw=none, fill=none] at (-18.1,5.6){$z$};
		\node [text=Brown,draw=none, fill=none] at (-7.5,7){$P^f_z$};
		\node [text=Emerald,draw=none, fill=none] at (-6.25,1.8){$H^f_z$};

		\node [text=Emerald,draw=none, fill=none] at (-6.2,0.8){$\mathbf{S}$};
		\node [text=red!50,draw=none, fill=none] at (-12.5,1.9){$\mathbf{B_1}$};
		\node [text=CornflowerBlue,draw=none,fill=none] at 
		(-17.3,0.8){$\mathbf{B_2}$};

		\foreach \from/\to in {x0/x1,x1/x2,x2/x3,x3/x4,x4/x5,x5/x6,x6/x7,x7/x8}
		\draw (\from) -- (\to);

		\draw [draw=red!50, line width=2pt]
		(-8.83,2.5) -- (-8.83,4.57) -- (-10.93,4.57) -- (-10.93,6) -- 
		(-16,6) -- (-16,2.5) -- cycle;

		\draw [draw=Emerald, line width=2pt]
		(-8.7,1.4) -- (-8.7,4.7) -- (-10.8,4.7) -- (-10.8,7.4) -- 
		(-3.7,7.4) -- (-3.7,1.4) -- cycle;

		\draw [draw=CornflowerBlue, line width=2pt]
		(-16.3,1.4) -- (-17.3,3.9) -- (-18.3,1.4) -- cycle;

	\end{tikzpicture} \end{center}

	


	\bigskip
	\bigskip
	\bigskip
%%%%%%%%%%%%%%%%%%%%%%%%%%%%%%%%%%%%%%%%%%%%%%%%%%%%%%%%%%%%%%%%%%
%%%%%%%%%%%%%%%%%%%%%%%%%%%%%%%%%%%%%%%%%%%%%%%%%%%%%%%%%%%%%%%%%%
%%%%%%%%%%%%%%%%%%%%%%%%%%%%%%%%%%%%%%%%%%%%%%%%%%%%%%%%%%%%%%%%%%

	\subsection{Algoritmo da dobra do diâmetro}
	\begin{algorithm}[H]
	\label{alg:dobraDiametro}
		\SetKwInOut{Input}{input}
		\SetKwInOut{Output}{output}

		\caption{}
		\Input{árvore $T$ com $n$ vértices e ${m\in[n]}$}
		\Output{}

	\end{algorithm}	